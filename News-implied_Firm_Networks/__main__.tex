%\documentclass[12pt,article]{memoir}
\documentclass[12pt,a4paper]{article}

%---------------------- MACROS -----------------------
\usepackage{/Users/jesusvillotamiranda/Documents/LaTeX/JVM_Macros}
%----------------------------------------------------

\usepackage{authblk} % to write the title with the authors
\usepackage{booktabs} % when using \documentclass[12pt,a4paper]{article}, you need to load this

%%%%%%%%%%%%%%%%%%%%%%%%%%%%%%%%%%%%%%%%%%%%%%%%%%%%%
%%%%%%%%%%%%%%%%%%%   TITLE     %%%%%%%%%%%%%%%%%%%%%
%%%%%%%%%%%%%%%%%%%%%%%%%%%%%%%%%%%%%%%%%%%%%%%%%%%%%

\title{\textsc{
{\LARGE Network Effects: Using Large Language Models to Map Complex Firm Relationships from News Data}
%\\
%{\large News- }
}}

\author[1]{
{ \bx \bx \bx Insert authors about here
%$^{\dagger}$
%\footnote{
%I am grateful to ...
%I acknowledge financial support from ...
%}
}

%\bx 
%%(\textsc{cemfi})
%{\small
%$\big{<}$
%\noindent $^{\dagger}$CEMFI, Calle Casado del Alisal, 5, 28014 Madrid, Spain 
%$\big{>}$
%
%$\big{<}$
%Email: \href{jesus.villota@cemfi.edu.es}{\texttt{jesus.villota@cemfi.edu.es}}
%$\big{>}$
%}
}

\date{}  % Remove date


%%%%%%%%%%%%%%%%%%%%%%%%%%%%%%%%%%%%%%%%%%%%%%%%%%%%%
\begin{document}
%%%%%%%%%%%%%%%%%%%%%%%%%%%%%%%%%%%%%%%%%%%%%%%%%%%%%
\maketitle
\thispagestyle{empty}  % Suppresses the page number on this page


\begin{abstract}
%----------------------------------------------------
%%\begin{abstract}
%%%%%%%%%%%%%%%%% THIS VERSION: 192 WORDS %%%%%%%%%%%%%%%%%%%%%
In this paper, we explore the incorporation of information in the stock market by building a trading strategy that trades clusters of news. First, we construct the clusters by applying KMeans to the vector embedding representations of the articles. We then propose two algorithms that select \textit{in-sample} the most profitable clusters for trading and project them \textit{out-of-sample} to evaluate the performance of the trading strategy. 
%\mx 
In the second part, we perform the clustering by feeding the news articles to a Large Language Model and asking it to classify the type, magnitude, and direction of the shocks implied by each article on the affected firms according to a predefined schema. We then apply the same algorithms for cluster selection as before and check the performance of our trading strategy.
%
The results show an inconsistent earnings profile in the test set for the KMeans clustering, while the LLM clustering is able to exploit the information of the articles in a profitable way, creating a consistent earnings profile. These results are robust to hyperparameter variability, such as the number of periods in the holding window and the upper bound on the number of traded clusters. 
%%%%%%%%%%%%%%%%%%%%%%%%%%%%%%%%%%%%%%%%%%%%%%%%%%%%%

%\end{abstract}
%----------------------------------------------------
This paper introduces a novel methodology for constructing and analyzing firm networks using Large Language Models (LLMs) applied to news article data. Unlike traditional approaches that rely on simple co-occurrence patterns, our method leverages LLMs to detect and classify meaningful economic relationships between firms. The framework identifies six distinct types of firm relationships: supplier-customer, competitive, partnership, mergers and acquisitions, legal disputes, and other interactions. We develop a mathematical framework that incorporates both directionality (asymmetric relationships) and reflexivity (self-relationships), resulting in a multi-layered network representation. Each layer corresponds to a specific relationship type, with edges weighted by LLM-derived relationship scores. Our methodology enables dynamic network analysis, tracking how firm relationships evolve over time and respond to external events. This enhanced network representation offers new insights into supply chain structures, competitive landscapes, and the propagation of economic shocks through firm networks.

\bx

\noindent\textbf{JEL Codes:}  C45, C55, G14, L14

\noindent\textbf{Keywords:} Firm Networks, Large Language Models, Natural Language Processing, Network Analysis, Corporate Relationships
\end{abstract}


%%%%%%%% TABLE OF CONTENTS %%%%%%%%%%%
\newpage
\tableofcontents
\thispagestyle{empty}  % Suppresses the page number on this page

\newpage
\setcounter{page}{1}
%%%%%%%%%%%% INTRODUCTION %%%%%%%%%%%%%%%%%%
%\section{Introduction, Literature Review \& Structure}
%----------------------------------------------------
%

%----------------------------------------------------
% 18 June 2024 | Raw text from my own writing
%----------------------------------------------------

\hspace{0.5cm} This paper aims to provide a novel and universal approach to analyzing the impact of business news on stock prices. Our approach is novel in that it is the first time that Large Language Models are employed to \textit{comprehensively} analyze the shocks described in business news articles for return prediction purposes, and it is universal in the sense that it does not rely on access to structured metadata from paid news portals, which, in general, are not widely accessible to the common researcher.

\mx 
Our database consists of a set of Spanish business news articles from DowJones spanning June 2020 to September 2021. Such articles are filtered in a way that allows us to extract the firms directly involved in them. In the first exercise, we convert the wording of the articles from text to high-dimensional vector embeddings by using a transformer model. Such representation captures the general contextual and semantic meaning of the text but is not able to capture the subtle nuances of the shocks described there on the affected firms. 

\mx 
We then cluster our news articles by applying the KMeans algorithm to the associated vector embeddings. This procedure delivers 26 clusters of business news, where each cluster usually pools together articles about a firm or set of firms in the same sector. 

\mx 
For each firm affected by an article, a market model is constructed on some window previous to the day where the article's information got incorporated into the market. Such model is then used to construct a beta-neutral strategy that extracts the abnormal returns of the firm after controlling for the market. 
%\mx 
By obtaining the metrics of this strategy and comparing them across clusters, we can obtain a measure of the profitability of each cluster. We then propose two algorithms that exploit this information to select the optimal clusters.
%to build a trading rule. 

\mx 
Finally, a trading rule is constructed by launching trades on the selected clusters over a specific holding period. By projecting the trading rule onto the test set, we obtain a measure of the profitability of the whole procedure out-of-sample. 

\mx 
The strategy based on KMeans clustering of vector embeddings is not able to generate a consistent earnings profile in the test set, which occurs due to the instability of the clustering method. Namely, the distribution of articles through clusters across data splits shows a very inconsistent pattern, which already hints at the fact that signals generated by the trading rule will not be generalizable out-of-sample.

\mx 
In the second part of the paper, we feed the news articles to a Large Language Model (LLM) and ask it to manually parse them. In particular, we ask the LLM to extract the affected firms and to individually classify the shock implied by the article in each affected firm based on a predefined schema. Such schema consists of a classification of news articles' shocks based on three categories: type (demand, supply, financial, technology, policy), magnitude (minor, major), and direction (positive, negative). We can then cluster the articles based on this classification. 

\mx
In this case, the distribution of articles through clusters is very stable across data splits, which indicates that the trading rule will generate signals that will be generalizable across data splits. Indeed, this is confirmed by the out-of-sample performance of the trading strategy, which shows a consistent earnings profile. These results are robust to hyperparameter variability. In particular, we show that the distribution of Sharpe Ratios in the test set for different choices of holding period length and maximum number of traded clusters is right-skewed and centered at positive values.

%----------------------------------------------------
% 29 April 2024 | Raw text from my own writing
%----------------------------------------------------

%This paper aims to provide a novel and universal approach to analyzing the impact of business news on stock prices. Our approach is novel in that it is the first time that Large Language Models are employed to comprehensively analyze the shocks described in business news articles for return prediction purposes, and it is universal in the sense that it does not rely on access to structured metadata from paid news portals, which, in general, are not widely accessible to the common researcher.


%
%\mx 
%We start with an unstructured corpus of textual data, namely Spanish business news articles, which undergoes preprocessing before being fed into the GPT API. Subsequently, we guide GPT in parsing these news articles and generating structured responses using "function calling", i.e., a predefined set of functions and parameters of our writing to rigorously instruct GPT on how to respond. 
%
%\mx
%Such functions prompt GPT to identify various attributes of the article, including its publication datetime, the type of information provided (new information, historical, analysis/comments, marketing) and its scope (firm-level, industry, global). If the scope of the article is at the firm level, we further ask GPT to list the firms that are directly affected by the events narrated therein. For each identified firm, we request GPT to provide its associated stock market ticker (if publicly traded; otherwise, it returns "NaN"), and further, to classify several aspects regarding how the shock pertains to the firm. Specifically, we prompt GPT to classify the type of shock (e.g., demand, supply, regulatory), the expected duration (short-term, long-term, permanent), the magnitude or relevance (minor, mild, major), and the expected impact on the firm's performance (positive, neutral, negative). Lastly, we obtain GPT's own trading signal by putting it in the shoes of a financial advisor tasked with deciding whether to \textit{Long} or \textit{Short} the stock associated with the firm affected by the news article.
%
%\mx 
%The methodological framework outlined above not only facilitates the identification of pertinent metadata but also generates a structured and comparable array of responses, enabling us to progress the analysis to a supervised stage, which consists of two parts.
%
%\mx 
%In the first part, we examine the predictability of stock returns in response to the shocks delineated in the news articles. To achieve this, we will conduct regressions of the stock returns of affected firms on the dummified shock classifications provided by GPT. This analysis will elucidate the direction and significance of shock predictors for return prediction.
%
%\mx 
%In the second part, we will assess the market timing capabilities of GPT through a market timing test. This involves contrasting GPT's decisions with the decisions that should have been made based on realized returns. In other words, we will juxtapose the stock return predictions of GPT, which inform decisions to long or short the stock, with the actual stock return performance of the respective stock.
%
%\mx 
%Finally, we will construct a set of Long-Short portfolios. One of these portfolios will trade shock signals, determining whether to go long or short based on the shock classifications provided by GPT. Another portfolio will be created using GPT's raw market timing capabilities, disregarding the deeper understanding of the news article implied by the shock analysis and categorization.
%
%\mx 
%All in all, our approach allows us to transition from an unsupervised learning procedure with unstructured data to a supervised learning procedure with structured data that enables us to study the stock return predictability of the shocks described by business news articles.
%
%
%%%%%%%%%%%%%%%%%%%%%%%%%%%%%%%%%%%%%%%%%%%%%%%%%%%%%%%%%%%%%%%%%%%%%%%%%%
%%%%%%%%%%%%%%%%%%%%%%%%%%%%%%%%%%%%%%%%%%%%%%%%%%%%%%%%%%%%%%%%%%%%%%%%%%
%%%%%%%%%%%%%%%%%%%%%%%%%%%%%%%%%%%%%%%%%%%%%%%%%%%%%%%%%%%%%%%%%%%%%%%%%%
%%%%%%%%%%%%%%%%%%%%%%%%%%%%%%%%%%%%%%%%%%%%%%%%%%%%%%%%%%%%%%%%%%%%%%%%%%
%
%%\mx 
%%The methodological framework described above not only facilitates the identification of relevant metadata but also produces a structured and comparable set of responses that allows us to advance the analysis further to a supervised stage,
%
%
%%In the first part, we analyze the stock return predictability of the shocks described in the news articles. For this purpose, we will regress the stock returns of the affected firms on the dummified shock classifications made by GPT. This will allow us to shed light on the direction and significance of shock predictors for return prediction. 
%%
%%For the second part, we will analyze the market timing capabilities of GPT by performing a market timing test. Here we will contrast the decision made by GPT with the decision that should have been taken based on the realized returns. In n other words, we will compare the stock return predictions of GPT (which underlie in the decision made to long or short the stock) and the actual stock return performance of the stock in question. 
%
%%Finally, we will construct a set of Long-Short portfolios. One of such portfolio will trade shock signals; that is, it will decide whether to long/short based on the shock classifications made by GPT. Another portfolio will be constructed using GPT's raw market timing abilities (that is, by simply asking GPT to long or short based on the news and without regard to the deeper understanding on the news article implied by the shock analysis and categorization)
%
%
%%in which we study the stock return predictability of the shock analysis and market timing signals generated by GPT.
%
%%Namely, the output from GPT's response consists of a classification of the shocks implied by the news articles. We launch a set of queries to GPT: we ask it to identify the publication datetime of the article, the type of article (set of possible answers: new information, historical, analysis or comments, marketing) and its scope (firm-level, industry, global). If the scope of the article is at the firm-level, then, we ask GPT to list the firms that are primarily and directly affected by the events narrated in the article. Then, for each firm in that list of firms, we prompt GPT to give us its associated stock market ticker (if the company is publicly traded, otherwise, it spits ``NaN'') and we further ask GPT to classify a set of aspects of how that shock relates to the firm in question. In particular, we ask it about the shock type (demand, supply, regulation,...), the expected duration of that shock (set of possible answers: short-term, long-term, permanent), the magnitude or relevance (set of possible answers: minor, mild, or major), and the direction in which that shock is expected to affect the firm's performance (set of possible answers: positive, neutral negative). Further, we ask GPT to provide a trading signal based on the news article; namely, we put GPT in the shoes of a financial advisor having to decide on whether to Long or Short the stock associated to the affected firm based on the events described in the news article. Once we have obtained a structured answer from GPT we can obtain the metadata of the identified firms to analyze the stock return predictability of the shock analysis and market timing signals generated by GPT.
%%
%
%
%
%%In particular, our approach departs from an unstructured dataset of textual data (a corpus of Spanish news articles) that is preprocessed and fed to GPT. At this stage, we direct GPT on how to parse these news articles and generate a structured response through ``function calling'' (i.e., writing a set of functions and parameters to precisely instruct GPT on to output its completions). This methodology allows us to not only identify the relevant metadata but also to obtain a structured and comparable output that places us in the right place to take the analysis to a supervised stage.
%%---
%%
%%At this stage, a set of functions is defined to instruct GPT on how to parse and analyze the news articles. 
%%
%%At this stage, we task GPT to produce a set of answers according to a function schema that we predefined in advance. 
%%
%%
%%---
%
%%
%%. Different from the previous literature, our procedures don't rely on having access to structured metadata from payable news portals, which are not widely accessible to the common researcher. 
%%
%%In other words, our approach is universal in that it doesn't require metadata from news portals that require subscriptions. 
%%
%%In this sense, we obtain all the information by asking GPT. This is an unsupervised learning approach. 
%%
%%How do we get around this? By delegating the identification of all the relevant metadata to GPT. However, this delegation only works if the right textual preprocessing is performed and the right questions are asked to GPT.  
%%
%%The modus operandi consists of starting from an unsupervised learning approach with unstructured textual data (news articles), which we then preprocess and feed to GPT. 
%%
%%Departing from unstructured news article data, we parse those articles by passing them through the GPT API. 
%%
%%
%%This leads GPT to produce a structured output that can then be employed to analyze the stock return predictability of GPT. 
%%
%
%
%
%%Namely, the output from GPT's response consists of a classification of the shocks implied by the news articles. We launch a set of queries to GPT: we ask it to identify the publication datetime of the article, the type of article (set of possible answers: new information, historical, analysis or comments, marketing) and its scope (firm-level, industry, global). If the scope of the article is at the firm-level, then, we ask GPT to list the firms that are primarily and directly affected by the events narrated in the article. Then, for each firm in that list of firms, we prompt GPT to give us its associated stock market ticker (if the company is publicly traded, otherwise, it spits ``NaN'') and we further ask GPT to classify a set of aspects of how that shock relates to the firm in question. In particular, we ask it about the shock type (demand, supply, regulation,...), the expected duration of that shock (set of possible answers: short-term, long-term, permanent), the magnitude or relevance (set of possible answers: minor, mild, or major), and the direction in which that shock is expected to affect the firm's performance (set of possible answers: positive, neutral negative). Further, we ask GPT to provide a trading signal based on the news article; namely, we put GPT in the shoes of a financial advisor having to decide on whether to Long or Short the stock associated to the affected firm based on the events described in the news article. Once we have obtained a structured answer from GPT we can obtain the metadata of the identified firms to analyze the stock return predictability of the shock analysis and market timing signals generated by GPT.
%
%
%% That is, our approach allows us to transition from an unsupervised learning with unstructured data to a supervised learning procedure with structured data. This latter procedure consists of two parts.
%
%
%%%%%%%%%%%%%%%%%%%%%%%%%%%%%%%%%%%%%%%%%%%%%%%%%%%%%%
%%----------------------------------------------------
%% 29 April 2024 | Parsed using GPT (asked it to simply rephrase my text)
%%----------------------------------------------------
%%%%%%%%%%%%%%%%%%%%%%%%%%%%%%%%%%%%%%%%%%%%%%%%%%%%%%
%
%%This paper endeavors to establish a methodical framework for the analysis of business news articles, aiming to transform unstructured news data into structured insights through the utilization of the GPT API.
%%
%%By instructing GPT through a series of predefined functions, we orchestrate the parsing of news articles to yield structured responses. This approach facilitates the extraction of comparable insights, primarily focusing on classifying the shocks conveyed within the articles. Our methodology involves querying GPT to ascertain key attributes of the articles, including publication datetime, article type (such as new information, historical context, analysis, or commentary), and scope (ranging from firm-level to industry or global perspectives).
%%
%%When the scope pertains to specific firms, we prompt GPT to identify the principal firms directly impacted by the events narrated in the article. Subsequently, for each identified firm, we solicit GPT to classify various aspects of the shock, encompassing its type (such as demand, supply, or regulatory), expected duration (ranging from short-term to long-term or permanent), magnitude or relevance (ranging from minor to major), and anticipated direction of impact on the firm's performance (positive, neutral, or negative). Additionally, we task GPT with generating trading signals based on the news articles, simulating the role of a financial advisor deciding whether to Long or Short the stock associated with the affected firm based on the article's content.
%%
%%The initial phase of our study focuses on analyzing the predictability of stock returns based on these shock classifications. Employing regression analysis, we explore the relationship between stock returns of affected firms and the identified shock predictors, examining both the direction and significance of these predictors.
%%
%%Subsequently, we delve into assessing the market timing capabilities of GPT through a rigorous market timing test. This involves contrasting GPT's decisions regarding stock returns with the ideal decisions that should have been made, thereby evaluating the effectiveness of GPT's predictions in real-world market scenarios.
%%
%%In the final phase, we construct a series of Long-Short portfolios to further explore the practical implications of GPT's insights. One such portfolio will trade based on shock signals, leveraging GPT's shock classifications to inform Long or Short decisions. Another portfolio will utilize GPT's raw market timing abilities, disregarding deeper shock analysis, and solely relying on GPT's instantaneous market predictions derived from news articles.
%
%
%
%%%%%%%%%%%%%%%%%%%%%%%%%%%%%%%%%%%%%%%%%%%%%%%%%%%%%%
%%----------------------------------------------------
%% 29 April 2024 | Parsed using GPT (asked it to do whatever it want with my text)
%%----------------------------------------------------
%%%%%%%%%%%%%%%%%%%%%%%%%%%%%%%%%%%%%%%%%%%%%%%%%%%%%%
%%
%%This paper presents a systematic framework for analyzing business news articles using the GPT API. Departing from unstructured data, we employ a structured approach by instructing GPT through predefined functions, enabling the extraction of comparable and analyzable responses.
%%
%%Our methodology begins by parsing articles through the GPT API, directing it to identify key attributes such as publication datetime, article type, and scope. For articles focused on individual firms, GPT identifies the primary affected firms and classifies aspects of the associated shock, including type, duration, magnitude, and expected impact direction.
%%
%%Furthermore, we task GPT with providing trading signals based on the news content, simulating the decision-making process of a financial advisor. This approach lays the foundation for our subsequent analyses.
%%
%%In the first part of the paper, we investigate the predictability of stock returns based on GPT's shock classifications. Through regression analysis, we examine the significance and directionality of these predictors.
%%
%%Subsequently, we assess GPT's market timing capabilities by comparing its stock return predictions with actual performance. This market timing test evaluates the efficacy of GPT's decision-making process in real-world scenarios.
%%
%%Finally, we construct Long-Short portfolios to further explore GPT's capabilities. One portfolio trades based on GPT's shock signals, while another relies solely on GPT's raw market timing abilities, providing insights into the impact of deeper news analysis.
%%
%%By employing this structured approach, we aim to enhance our understanding of the relationship between business news content and stock market movements, leveraging the capabilities of advanced language models like GPT. 
%%
%%
%%
%%

%----------------------------------------------------
%----------------------------------------------------
%%%%%%%%%%%%%%%%%%%%%%%%%%%%%%%%%%%%%%%%%%%%%%%%%%%%%%
%%%%%%%%%%% INTRODUCING THE LITERATURE %%%%%%%%%%%%%%
%%%%%%%%%%%%%%%%%%%%%%%%%%%%%%%%%%%%%%%%%%%%%%%%%%%%%
The foundation of textual analysis in finance, particularly in the predictive analysis of stock returns using economic news, is built upon several key studies that have shaped the current understanding and methodologies employed in this field. These seminal works not only introduced innovative approaches to analyzing textual data but also laid the groundwork for subsequent research that leverages advanced computational techniques.
%%%%%%%%%%%%%%%%%%%%%%%%%%%%%%%%%%%%%%%%%%%%%%%%%%%%%
%%%%%%%%%%%%%%%%%%%%%%%%%%%%%%%%%%%%%%%%%%%%%%%%%%%%%
%%%%%%%%%%%%%%%%%%%%%%%%%%%%%%%%%%%%%%%%%%%%%%%%%%%%%

\mx
\cite{tetlock2007giving}
's study is a pivotal contribution to textual analysis in finance. He examines the predictive power of negative words in a popular column from The Wall Street Journal on subsequent stock returns. His findings indicate that high levels of media pessimism predict lower future returns, suggesting that media sentiment embedded in news content can significantly influence market prices. This research underscores the role of media tone and sentiment in shaping investor behavior and market trends.

\mx
\cite{fang2009media}
explore the impact of media coverage on the stock returns of publicly traded firms. They find that stocks with less media coverage generate higher future returns compared to those with more extensive coverage. This effect is attributed to the media's role in disseminating information to the public, which influences investor awareness and risk perceptions. Their study provides crucial insights into how variations in media attention can affect market dynamics and investor behavior.

\mx
\cite{bollen2011twitter} 
extend the domain of textual analysis into the realm of social media, demonstrating how sentiment derived from Twitter feeds can forecast stock market movements. They analyze the correlation between collective mood states expressed on Twitter and the Dow Jones Industrial Average. Their results suggest that certain mood dimensions on Twitter can precede and predict changes in stock prices, highlighting the increasing relevance of social media sentiment in financial forecasting.

\mx
\cite{jegadeesh2013word} 
introduce a refined content analysis technique that quantifies the informational value of words in financial documents, such as 10-K filings. Their method prioritizes term relevance based on the market's response to these filings, moving beyond traditional sentiment dictionaries to identify terms that significantly impact stock returns. The study demonstrates that the market reacts more predictively to the nuanced term-weighting approach, highlighting its effectiveness in extracting financially relevant information from corporate disclosures. This innovative methodology provides deeper insights into the influence of textual nuances on market behavior.

\mx
\cite{baker2016measuring}
introduce the Economic Policy Uncertainty Index, a novel quantitative measure constructed from the frequency of newspaper coverage concerning economic uncertainty and policy matters. The authors employ this index to explore the broader economic impacts of policy uncertainty, demonstrating its significant correlation with reduced macroeconomic performance, such as lower investment levels and reduced employment rates. While not directly focused on stock market predictability, this research is pivotal for its methodological innovation in quantifying an abstract concept like uncertainty using textual analysis. The EPU Index's ability to capture the macroeconomic climate influenced by policy decisions makes it a crucial tool for understanding market conditions that indirectly affect stock valuations and investor behavior, thus providing essential context for any analysis related to financial markets and economic forecasting.


%%%%%%%%%%%%%%%%%%%%%%%%%%%%%%%%%%%%%%%%%%%%%%%%%%%%%
%%%%%%%%%%%%%%%%%   TRANSITION   %%%%%%%%%%%%%%%%%%%% 
%%%%%%%%%%%%%%%%%%%%%%%%%%%%%%%%%%%%%%%%%%%%%%%%%%%%%
As the field evolved, newer research began to address some of the limitations and assumptions inherent in earlier studies. More recently, advancements in machine learning and natural language processing have allowed researchers to dissect and leverage textual data with unprecedented granularity and accuracy. This shift towards more sophisticated models is exemplified by recent works that integrate deep learning technologies to parse and analyze economic news for market predictions more effectively.
%%%%%%%%%%%%%%%%%%%%%%%%%%%%%%%%%%%%%%%%%%%%%%%%%%%%
%%%%%%%%%%%%%%%%%%%%%%%%%%%%%%%%%%%%%%%%%%%%%%%%%%%%%
%%%%%%%%%%%%%%%%%%%%%%%%%%%%%%%%%%%%%%%%%%%%%%%%%%%%%

\mx 
%------------------  GPT PARSED ---------------------
% Ke, Z. T., Kelly, B. T., & Xiu, D. (2019). Predicting returns with text data (No. w26186). National Bureau of Economic Research.
%----------------------------------------------------

\cite{ke2019predicting} introduce a novel text-mining methodology that integrates machine learning techniques to predict asset returns from financial news articles. Their method, termed Sentiment Extraction via Screening and Topic Modeling (SESTM), innovates by constructing a sentiment scoring model tailored specifically for return prediction. SESTM operates through three steps: predictive screening to identify relevant terms, assigning weights to these terms using a supervised topic model, and aggregating these into an article-level predictive score through penalized likelihood. The empirical analysis, which leverages the Dow Jones Newswires, demonstrates that this model effectively extracts predictive signals from news content, highlighting that news assimilation into prices occurs with a delay, particularly for smaller and more volatile firms. While the SESTM methodology showcases significant advancements in return predictability using textual data, it primarily focuses on extracting general sentiment rather than dissecting the specific financial impacts of news content on stock prices. This approach may overlook the nuanced ways in which individual news items affect specific firms, an area where the methodology in this paper aims to contribute. 

%By employing a function schema that instructs GPT to analyze and parse news for detailed, firm-specific financial impacts, this paper proposes a more targeted analysis, enhancing the granularity and applicability of news-driven financial predictions. This method not only deepens the level of textual analysis beyond sentiment scoring but also aligns closely with practical needs for precision in financial decision-making.


\mx 
%------------------  GPT PARSED ---------------------
% Baker, S., Bloom, N., Davis, S. J., & Sammon, M. C. (2021). What triggers stock market jumps?. [National Bureau of Economic Research]
%----------------------------------------------------

%\cite{baker2021triggers}
%They integrate news text into macro-finance analyses using carefully curated researcher inputs in place of statistical models. (i.e: train graduate students to read news article and ask them to extract the sentiment from many news articles. This is an expensive and brute force approach. May be contaminated by human error and biases)
%Interesting: They analyze around 6000 news, but using human readers! In our case, we can do this for a big amount of news using GPT

\cite{baker2021triggers} utilizes a detailed analysis of newspaper accounts to determine the proximate causes of significant stock market jumps. Their method employs human readers to categorize 6,200 instances of market jumps from 16 national markets into 17 predefined categories, based on journalistic accounts. While this approach has the advantage of detailed human interpretation, it is susceptible to several issues: the cost and time required to train and deploy human coders are significant, and human bias may affect the consistency and neutrality of data categorization, as readers may inadvertently favor narratives that fit with prevailing economic theories or personal biases. Additionally, human coding may lack scalability and flexibility when adapting to new information or changing market dynamics. Using a Large Language Model for parsing news offers a compelling alternative. LLMs can process large volumes of data at scale and with consistent criteria, reducing the subjective bias introduced by human readers thus making it a more robust and cost-effective tool for financial analysis compared to the labor-intensive process of human coding.


%\mx 
%%------------------  GPT PARSED ---------------------
%% Kelly, B., Manela, A., & Moreira, A. (2021). Text selection. Journal of Business & Economic Statistics
%%----------------------------------------------------
%\cite{kelly2021text} introduce a novel approach to economic textual analysis with their Hurdle Distributed Multinomial Regression (HDMR) model. This model innovatively addresses the issue of sparsity in high-dimensional text data by focusing on the inclusion rather than frequency of phrases, aiming to capture economically significant phrases from a broad corpus. They demonstrate the model's efficacy through applications like analyzing U.S. Congressional speeches for partisanship and forecasting economic indicators from newspaper text, showing a marked improvement in predictive performance and interpretability over traditional models. However, while HDMR effectively handles text sparsity and enhances model interpretability, it may fall short in adapting to new data or evolving linguistic trends due to its static nature. Each phrase's significance is determined at the time of model training, potentially overlooking emerging terms or phrases that could gain economic significance later on. In contrast, the methodology in this paper, utilizing the dynamic capabilities of the GPT model, continuously adapts to new language usage and evolving contexts, maintaining its relevance and accuracy over time in financial news analysis. This adaptability makes it particularly suited to the fast-paced environment of stock market forecasting, where the relevance and impact of news can shift rapidly.



\mx 
%------------------  GPT PARSED ---------------------
%Jiang, H., Li, S. Z., & Wang, H. (2021). Pervasive underreaction: Evidence from high-frequency data. Journal of Financial Economics, 141(2), 573-599.
%----------------------------------------------------
%\cite{jiang2021pervasive} decompose daily stock returns into news-and
%non-news-driven components, and uncover evidence of pervasive stock market underreaction to firm news.

\cite{jiang2021pervasive} delve into the dynamics of stock market reactions to firm news, proposing a novel approach to decompose daily stock returns into news-driven and non-news-driven components. Utilizing high-frequency data, the authors analyze the intraday price movements following firm-specific news, identifying a significant underreaction to news events. They document that stock prices tend to continue moving in the direction of the initial reaction for several days without reversing, and develop a trading strategy that exploits this return drift, yielding high abnormal returns even after accounting for transaction costs. While the study presents robust findings on market underreaction and its profitable exploitation using high-frequency data, it primarily focuses on the aggregate behavior of the market rather than the specific impacts on individual firms. In this paper, by looking at the specific impacts on individual firms we obtain tailored insights for individual stock predictions, potentially providing a deeper understanding of the financial implications of news events on specific firms. 

\mx 
%------------------  GPT PARSED ---------------------
% Bybee, L., Kelly, B. T., Manela, A., & Xiu, D. (2021). Business news and business cycles (No. w29344). National Bureau of Economic Research.	
%----------------------------------------------------

%\cite{bybee2021business} also perform textual analyisis of business news, but  in their case, they employ a topic model based on a bag-of-words approach, which is an unsupervised tecnhique for dimension reduction and clustering. Hence, their analysis of news doesn't take into accound the contextual nature and semantic relationships of text. 
%*-*-*-*-*-*-*-*-*-*-
%\cite{bybee2021business} employ Latent Dirichlet Allocation (LDA) to distill the content of approximately 800,000 articles from The Wall Street Journal spanning from 1984 to 2017 into a manageable number of topics that represent distinct thematic elements of business news. These topics are then used to measure the share of media attention across various economic issues over time. The study demonstrates that shifts in the thematic focus of news coverage closely correlate with macroeconomic activity and can significantly explain variations in aggregate stock market returns. This methodology showcases the potential of text-based data to capture the latent states of economic conditions that are not directly observable through traditional economic indicators.
% However, a notable limitation of LDA is its bag-of-words approach, which overlooks the context in which terms appear. This can lead to a loss of nuanced meaning since the semantic relationships and dependencies between words are not considered, potentially reducing the accuracy of interpreting the impact of news.
%*-*-*-*-*-*-*-*-*-*-
\cite{bybee2021business} employ Latent Dirichlet Allocation (LDA) to process approximately 800,000 articles from The Wall Street Journal spanning from 1984 to 2017. By distilling large volumes of text into interpretable topical themes, they quantify the proportion of news attention allocated to each theme over time, effectively demonstrating that variations in these thematic exposures closely track economic activities and can explain about 25\% of aggregate stock market returns. This approach effectively harnesses the vast information contained in news text to model macroeconomic dynamics and demonstrates the significant role of media in shaping economic perceptions. However, the methodology relies on a bag-of-words model, which overlooks the syntactic and contextual nuances of language, potentially compromising the depth and accuracy of the analysis. In contrast, the novel methodology proposed in this paper, which exploits the capabilities of LLMs to parse news articles, maintains the contextual integrity of data, allowing for a more nuanced and precise understanding of how specific news articles influence individual stock prices.



\mx 
%------------------  GPT PARSED ---------------------
% Expected Returns and Large Language Models (B. Kelly, D. Xiu) SSRN
%----------------------------------------------------

%\cite{chen2022expected} do a thorough job in analyzing the market timing abilities of LLMs, however, in their quest for explainability, they employ old-fashioned technology that is completely deprecated. In their paper, they renegate of GPT as a tool for explicit generation of trading sginals, since "it has not been trained for this purpose". However, they do recognize the promising ability of GPT as a news parser, which could then be used to launch optimal trading signals.
\cite{chen2022expected} present a comprehensive analysis of market timing capabilities using LLMs, emphasizing their efficacy in extracting and modeling stock returns from financial news. The authors maintain a reliance on more traditional technologies such as BERT, RoBERTa, and OPT, for explicating model behaviors, potentially limiting the adaptability and future readiness of their approach.
%However, the authors acknowledge the potential of employing more sophisticated LLMs, though they 
%Furthermore, they explicitly acknowledge limitations in using GPT for direct trading signal generation, citing its original training not being aligned with specific financial tasks, yet they recognize GPT's potential as a sophisticated news parser.
In contrast, this paper leverages the parsing capabilities of cutting-edge LLMs (we employ the LLaMA-3 model, released in April 2024) and we employ a function-calling approach to tailor and structure the economic analysis of business news directly for market prediction. Our innovative use of function calls allows us to transform the raw analytical power of LLMs into a targeted tool for generating actionable trading insights. 
%By doing so, we not only harnesses the advanced contextual comprehension but also directs its output to effectively inform trading decisions, showcasing a practical and forward-thinking application of LLMs in financial markets. This method stands to offer more direct and dynamic market insights compared to the somewhat static and general-purpose models discussed in \cite{chen2022expected}, presenting a novel pathway to integrating AI in financial analysis and decision-making.

\mx 
%------------------  GPT PARSED ---------------------
% Bybee, L.. (2023).  The Ghost in the Machine: Generating Beliefs with Large Language Models
%----------------------------------------------------

\cite{bybee2023ghost} use GPT-3.5 to generate economic expectations from historical news data from The Wall Street Journal. By feeding GPT-3.5 with past news articles, the author creates a structured time series of economic beliefs that closely match traditional economic surveys. His study demonstrates the potential of LLMs to replicate and extend traditional survey measures by analyzing aggregate economic behavior over extensive historical periods. 
While \cite{bybee2023ghost} offers a robust framework for examining aggregate economic behaviors through a macro lens, the methodology in this paper adopts a different approach by focusing on the firm-specific impact of business news on stock prices. This involves a detailed parsing of news content to extract actionable financial insights directly related to market behavior by targeting individual stock reactions rather than broad economic sentiment.
%, this paper offers a more granular and direct analysis, better suited to investors and financial analysts seeking precise cues from daily news flows.


\mx 
%------------------  GPT PARSED ---------------------
% Lopez-Lira, A., & Tang, Y. (2023). Can chatgpt forecast stock price movements? return predictability and large language models. arXiv preprint arXiv:2304.07619.
%----------------------------------------------------

%More recently, \cite{lopez2023can} employed ChatGPT to make stock market predictions based on news headlines. However, these trading signals are overly limited by the nature of their dataset, news headlines, compared to using full news articles as we do in this paper. Also, their data doesn't allow to test for lookahead bias in GPT, since they only consider data after the last training period of GPT (September 2021). Furthermore, aware of the restriction imposed by working only with headlines, the authors limit themselves to asking ChatGPT to complete a sentence after having fed it with an article; such masked phrase is ``This is \_ news''. They extract the trading signal from such completion: "good", "no", "bad". Their methodology is unsophisticated and does not 
%make justice to the potential of state-of-the-art LLMs in timing the market.
%By designing a function schema that guides GPT in parsing the news, we obtain detailed information on the nature of the news article in order to obtain a much higher-quality trading signal from GPT.


More recently, \cite{lopez2023can} investigated the use of ChatGPT for stock market predictions by analyzing news headlines. They employed a dataset consisting solely of headlines from major news sources, analyzed post-September 2021, to avoid overlap with ChatGPT's training data. 
%This temporal restriction, however, prevents them from testis lookahead bias, an interesting exercise in this type of analysis.
%Furthermore, recognizing the constraints of working solely with headlines, the authors confine their approach to prompting ChatGPT to complete a sentence after having fed it with a headline. The masked phrase used is "This is \_ news", and they derive the trading signal from such completion: "good," "no," "bad." This methodology, while straightforward, fails to fully capitalize on the sophisticated capabilities of state-of-the-art large language models (LLMs) in market timing.
The authors' methodology involved prompting ChatGPT to classify these headlines as \qquote{good}, \qquote{neutral}, or \qquote{bad} for stock prices, based on a simplistic sentence-completion task where the model fills in a masked phrase, \qquote{This is \_ news}. This approach does not fully harness the capabilities of state-of-the-art LLMs for deep textual analysis and market timing. It also confines the analysis to the superficial information available in headlines rather than detailed reports. In contrast, this paper employs a more sophisticated method by designing a function schema that guides the LLM in parsing and analyzing full news articles. This methodology not only allows for a deeper and more nuanced understanding of news content but also enhances the quality of the trading signals derived from the LLM's interpretation. By using full articles instead of headlines, we obtain more reliable and actionable insights for market timing decisions.


\mx 
%----------------------------------------------------
% ChatGPT, Stock Market Predictability and Links to the Macroeconomy. SSRN
%----------------------------------------------------
%\cite{chen2023chatgpt}
%Only employ news headlines and alerts from the Wall Street Journal and focus on the examination of the timing ability of ChatGPT on the aggregate stock market (and not on the specific firmst affected by the news). Their analysis is based on a single prompt where ChatGPT is asked to read the news and say whether the stock market would GO UP, GO DOWN or UNKNOWN. Again, they don't exploit GPT's intrinsic abilities to produce a rigorous evaluation of news content, and they simply judge this technology on its capacity to ''time the aggregate stock market'', an ability for which it has not been trained and is not expected to excel at. In contrast, in this paper, we direct GPT's through a function schema that rigorously instructs it to analyze and parse the news. Our approach permits a targeted analysis that informs better trading decisions.

\cite{chen2023chatgpt} utilize news headlines and alerts from the Wall Street Journal to assess the aggregate market timing capabilities of ChatGPT, disregarding the effects on specific firms mentioned in the news.  
Their methodology hinges on a singular prompt: ChatGPT is tasked with reading a headline and predicting whether the stock market will \qquote{go up}, \qquote{go down}, or remain \qquote{unknown}.
This approach fails to leverage GPT's sophisticated abilities for a deep evaluation of news content. Instead, it narrowly evaluates the model's ability to time the market, an area outside its training and not aligned with its core strengths. Conversely, this paper exploits the use of advanced LLMs by employing a function schema that meticulously guides the model to comprehensively analyze and parse news articles. This structured methodology fosters a more precise and targeted analysis, yielding superior trading insights and decisions based on firm-specific impacts rather than broad market movements.

 

%----------------------------------------------------
%----------------------------------------------------
%\input{txt_structure.tex}
%----------------------------------------------------

%%%%%%%%%%%%%%%%%%% DATA %%%%%%%%%%%%%%%%%%%%%%%%
%\section{Data}
%----------------------------------------------------
%\input{txt_data.tex}
%----------------------------------------------------



%\newpage
%%%%%%%%%%%%%%%%% METHODOLOGY %%%%%%%%%%%%%%%%%%%%%%
\section{Methodology}

Different from \cite{hu2021networks}: 
\begin{itemize}
  \item instead of considering that news articles embed a leader-follower relationship, we don't impose any structure in the news articles
  \item we perform NER in a more realistic way, by having an LLM parse the news articles and extracting the firms that it considers as \qquote{directly affected by the news articles}. The problen with \cite{hu2021networks}'s NER is that they need to assume that every firm mentioned in a news article is relevant to the news article. This is actually not the case in most news articles, where many firms are mentioned contextually, or even more extreme, sometimes there is no relationship going on between the firms mentioned in the article. For example, we could have a news article like this: \qquote{Moodys lowers the credit rating of Banco Santander}. It's clear that this article is not talking about the existence of a relationship between Moodys and Banco Santander, however, in \cite{hu2021networks}'s logic, these article defines a connection between those two firms.
\end{itemize}

Our methodology is less restrictive and imposes no structure on the treatment of business news articles. 



%%%%%%%%%%%%%%%%%%%%%%%%%%%%%%%%%%%%%%%%%%%%%%%%%%%%%
\begin{quote}
%%%%%%%%%%%%%%%%%%%%%%%%%%%%%%%%%%%%%%%%%%%%%%%%%%%%%

Given a set of textual news articles aggregated in period $T, \D_T:=\left\{m_1, \ldots, m_d\right\}$ and a universe of $n$ firms $\F :=\{1, \ldots, n\}$, we identify the set of news linkage pairs $l^{(i,j)}_T$ between firms $i, j \in \F $ as:
$$
l^{(i,j)}_T
%\stackrel{\text { def }}{=}
:=
 \bigcup_{m_d \in \D_T}\9{ (i, j) ~\bigg{|}~ 
m_d \t{ describes a relationship between }i, j \in \F ~\t{}
% j \text { in } m_d \text { title, } i \text { in } m_d \text { headline, } i \neq j
}
$$


With the well-defined news linkage pairs, we then define the \qquote{News-implied Firm Network} at time period $T$ by the adjacency matrix 
%Given a weighted direct graph $\mathcal{G}=(\mathcal{V}, \mathcal{E})$ 
$$
\mathcal{W}_T := 
\2{
\begin{array}{ccc}
	|l^{(1,1)}_T| 		& \cdots  	& |l^{(1,n)}_T|
	\\
	\vdots				& \ddots 	& \vdots 
	\\
	|l^{(n,1)}_T|		& \cdots  	& |l^{(n,n)}_T|
\end{array}
}
$$
So far we have not imposed any hierarchy, so $\mathcal{W}_T$ is an asymmetric matrix with all zero diagonal values. However, we could further ask the LLM to give us the structure of the relationship between $i,j$. Depending on this relationship, we can define the following type of relationships: 
\begin{itemize}
  \item $i \sim j \iff i$ and $j$ are competitors within the same industry 
  \item $i \succ j \iff i $ is the supplier of $j$
  \item $i \bowtie j \iff $ in the rest of the cases
\end{itemize}

%%%%%%%%%%%%%%%%%%%%%%%%%%%%%%%%%%%%%%%%%%%%%%%%%%%%%
%%%%%%%%%%%%%%%%%%%%%%%%%%%%%%%%%%%%%%%%%%%%%%%%%%%%%

\Vhrulefill
{\center \href{https://chatgpt.com/share/66f342fc-07f0-800d-9210-0506f9c26169}{Conversation w/ ChatGPT}
\par}
\Vhrulefill

%%%%%%%%%%%%%%%%%%%%%%%%%%%%%%%%%%%%%%%%%%%%%%%%%%%%%
%%%%%%%%%%%%%%%%%%%%%%%%%%%%%%%%%%%%%%%%%%%%%%%%%%%%%

\subsection{Introduction}

The increasing availability of textual data from business news articles provides a rich source of information for studying the relationships between firms. Traditionally, firm networks inferred from such data rely on simple co-occurrence models, where firms are assumed to be connected if they are mentioned together in an article. 

%----------------------------------------------------
\begin{quote}
In particular, 
\begin{itemize}
  \item Let $\mathcal{F}=\left\{F_1, F_2, \ldots, F_n\right\}$ represent the set of $n$ firms you are analyzing.
  \item Let $\mathcal{A}=\left\{A_1, A_2, \ldots, A_m\right\}$ represent the set of $m$ news articles that mention these firms.
\end{itemize}
We assume that the news articles can be mapped to specific dates, allowing for a time dimension if needed, i.e., $A_i(t)$, where $t$ represents the publication date of article $A_i$.

\textit{Firm-Article Matrix (Incidence Matrix)}

\begin{itemize}
  \item Define an incidence matrix $M \in\{0,1\}^{n \times m}$, where the entry $M_{i j}=1$ if firm $F_i$ is mentioned in article $A_j$, and $M_{i j}=0$ otherwise.
  \item This matrix allows us to encode which firms are co-mentioned in the same articles.
\end{itemize}

\textit{Co-occurrence Matrix}

From the incidence matrix $M$, we can construct a co-occurrence matrix $C \in \mathbb{R}^{n \times n}$, where each entry $C^{i j}$ captures the number of articles in which firms $F_i$ and $F_j$ are co-mentioned.
Mathematically, this can be expressed as:
$$
C=M M^T
$$
Here, $C^{i j}$ counts the number of articles that mention both firm $F_i$ and firm $F_j$.


\textit{Weighted Network Representation}

The co-occurrence matrix $C$ can be used to define a weighted undirected graph $G=$ $(\mathcal{F}, \mathcal{E}, w)$, where:
\begin{itemize}
  \item $\mathcal{F}$ is the set of firms (nodes).
  \item $\mathcal{E} \subseteq \mathcal{F} \times \mathcal{F}$ is the set of edges between firms, where an edge exists between firms $F i$ and $F^j$ if $C^{i j}>0$ (i.e., they have been co-mentioned in at least one article).
  \item $w: \mathcal{E} \rightarrow \mathbb{R}+$ is the weight function, where the weight of the edge between firms $F_i$ and $F j$ is given by $w\left(F_i, F^j\right)=C^{i j}$. This weight represents the strength of the connection between the two firms, based on the number of co-occurrences in news articles.
\end{itemize}

\end{quote}
%----------------------------------------------------




However, this approach does not account for the nature or type of relationships between firms, nor does it consider the directionality or complexity of those relationships.

In this paper, I propose a novel methodology for constructing firm networks using Large Language Models (LLMs) to analyze the textual content of news articles. The LLM is tasked with two goals: (1) determining whether a substantive relationship exists between a pair of firms based on the context provided in the article, and (2) classifying the type of relationship (e.g., supplier-customer, competitor, partnership). Additionally, I incorporate the \textbf{directionality} of certain types of relationships, such as supplier-customer or mergers and acquisitions (M\&A), where relationships are inherently asymmetric. In particular, directionality refers to relationships between different firms where $F_i \rightarrow F_j$ but $F_j \not \rightarrow F_i$. I also consider \textbf{reflexivity}, where firms can have self-relations, such as internal restructuring or stock buybacks. In this case $F_i \leftrightarrow F_i$
The methodology yields a nuanced and comprehensive firm network that captures both the strength and type of relationships between firms, providing a powerful tool for analyzing firm interactions, market dynamics, and the effects of external shocks.

\subsection{Mathematical Framework for Firm Networks Using LLMs}

\subsection{Firm Set and News Articles}

Let $ \F = \{F_1, F_2, \dots, F_n\} $ represent the set of $ n $ firms under consideration. Each firm is potentially mentioned in a set of news articles $ \mathcal{A} = \{A_1, A_2, \dots, A_m\} $, where $ m $ denotes the number of articles. Each article $ A_i \in \mathcal{A} $ contains textual content $ T(A_i) $ and is published on a specific date $ t(A_i) $.

\subsection{LLM-based Relationship Detection and Classification}

For each article $ A_i $, the LLM processes the textual content $ T(A_i) $ and performs two key tasks:
\begin{enumerate}
    \item \textbf{Relationship Detection}: The LLM determines whether there is a substantive relationship between a pair of firms $(F_i,F_j)\in\F\times\F$ based on the context provided in article $A_k$.
    \item \textbf{Relationship Classification}: If a relationship exists, the LLM classifies the relationship between $(F_i,F_j)$ described in article $A_k$ into a relationship type $ r_{ij}(A_k) \in \mathcal{T} $, where $\T=\{\t{Supplier, Competitor, Partnership, M\&A, Legal, Other}\}$ is the set of relationship types. Note that some relationships are \qquote{directional}, while others are not. In particular:
%\begin{itemize}
%  \item Supplier: One firm supplies goods or services to another.
%  \item Competitor: Firms operate in the same industry and compete for market share.
%  \item Partnership: Firms collaborate on a joint project or initiative.
%  \item Legal Dispute: Firms are involved in a legal battle.
%  \item Mergers \& Acquisitions: Firms are involved in a merger or acquisition event.
%\end{itemize}
    \begin{itemize}
        \item \textit{Supplier-Customer}: $ F_i \to F_j $, where $ F_i $ is the supplier and $ F_j $ is the customer.
        \item \textit{Competitor}: $ F_i \leftrightarrow F_j $, where both firms compete for market share.
        \item \textit{Partnership}: $ F_i \leftrightarrow F_j $, where the firms collaborate on a project or initiative.
        \item \textit{Mergers \& Acquisitions (M\&A)}: $ F_i \to F_j $, where firm $ F_i $ absorbs or acquires firm $ F_j $.
        \item \textit{Legal Dispute}: $ F_i \to F_j $, where firm $ F_i $ sues or takes legal action against firm $ F_j $.
        \item \textit{Other}: contains the rest of relationships that the LLM was unable to clasify in the previous categories. For simplicity, we make this an undirected relationship, so $F_i\leftrightarrow F_j$. 
    \end{itemize}
\end{enumerate}

Note that in these defitions, order matters, as we are always considering that $F_i \to F_j$ in any directional relationship between any $(F_i, F_j)\in\F\times\F$. 

The LLM also assigns a relationship score $ \text{LLM\_score}(A_k, F_i, F_j, r) $, reflecting the confidence in the existence and strength of the relationship $ r_{ij}(A_k) $ between firms $ F_i $ and $ F_j $ based on article $ A_k $.

\subsection{General Relationship Matrix}

Let $\mathcal{R}\left(A_i\right) \subseteq \F \times \F$ represent the set of firm pairs $\left(F_i, F_j\right)$ that the LLM determines to be related based on the content of article $A_i$. The task of the LLM is to analyze the article $T\left(A_i\right)$ and determine when a meaningful relationship or event connects the firms.

For each article $A_i$, the LLM processes the text $T\left(A_i\right)$ and returns a set of firm relationships: 
$$
\mathcal{R}(A_i)
=
\9{
\left(F_i, F_j\right) \in \F \times \F
\mid 
\t{LLM concludes $F_i$ and $F_j$ are economically tied in $A_i$}
}
$$

The key here is that $\mathcal{R}\left(A_i\right)$ is determined by the LLM's understanding of the text, identifying cases where firms are tied by contracts, joint ventures, lawsuits, partnerships, or other significant business events, rather than simple co-mentioning.

From the set of firm relationships across all articles, we can construct a relationship matrix $R \in \mathbb{R}^{n \times n}$, where each entry $R_{i j}$ quantifies the strength of the relationship between firm $F_i$ and firm $F_j$. The entry $R_{i j}$ is computed as:

$$
R_{i j}
=
\sum_{k=1}^m \I{(F_i, F_j) \in \mathcal{R}(A_k) }
\cdot 
\mathrm{LLM} \_\operatorname{score}\left(A_k, F_i, F_j\right)
$$


\subsection{Type-Specific Relationship Matrix}
Since we have richer information about the relationship type, we can define a relationship matrix that is specific to each type of relationship $r\in\T$. 
For each article $A_i$, we now have: 
$$
\mathcal{R}(A_i)
=
\9{
\4{F_i, F_j, r_{ij}(A_k)} \in \F \times \F \times \T
~\bigg{|}~
\t{LLM detects and classifies a relationship}
}
$$
For each relationship type $ r \in \mathcal{T} $, we define a relationship matrix $ R^r \in \mathbb{R}^{n \times n} $, where the entry $ R^r_{ij} $ quantifies the strength of the relationship $ r $ between firms $ F_i $ and $ F_j $.

$$
R^r_{ij} = \sum_{k=1}^{m} 
\I{(F_i, F_j, r_{ij}(A_k)) \in \mathcal{R}(A_k)} 
\cdot
\text{LLM\_score}(A_k, F_i, F_j, r_{ij}(A_k)),
$$

where:
\begin{itemize}
    \item $\I{(F_i, F_j, r_{ij}(A_k)) \in \mathcal{R}(A_k)}$ is an indicator function that equals 1 if article $ A_k $ identifies relationship $ r $ between firms $ F_i $ and $ F_j $, and 0 otherwise.
    \item $ \text{LLM\_score}(A_k, F_i, F_j, r_{ij}(A_k)) $ is the score provided by the LLM that quantifies the strength of the relationship.
\end{itemize}

For some relationship types, such as \textit{supplier-customer}, the matrix $ R^r $ is \textbf{asymmetric}, meaning $ R^r_{ij} \neq R^r_{ji} $. In contrast, for \textit{competitor} or \textit{partnership} relationships, the matrix $ R^r $ is \textbf{symmetric}, meaning $ R^r_{ij} = R^r_{ji} $.

\subsection{Reflexivity in Firm Relationships}

In addition to relationships between firms, we also consider \textbf{reflexive relationships}, where a firm $ F_i $ has a relationship with itself, denoted $ F_i \leftrightarrow F_i $. Reflexivity can capture internal actions such as:
\begin{itemize}
    \item \textit{Firm Restructuring}: Internal reorganization or governance changes.
    \item \textit{Stock Buybacks}: Financial actions where a firm repurchases its own shares.
    \item \textit{Internal Legal Actions}: Actions that affect a firm's own internal compliance or governance.
\end{itemize}

The reflexive relationships are represented in the diagonal elements $ R^r_{ii} $ of the relationship matrix for each type $ r $:

$$
R^r_{ii} = \sum_{k=1}^{m} 
\I{(F_i, F_j, r_{ij}(A_k)) \in \mathcal{R}(A_k)}
\cdot \text{LLM\_score}(A_k, F_i, F_i, r_{ij}(A_k)),
$$

where the diagonal element $ R^r_{ii} $ quantifies the strength of firm $ F_i $'s reflexive relationship under relationship type $r$.

%%%%%%%%%%%%%%%%%%%%%%%%%%%%%%%%%%%%%%%%%%%%%%%%%%%%%
%Mathematically, reflexivity would correspond to the diagonal elements of the relationship matrix $R^r$. 
Specifically:
\begin{itemize}
  \item If $R_{i i}^r>0$, firm $F_i$ has a reflexive relationship in the context of relationship type $r$ (e.g., self-influence or self-reference).
  \item If $R_{i i}^r=0$, firm $F_i$ does not have a reflexive relationship.

\end{itemize}

%%%%%%%%%%%%%%%%%%%%%%%%%%%%%%%%%%%%%%%%%%%%%%%%%%%%%

\subsection{Network Representation with Directionality and Reflexivity}

The firm network is constructed as a \textbf{multi-layered graph} $G=\{G^r\}_{r\in\T}$
%, where each $G^r=\left(\F, \mathcal{E}^r, w^r\right)$ is a relationship-specific layer 
composed of relationship-specific layers $G^r=\left(\F, \mathcal{E}^r, w^r\right)$, where:
\begin{itemize}
  \item $\F$ is the set of firms (nodes).
  \item $\mathcal{E}^r \subseteq \F \times \F$ is the set of edges representing relationships of type $r$, where an edge exists between $F_i$ and $F_j$ if $R_{i j}^r>0$. Depending on the type of relationship $r \in \mathcal{T}$, the edges may be \textbf{directed}, \textbf{undirected} or \textbf{looping}:
\begin{itemize}
    \item \textit{Directed edges}: For relationships such as supplier-customer, M\&A, and legal disputes, the edges are directed, representing asymmetric relationships where $ F_i \to F_j $ but not necessarily $ F_j \to F_i $.
    \item \textit{Undirected edges}: For symmetric relationships like competition and partnerships, the edges are undirected, meaning $ F_i \leftrightarrow F_j $.
    \item \textit{Looping edges}: For reflexive relationships where $F_i\leftrightarrow F_i$. The weight of the self-loop reflects the strength of the firm's internal actions.
\end{itemize}
  \item $w^r: \mathcal{E}^r \rightarrow \mathbb{R}_{+}$ is the weight function, where the weight of the edge between $F_i$ and $F_j$ is given by $w^r\left(F_i, F_j\right)=R_{i j}^r$, representing the strength of relationship type $r$ between the two firms.
\end{itemize}

This creates multiple layers of networks, each representing a different type of firm interaction.

%The firm network is constructed as a \textbf{multi-layered graph} $ G = (\F, \mathcal{E}, w) $, where:
%\begin{itemize}
%    \item $ \F $ represents the set of firms (nodes).
%    \item $ \mathcal{E} \subseteq \F \times \F $ represents the set of edges, with an edge $ (F_i, F_j) $ indicating a relationship between firms $ F_i $ and $ F_j $.
%\begin{itemize}
%  \item Directed edges $\mathcal{E}_D \subseteq \F \times \F$ represent directional relationships. For example, if firm $F_i$ supplies firm $F_j$, then $\left(F_i, F_j\right) \in \mathcal{E}_D$.
%  \item Undirected edges $\mathcal{E}_U \subseteq \F \times \F$ represent symmetric relationships. For example, if firms $F_i$ and $F_j$ are competitors, then $\left(F_i, F_j\right) \in \mathcal{E}_U$.
%\end{itemize}
%    \item $ w: \mathcal{E} \to \mathbb{R}_+ $ is a weight function that assigns a weight $ w(F_i, F_j) = R^r_{ij} $, reflecting the strength of the relationship.
%\end{itemize}


\subsection{Dynamic Networks and Temporal Analysis}

To capture how relationships between firms evolve over time, we introduce a \textbf{time-varying relationship matrix} for each relationship type $ r $:

$$
R^r_{ij}(t) = \sum_{k=1}^{m} \I{(F_i, F_j, r_{ij}(A_k)) \in \mathcal{R}(A_k)} \cdot \text{LLM\_score}(A_k, F_i, F_j, r_{ij}(A_k)) \cdot \I{t(A_k) = t}.
$$

This matrix captures the relationships at a specific time $ t $ based on the articles published during that time period. The resulting \textbf{dynamic network} $ G(t) $ allows for the study of how firm interactions and network structures evolve in response to economic events, mergers, or external shocks.

\subsection{Analytical Methods and Potential Insights}

With the constructed network, several analyses can be performed to extract valuable insights:
\begin{itemize}
    \item \textbf{Centrality Measures}: Compute various centrality measures (degree, eigenvector, betweenness) to identify key firms within specific relationship networks. For example, firms central in the "supplier" network might play crucial roles in supply chains, while those central in the "competitor" network might dominate their industries.
    \item \textbf{Community Detection}: Apply community detection algorithms to identify clusters of firms that are closely related. Different clusters may emerge in different layers, such as a supply chain ecosystem or a competitive industry cluster.
    \item \textbf{Temporal Analysis}: Track the evolution of firm relationships over time, analyzing how major economic events (e.g., financial crises, regulatory changes) impact the structure and strength of firm networks.
    \item \textbf{Impact of Reflexivity}: Study firms with significant self-relations (high diagonal values in the relationship matrix) to understand how internal actions, such as restructuring or stock buybacks, affect firm performance and market position.
\end{itemize}

\subsection{Conclusion}

This paper introduces a comprehensive and nuanced approach to constructing firm networks from business news articles by leveraging LLMs to detect and classify firm relationships. By incorporating relationship types, directionality, and reflexivity, the proposed methodology provides a rich framework for analyzing firm interactions and market dynamics. The resulting multi-layered, directed- and undirected network allows for advanced analysis of firm relationships, providing insights into how firms navigate competitive environments, form partnerships, and manage internal actions.

This approach opens the door to further research into how firm networks evolve, how firms' internal and external actions affect their market position, and how different types of firm interactions impact the broader economic environment.






%%%%%%%%%%%%%%%%%%%%%%%%%%%%%%%%%%%%%%%%%%%%%%%%%%%%%
\end{quote}
%%%%%%%%%%%%%%%%%%%%%%%%%%%%%%%%%%%%%%%%%%%%%%%%%%%%%




%%%%%%%%%%%%%%%%% RESULTS %%%%%%%%%%%%%%%%%%%%%%
%\section{Results}

%%%%%%%%%%%%%%%%% CONCLUSION %%%%%%%%%%%%%%%%%%%%%%
%\section{Conclusion}
%%\section{Conclusion}

\hspace{0.5cm} In this paper we explore the incorporation of information in the stock market via a Trading strategy that trades clusters of news. In particular, for a set of Spanish business news articles from July 2020 to Septemeber 2021, we perform KMeans clustering and select the optimal clusters for trading using two proposed algorithms: \textit{Greedy} and \textit{Stable}. Both the clustering of articles and the selection of optimal clusters are done \textit{in-sample}, while the performance evaluation of the trading strategy is done by projecting the trading rule onto the test set. The results of this exercise show inconsistent earnings profile \textit{out-of-sample}, indicating that this strategy is not able to exploit news articles' information for profitable trading.

\mx 
In the second part, we feed the news articles to a Large Language Model and ask it to parse them according to a predefined schema. Such schema consists of identifying the firms affected by the articles and classifying the shocks implied by the article on such firms by their type, magnitude, and direction. Clustering based on the classification of shocks made by the LLM generates a more stable distribution of articles through clusters over data splits and provides a consistent profile of earnings in the test set. These results are robust to the choice of hyperparameters (the holding period length of the trading strategy and the amount of selected clusters for trading).

\mx 
The results of this paper show a promising avenue. LLMs can help predict market reactions to news by using a simple classification schema based on a shock analysis of the events narrated in the articles. 




%This paper introduces a comprehensive and nuanced approach to constructing firm networks from business news articles by leveraging LLMs to detect and classify firm relationships. By incorporating relationship types, directionality, and reflexivity, the proposed methodology provides a rich framework for analyzing firm interactions and market dynamics. The resulting multi-layered, directed- and undirected network allows for advanced analysis of firm relationships, providing insights into how firms navigate competitive environments, form partnerships, and manage internal actions.



%%%%%%%%%%%%%%%%%% BIBLIOGRAPHY %%%%%%%%%%%%%%%%%%%%%%
\bibliography{bib_references.bib}
\bibliographystyle{plain}


%\newpage
\appendix
%%%%%%%%%%%%%%%%%%%%%%%%%%%%%%%%%%%%%%%%%%%%%%%%%%%%%
%\section{Appendix}
%%----------------------------------------------------
\subsection{Cointegration Meets Synthetic Controls: A Formal Equivalence}
\label{sec:cointegration_meets_synthetic_controls}

In this appendix section, we develop a formal argument showing how, under some stringent assumptions, our notion of \emph{synthetic control} can be viewed as a special case of \emph{cointegration}. This connection underlies the intuition that, when one normalizes the first variable of a cointegrated system to 1, the remaining cointegration relationships effectively produce the \emph{synthetic} version of the first variable when the cointegration vector satisfies a specific restriction. 
%Here, we adopt a rigorous perspective aimed at bridging the econometric concept of cointegration with methodologies employed in the synthetic control literature (e.g., Abadie and Gardeazabal).

%\subsection{Cointegration}

Let $\{y_{i,t} \}_{t=1}^{T}$ denote the time series sequence of log-prices for each asset $i\in\{1,\ldots,N\}$.
%
Throughout, we assume each $y_{i,t}$ is an $I(1)$ process (integrated of order 1). 
%
Formally, an $I(1)$ process is one that becomes \emph{stationary} (and typically ergodic) upon differencing once:
$\Delta y_{i,t} := y_{i,t} - y_{i,t-1} \sim I(0).$
%
The notion of cointegration, due to Engle and Granger, is central in analyzing potentially long-run equilibria among these variables.

%----------------------------------------------------
\begin{definition}[Engle and Granger (1987)]
The components of $\mbf{y}_t:=[y_{1t}, ..., y_{Nt}]$ are said to be cointegrated of order $d$, $b$, denoted $\mbf{y}_t \sim CI(d,b)$, if (a) all components of $\mbf{y}_t$ are $I(d)$ and (b) a vector $\b{\beta}\neq 0$ exists so that $\b{\beta}'\mbf{y}_t \sim I(d-b)$, $b > 0$. The vector $\b{\beta}$ is called the cointegrating vector.
\end{definition}
%----------------------------------------------------
%\begin{definition}[Campbell and Perron (1991)]
%An $(n \times 1)$ vector of variables $\mbf{y}_t$ is said to be cointegrated if at least one nonzero $n$-element vector $\b{\beta}_i$ exists such that $\b{\beta}'_i\mbf{y}_t$ is trend-stationary. $\b{\beta}_i$ is called a cointegrating vector. If $r$ such linearly independent vectors $\b{\beta}_i(i = 1,\ldots,r)$ exist, we say that $\{\mbf{y}_t\}$ is cointegrated with cointegrating rank $r$. We then define the $(n \times r)$ matrix of cointegrating vectors $\b{\beta} = (\b{\beta}_1,\ldots,\b{\beta}_r)$. The $r$ elements of the vector $\b{\beta}'\mbf{y}_t$ are trend-stationary, and $\b{\beta}$ is called the cointegrating matrix.
%\end{definition}
%----------------------------------------------------

%\subsection{Synthetic Control}
%\begin{definition}
%In a synthetic control problem we have a target element $y_1$ that we seek to mimick through a linear combination of elements in a donor pool $\mbf y_{2:n}=(y_2,...,y_n)$ with weigths $\mbf w=\arg\underset{w\in\W}{\min} \sum_{t=1}^T (y_1-\mbf w'\mbf y_{2:n})^2$ where $\mathcal{W} := \{\mbf{w}\in\mathbb{R}_{+}^{n-1}: \sum_{j=2}^n w_j=1\}$.
%\end{definition}

\begin{definition}[Synthetic Control]\label{def:synthetic_control}
Let $\{y_1, y_2, \dots, y_n\}$ be a collection of random variables, where $y_1$ is the ``target'' variable and 
$\mathbf{y}_{2:n} = (y_2,\dots,y_n)$ constitute the ``donor pool''. A \emph{synthetic control} for 
$y_1$ is constructed by choosing weights $\mathbf{w}$ in the $(n-1)$-dimensional space
$\mathcal{W} := \{\mbf{w}\in\mathbb{R}_{+}^{n-1}: \sum_{j=2}^n w_j=1\}$
that satisfy
%to minimize the sum of squared deviations over $T$ observations:
$\mbf w=\arg\underset{w\in\W}{\min} 
\sum_{t=1}^T 
(y_{1,t}-\mbf w'\mbf y_{2:n,t})^2$.
%Because $\mathcal{W}$ is precisely the convex hull of the standard basis vectors in $\mathbb{R}^{n-1}$, the resulting synthetic control $\mathbf{w}'\,\mathbf{y}_{2:n}$ is a convex combination of the donor pool elements $(y_2,\dots,y_n)$.
\end{definition}


%\subsection{Equivalence}
Given that cointegration relationships prevail up to scale and sign changes, then, under suitable conditions on the cointegration vector, there exists a nontrivial constant $\kappa$ that allows us to reinterpret the cointegration relationship as one of a synthetic control. In particular,
%----------------------------------------------------
\begin{proposition} 
For a cointegrated vector $\mbf y$  with rank $r$, if (at least) one of the cointegrating vectors $\b \beta$ satisfies the restriction
$\mathcal R=
\{
%\begin{array}{ll}
%\b \beta > 0, ~
\mbf 1' \b \beta  = 0
%\\
%\beta_j \geq 0, j\neq i
%\end{array}
\}$,
%( i.e, that its components 
%are nonnegative and 
%sum to 1
%, and at least one of them is strictly positive). 
 then we can scale the cointegration vector by $\kappa=1/\beta_i$ such that $\kappa \b \beta ' \mbf y$ is stationary and describes a \qquote{synthetic control} relationship (as per \cref{def:synthetic_control}) between $y_i$ and $\mbf y_{-i}$. 
\end{proposition}
%----------------------------------------------------

\begin{proof}
The proof is straightforward. For a cointegration vector $\b \beta$ where $\mathcal R$ holds, we have that $\mbf 1'\b \beta = \sum_{j=1}^n \beta_j= 0$, which trivially implies $\beta_i = -\sum_{j\neq i}\beta _j$. For the sake of the proof, set that $\beta_i$ to the first component ($\beta_1$). Then
$\beta_1 = -\sum_{j=2}^n \beta_j$ and $\kappa = (\beta_1)^{-1} = -(\sum_{j=2}^n \beta_j)^{-1}$
$$
\kappa \b \beta' \mbf y 
= 
\frac{1}{\beta_1} [\beta_1 ~~ \b \beta_{2:n}] \mbf y_t 
=
\2{1 ~~ \frac{-\b \beta'_{2:n}}{\sum_{j=2}^n\beta_j}}
\2{\v{y_1 \\ \mbf y_{2:n}}}
=
y_1 - \frac{\beta_2}{\sum_{j=2}^n \beta_j}y_2 - \cdots - 
\frac{\beta_n}{\sum_{j=2}^n \beta_j} y_n
\sim I(0)
$$
describes a stationary cointegration relationship in $\mbf y$, and since
\begin{align*}
y_1 
&= \frac{\beta_2}{\sum_{j=2}^n \beta_j}y_2 + \cdots + 
\frac{\beta_n}{\sum_{j=2}^n \beta_j} y_n + \eps 
\\&= \mbf w' \mbf y_{2:n} + \eps 
%, \t{~~where~} \eps\sim I(0)
\end{align*}
with $\eps\sim I(0)$ and $\mbf w:=\1{\frac{\beta_2}{\sum_{j=2}^n \beta_j}, ..., \frac{\beta_n}{\sum_{j=2}^n \beta_j}}'\in \W$, then this relationship is endowed with a synthetic control structure. A similar reasoning applies to any other $\beta_i$ different from $\beta_1$.
\end{proof}
%----------------------------------------------------

%----------------------------------------------------
%%%%%%%%%%%%%%%%%%%%%%%%%%%%%%%%%%%%%%%%%%%%%%%%%%%%%
% DISCUSSION: CARDINALITY-CONSTRAINED SYNTHETIC CONTROL
%%%%%%%%%%%%%%%%%%%%%%%%%%%%%%%%%%%%%%%%%%%%%%%%%%%%%
\subsection{Why not use a cardinality-constrained Synthetic Control?}
\label{sec:discussion_card_constr}

While the $\ell_1$-regularized approach provides a computationally efficient and convex framework for constructing sparse synthetic controls, it is worth considering alternative methods that directly impose sparsity through cardinality constraints. A natural alternative is to solve a cardinality-constrained quadratic program, which explicitly limits the number of non-zero weights in the synthetic asset. Formally, this can be expressed as:
\begin{equation*}
\mathbf{w}^* = \argmin_{\mathbf{w} \in \R^{N}} \sum_{t=1}^T \left(y_{t} - \sum_{i=1}^N w_i x_{it}\right)^2 
\quad \text{s.t.} \quad 
\left|
\begin{array}{ll}
	\mbf 1^\top \mbf w &= 1 \\
	\norm{\mathbf{w}}_0 &\leq K
\end{array}
\right.
\end{equation*}

where $\|\mathbf{w}\|_0 := \sum_{i=1}^N \mathbb{I}\{w_i \neq 0\}$ counts the number of non-zero elements in $\mathbf{w}$, and $K$ is a user-defined sparsity level. This formulation directly enforces sparsity by restricting the synthetic asset to be constructed from at most $K$ donor assets. However, the cardinality constraint introduces significant computational challenges, as the problem becomes NP-hard due to its combinatorial nature. Below, we discuss two approaches to approximate this problem and their limitations.

\subsubsection{Mixed-Integer Programming Approach}
One way to tackle the cardinality-constrained problem is to reformulate it as a mixed-integer quadratic program (MIQP). This involves introducing binary variables $z_i \in \{0, 1\}$ for $i = 1, \dots, N$, where $z_i = 1$ indicates that the $i$-th asset is included in the synthetic control, and $z_i = 0$ otherwise. The problem can then be rewritten as:
\begin{equation*}
\mathbf{w}^*, \mathbf{z}^* 
= 
\left[
\begin{array}{rlll}
\underset{\mathbf{w} \in \R^{N},~\mathbf{z} \in \{0, 1\}^N}{\arg\min}
&
\sum_{t=1}^T \left(y_{t} - \sum_{i=1}^N w_i x_{it}\right)^2
%\norm{\mathbf{y} - \mathbf{X}\mathbf{w}}_2^2
\\
\text{s.t.}  &
\left|
\begin{array}{ll}
\mathbf{1}^\top \mathbf{w} &= 1, \\
\sum_{i=1}^N z_i &\leq K, \\
|w_i| &\leq M z_i \quad \text{for } i = 1, \dots, N,
\end{array}
\right.
\end{array}
\right]
\end{equation*}

where $M$ is a sufficiently large constant that bounds the magnitude of the weights. The constraint $|w_i| \leq M z_i$ ensures that $w_i$ can only be non-zero if $z_i = 1$. While this formulation is exact, it is computationally intensive, especially for large donor pools, as it requires solving a mixed-integer program. The computational complexity grows exponentially with the number of assets, making it impractical for high-dimensional settings.

\subsubsection{Two-Step Heuristic Procedure}
An alternative approach is to use a two-step heuristic procedure that approximates the cardinality-constrained solution without requiring mixed-integer programming. This procedure proceeds as follows:

\begin{enumerate}
\item \textbf{Solve the full least squares problem:} First, solve the unconstrained least squares problem to obtain an initial weight vector:
\begin{equation*}
\mathbf{w}^{(1)} 
= 
\argmin_{\mathbf{w} \in \mathbb{R}^{N}} 
\norm{\mathbf{y} - \mathbf{X}\mathbf{w}}_2^2
\quad \text{s.t.} \quad \mathbf{1}^\top \mbf w = 1.
\end{equation*}

\item \textbf{Select the $K$ largest weights:} Identify the $K$ largest weights (in absolute value) from $\mathbf{w}^{(1)}$ and define the support set:
\begin{equation*}
\mathcal{I} := \{i : |w_i^{(1)}| \text{ is among the $K$ largest}\}.
\end{equation*}

\item \textbf{Solve the restricted program:} Solve the least squares problem restricted to the support set $\mathcal{I}$:
\begin{equation*}
	\mbf w^{(2)} = \arg \min_{\mbf w_{\mathcal I} \in \mathbb{R}^K} \norm{\mbf y - \mbf X_{\mathcal I}\mbf w_{\mathcal I}}_{2}^{2}
\quad \text{s.t.} \quad 
\mbf 1^\top \mbf w_{\mathcal I} = 1,
\end{equation*}
where $\mbf X_{\mathcal{I}} \in \mathbb{R}^{T \times K}$ is the restricted donor matrix and $\mbf w_{\mathcal{I}} \in \mathbb{R}^{K}$ is the restricted weight vector.

\item \textbf{Construct the full weight vector:} Embed the optimized restricted weights back into the original $N$-dimensional space:
\begin{equation*}
	w^*_i = 
\begin{cases}
w^{(2)}_j & \text{if } i = \mathcal{I}_j, \\
0 & \text{otherwise}.
\end{cases}
\end{equation*}
\end{enumerate}

While this heuristic is computationally efficient, it has several drawbacks. First, the initial least squares solution $\mathbf{w}^{(1)}$ may not provide a good indication of which assets are most relevant, especially in the presence of multicollinearity or noise. Second, the procedure can lead to extreme weights (both positive and negative) in the final solution, resulting in a highly leveraged portfolio that may not be practical for trading. This is because the restricted optimization step does not impose any bounds on the magnitude of the weights, allowing for large positive and negative values that cancel each other out to satisfy the unit sum constraint.

\subsubsection{Comparison with $\ell_1$-Regularized Approach}
In contrast to the cardinality-constrained approaches, the $\ell_1$-regularized method provides a more balanced trade-off between sparsity and computational efficiency. By shrinking some weights exactly to zero, the $\ell_1$ penalty achieves sparsity without requiring explicit cardinality constraints. Moreover, the convex nature of the problem ensures that it can be solved efficiently using proximal algorithms or quadratic programming techniques, even for high-dimensional donor pools. Additionally, the regularization parameter $\lambda$ provides fine-grained control over the sparsity level, allowing the user to tune the solution based on their specific requirements.

In practice, we found that the $\ell_1$-regularized approach yields more stable and interpretable synthetic controls compared to the cardinality-constrained methods. The latter often produce highly leveraged portfolios with extreme weights, which are undesirable in a trading context. Furthermore, the computational advantages of the $\ell_1$-regularized approach make it more suitable for real-world applications, where scalability and robustness are critical.

In conclusion, while cardinality-constrained formulations offer a conceptually appealing way to enforce sparsity, their practical limitations make them less attractive for constructing synthetic controls in pairs trading. The $\ell_1$-regularized approach strikes a better balance between sparsity, interpretability, and computational efficiency, making it the preferred choice for our application.



%\subsection{Cardinality-Constrained Programming}
%\subsubsection{Mixed Integer Quadratic Programming}
%The cardinality-constrained problem can be reformulated as a mixed-integer quadratic program (MIQP) by introducing binary variables $z_i \in \{0,1\}$ that indicate whether asset $i$ is included in the synthetic control:
%
%\begin{equation*}
%\begin{array}{ll}
%\min_{\mathbf{w}, \mathbf{z}} & \sum_{t=1}^T \left(y_{t} - \sum_{i=1}^N w_i x_{it}\right)^2 \\
%\text{subject to} & \mathbf{1}^\top \mathbf{w} = 1 \\
%& -Mz_i \leq w_i \leq Mz_i, \quad i = 1,\ldots,N \\
%& \sum_{i=1}^N z_i \leq K \\
%& z_i \in \{0,1\}, \quad i = 1,\ldots,N
%\end{array}
%\end{equation*}
%
%where $M$ is a sufficiently large constant that bounds the absolute values of the weights. The binary constraints $z_i \in \{0,1\}$ make this problem NP-hard, requiring branch-and-bound techniques for its solution. While modern MIQP solvers (e.g., Gurobi, CPLEX) can handle problems of moderate size, the computational burden becomes prohibitive for large donor pools or when the optimization needs to be performed repeatedly, as in our rolling-window implementation of the pairs trading strategy.
%
%\subsubsection{Iterative Weight Thresholding}
%An alternative approach, which we refer to as Iterative Weight Thresholding (IWT), follows a sequential procedure that approximates the cardinality-constrained solution. The method consists of solving a sequence of unconstrained problems, progressively focusing on the most relevant assets:
%
%\begin{enumerate}
%\item Solve the full least squares problem
%\begin{equation*}
%\mathbf{w}^{(1)} = \argmin_{\mathbf{w} \in \mathbb{R}^{N}} \norm{\mathbf{y} - \mathbf{X}\mathbf{w}}_2^2
%\quad \text{s.t.} \quad \mathbf{1}^\top \mbf w=1
%\end{equation*}
%
%\item Select the $K$ largest weights (in absolute value) from $\mbf w^{(1)}$ to form the support set
%$$\mathcal I:=\{i : |w_i^{(1)}| \text{~among~} K \text{~largest}\}$$
%
%\item Solve the restricted program on support $\mathcal I$
%\begin{equation*}
%\mbf w^{(2)} = \arg \min_{\mbf w_{\mathcal I}\in \mathbb{R}^K} \norm{\mbf y - \mbf X_{\mathcal I}\mbf w_{\mathcal I}}_{2}^{2}
%\quad \text{s.t.} \quad 
%\mbf 1\' \mbf w_{\mathcal I} = 1
%\end{equation*}
%where $\mbf X_{\mathcal{I}} \in \mathbb{R}^{T \times K}$ is the restricted donor matrix and $\mbf w_{\mathcal{I}} \in \mathbb{R}^{K}$ is the restricted weight vector.
%
%\item Construct the full weight vector $\mbf w^* \in \mathbb{R}^{N}$ by embedding the optimized restricted weights:
%\begin{equation*}
%w^*_i = 
%\begin{cases}
%w^{(2)}_j & \text{if}~ i = \mathcal I_j \\
%0 & \text{otherwise}
%\end{cases}
%\end{equation*}
%\end{enumerate}
%
%While both approaches-MIQP and IWT-can theoretically achieve the desired cardinality constraint, they present significant practical challenges in our pairs trading context. The MIQP formulation, although exact, becomes computationally intractable for large donor pools and high-frequency rebalancing. The IWT approach, while computationally efficient, tends to produce extreme weights in our empirical implementation. Specifically, when applying IWT to our dataset, we observed weights often exceeding $\pm 500\%$ of the portfolio value, resulting in highly leveraged positions that would be impractical to implement due to transaction costs and risk management constraints.
%
%These limitations motivate our choice of the $\ell_1$-regularized approach presented in the main text. The lasso penalty provides a convex relaxation of the cardinality constraint that is both computationally efficient and tends to produce more reasonable weight allocations. Moreover, the continuous nature of the $\ell_1$ penalty allows for smoother transitions in portfolio weights over time, reducing turnover compared to the discrete selection methods discussed above.
%----------------------------------------------------

%----------------------------------------------------
%%%%%%%%%%%%%%%%%%%%%%%%%%%%%%%%%%%%%%%%%%%%%%%%%%%%%
\subsection{Algorithms}
%%%%%%%%%%%%%%%%%%%%%%%%%%%%%%%%%%%%%%%%%%%%%%%%%%%%%




%\begin{algorithm}[H]
\caption{Mispricing Detective}
\label{alg:mispricing_detective}
\begin{algorithmic}[1]
%----------------------------------------------------
\Require
price time series of the target asset $\{p^{\t{trgt}}_{t}\}_{t \in \T}$;
price time series of the synthetic pool $\{p^{\t{synth}}_{t}\}_{i\in\D, t \in \T}$

%----------------------------------------------------
\mx 
\Ensure cumulative mispricing indices 
$\{
CMI^{\t{trgt}|\t{synth}}
\}_{t \in \T}
, 
\{
CMI^{\t{synth}|\t{trgt}}
\}_{t \in \T}$
%----------------------------------------------------
\mx 
\Function{SyntheticControlBuilder}{$\{p^{\t{trgt}}_{t}\}_{t \in \T}, \{p^{i}_{t}\}_{i\in\D, t \in \T}$}
$$
~~\{ \hat w^{i} \}_{i\in \D}
\gets  \arg \min_
% {\mbf w} 
{\{ \hat w^i \}_{i\in\D}}
\bigg{\{}
\sum_{t\in \mathcal T^{tr}}
\left(
\log p_{t}^\t{trgt}
- 
\sum_{i\in \mathcal D}
w^i \log p_{t}^i
\right)^2
+ 
\lambda \sum_{i\in\mathcal D} |w^i|
\bigg{\}}
\quad 
\t{s.t.}
\quad 
\sum_{i\in\mathcal D} w^i=1
%\\[1em]
%\mbf w_{\t{sc}} &= \{w_i^* : w_i^* \geq 10^{-1}\}
.
$$

$
\{ p^{\t{synth}}_t \}_{t\in\T} \gets  
\{\sum_{i\in \mathcal D} \hat w^i p_{t}^i \}_{t\in\T}
$
%----------------------------------------------------
\mx 
    \State \Return $\{ p^{\t{synth}}_t \}_{t\in\T}$
\EndFunction
\end{algorithmic}
\end{algorithm}
%----------------------------------------------------


%----------------------------------------------------
\subsection{Barra model}


Before proceeding with the formulation of our pairs trading approach, it is essential to understand the economic drivers behind the relative performance between our target and synthetic assets. While our sparse synthetic control methodology creates a close replicating portfolio, any pairs trading strategy fundamentally relies on exploiting temporary divergences in pricing relationships. Therefore, a rigorous factor-based analysis provides critical insights into the structural sources of these divergences.

To this end, we employ the Barra factor model framework, which decomposes the returns of both target and synthetic assets into systematic components (fundamental and industry factors) and idiosyncratic components. This decomposition serves multiple purposes in our research:

First, it provides economic interpretation of the spread returns by quantifying how much of the relative performance is driven by different factor exposures versus pure alpha. Second, it validates the quality of our synthetic control construction by measuring how closely the factor exposures match between target and synthetic assets. Third, it identifies specific factor tilts that persist even after optimization, potentially representing structural drivers of spread returns that our trading strategy can exploit.

By understanding these factor relationships before developing trading signals, we can distinguish between transient mispricings (which represent opportunities) and permanent structural differences (which represent risks). This factor-based foundation also helps explain why certain statistical relationships identified by our copula models may exist, providing theoretical underpinning to the empirical patterns we observe in subsequent sections.

%The Barra model thus serves as a critical bridge between our synthetic asset construction and trading signal generation, ensuring that our overall approach is grounded in economic intuition rather than relying solely on statistical patterns.

The Barra model for our target and synthetic asset may be written as
\begin{align*}
\2{\v{r_t \\ r_t^*}} =
\2{\v{\alpha \\ \alpha^*}} 
+ 
\2{\v{\b \beta\' \\ \b \beta^{*\'}}} 
\mbf f_t
+
\2{\v{\b \gamma\' \\ \b \gamma^{*\'}}} 
\mbf i_t
+
\2{\v{\eps_t \\ \eps_t^*}} 
\end{align*}

where we consider $K=8$ fundamental factors $\mbf f_t$ (i.e.: $\b \beta, \b \beta^*, \mbf f_t \in \mathbb R^K$) and $M=17$ industry factors $\mbf i_t$ (i.e.: $\b \gamma, \b \gamma^*, \mbf i_t \in \mathbb R^M$).
%
The \qquote{active return} between the target and synthetic asset is given by:
$$
\dot{r}_t := r_t - r_t^* = (\alpha - \alpha^*) + (\b \beta - \b \beta^*)\'\mbf f_t +  (\b \gamma - \b \gamma^*)\'\mbf i_t + (\eps_t - \eps_t^*)
.
$$
Now defining the \textit{relative alpha, beta and gamma}, respectively, as
$
\dot{\alpha}:= (\alpha - \alpha^*),
\dot{\b \beta} := (\b \beta - \b \beta^*),
\dot{\b \gamma} := (\b \gamma - \b \gamma^*)
$
and setting $\dot{\eps}_t := (\eps_t - \eps_t^*)$, we may write the model in terms of the portfolio's active return
\begin{equation}\label{eq:barra}
\dot{r}_t = \dot{\alpha} + \dot{\b \beta}\' \mbf f_t + \dot{\b \gamma}\' \mbf i_t + \dot{\eps}_t
.
\end{equation}

In \cref{fig:factor_corr_matrix} we show the factor correlation matrix $\t{Corr}(\mbf X)\in\mathbb{R}^{J\times J}$ of all the factors
$$
\mbf{X} 
= \2{\v{\mbf f_1\', \\ \vdots \\ \mbf f_T\', } 
~
% \c 
\v{\mbf i_1\' \\ \vdots \\ \mbf i_T\' }}
\in\mathbb R^{T\times J}
,
$$
where $J=K+M$. 
In our application we are using 
[\texttt{MKT\_RF}, \texttt{SMB}, \texttt{HML}, \texttt{RMW}, \texttt{CMA}, \texttt{MOM}, \texttt{ST\_REV}, \texttt{LT\_REV}] as the fundamental factors, and [\texttt{Food}, \texttt{Mines}, \texttt{Oil}, \texttt{Clths}, \texttt{Durbl}, \texttt{Chems}, \texttt{Cnsum}, \texttt{Cnstr}, \texttt{Steel}, \texttt{FabPr}, \texttt{Machn}, \texttt{Cars}, \texttt{Trans}, \texttt{Utils}, \texttt{Rtail}, \texttt{Finan}, \texttt{Other}] as the industry factors. 
As we can see, correlations are very high among factors, specially among industry factors, which means that regular OLS estimation of \cref{eq:barra} will deliver highly unstable coefficients due to multicollinearity.



%==============[	  FACTOR CORRELATION MATRIX  ]==============
\inserthere{fig:factor_corr_matrix}
\begin{figure}[H]
  \centering
  \caption{Factor Correlation Matrix}
  %----------------------------------------------------
  \begin{subfigure}{\textwidth}
  \centering	
  \caption{Train}
  \includegraphics[scale=0.5]{/Users/jesusvillotamiranda/Library/CloudStorage/OneDrive-UniversidaddeLaRioja/GitHub/Repository/arbitragelab-master/__OUTPUT_TeX__/figures/Factor_Correlation_Matrix_(17_Ind)_Train.pdf}
  \label{subfig:factor_corr_matrix_train}
  \end{subfigure}

	\vspace{0.5cm} % Adjust the spacing as needed

  \begin{subfigure}{\textwidth}
  \centering
  \caption{Test}
  \includegraphics[scale=0.5]{/Users/jesusvillotamiranda/Library/CloudStorage/OneDrive-UniversidaddeLaRioja/GitHub/Repository/arbitragelab-master/__OUTPUT_TeX__/figures/Factor_Correlation_Matrix_(17_Ind)_Test.pdf}
  \label{subfig:factor_corr_matrix_test}
  \end{subfigure}
%----------------------------------------------------
\label{fig:factor_corr_matrix}
\end{figure}


Hence, to properly estimate the model parameters, we employ an orthogonal regression approach based on Principal Component Analysis (PCA), which will allow us to obtain more stable estimates of the factor exposures. The implementation follows these steps. 

%==============[	  Standardization  ]==============
%First, we standardize all factors (both fundamental and industry factors) to have zero mean and unit variance, which yields
%$
%%\tilde{\mbf{x}}_t = (\mbf{x}_t - \bar{\mbf{x}}) / \sigma_{\mbf{x}}.
%\tilde{\mbf X}\in \mathbb{R}^{T\times J}.
%$
%
%==============[	  Principal Component Analysis  ]==============
First, we compute the covariance matrix of the factors $\mbf \Sigma := \text{Cov}({\mbf{X}})\in \mathbb R^{J\times J}$ and obtain its eigendecomposition
$
\mbf \Sigma \mbf V = \mbf \Lambda \mbf V,
$
where $\mbf \Lambda:=\diag(\lambda_1,...,\lambda_J)\in \mathbb R^{J\times J}$ are the eigenvalues and $\mbf{V} := [\mbf{v}_1,...,\mbf{v}_J]\in \mathbb R^{J\times J}$ are the corresponding eigenvectors, both sorted in descending order of the $\lambda$'s. 
The principal components are given by:
$
%\mbf{p}_t = \mbf{V}\' \tilde{\mbf{x}}_t.
\mbf P = {\mbf X} \mbf V \in \mathbb{R}^{T\times J}.
$

%========[	  Principal Component Regression (Unrestricted)  ]=========
Second, we regress the active returns onto the principal components
$$
\dot{r}_t = a^{(u)} + \sum_{i=1}^J b^{(u)}_i p_{t,i} + \nu_t^{(u)}
%\dot{r}_t = a^{(u)} + \mbf p_t\' \b b^{(u)} + \nu_t^{(u)}
$$
where $a^{(u)}$ is the \qquote{unrestricted} intercept, $\b b^{(u)}:=[b^{(u)}_1,...,b^{(u)}_J]$ are the \qquote{unrestricted} coefficients for each principal component, and $\nu_t$ is the error term.
%
%==============[	  Selection of Significant Components  ]==============
We keep only the statistically significant principal components at the $0.05$ significance level:
$
\mathcal{S} := \{i : p\text{-value}(b_i^{(u)}) < 0.05\}.
$
%
%========[	  Principal Component Regression (Restricted)  ]=========
Then, we estimate a restricted model using only the significant principal components
$$
\dot{r}_t = a^{(r)} + \sum_{i \in \mathcal{S}} b_i^{(r)} p_{t,i} + \nu_t^{(r)}.
$$
    
%==============[	  Transformation to Factor Space  ]==============
Finally, we transform the coefficients back to original factor space. Let $\b b^{(r)}\in \mathbb R^J$ denote the vector filled with $b_i^{(r)}$ if $i\in \mathcal S$ and 0 otherwise. Then, we can write
$
\dot r_t 
= a^{(r)} + \mbf p_t\' \b b^{(r)}+\nu_t^{(r)} 
= a^{(r)} + {\mbf x}_t\' \mbf V \b b^{(r)}+\nu_t^{(r)}
,
$
where $\mbf p_t$ and $\mbf x_t$ are rows of $\mbf P$ and $\mbf X$, respectively (given as column vectors).
Thus, by setting $\dot \alpha = a^{(r)}$ and
$
% 2{\v{\dot{\b \beta} \\ \dot{\b \gamma}}} = \mbf V \b b^{(r)}
[\v{\dot{\b \beta} ~ \dot{\b \gamma}}]\' = \mbf V \b b^{(r)}
$
we recover alpha and the factor betas and gammas while avoiding the instability due to multicollinearity.
%==============[	  HAC Standard Errors  ]==============
Both the unrestricted and restricted models are estimated with Heteroskedasticity and Autocorrelation Consistent (HAC) standard errors using a maximum lag of 5 periods to account for potential serial correlation and heteroskedasticity in the residuals. 

%==============[	  Advantages of this procedure  ]==============
%This approach offers several advantages. First, by using orthogonal principal components, we eliminate multicollinearity concerns. Second, by selecting only significant components, we reduce dimensionality and potential overfitting. Finally, the transformation back to the original factor space allows for direct interpretation of the factor exposures $\dot{\b  \beta}$ and $\dot{\b \gamma}$ in our active return decomposition model.

%For robustness, we repeat this procedure with various industry factor classifications, ranging from 10 to 49 industries, to assess the sensitivity of our results to the industry granularity. If no principal components are found to be statistically significant at the 5\% level, we default to an intercept-only model, effectively attributing the entire active return to the $\dot{\alpha}$ term.

%----------------------------------------------------
\input{/Users/jesusvillotamiranda/Library/CloudStorage/OneDrive-UniversidaddeLaRioja/GitHub/Repository/arbitragelab-master/__OUTPUT__/barra_table.tex}
%----------------------------------------------------


\cref{tab:barra_model} presents the results of our Barra model decomposition using the Principal Component Regression (PCR) approach described above. This table provides insights into the factor exposures that drive the relative performance between our target and synthetic assets.

The relative alpha ($\dot \alpha$) represents the portion of active returns not explained by factor exposures. In both the training and testing periods, the relative alpha is not statistically significant ($p$-values of 0.6766 and 0.1419, respectively), suggesting that factor exposures, rather than idiosyncratic effects, are the primary drivers of the spread between target and synthetic assets. The alpha changes from -0.0002 in the training period to 0.0009 in the testing period, though this difference remains statistically insignificant.

Among the fundamental factors, several notable patterns emerge. The profitability factor (\texttt{RMW}) exhibits the largest positive exposure in the training period (0.6985), indicating that the target asset tends to outperform the synthetic asset when profitability is rewarded in the market. In the testing period, the most substantial exposure shifts to the long-term reversal factor (\texttt{LT\_REV}) with a coefficient of 0.4929. Interestingly, the value factor (\texttt{HML}) shows a consistent negative exposure across both periods, becoming more pronounced in the testing period (-0.4563). This suggests that the target asset tends to underperform the synthetic asset during periods when value stocks outperform growth stocks.

The industry factor exposures reveal significant sector-based drivers of the spread returns. The machinery sector (\texttt{Machn}) shows the strongest positive exposure in both periods, dramatically increasing from 0.4337 in training to 1.0429 in testing. This indicates that the target asset has significantly higher exposure to the machinery sector compared to the synthetic asset. Conversely, the retail sector (\texttt{Rtail}) demonstrates the most substantial negative exposure in both periods (-0.5772 and -0.6114), suggesting that the synthetic asset has greater retail exposure than the target. Other sectors with notable negative exposures include \texttt{Steel} (-0.3551 and -0.4512) and \texttt{Durables} (-0.2761 and -0.1056).

%Some industry exposures exhibit interesting shifts between periods. The transportation sector (\texttt{Trans}) reverses from -0.1163 in training to 0.2022 in testing, while the construction sector (\texttt{Cnstr}) shifts from a substantial positive exposure (0.2890) to a slight negative one (-0.0143). These changes reflect the dynamic nature of sector relationships between the target and synthetic assets over time.

The model's explanatory power improves markedly from the training to testing period, with the adjusted $R^2$ increasing from 0.0687 to 0.2525. Both models are statistically significant based on their $F$-statistics (10.5893 and 19.7223) with $p$-values below 0.0001. This improvement in fit suggests that the factor structure identified during the training period becomes more pronounced during the testing period, potentially enhancing the strategy's effectiveness.

These results demonstrate that while our synthetic control methodology successfully creates a close match to the target asset, systematic differences in factor exposures remain that can be exploited by our trading strategy. In particular, the significant exposures to fundamental factors like profitability and long-term reversal, alongside pronounced industry tilts, represent potential sources of alpha that our pairs trading approach can capitalize on.
%----------------------------------------------------



%%%%%%%%%%%%%%%%%%%%%%%%%%%%%%%%%%%%%%%%%%%%%%%%%%%%%
\end{document}
%%%%%%%%%%%%%%%%%%%%%%%%%%%%%%%%%%%%%%%%%%%%%%%%%%%%%