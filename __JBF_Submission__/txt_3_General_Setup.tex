\section{Mathematical Treatment of News Articles}

Our dataset consists of $N=2,613$ Spanish business news articles 
 sourced from DowJones and spanning the period from 2020/06/24 to 2021/09/30. 
 We denote as $\mathcal D$ the set of all articles in our sample.
 % from DowJones for the period 2020-6-24 to 2021-09-30. 
 These articles have been specifically filtered to reference firms listed on the IBEX-35.
% These articles have been specifically filtered to be explicitly referred to firms that are listed in the IBEX-35.
 Let $\F_{\t{IBEX35}}$ denote the universe of such firms. 
 Each article $i \in \mathcal{D}$ is a textual document detailing an event that directly pertains to a subset of firms $\mathcal{F}^i \subseteq \F_{\t{IBEX35}}$.
% Each article $i\in \D$ is a textual document narrating an event that directly involves a set of firms %$\F^i \subseteq \F_{\t{IBEX35}}$.
%
The publication date and time of each article are represented as $\mathcal{Y}_0^i = \langle d_0^i, t_0^i \rangle$, where $d_0^i$ captures the date 
(YYYY-MM-DD) 
and $t_0^i$ captures the time
 (HH:MM) 
 of publication. 
%This datetime representation emphasizes 
Therefore we observe
the moment at which $\mathcal{F}^i$ receives the \qquote{treatment} of public news dissemination. 
%The information conveyed by an article can then be represented in a tuple 
%$
%\langle i, \F^i , \mathcal Y_0^i \rangle 
%$

%The article is published at some datetime represented by a tuple $\mathcal Y_0^i :=\angl{d_0^i, t_0^i}$, where $d_0^i$ represents the date component (YYYY-MM-DD) and $t_0^i$ represents the time component (HH:MM:SS.SSS). We use a 0-subscript to emphasize that $\mathcal Y_0^i$ is the day and time in which $\mathcal F^i$  receives  ``treatment'' (publication of a news article directly involving $\mathcal F^i$).
%%``treatment date'' or day in which a news article is published regarding firm $j$ (firm $j$ receives treatment) 
%

%\mx 
\subsubsection*{Data Splitting}
For robust model development and evaluation, the dataset is partitioned into three sequential subsets: training, validation, and test:
$
\D := \D^{tr} \cup \D^{val} \cup \D^{test}.
$
Define $N_{split}:=|\D^{split}|$ for $split\in\3{tr,val,test}$, where $\abs{\cd}$ denotes the cardinality of a set. 
%Then, it follows that
%%with $N_{tr}:=\abs{\D^{tr}}, N_{val}:=\abs{\D^{val}}$, N_{test}:=\abs{\D^{test}}$ 
%$\abs{\D}= N_{tr} + N_{val} + N_{test} = N$. 
%----------------------------------------------------
%The training and validation samples represent 80\% of the total sample $(\frac{N_{tr}+N_{val}}{N} = 0.8)$ and are used to design the optimal trading strategy, while the test sample represents the remaining 20\% $(\frac{N_{test}}{N} = 0.2)$ and is used to evaluate the \textit{out-of-sample} performance of the strategy. 
%--------- CHAT GPT OPTION ---------%
The training and validation sets collectively comprise 80\% of the total dataset $(\frac{N_{\text{tr}} + N_{\text{val}}}{N} = 0.8)$ and are instrumental in constructing and fine-tuning the trading strategy. The remaining 20\% $(\frac{N_{\text{test}}}{N} = 0.2)$ is reserved for out-of-sample testing to assess the performance and generalizability of the strategy under unseen conditions.
%----------------------------------------------------

%%%%%%%%%%%%%%%%%%%%%%%%%%%%%%%%%%%%%%%%%%%%%%%%%%%%%
%%%%%%%%%%%%%%%%%%%%%%%%%%%%%%%%%%%%%%%%%%%%%%%%%%%%%
\subsubsection*{Effective treatment day}
We are interested in examining the impact of each news article $i\in\D$ on the stock price of the firms that are affected directly (i.e.: all $j\in\F^i$). Since the publication datetime is not necessarily a trading datetime, we cannot directly gauge such an effect by looking at $\mathcal Y_0^i$. 
For this reason, we need to work through some definitions. 
%
%$(\tilde{d}_0^i)$ as the date at which news article $i$ events can be incorporated into the stock prices of $\F^i$. 
%
Let $\T$ denote the set of all datetimes in the sample timeline and let $\widetilde{\T}\subset \T$ be the subset of Spanish trading datetimes associated to our sample.
\begin{align*}
\widetilde{\T} := 
\3{\angl{d,t} \mid d \in \tilde{\mathfrak d} ~\wedge~ t\in \tilde{\mathfrak t}}
,
\end{align*}
where 
$
\tilde{\mathfrak{d}}:=\{\tilde{\mathfrak{d}}_{[1]},\tilde{\mathfrak{d}}_{[2]}, \ldots \tilde{\mathfrak{d}}_{[n]}\}
$
is the ordered set of week and non-festive days according to the Spanish calendar in our data timeline,
%, and $\tilde{\mathfrak t}$ are the trading hours 
and 
$\tilde{\mathfrak t}:=
\{t \mid \t{09:30} \leq t \leq  \t{17:30}\}
%\{t \mid t\in\t[\t{09:30}, \t{17:30}]\}
$
 are the Spanish stock market trading hours. 
Note that we use tildes to emphasize that we are considering trading dates or times. 

%We can now define a set of functions that will become useful as we work with trading days.
%We now introduce some functions that will prove useful now and later on for easily handling trading days. 

%%----------------------------------------------------
%
\textbf{Index Function}. 
Given a finite ordered set $\mathcal{Z}=\left\{z_1, z_2, \ldots, z_n\right\}$, the index function 
$\mathbb{I}_{\mathcal{Z}}: \mathcal{Z} \rightarrow\{1,2, \ldots,|\mathcal{Z}|\}$
%$\mathbb{I}_{\mathcal{Z}}(z)$
 maps an element $z\in\Z$ to its position in the ordered set $\mathcal{Z}$. Formally:
$$
\mathbb{I}_{\mathcal{Z}}(z_{\ell})=\ell 
%\quad \text { if and only if } \quad z=z_{\ell} 
\quad \text { for } \quad \ell \in\{1,2, \ldots,|\mathcal{Z}|\}
$$
%where $z_{\ell}$ denotes the ${\ell}$-th element of the ordered set $\mathcal{Z}$.

\textbf{Inverse Index Function}. 
The inverse index function 
$\mathbb{I}_{\mathcal{Z}}^{-1}:\{1,2, \ldots,|\mathcal{Z}|\} \rightarrow \mathcal{Z}$
%$\mathbb{I}_{\mathcal{Z}}^{-1}({\ell})$
 retrieves the element $z \in \mathcal{Z}$ corresponding to a given index ${\ell}$.
Formally:
$$
\mathbb{I}_{\mathcal{Z}}^{-1}({\ell})=z_{\ell} \quad \text { for } \quad {\ell} \in\{1,2, \ldots,|\mathcal{Z}|\}
$$

%Properties
%1. Index Function: $\mathbb{I}_{\mathcal{Z}}: \mathcal{Z} \rightarrow\{1,2, \ldots,|\mathcal{Z}|\}$
%2. Inverse Index Function: $\mathbb{I}_{\mathcal{Z}}^{-1}:\{1,2, \ldots,|\mathcal{Z}|\} \rightarrow \mathcal{Z}$

%%%%%%%%%%%%%%%%%%%%%%%%%%%%%%%%%%%%%%%%%%%%%%%%%%%%%%
%%%%%%%%%%%%%%%% Updated: 17th July %%%%%%%%%%%%%%%%%%
%%%%%%%%%%%%%%%%%%%%%%%%%%%%%%%%%%%%%%%%%%%%%%%%%%%%%%
\begin{table}[H]
\centering
\renewcommand{\arraystretch}{1.4}
{\small
\begin{tabular}{L{9.9cm}|L{6.5cm}}
\hline
\multicolumn{1}{c|}{\textit{Function}}  & \multicolumn{1}{c}{\textit{Definition}}\\ \hline
%\toprule 
%\hline
%----------------------------------------------------
\textbf{Index Function} 

$~\bullet~$ Maps an element $z\in\Z$ to its position in $\Z$

$~\bullet~$ $\mathbb{I}_{\Z}: \Z \to 
\3{1,2,...,\abs{\Z}}
%\3{n\in \mathbb{N} \mid 1\leq n \leq \abs{\Z}}
$ 
&
$
~~\mathbb{I}_{\Z}(z) 
=
b, \qquad z = \Z_{[b]}
%\begin{cases} 
%	k & \text{if~~~} d = \tilde{\mathfrak{d}}_{[k]} \in \tilde{\mathfrak{d}} 
%	\\ 
%	\emptyset & \text{if~~~} d \medskip\notin \tilde{\mathfrak{d}} 
%\end{cases}
$
\\ 
\hline 
%----------------------------------------------------
\textbf{Inverse Index Function}

$~\bullet~$ Retrieves the element $z\in\Z$ corresponding to an index

$~\bullet~$ $\mathbb{I}^{-1}_{\Z}: \{1, 2, \ldots, 
\abs{Z}
%|\tilde{\mathfrak{d}}|
\} \to \Z$ 
&
~~$\mathbb{I}^{-1}_{\Z}(b) 
= 
\Z_{[b]}  
,\quad
b \in \{1, 2, \ldots, \abs{Z}\} 
%\begin{cases} 
%	\tilde{\mathfrak{d}}_{[k]} & \text{if~~~} k \in \{1, 2, \ldots, 
%	Z
%	\} 
%	\\ 
%	\emptyset & \text{if~~~} k \notin \{1, 2, \ldots, 
%	Z
%	\} 
%\end{cases}
$
\\ 
\hline
%----------------------------------------------------
\textbf{Closest Trading Day Function} 


$~\bullet~$ Returns the next closest trading day to $d\in\mathfrak{d}$ within $\tilde{\mathfrak{d}}$

$~\bullet~$ $\Lambda: \mathfrak{d} \to \tilde{\mathfrak{d}}$
&
~~
$\Lambda(d) 
:= 
\min \3{ \tilde{d} \in \tilde{\mathfrak{d}} \mid \tilde{d} \geq d }$
\\
\hline
\end{tabular}
}
\end{table}
%%%%%%%%%%%%%%%%%%%%%%%%%%%%%%%%%%%%%%%%%%%%%%%%%%%%%




%%%%%%%%%%%%%%%%%%%%%%%%%%%%%%%%%%%%%%%%%%%%%%%%%%%%%%%
%%%%%%% This was the table used before 17th July %%%%%% 
%%%%%%%%%%%%%%%%%%%%%%%%%%%%%%%%%%%%%%%%%%%%%%%%%%%%%%%
%
%\begin{table}[H]
%\centering
%\renewcommand{\arraystretch}{1.4}
%{\small
%\begin{tabular}{L{9.9cm}|L{6.5cm}}
%\hline
%\multicolumn{1}{c|}{\textit{Function}}  & \multicolumn{1}{c}{\textit{Definition}}\\ \hline
%%\toprule 
%%\hline
%%----------------------------------------------------
%\textbf{Index Function} 
%
%$~\bullet~$ Maps a trading day $d \in \tilde{\mathfrak{d}}$ to its position in $\tilde{\mathfrak{d}}$
%
%$~\bullet~$ $\mathbb{I}_{\tilde{\mathfrak{d}}}: \tilde{\mathfrak{d}} \to \{1, 2, \ldots, Z
%\}$ 
%&
%$
%~~\mathbb{I}_{\tilde{\mathfrak{d}}}(d) 
%=
%k, \qquad d = \tilde{\mathfrak{d}}_{[k]} \in \tilde{\mathfrak{d}} 
%%\begin{cases} 
%%	k & \text{if~~~} d = \tilde{\mathfrak{d}}_{[k]} \in \tilde{\mathfrak{d}} 
%%	\\ 
%%	\emptyset & \text{if~~~} d \medskip\notin \tilde{\mathfrak{d}} 
%%\end{cases}
%$
%\\ 
%\hline 
%%----------------------------------------------------
%\textbf{Inverse Index Function}
%
%$~\bullet~$ Retrieves the trading day corresponding to an index
%
%$~\bullet~$ $\mathbb{D}_{\tilde{\mathfrak{d}}}: \{1, 2, \ldots, 
%Z
%%|\tilde{\mathfrak{d}}|
%\} \to \tilde{\mathfrak{d}}$ 
%&
%~~$\mathbb{D}_{\tilde{\mathfrak{d}}}(k) 
%= 
%\tilde{\mathfrak{d}}_{[k]}  
%,\quad
%k \in \{1, 2, \ldots, Z\} 
%%\begin{cases} 
%%	\tilde{\mathfrak{d}}_{[k]} & \text{if~~~} k \in \{1, 2, \ldots, 
%%	Z
%%	\} 
%%	\\ 
%%	\emptyset & \text{if~~~} k \notin \{1, 2, \ldots, 
%%	Z
%%	\} 
%%\end{cases}
%$
%\\ 
%\hline
%%----------------------------------------------------
%\textbf{Closest Trading Day Function} 
%
%
%$~\bullet~$ Returns the next closest trading day to $d\in\mathfrak{d}$ within $\tilde{\mathfrak{d}}$
%
%$~\bullet~$ $\Lambda: \mathfrak{d} \to \tilde{\mathfrak{d}}$
%&
%~~$\Lambda(d) 
%:= 
%\min \left\{ \tilde{d} \in \tilde{\mathfrak{d}} \mid \tilde{d} \geq d \right\}$
%\\
%\hline
%\end{tabular}
%}
%\end{table}
%%%%%%%%%%%%%%%%%%%%%%%%%%%%%%%%%%%%%%%%%%%%%%%%%%%%%%

%%----------------------------------------------------

\bx 
Throughout our analysis, we will work with daily stock market closing data for each trading day. However, we will exploit the time component of $\mathcal Y_0^i$ to assign an \qquote{effective treatment date} to each article. Namely, define $\tilde d_0^i$ as the day at which article $i$'s information can be incorporated into the stock market; then, $\tilde d_0^i$ is the publication date if the article was published on a trading day before the stock market was closed, and is equal to the next closest trading day otherwise. 
To compute the next closest trading day to $d\in\mathfrak d$ within $\tilde{\mathfrak d}$, we need to work with a function $\Lambda:\mathfrak d \to \tilde{\mathfrak d}$ such that  
$\Lambda(d) 
:= 
\min \{ \tilde{d} \in \tilde{\mathfrak{d}} \mid \tilde{d} \geq d \}$. 
Thus, now we can define:
\begin{align*}
\tilde{d}_0^i :=
\mycases{llll}{
d_0^i & \IF & d_0^i \in \tilde{\mathfrak d} ~\wedge~t_0^i < \t{17:30}
\\
\Lambda(d_0^i)
& \IF & d_0^i \not \in \tilde{\mathfrak d} ~\vee~t_0^i \geq  \t{17:30}
}
.
\end{align*}

