\subsection{Literature Review} \label{sec:lierature_review}

%==============[	  Classics  ]==============
%Pairs trading has emerged as a cornerstone of statistical arbitrage strategies, with a rich history in both academic research and practical applications. 
The foundational work of \cite{Gatev2006} provided the first comprehensive academic study of pairs trading, documenting significant excess returns of up to 11\% annually for self-financing portfolios over a 40-year period from 1962 to 2002. This seminal paper was complemented by the theoretical framework developed in \cite{Elliott2005}, which introduced a mean-reverting Gaussian Markov chain model for spread dynamics and established analytical methods for parameter estimation using the EM algorithm.

%==============[	  Empirical investigations  ]==============
Empirical investigations have thoroughly examined the profitability of pairs trading across different markets and time periods. For instance, \cite{Chen2019} reported large abnormal returns driven by short-term reversals and pairs momentum effects, while \cite{Do2010} showed that simple pairs trading remains viable in turbulent periods despite a general profitability decline in later years. In a UK-centric study, \cite{Bowen2014} recorded moderate annual returns once risk and liquidity were accounted for. Large-scale assessments in \cite{Krauss2016} and \cite{Rad2016} confirmed that distance, cointegration, and copula-based strategies can yield significant alpha but exhibit important differences regarding convergence speed and trading frequencies.

%==============[	  Cointegration  ]==============
A popular way to identify and exploit persistent relationships in pairs trading has involved cointegration analysis. \cite{vidyamurthy2004pairs} stands out as a seminal reference, detailing how cointegration can be applied to detect mean-reverting spreads in equity markets. 
Subsequent research has explored various aspects of this approach: \cite{Caldeira2013} demonstrated the effectiveness of cointegration-based selection methods in the Brazilian market, while \cite{Huck2014} provided evidence that cointegration-based strategies outperform distance-based methods. \cite{Cartea2015} extended the framework by incorporating optimal dynamic investment strategies, and \cite{Lintilhac2016} applied these techniques to cryptocurrency markets.

%==============[	  Copulas  ]==============
A growing strand of research leverages copulas to model more general dependencies beyond linear correlation.
\cite{Min2010} introduced Bayesian inference for multivariate copulas using pair-copula constructions, while \cite{stander2013trading} offer a copula-based approach for detecting relative mispricing. 
Extensions in \cite{Liew2013} and \cite{Xie2016} underscore that copulas outperform distance-of-prices rules in capturing tail dependencies.
Multi-dimensional variants have been proposed (e.g., \cite{lau2016multi}) to incorporate three or more assets into a single framework. Further refinements, like those introduced in \cite{Krauss2017} and \cite{zhi2017dynamic}, combine t-copulas or dynamic copula-GARCH models with individualized thresholds for improved risk-adjusted returns. In the high-frequency domain, \cite{Chu2018} showed that copula-based mispricing indices can be coupled with deep learning for profitability enhancements. Recent efforts also explore mixed copulas (\cite{SabinodaSilva2023}), ARMA-GARCH approaches (\cite{Wang2023}), and copulas specialized for cointegrated assets (\cite{He2024}), culminating in improved alpha extraction. Finally, \cite{Tadi2025} proposes reference-asset-based copula trading specifically for cryptocurrencies. 
%

%==============[	  Didactic resources  ]==============
Practical guidance and pedagogical discussions on pairs trading can be found in \cite{hudsonthames2024}, which provides a broad compendium of methods, from classical cointegration to machine learning-based selection. On a methodological note, \cite{alexander2008market} offers valuable introductions to both cointegration analysis and copula applications in financial markets, particularly in chapters II.5 and II.6.


%==============[	  Alternative approaches  ]==============
Beyond cointegration or copula methodologies, several innovative techniques have surfaced. 
%New approaches to modeling and parameter estimation for pairs trading appear in \cite{do2006new} and \cite{Zeng2014}, with the latter introducing threshold-based mean-reversion strategies. 
\cite{do2006new} developed a stochastic residual spread model, while \cite{Zeng2014} focused on optimal threshold determination. 
In more recent research, \cite{Sarmento2020} incorporates machine learning (OPTICS clustering) to constrain search space, while \cite{Johansson2024} leverages convex-concave optimization for multi-asset statistical arbitrage. Reinforcement learning is featured in \cite{Han2023} for automated pair selection, and \cite{qureshi2024pairs} employs a graphical matching approach to reduce overlap among chosen pairs. Further, \cite{Roychoudhury2023} couples clustering with deep RL for equity indices, whereas \cite{Rotondi2025} applies a partial correlation-based distance to cluster promising trading candidates.
%Alternative approaches to traditional pairs trading have been proposed. \cite{do2006new} developed a stochastic residual spread model, while \cite{Zeng2014} focused on optimal threshold determination. Machine learning applications have gained prominence, with \cite{Sarmento2020} utilizing LSTM networks, \cite{Han2023} employing unsupervised learning for pair selection, and \cite{Roychoudhury2023} combining clustering with deep reinforcement learning. Novel optimization approaches include the convex-concave framework of \cite{Johansson2024}, the graphical matching approach of \cite{qureshi2024pairs}, and the clustering-based methodology of \cite{Rotondi2025}.
%

%==============[	  Synthetic Controls / Index-Tracking  ]==============
The method of replicating a target asset's returns by constructing a portfolio of contributor assets is reminiscent of index-tracking procedures. Classic treatments connecting cointegration analysis and hedging tasks (e.g., \cite{Alexander1999} and \cite{Alexander2002}) lay theoretical groundwork for such an approach. Subsequent refinements in \cite{Alexander2005a} and \cite{Alexander2005b} investigate how cointegration outperforms traditional techniques in crafting robust index trackers and exploiting time-varying market regimes. Complementary research (e.g., \cite{Shu2020}) shows that sparse solutions across a large universe can reduce transaction costs, an idea further corroborated in \cite{Bradrania2021}, where machine learning identifies dynamic selection methods for index constituents. These frameworks illustrate how synthetic control concepts provide a flexible foundation for building market-neutral positions or tracking assets with fewer assumptions.
%Our synthetic control methodology draws inspiration from the index tracking literature. \cite{Alexander1999} pioneered the application of cointegration to tracking problems, while \cite{Alexander2002} and \cite{Alexander2005a} developed enhanced indexing strategies. \cite{Alexander2005b} explored market regime effects, and recent work by \cite{Shu2020} introduced adaptive elastic net methods for high-dimensional tracking. \cite{Bradrania2021} incorporated machine learning for state-dependent stock selection, demonstrating the evolving sophistication of tracking methodologies.