\section{Pairs Trading Strategy via Mispricing Indices (MI)} \label{sec:trading_strategy}

In this section, we adapt the mispricing index (MI) strategy from \cite{Xie2016} to our setting, wherein we trade a target asset (with returns $R_t$) against its synthetic counterpart (with returns $R_t^*$). While the strategy might initially appear unconventional, it hinges on interpreting conditional probabilities of daily returns as an evolving measure of relative mispricing. Below, we detail the essential components of the approach and how trading positions are opened and closed.

%%%%%%%%%%%%%%%%%%%%%%%%%%%%%%%%%%%%%%%%%%%%%%%%%%%%%
% MISPRICING INDICES
%%%%%%%%%%%%%%%%%%%%%%%%%%%%%%%%%%%%%%%%%%%%%%%%%%%%%
\subsection{Mispricing Index (MI), Flags and Cumulative Mispricing Index (CMI)}

%==============[	  MI  ]==============
On each trading day $t$, let $r_t$ and $r_t^*$ respectively denote the realized returns for the target and synthetic assets. We define two conditional mispricing indices,
\begin{align*}
%MI_t
MI_t^{R \mid R^*} 
&:= \mathbb{P}(R_t \leq r_t \mid R_t^* = r_t^*)
= 
%\left.
\frac{\partial C_{\hat{\theta}}(F_R(r_t), F_{R^*}(r_t^*))}{\partial F_{R^*}(r_t^*)}
%\right|_{r_t,r_t^*}
,
\\[0.4em]
%MI_t^*
MI_t^{R^* \mid R} 
&:= \mathbb{P}(R_t^* \leq r_t^* \mid R_t = r_t)
= 
%\left.
\frac{\partial C_{\hat{\theta}}(F_R(r_t), F_{R^*}(r_t^*))}{\partial F_R(r_t)}
%\right|_{r_t,r_t^*}
.
\end{align*}

The quantity $MI_t^{R \mid R^*}$ measures how ``mispriced'' the target asset appears when conditioned on that day's synthetic return, whereas $MI_t^{R^* \mid R}$ does the same for the synthetic asset when conditioned on the target return.
%
%==============[	  FLAGS  ]==============
Since a single day's mispricing index reflects only an instantaneous view, we accumulate daily signals over time to gauge how much the returns have gradually driven prices apart (or together). We define a \emph{flag} series for each asset, defined as a running sum of daily deviations from $0.5$\footnote{The subtraction of $0.5$ centers the cumulative sum so that deviations from zero reflect mispricing.}. Let $\text{Flag}_{R}(0)=\text{Flag}_{R^*}(0)=0$, then, for $t=1, ..., T$ we have
%\begin{align*}
$$\begin{array}{llll}
\mathrm{Flag}^{R}_{t} 
&= \mathrm{Flag}^{R}_{t-1} + (MI_t^{R \mid R^*} - 0.5)
&=\sum_{s=1}^t (MI_s^{R \mid R^*} - 0.5),
%&&\text{Flag}_{R}(0) = 0,
\\[0.2em]
\mathrm{Flag}^{R^*}_{t}
&= \mathrm{Flag}^{R^*}_{t-1} + (MI_t^{R^* \mid R} - 0.5)
&= \sum_{s=1}^t (MI_s^{R^* \mid R} - 0.5)
%&&\text{Flag}_{R^*}(0) = 0
.
\end{array}$$
%\end{align*}
%Equivalently, each raw flag is a cumulative sum over the trading horizon:
%\begin{equation}
%\mathrm{Flag}^{R}_{t} = \sum_{s=1}^t (MI_s^{R \mid R^*} - 0.5),
%\qquad
%\mathrm{Flag}^{R^*}_{t}= \sum_{s=1}^t (MI_s^{R^* \mid R} - 0.5).
%\end{equation}
Similar to plotting cumulative returns, these raw flags track the net effect of mispricing signals over time. 
%However, the \emph{real flag} series, $\text{CMI}_R$ and $\text{CMI}_{R^*}$, will be \emph{reset to zero} whenever an open position is fully closed (i.e., after an exit signal). This reset ensures that newly detected mispricings are measured from a fresh baseline rather than mixing with completed trades.
%


%==============[	  CMI  ]==============

To prevent the compounding of stale mispricing signals, we formally define a Cumulative Mispricing Index (CMI) as the reset-adjusted flag series through the recursive relationship:
$$
\mathrm{CMI}^{R}_{t} =
\begin{cases}
\mathrm{CMI}^{R}_{t-1} + (MI_t^{R\mid R^*} - 0.5), & \text{if no position reset occurs at time } t,\\
0, & \text{if a position is closed at } t,
\end{cases}
$$
$$
\mathrm{CMI}^{R^*}_{t} =
\begin{cases}
\mathrm{CMI}^{R^*}_{t-1} + (MI_t^{R^*\mid R} - 0.5), & \text{if no position reset occurs at time } t,\\
0, & \text{if a position is closed at } t,
\end{cases}
$$
where \(\mathrm{CMI}^R_{0}= \mathrm{CMI}^{R^*}_{0}=0\). 
Unlike the raw flags that accrue continuously, each CMI absorbs daily mispricing signals only until a trade is exited, at which point it is reset to zero. This mechanism ensures that any fresh mispricing accumulates from a ``clean slate,'' thereby preventing the influence of past, already-traded mispricing from compounding future signals.

%==============[	  Algorithms  ]==============
We formally present the procedures to compute the mispricing index and update the cumulative mispricing indices in \cref{alg:mispricing_indices} and \cref{alg:cmi_update}

%----------------------------------------------------
% To track active mispricing signals, we introduce the Cumulative Mispricing Index (CMI), which resets upon trading activity. Let $\tau_k$ denote the sequence of times when trades are closed (with $\tau_0 = 0$). For any time $t \in [\tau_k, \tau_{k+1})$, the CMIs are defined as:
%
%\begin{equation*}
%\text{CMI}_{R}(t) = \sum_{s=\tau_k + 1}^t (MI_s^{R \mid R^*} - 0.5),
%\end{equation*}
%
%\begin{equation*}
%\text{CMI}_{R^*}(t) = \sum_{s=\tau_k + 1}^t (MI_s^{R^* \mid R} - 0.5).
%\end{equation*}
%
%These indices maintain a running tally of mispricing signals since the most recent trade closure, providing a cleaner measure of emerging price divergences.




%%%%%%%%%%%%%%%%%%%%%%%%%%%%%%%%%%%%%%%%%%%%%%%%%%%%%
% TRADING STRATEGY 
%%%%%%%%%%%%%%%%%%%%%%%%%%%%%%%%%%%%%%%%%%%%%%%%%%%%%
\subsection{Trading Logic}

We implement a dollar-neutral trading strategy that capitalizes on relative mispricing signals between the target and synthetic assets. The trading rule ($TR$) we employ builds upon the frameworks of \cite{Xie2016} and \cite{Rad2016}, incorporating their key insights about signal combination logic. While \cite{Xie2016} originally proposed an \qquote{or-or} framework, where trades are initiated when either asset shows mispricing and closed when either asset exhibits correction, \cite{Rad2016} demonstrated that a more conservative \qquote{and-or} approach yields more robust performance. This latter approach requires concurrent mispricing signals from both assets to open positions while maintaining a sensitive exit strategy where correction in either asset triggers position closure.

Let $D_l$ and $D_u$ denote the lower and upper thresholds for opening positions, and $S_l$ and $S_u$ the lower and upper stop-loss boundaries. Starting with $TR_0=0$, for $t=1,...,T$, the trading rule evolves as follows:
%
%==============[	  FORMULA  ]==============
\begin{align*}\label{eq:trading_rule}
&TR_t(\t{CMI}_t^R, \t{CMI}_t^{R^*}, TR_{t-1}; D_l, D_u, S_l, S_u) 
=
\\[0.2em]\nonumber
&\mycases{clllll}{
+1 & \IF ~  
(\t{CMI}_t^R \leq  D_l 
~\t{and}~ 
\t{CMI}_t^{R^*} \geq D_u)
\\
-1 & \IF ~ 
(\t{CMI}_t^R \geq D_u 
~\t{and}~ 
\t{CMI}_t^{R^*} \leq D_l)
\\
0 & \IF~
%\t{either} 
$$
%\mycases{llllll}{
\begin{cases}
\biggl\{
TR_{t-1}=1 
~~~\t{and}~ 
\bigl[
(\ub{\t{CMI}_t^R\geq 0 ~\t{or}~ \t{CMI}_t^{R^*}\leq 0}{\t{take profit}})
~\t{or}~
(\ub{\t{CMI}_t^R\leq S_l ~\t{or}~ \t{CMI}_t^{R^*}\geq S_u}{\t{stop loss}})
\bigr]
\biggr\}
,\t{or}
\\
\biggl\{
TR_{t-1}=-1 
~\t{and}~ 
\bigl[
(\ub{\t{CMI}_t^R\leq 0 ~\t{or}~ \t{CMI}_t^{R^*}\geq 0}{\t{take profit}})
~\t{or}~
(\ub{\t{CMI}_t^R\geq S_u ~\t{or}~ \t{CMI}_t^{R^*}\leq S_l}{\t{stop loss}})
\bigr]
\biggr\}
%}
\end{cases}
$$
\\
TR_{t-1} & \t{otherwise}
}
\end{align*}

%==============[	  Explanation of formula  ]==============
That is, at the beginning of each trading day \(t\), observe the current values of both mispricing indicators, \(\mathrm{CMI}_t^R\) (for the target asset) and \(\mathrm{CMI}_t^{R^*}\) (for the synthetic). The trading rule \(TR_t\) can take one of three values: \(+1\), \(-1\), or \(0\), indicating a \qquote{long-short}, \qquote{short-long}, or \qquote{flat} position, respectively. When no position is open (i.e., \(TR_{t-1} = 0\)), the rule opens a position only if there is simultaneous mispricing in both assets according to the thresholds \(D_l\) and \(D_u\). 
%----------------------------------------------------
Specifically,
\begin{itemize}
  \item \textbf{Long target/Short synthetic (+1)}: Entered when both CMIs indicate the target asset is underpriced relative to the synthetic ($\text{CMI}_t^R \leq D_l$ \textbf{and} $\text{CMI}_t^{R^*} \geq D_u$).
  \item \textbf{Short target/Long synthetic (-1)}: Entered when both CMIs indicate the target asset is overpriced relative to the synthetic ($\text{CMI}_t^R \geq D_u$ \textbf{and} $\text{CMI}_t^{R^*} \leq D_l$).
\end{itemize} 
%----------------------------------------------------
%if \(\mathrm{CMI}_t^R\) falls below \(D_l\) at the same time that \(\mathrm{CMI}_t^{R^*}\) rises above \(D_u\), then \(TR_t\) is set to \(+1\). Conversely, if \(\mathrm{CMI}_t^R\) rises above \(D_u\) while \(\mathrm{CMI}_t^{R^*}\) falls below \(D_l\), \(TR_t\) is set to \(-1\).

Once a position is open (either \(TR_{t-1} = +1\) or \(TR_{t-1} = -1\)), the logic checks each day whether the mispricing has corrected enough to trigger a take-profit condition or crossed critical boundaries that trigger a stop-loss. These checks apply to either of the two mispricing indices, so if correction or a stop-loss occurs in any one of them, the entire position is closed. Mathematically, this is captured by the \qquote{OR} clauses in the formula, which evaluate whether \(\mathrm{CMI}_t^R\) or \(\mathrm{CMI}_t^{R^*}\) has crossed the zero line (for take-profit) or moved beyond the \((S_l, S_u)\) band (for stop-loss). If one of these events occurs, then \(TR_t\) is set to \(0\), and the mispricing indices are both reset to zero for the next trading day. 
%----------------------------------------------------
%Specifically:
%\begin{itemize}
%\item \textit{Exit position} (0): An existing position is closed under two scenarios:
%\begin{itemize}
%  \item Take profit: When \textbf{either} CMI reverts to zero, suggesting the mispricing has been corrected
%  \item Stop loss: When \textbf{either} CMI reaches its respective stop-loss boundary
%\end{itemize}
%\end{itemize}
%----------------------------------------------------
If neither a take-profit nor a stop-loss threshold is met, then the position remains unchanged, meaning \(TR_t\) simply inherits the previous value \(TR_{t-1}\).

Intuitively, when both indicators are simultaneously misaligned (one significantly high and the other significantly low), the strategy deems it a strong signal to open a dollar-neutral position that is long the \qquote{undervalued} side and short the \qquote{overvalued} side. As soon as either index crosses back toward zero (suggesting partial correction of that asset's mispricing) or breaches a stop-loss boundary (indicating that the trade is moving unfavorably), the position is liquidated. This \qquote{and-or} logic helps filter out noise in the daily movements and more reliably captures episodes in which both assets appear to be drifting apart (opening a trade) and then swiftly catches at least one side reverting (closing the trade). 
%
%==============[	  Algorithm  ]==============
We formally present this procedure in \cref{alg:mi_pairs_trading}

%==============[	  Parametric choice  ]==============
As in \cite{Xie2016}, we set $(D_l, D_u)=(-0.6,0.6)$ and $(S_u,S_l)=(-2,2)$ and we will explore sensitivity to other parametric choices in future robustness checks.




%The intuition behind these rules is that the cumulative mispricing captured by the flag series reflects how returns may drive prices apart or together over time. Because trading is ultimately conducted on prices, it becomes necessary to integrate the daily return information in a cumulative measure and to reset it upon entering a new trade.


%----------------------------------------------------
%----------------------------------------------------
%Starting with $TR_0=0$, the strategy generates three possible positions at each time $t=1,...,T$:
%\begin{itemize}
%  \item Long target/Short synthetic (+1): Entered when both CMIs indicate the target asset is underpriced relative to the synthetic ($\text{CMI}_t^R \leq D_l$ \textbf{and} $\text{CMI}_t^{R^*} \geq D_u$).
%  \item Short target/Long synthetic (-1): Entered when both CMIs indicate the target asset is overpriced relative to the synthetic ($\text{CMI}_t^R \geq D_u$ \textbf{and} $\text{CMI}_t^{R^*} \leq D_l$).
%  \item Exit position (0): An existing position is closed under two scenarios:
%\begin{itemize}
%  \item Take profit: When \textbf{either} CMI reverts to zero, suggesting the mispricing has been corrected
%  \item Stop loss: When \textbf{either} CMI reaches its respective stop-loss boundary
%\end{itemize}
%\end{itemize}
% If none of the conditions for opening or closing a position are met, the previous trading signal ($TR_{t-1}$) is maintained. Formally, the trading rule is defined as:
%----------------------------------------------------
