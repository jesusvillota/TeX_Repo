\inserthere{tab:copula_fit}
\begin{table}[H]
\centering
\caption{Copula Fitting Results}
\label{tab:copula_fit}
\begin{tabular}{lrrrr}
\toprule
Copula & SIC & AIC & HQIC \\
\toprule
Joe & -671.50 & -677.39 & -675.26 \\
Clayton & -1168.92 & -1174.80 & -1172.67 \\
Gumbel & -1210.02 & -1215.90 & -1213.78 \\
Frank & -1212.68 & -1218.56 & -1216.43 \\
Gaussian & -1337.69 & -1343.57 & -1341.44 \\
N14 & -1425.18 & -1431.06 & -1428.94 \\
Student & -1427.05 & -1432.94 & -1430.81 \\
\bottomrule
\end{tabular}
\label{tab:copula_fits}
%----------------------------------------------------
\vspace{0.5cm}
\begin{minipage}{\textwidth}
\setlength{\parindent}{0pt}
\small\textit{Note: 
This table reports goodness-of-fit measures for various copula specifications used to model the dependence structure between the target and synthetic asset returns. The evaluation metrics include the Schwarz Information Criterion (SIC), Akaike Information Criterion (AIC), and Hannan-Quinn Information Criterion (HQIC). All criteria balance model fit against complexity, with lower values indicating better models. The Student-t copula achieves the best fit across all three criteria, followed closely by the N14 mixed copula, suggesting that the dependence structure exhibits tail dependence and asymmetry.
}
\end{minipage}
%----------------------------------------------------
\end{table}