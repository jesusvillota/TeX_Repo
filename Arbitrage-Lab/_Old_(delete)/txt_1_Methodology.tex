


%%%%%%%%%%%%%%%%%%%%%%%%%%%%%%%%%%%%%%%%%%%%%%%%%%%%%
%%%%%%%%%%%%%%%%%%%%%%%%%%%%%%%%%%%%%%%%%%%%%%%%%%%%%
%%%%%%%%%%%%%%%%%%%%%%%%%%%%%%%%%%%%%%%%%%%%%%%%%%%%%
%%%%%%%%%%%%%%%%%%%%%%%%%%%%%%%%%%%%%%%%%%%%%%%%%%%%%
%%%%%%%%%%%%%%%%%%%%%%%%%%%%%%%%%%%%%%%%%%%%%%%%%%%%%
%%%%%%%%%%%%%%%%%%%%%%%%%%%%%%%%%%%%%%%%%%%%%%%%%%%%%
%%%%%%%%%%%%%%%%%%%%%%%%%%%%%%%%%%%%%%%%%%%%%%%%%%%%%
%%%%%%%%%%%%%%%%%%%%%%%%%%%%%%%%%%%%%%%%%%%%%%%%%%%%%
%%%%%%%%%%%%%%%%%%%%%%%%%%%%%%%%%%%%%%%%%%%%%%%%%%%%%
%%%%%%%%%%%%%%%%%%%%%%%%%%%%%%%%%%%%%%%%%%%%%%%%%%%%%






\section{Methodology}
\label{sec:methodology}

\subsection{Synthetic Asset Construction}
The core component of our pairs trading strategy involves constructing a synthetic asset that replicates the price behavior of a target security (AAPL) using a combination of assets from a donor pool. We employ a two-stage optimization approach with cardinality constraints:

\begin{enumerate}
    \item \textbf{Initial Unconstrained Optimization:} Solve the convex optimization problem:
    \begin{equation}
        \min_{\mathbf{w}} \sum_{t=1}^T \left(y_t - \sum_{i=1}^N w_i x_{i,t}\right)^2 \quad \text{s.t.} \quad \sum_{i=1}^N w_i = 1
    \end{equation}
    where $ y_t $ represents the target asset's log-price and $ x_{i,t} $ the donor assets' log-prices.
    
    \item \textbf{Cardinality-Constrained Refinement:} Select the top $ k $ assets with largest absolute weights from the initial solution, then resolve the optimization problem restricted to these $ k $ assets. The optimal $ k $ is determined through time-series cross-validation with 5 splits, minimizing mean squared error.
\end{enumerate}

The final synthetic asset price series $ S_t $ is computed as:
\begin{equation}
    S_t = \sum_{i=1}^k w_i^* x_{i,t}^*
\end{equation}
where $ w_i^* $ are the optimized weights and $ x_{i,t}^* $ the selected donor assets' log-prices.

\subsection{Copula Model Specification}
We model the dependence structure between the target asset and synthetic asset returns using copula functions. The methodology proceeds as follows:

\begin{enumerate}
    \item \textbf{Marginal Distribution Transformation:} Convert returns to uniform marginals using empirical CDFs:
    \begin{equation}
        U_t = F_{\text{emp}}(r_t^{A1}), \quad V_t = F_{\text{emp}}(r_t^{A2})
    \end{equation}
    
    \item \textbf{Copula Selection:} Evaluate multiple Archimedean and elliptical copulas (Gumbel, Clayton, Frank, Joe, N14, Gaussian, and Student-t) using maximum likelihood estimation. The log-likelihood for a copula $ C_\theta $ with parameter $ \theta $ is:
    \begin{equation}
        \mathcal{L}(\theta) = \sum_{t=1}^T \ln c_\theta(U_t, V_t)
    \end{equation}
    where $ c_\theta $ is the copula density function.
\end{enumerate}

\subsection{Trading Strategy Implementation}
The trading strategy employs a dollar-neutral approach based on copula-derived mispricing indices:

\begin{enumerate}
    \item \textbf{Signal Generation:} Compute mispricing indices $ M_t $ using the conditional copula probabilities:
    \begin{align}
        M_t^{A1} &= P(U_t \leq u | V_t = v) \\
        M_t^{A2} &= P(V_t \leq v | U_t = u)
    \end{align}
    
    \item \textbf{Position Management:} Implement trading rules with thresholds:
    \begin{itemize}
        \item Open positions when $ |M_t| > 0.5 $ (entry threshold)
        \item Close positions when $ |M_t| < 0.2 $ (exit threshold)
        \item Stop-loss triggers at $ |M_t| > 2.0 $
    \end{itemize}
    
    \item \textbf{Portfolio Construction:} Maintain dollar-neutral exposure through dynamic position sizing:
    \begin{equation}
        \Delta \text{Position}_t = \begin{cases}
            \text{Long A1/Short A2} & M_t < -0.5 \\
            \text{Short A1/Long A2} & M_t > 0.5 \\
            0 & \text{otherwise}
        \end{cases}
    \end{equation}
\end{enumerate}

\subsection{Performance Evaluation}
We assess strategy effectiveness through:
\begin{itemize}
    \item In-sample/out-of-sample division (70\%-30\% split)
    \item Cumulative returns calculation:
    \begin{equation}
        R_{\text{cum}} = \sum_{t=1}^T (r_t^{A1} \cdot w_t^{A1} + r_t^{A2} \cdot w_t^{A2})
    \end{equation}
    \item Risk-adjusted metrics: Sharpe ratio and maximum drawdown
\end{itemize}

%%%%%%%%%%%%%%%%%%%%%%%%%%%%%%%%%%%%%%%%%%%%%%%%%%%%%
%%%%%%%%%%%%%%%%%%%%%%%%%%%%%%%%%%%%%%%%%%%%%%%%%%%%%
%%%%%%%%%%%%%%%%%%%%%%%%%%%%%%%%%%%%%%%%%%%%%%%%%%%%%
%%%%%%%%%%%%%%%%%%%%%%%%%%%%%%%%%%%%%%%%%%%%%%%%%%%%%
%%%%%%%%%%%%%%%%%%%%%%%%%%%%%%%%%%%%%%%%%%%%%%%%%%%%%
%%%%%%%%%%%%%%%%%%%%%%%%%%%%%%%%%%%%%%%%%%%%%%%%%%%%%
%%%%%%%%%%%%%%%%%%%%%%%%%%%%%%%%%%%%%%%%%%%%%%%%%%%%%
%%%%%%%%%%%%%%%%%%%%%%%%%%%%%%%%%%%%%%%%%%%%%%%%%%%%%
%%%%%%%%%%%%%%%%%%%%%%%%%%%%%%%%%%%%%%%%%%%%%%%%%%%%%
%%%%%%%%%%%%%%%%%%%%%%%%%%%%%%%%%%%%%%%%%%%%%%%%%%%%%



%\section{Methodology}
%
%\subsection{Synthetic Asset Construction}
%Our methodology begins with the construction of a synthetic asset that closely tracks our target asset. We employ a cardinality-constrained approach to synthetic control, which optimizes the following problem:
%
%\begin{equation}
%\begin{aligned}
%\min_{w} & \sum_{t=1}^{T} (y_t - X_t w)^2 \\
%\text{s.t. } & \sum_{i=1}^{N} w_i = 1 \\
%& \|\{i: w_i \neq 0\}\|_0 \leq k
%\end{aligned}
%\end{equation}
%
%where $y_t$ represents the log-price of the target asset at time $t$, $X_t$ is the vector of log-prices for the donor pool assets at time $t$, $w$ is the weight vector, and $k$ is the maximum number of assets allowed in the synthetic control (cardinality constraint).
%
%The optimization is solved in two stages:
%\begin{enumerate}
%    \item First, we solve the unconstrained problem to identify the most important assets based on their absolute weights.
%    \item Second, we resolve the optimization problem using only the top $k$ assets.
%\end{enumerate}
%
%\subsection{Copula-Based Pairs Trading Framework}
%Having constructed the synthetic asset, we implement a copula-based pairs trading strategy that captures the complex dependence structure between the target asset and its synthetic counterpart. The strategy consists of several key components:
%
%\subsubsection{Return Series Transformation}
%Let $R_1$ and $R_2$ denote the return series of the target and synthetic assets, respectively. We transform these returns into uniform variates using their empirical cumulative distribution functions (ECDFs):
%
%\begin{equation}
%U_1 = F_1(R_1), \quad U_2 = F_2(R_2)
%\end{equation}
%
%where $F_1$ and $F_2$ are the ECDFs constructed using linear interpolation.
%
%\subsubsection{Copula Selection and Fitting}
%We evaluate multiple copula families to capture the dependence structure:
%\begin{itemize}
%    \item Archimedean copulas: Gumbel, Clayton, Frank, Joe, N14
%    \item Elliptical copulas: Gaussian, Student-t
%\end{itemize}
%
%The optimal copula is selected based on maximum likelihood estimation:
%
%\begin{equation}
%\hat{\theta} = \argmax_{\theta} \sum_{t=1}^{T} \log c(U_{1,t}, U_{2,t}; \theta)
%\end{equation}
%
%where $c(\cdot,\cdot;\theta)$ is the copula density with parameter $\theta$.
%
%\subsubsection{Trading Signal Generation}
%We employ a mispricing index (MPI) approach to generate trading signals. The MPI is calculated using:
%
%\begin{equation}
%\text{MPI}_t = \frac{\partial}{\partial u_1} C(u_1, u_2) - \frac{\partial}{\partial u_2} C(u_1, u_2)
%\end{equation}
%
%where $C$ is the fitted copula and $\frac{\partial}{\partial u_i}$ represents the partial derivative with respect to the $i$-th argument.
%
%Trading positions are determined using a flag-based system:
%\begin{equation}
%\text{Position}_t = 
%\begin{cases}
%1 & \text{if Flag}_{1,t} > \alpha \text{ and Flag}_{2,t} < -\alpha \\
%-1 & \text{if Flag}_{1,t} < -\alpha \text{ and Flag}_{2,t} > \alpha \\
%0 & \text{otherwise}
%\end{cases}
%\end{equation}
%
%where $\alpha$ is the threshold parameter and Flags are cumulative measures of the MPI.
%
%\subsection{Portfolio Construction}
%We implement a dollar-neutral strategy where the position sizes are determined by:
%
%\begin{equation}
%\begin{aligned}
%\text{Units}_{1,t} &= \text{Position}_t \cdot \frac{1}{2P_{1,t}} \\
%\text{Units}_{2,t} &= -\text{Position}_t \cdot \frac{1}{2P_{2,t}}
%\end{aligned}
%\end{equation}
%
%where $P_{i,t}$ represents the price of asset $i$ at time $t$. The portfolio return at time $t$ is calculated as:
%
%\begin{equation}
%R_{p,t} = \text{Units}_{1,t} \cdot R_{1,t} + \text{Units}_{2,t} \cdot R_{2,t}
%\end{equation}



%%%%%%%%%%%%%%%%%%%%%%%%%%%%%%%%%%%%%%%%%%%%%%%%%%%%%
%%%%%%%%%%%%%%%%%%%%%%%%%%%%%%%%%%%%%%%%%%%%%%%%%%%%%
%%%%%%%%%%%%%%%%%%%%%%%%%%%%%%%%%%%%%%%%%%%%%%%%%%%%%
%%%%%%%%%%%%%%%%%%%%%%%%%%%%%%%%%%%%%%%%%%%%%%%%%%%%%
%%%%%%%%%%%%%%%%%%%%%%%%%%%%%%%%%%%%%%%%%%%%%%%%%%%%%
%%%%%%%%%%%%%%%%%%%%%%%%%%%%%%%%%%%%%%%%%%%%%%%%%%%%%
%%%%%%%%%%%%%%%%%%%%%%%%%%%%%%%%%%%%%%%%%%%%%%%%%%%%%
%%%%%%%%%%%%%%%%%%%%%%%%%%%%%%%%%%%%%%%%%%%%%%%%%%%%%
%%%%%%%%%%%%%%%%%%%%%%%%%%%%%%%%%%%%%%%%%%%%%%%%%%%%%
%%%%%%%%%%%%%%%%%%%%%%%%%%%%%%%%%%%%%%%%%%%%%%%%%%%%%



\section{Methodology}\label{sec:methodology}

In this section, we outline the methodology employed for constructing a copula-based pairs trading strategy in which one of the two assets is a synthetic reference asset. The approach consists of four main steps: (i) data preprocessing and partitioning, (ii) construction of the synthetic control asset via a cardinality-constrained optimization, (iii) copula fitting to capture dependence, and (iv) generating trading signals based on mispricing indices derived from the fitted copula.

\subsection{Data Preprocessing}
We start by gathering price data (log prices, in particular) for a target symbol, denoted $A_1$, and a donor pool of other assets. The time horizon is split into two segments: a training period and an out-of-sample (testing) period. Formally, we partition the overall timeline $\{ t_1, t_2, \ldots, t_T \}$ into:

\begin{itemize}
    \item A \textit{training set}, $\{ t_1, \ldots, t_{k} \}$, used to fit models.
    \item A \textit{testing set}, $\{ t_{k+1}, \ldots, t_T \}$, used for out-of-sample validation.
\end{itemize}

We collect the target symbol's log prices, $\{ p_{A_1}(t_i) \}$, and the log prices of the donor pool $\{ p_{j}(t_i) \}_{j \in \mathcal{D}}$. As shown in the code, we store these values in data frames and pivot them such that each row corresponds to a timestamp and each column to a specific symbol's log price.

\subsection{Synthetic Asset Construction}
The objective is to create a single synthetic asset $A_2^\text{(syn)}$ that tracks the target symbol $A_1$ closely. Formally, let $\{ x_i \}$ be the vector of stacked donor asset log prices at time $t_i$ (one component for each asset in the donor pool). We aim to find weights $\mathbf{w} = (w_1, \ldots, w_N)$ such that the synthetic price 
$
   p_{A_2}^\text{(syn)}(t_i) \;=\; \sum_{j=1}^{N} w_j\, x_{j}(t_i)
$
tracks $ p_{A_1}(t_i)$ in-sample while restricting the total number of assets in the combination. 

\paragraph{Cardinality-Constrained Optimization.}
To realize this, we solve a two-stage least squares problem. First, we solve
$
   \min_{\mathbf{w}} \sum_{i\,\in\,\text{train}} \bigl(p_{A_1}(t_i) \;-\; \mathbf{w}^\top \mathbf{x}_i\bigr)^2,
   \quad \text{subject to} \quad \sum_{j=1}^N w_j \;=\; 1,
$
and subsequently select the top $\max\_assets$ contributors (in absolute weight). We re-solve the minimization using only that restricted subset of donor assets. This step ensures interpretability by limiting the model to a small number of donor assets. The outcome of this step is a fitted weight vector $\widehat{\mathbf{w}}$, from which we then derive the in-sample synthetic log prices,
$
    p_{A_2}^\text{(syn)}(t_i) \;=\; \widehat{\mathbf{w}}^\top \mathbf{x}_i.
$

\subsection{Copula Modeling and Fitting}
Once we obtain the pair $\{ A_1, A_2^\text{(syn)} \}$, we transform the price series into returns for modeling dependence. Let
$
   r_{A_1}(t_i) \;=\; p_{A_1}(t_i) - p_{A_1}(t_{i-1}), 
   \quad
   r_{A_2}(t_i) \;=\; p_{A_2}^\text{(syn)}(t_i) - p_{A_2}^\text{(syn)}(t_{i-1}).
$
We then estimate the empirical marginal cumulative distribution functions (CDFs) $\widehat{F}_{A_1}$ and $\widehat{F}_{A_2}$ for the training period. By applying these empirical CDFs, we map the returns into uniforms, facilitating copula fitting.

A variety of copula families may be fitted (Gumbel, Clayton, Frank, Joe, N14, Gaussian, Student-$t$), each capturing different forms of dependence and tail behavior. Let 
$
   (u_i, v_i) \;=\; \bigl(\widehat{F}_{A_1}(r_{A_1}(t_i)), \,\widehat{F}_{A_2}(r_{A_2}(t_i))\bigr)
$
refer to the pairs of transformed returns. The likelihood of each copula family $\mathcal{C}_\theta$ for parameters $\theta$ is maximized:
$
   \hat{\theta} 
   \;=\; 
   \underset{\theta}{\arg\max}
   \sum_{i \in \text{train}} 
   \log \Bigl( c_\theta\bigl(u_i, v_i\bigr) \Bigr),
$
where $c_\theta$ is the copula density. Based on goodness-of-fit criteria (e.g., log-likelihood) and tail-dependence considerations, we select one or more copulas (e.g., Student-$t$ and N14) to generate trading signals.

\subsection{Trading Strategy and Signal Generation}
We employ a mispricing index (MPI) signal derived from the fitted copula. In essence, the MPI measures how much the current (transformed) return pair $(u, v)$ deviates from its expected relationship specified by the copula. Larger deviations correspond to stronger over- or under-valuation between $A_1$ and $A_2^\text{(syn)}$. 

Concretely, we:
\begin{enumerate}
    \item Compute $\{ r_{A_1}(t_i),\,r_{A_2}(t_i)\}$ in the test period.
    \item Transform returns via the training-set CDFs, yielding $\{ (u_i, v_i)\}$.
    \item Use the fitted copula to estimate the MPI.  
    \item Define a set of threshold-based trading rules:
    \begin{itemize}
        \item \textit{Open} a position if the MPI crosses a specified boundary (indicating mispricing).
        \item \textit{Close} or \textit{exit} the position once the MPI reverts to a certain level or hits a stop-loss.
    \end{itemize}
    \item The position sizes are maintained \textit{dollar-neutral} (or value-neutral), such that the capital invested long in one asset matches the capital invested short in the other.
\end{enumerate}

\paragraph{Equity Curve Computation.}
Finally, we track profit and loss (P\&L) based on the positions held. Let $\{ \pi_{A_1}(t), \pi_{A_2}(t)\}$ be the time series of positions (in shares) for each asset, determined from the MPI-based trading signals. The daily P\&L can be approximated by:
$
   \Delta \text{P\&L}(t) 
   \;=\; \pi_{A_1}(t) \, \bigl[r_{A_1}(t)\bigr] \;+\; \pi_{A_2}(t) \, \bigl[r_{A_2}(t)\bigr].
$
Cumulative sums (or exponentiated if returns are log-based) of these daily changes form the so-called \textit{equity curve}. We compare equity curves for different copulas to evaluate strategy performance, risk exposure, and stability out of sample.


%%%%%%%%%%%%%%%%%%%%%%%%%%%%%%%%%%%%%%%%%%%%%%%%%%%%%
%%%%%%%%%%%%%%%%%%%%%%%%%%%%%%%%%%%%%%%%%%%%%%%%%%%%%
%%%%%%%%%%%%%%%%%%%%%%%%%%%%%%%%%%%%%%%%%%%%%%%%%%%%%
%%%%%%%%%%%%%%%%%%%%%%%%%%%%%%%%%%%%%%%%%%%%%%%%%%%%%
%%%%%%%%%%%%%%%%%%%%%%%%%%%%%%%%%%%%%%%%%%%%%%%%%%%%%
%%%%%%%%%%%%%%%%%%%%%%%%%%%%%%%%%%%%%%%%%%%%%%%%%%%%%
%%%%%%%%%%%%%%%%%%%%%%%%%%%%%%%%%%%%%%%%%%%%%%%%%%%%%
%%%%%%%%%%%%%%%%%%%%%%%%%%%%%%%%%%%%%%%%%%%%%%%%%%%%%
%%%%%%%%%%%%%%%%%%%%%%%%%%%%%%%%%%%%%%%%%%%%%%%%%%%%%
%%%%%%%%%%%%%%%%%%%%%%%%%%%%%%%%%%%%%%%%%%%%%%%%%%%%%

%\section{Methodology}
%\subsection{Synthetic Asset Construction}
%We construct a synthetic asset $A_2$ designed to replicate the price movements of the target asset $A_1$ (AAPL) using a donor pool of assets. The synthetic asset is created through an Elastic Net-regularized regression model that solves:
%
%\begin{equation}
%\min_{\mathbf{w}} \| \mathbf{y} - \mathbf{Xw} \|^2 + \alpha \left( \lambda \|\mathbf{w}\|_1 + (1-\lambda)\|\mathbf{w}\|^2 \right)
%\end{equation}
%
%subject to:
%\begin{equation}
%\sum w_i = 1 \quad \text{and} \quad w_i \geq 0
%\end{equation}
%
%where $\mathbf{y}$ represents the target asset returns, $\mathbf{X}$ the donor pool returns, and $\mathbf{w}$ the portfolio weights. A cardinality-constrained variant limits the number of non-zero weights to enhance interpretability.
%
%\subsection{Copula-Based Mispricing Detection}
%\subsubsection{Returns Transformation}
%For assets $A_1$ and $A_2$, we calculate daily log returns:
%\begin{equation}
%r_t = \ln(p_t) - \ln(p_{t-1})
%\end{equation}
%
%Empirical CDFs $F_{A_1}$ and $F_{A_2}$ are constructed from training period returns for probability integral transformation:
%\begin{equation}
%u_t = F_{A_1}(r_t^{A_1}), \quad v_t = F_{A_2}(r_t^{A_2})
%\end{equation}
%
%\subsubsection{Copula Fitting}
%We model the dependency structure using several Archimedean and elliptical copulas:
%\begin{equation}
%\mathcal{C}(u,v) = \Phi_\theta(u,v)
%\end{equation}
%
%where $\Phi_\theta$ represents the copula function with parameters $\theta$, selected via maximum likelihood estimation. The best-fitting copula is chosen based on log-likelihood scores and tail dependency characteristics.
%
%\subsection{Mispricing Index and Trading Signals}
%\subsubsection{Mispricing Index (MPI)}
%The conditional probability MPI is calculated as:
%\begin{equation}
%MI_t^{A_1|A_2} = P(R_t^{A_1} \leq r_t^{A_1} | R_t^{A_2} = r_t^{A_2})
%\end{equation}
%\begin{equation}
%MI_t^{A_2|A_1} = P(R_t^{A_2} \leq r_t^{A_2} | R_t^{A_1} = r_t^{A_1})
%\end{equation}
%
%\subsubsection{Flag Accumulation}
%We maintain cumulative mispricing indicators:
%\begin{equation}
%Flag_{A_1}^*(t) = \sum_{s=1}^t (MI_s^{A_1|A_2} - 0.5)
%\end{equation}
%\begin{equation}
%Flag_{A_2}^*(t) = \sum_{s=1}^t (MI_s^{A_2|A_1} - 0.5)
%\end{equation}
%
%These raw flags are reset to zero upon position closure.
%
%\subsection{Trading Rules}
%\subsubsection{Entry/Exit Conditions}
%A dollar-neutral pairs trading strategy is implemented with:
%\begin{itemize}
%\item Entry when flags cross $\pm D$ thresholds ($D = 0.2-0.6$)
%\item Exit when flags revert to 0 or reach stop-loss levels $\pm S$ ($S = 2$)
%\item Position sizing maintains equal dollar amounts in long/short legs
%\end{itemize}
%
%The strategy employs four decision rules for signal combinations:
%\begin{equation}
%\text{Position}_t = \begin{cases}
%+1 & \text{if } Flag_{A_1} \leq -D \text{ OR } Flag_{A_2} \geq D \\
%-1 & \text{if } Flag_{A_1} \geq D \text{ OR } Flag_{A_2} \leq -D \\
%0 & \text{otherwise}
%\end{cases}
%\end{equation}
%
%\subsection{Performance Measurement}
%Strategy returns are calculated as:
%\begin{equation}
%R_t^{strategy} = \frac{1}{2}(r_t^{A_1} - r_t^{A_2}) \times \text{Position}_{t-1}
%\end{equation}
%
%Equity curves are computed through cumulative sum of daily returns, with positions shifted one period to avoid look-ahead bias.
%
%%\end{document}
%
%
%
%%\section*{Two-Stage Optimization Approach}
%%
%%In this formulation, we select a subset of assets using a two-stage procedure.
%%
%%\subsection*{Stage 1: Unrestricted Optimization}
%%
%%We first solve the following optimization problem, where $ y \in \mathbb{R}^T $ is the target time series, $ X \in \mathbb{R}^{T \times N} $ is the donor pool data, and $ w \in \mathbb{R}^N $ represents the weight vector:
%%\begin{equation*}
%%	\begin{array}{ll}
%%\text{minimize}_{w} & \| y - Xw \|_2^2 \\
%%\text{subject to} & \displaystyle \sum_{i=1}^{N} w_i = 1.
%%\end{array}
%%\end{equation*}
%%
%%Let $\hat{w}$ denote the solution of the above problem. We then select a subset of indices $ I \subset \{1,2,\dots,N\} $ with $|I| = k$ (where $ k $ is the maximum number of assets allowed), corresponding to the $ k $ largest values in absolute terms:
%%$
%%I = \left\{ i \in \{1,\dots,N\} \;:\; i \text{ is among the indices corresponding to the } k \text{ largest values of } \{ |\hat{w}_1|, \dots, |\hat{w}_N| \} \right\}.
%%$
%%
%%\subsection*{Stage 2: Restricted Optimization}
%%
%%Let $ X_I \in \mathbb{R}^{T \times k} $ be the submatrix of $ X $ that contains only the columns indexed by $ I $, and let $ w_I \in \mathbb{R}^k $ be the corresponding weight vector. We then solve:
%%\begin{equation*}
%%	\begin{array}{ll}
%%\text{minimize}_{w_I} & \| y - X_I w_I \|_2^2 \\
%%\text{subject to} & \displaystyle \sum_{i \in I} (w_I)_i = 1.
%%\end{array}
%%\end{equation*}The final full weight vector $ w \in \mathbb{R}^N $ is defined by:
%%%$
%%%w_i =
%%%\begin{cases}
%%%(w_I)_i, & \text{if } i \in I,\[1mm]
%%%0, & \text{otherwise.}
%%%\end{cases}
%%%$
%%
%%%\end{document}