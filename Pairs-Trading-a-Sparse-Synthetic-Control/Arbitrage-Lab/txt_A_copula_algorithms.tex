%%%%%%%%%%%%%%%%%%%%%%%%%%%%%%%%%%%%%%%%%%%%%%%%%%%%%
%%%%%%%%%%%%%%%%%%%%%%%%%%%%%%%%%%%%%%%%%%%%%%%%%%%%%
%%%%%%%%%%%%%%%%% COPULA DENSITIES %%%%%%%%%%%%%%%%%%
%%%%%%%%%%%%%%%%%%%%%%%%%%%%%%%%%%%%%%%%%%%%%%%%%%%%%
%%%%%%%%%%%%%%%%%%%%%%%%%%%%%%%%%%%%%%%%%%%%%%%%%%%%%

\subsection{Copula Densities}\label{subsec:copula_densities}
\begin{itemize}
\item \textbf{Elliptical Copulas:}
   \begin{itemize}
   \item \textbf{Gaussian}: $\Theta = \{\rho \in (-1,1)\}$ with density
%$$
%c_\rho^{Gauss}(u,v) = \frac{1}{\sqrt{1-\rho^2}} \exp\left(-\frac{\zeta_u^2 + \zeta_v^2 - 2\rho\zeta_u\zeta_v}{2(1-\rho^2)} + \frac{\zeta_u^2 + \zeta_v^2}{2}\right)
%$$
$$\displaystyle
c(u, v) = \frac{1}{\sqrt{1-\rho^2}} \exp\left(-\frac{\rho^2(x^2 + y^2) - 2\rho xy}{2(1-\rho^2)}\right)
$$
   where $x = \Phi^{-1}(u)$, $y = \Phi^{-1}(v)$ and $\Phi$ is the standard normal CDF.
   
   \item \textbf{Student-$t$}: $\Theta = \{\rho \in (-1,1), \nu > 2\}$ with density
%   \[
%   c_{\rho,\nu}^{t}(u,v) = \frac{\Gamma\left(\frac{\nu+2}{2}\right)\Gamma\left(\frac{\nu}{2}\right)}{\sqrt{1-\rho^2}\Gamma\left(\frac{\nu+1}{2}\right)^2} 
%   \frac{\left(1 + \frac{\zeta_u^2 + \zeta_v^2 - 2\rho\zeta_u\zeta_v}{\nu(1-\rho^2)}\right)^{-(\nu+2)/2}}{\prod_{i\in\{u,v\}} \left(1 + \frac{\zeta_i^2}{\nu}\right)^{-(\nu+1)/2}}
%   \]
$$\displaystyle
c(u,v) = \frac{1}{\sqrt{1-\rho^2}} \cdot \frac{\Gamma(\frac{\nu+2}{2})\Gamma(\frac{\nu}{2})}{\Gamma(\frac{\nu+1}{2})^2} \cdot \frac{\left(1 + \frac{x^2 + y^2 - 2\rho xy}{\nu(1-\rho^2)}\right)^{-\frac{\nu+2}{2}}}{\left(1 + \frac{x^2}{\nu}\right)^{-\frac{\nu+1}{2}} \cdot \left(1 + \frac{y^2}{\nu}\right)^{-\frac{\nu+1}{2}}}
$$
where $x = t_\nu^{-1}(u)$, $y = t_\nu^{-1}(v)$ and $t_\nu$ is the Student-$t$ CDF.
   \end{itemize}	

\item \textbf{Archimedean Copulas:} 
%For generator function $\psi_\theta$, 
%$$
%C_\theta(u,v) = \psi^{-1}_\theta(\psi_\theta(u) + \psi_\theta(v))
%$$
    \begin{itemize}
    \item \textbf{Clayton}: $\Theta = [-1, \infty)\setminus \{0\}$ with density
%$\psi_\theta(t) = \frac{1}{\theta}(t^{-\theta}-1)$
$$\displaystyle
c(u,v) = (1+\theta)(uv)^{-1-\theta}(u^{-\theta} + v^{-\theta} - 1)^{-2-\frac{1}{\theta}}
$$
    \item \textbf{Gumbel}: $\Theta = [1, \infty)$ with density
$$
c(u, v) = 
\frac{1}{u \cdot v} 
\cdot 
\exp(-\Upsilon) 
\cdot 
\frac{\varphi(u)}{-\log u} 
\cdot 
\frac{\varphi(v)}{-\log v} 
\cdot 
(\theta + \Upsilon - 1)
\cdot 
\Upsilon^{1-2\theta}% EQUIVALENT TO % (\varphi(u) + \varphi(v))^{\frac{1}{\theta} - 2}
$$
where
$\varphi(x):=(-\log x)^{\theta}$,
and
$\Upsilon := [\varphi(u) + \varphi(v)]^{\frac{1}{\theta}}$
%$\psi_\theta(t) = \exp(-t^{1/\theta})$
    \item \textbf{Frank}: $\Theta = \mathbb{R}\setminus\{0\}$ with density
$$\displaystyle
c(u, v) = 
\frac{e^{\theta} \cdot e^{u\theta} \cdot e^{v\theta} \cdot (e^{\theta} - 1) \cdot \theta}{(e^{\theta} + e^{u\theta} \cdot e^{v\theta} - e^{\theta} \cdot e^{u\theta} - e^{\theta} \cdot e^{v\theta})^2}
$$
%$\psi_\theta(t) = -\frac{1}{\theta}\ln\left(1 - (1 - e^{-\theta})e^{-t}\right)$
    \item \textbf{Joe}: $\Theta = [1, \infty)$ with density
$$\displaystyle
c(u, v) = 
\frac{\zeta(u)}{1-u} 
\cdot 
\frac{\zeta(v)}{1-v} 
\cdot 
[\zeta(u) + \zeta(v) - \zeta(u)\zeta(v)]^{\frac{1}{\theta} - 2} 
\cdot 
\{\theta - [\zeta(u) - 1] \cdot [\zeta(v) - 1]\}
\\
$$
where
$\zeta(x):=(1-x)^{\theta}$.
%$\tilde{u} = (1-u)^{\theta}$ and $\tilde{v} = (1-v)^{\theta}$.
%$\psi_\theta(t) = 1 - (1 - e^{-t})^{1/\theta}$
    \end{itemize}

    \begin{itemize}
    \item \textbf{N14} \footnote{For more details about this copula family, see \cite{Nelsen2006}; p. 116.}: $\Theta = [1,\infty)$ with density
%See 
    
$$\displaystyle
c(u, v) = 
\frac{
	\varrho(u) 
	\cdot 
	\varrho(v) 
	\cdot 
	(\xi - 1) 
	\cdot 
	[-1 + \theta + 2 \theta \cdot (\xi - 1)]
}
{
	[\varrho(u) + \varrho(v)]^2 
	\cdot
	\xi^{2 + \theta} 
	\cdot 
	u \cdot v 
	\cdot 
	\eta(u) \cdot \eta(v) \cdot \theta}
$$
where 
$\eta(x) := -1 + x^{1/\theta}$,
$\varrho(x) := (-1 + x^{-1/\theta})^{\theta}$ and 
$\xi := 1 + [\varrho(u) + \varrho(v)]^{1/\theta}$.
    \end{itemize}
    
%----------------------------------------------------
%\item \textbf{Mixed Copulas:}
%\begin{itemize}
%  \item CTG
%  \item CFG
%\end{itemize}
%----------------------------------------------------
    
\end{itemize}



%%%%%%%%%%%%%%%%%%%%%%%%%%%%%%%%%%%%%%%%%%%%%%%%%%%%%
%%%%%%%%%%%%%%%%%%%%%%%%%%%%%%%%%%%%%%%%%%%%%%%%%%%%%
%%%%%%%%%%%%%% CALIBRATION ALGORITHMS %%%%%%%%%%%%%%%
%%%%%%%%%%%%%%%%%%%%%%%%%%%%%%%%%%%%%%%%%%%%%%%%%%%%%
%%%%%%%%%%%%%%%%%%%%%%%%%%%%%%%%%%%%%%%%%%%%%%%%%%%%%

\subsection{Copula calibration algorithms} \label{subsec:copula_calibration}

\begin{algorithm}[H]
\label{alg:calibrating_archimedean}
\caption{Calibrating Archimedean Copulas}
\begin{algorithmic}[1]
\Require Pseudo-observations from training data: $\mbf u,\mbf v\in [0,1]^{T^{tr}}$
\State Compute $N_C \gets \sum_{i=1}^{T^{tr}}\sum_{j=i+1}^{T^{tr}} \I{(u_i-u_j)(v_i-v_j) > 0}$
\State Compute $N_D \gets \sum_{i=1}^{T^{tr}}\sum_{j=i+1}^{T^{tr}} \I{(u_i-u_j)(v_i-v_j) < 0}$
\State Compute Kendall's $\displaystyle \hat \tau \gets \frac{2(N_C-N_D)}{n(n-1)}$
\State Retrieve $\hat \theta \gets f^{-1}(\hat \tau)$
\Ensure $\hat \theta $
\end{algorithmic}
\end{algorithm}

Where we leverage the fact that, for any Archimedean copula $C$ with generator $\varphi$, Kendall's tau verifies (\cite{alexander2008market}; p.280):
$$
\tau 
= f(\theta):=
%\1{
1+4 \int_{0}^{1} \frac{\varphi(t;\theta)}{\varphi'(t;\theta)}dt
%}
$$

And for the special case of the Frank copula, we have:
$$
\tau = g(\theta) := 1+\frac{4}{\theta}[D_1(\theta)-1]
$$
where $D_1(\theta)$ is the Debye function defined as:
$
%\displaystyle
D_1(\theta)=\frac{1}{\theta}\int_{0}^{\infty} \frac{t}{e^t-1}dt
$.
Therefore, we can retrieve $\theta$ from the inversion of these relationships:

\vspace{0.2cm}
%----------------------------------------------------
{\centering
\renewcommand{\arraystretch}{1.2}  
\begin{tabular}{ccll}
%\toprule
\textit{Clayton} && 
$\displaystyle 
\theta = {2\tau}\1{1-\tau}^{-1}
$ 
& Analytical inverse
\\
\textit{Frank} && 
$\displaystyle 
\theta = g^{-1}(\tau)
$ 
& Numerical inverse
\\
\textit{Gumbel} && 
$\displaystyle 
\theta = \1{1-\tau}^{-1}
$ 
& Analytical inverse
\\
\textit{Joe} && 
$\displaystyle 
\theta = f^{-1}(\tau) 
$
& Numerical inverse
\\
\textit{N14} && 
$\displaystyle
\theta = \fracc{1+\tau}{2-2\tau}
$
& Analytical inverse
\\
%\bottomrule
\end{tabular}
\par}
%----------------------------------------------------



%==============[	  Calibrating Gaussian copula  ]==============
\begin{algorithm}[H]
\label{alg:calibrating_gaussian}
\caption{Calibrating Gaussian Copula}
\begin{algorithmic}[1]
\Require Pseudo-observations from training data: $\mbf u,\mbf v\in [0,1]^{T^{tr}}$
\State Transform $\mbf u$ and $\mbf v$ into standard normal variates $\mbf x=\Phi^{-1}(\mbf u)$ and $\mbf y=\Phi^{-1}(\mbf v)$%, where $\Phi^{-1}$ is applied element-wise.
\State Obtain the empirical covariance matrix 
$$
\widehat{\mbf \Sigma} 
:=\begin{bmatrix} 
			\hat \sigma_{x}^2 & \hat \sigma_{yx} \\
			\hat \sigma_{xy} & \hat \sigma_{y}^2
		\end{bmatrix}
= \frac{1}{T_{tr}-1}\begin{bmatrix} \mbf x^\top\mbf x & \mbf x^\top\mbf y \\ \mbf y^\top\mbf x & \mbf y^\top\mbf y \end{bmatrix}$$
\State Set $\displaystyle \hat \rho \gets \frac{\hat \sigma_{xy}}{\hat \sigma_x \hat \sigma_y}$
\Ensure $\hat \rho$
\end{algorithmic}
\end{algorithm}


%==============[	  Calibrating Student-t copula  ]==============
\begin{algorithm}[H]
\label{alg:calibrating_student-t}
\caption{Calibrating the Student-$t$ Copula}
\begin{algorithmic}[1]
\Require Pseudo-observations from training data: $\mbf u,\mbf v\in [0,1]^{T^{tr}}$
\For{each $\nu\in\mathcal V:=[1,15]$}
	\State Transform $\mbf u$ and $\mbf v$ into Student-$t$ variates: % using the inverse Student-$t$ CDF: 
$\mbf x_{\nu}=t_{\nu}^{-1}(\mbf u)$;~~$\mbf y_{\nu}=t_{\nu}^{-1}(\mbf v)$
%, where $t_{\nu}^{-1}$ is the inverse CDF of a Student-$t$ distribution with $\nu$ degrees of freedom.
	\State Obtain the empirical covariance matrix 
		$$
		\widehat{\mbf \Sigma}(\nu)
		:=
		\begin{bmatrix} 
			\hat \sigma_{x}^2(\nu) & \hat \sigma_{yx}(\nu) \\
			\hat \sigma_{xy}(\nu) & \hat \sigma_{y}^2(\nu)
		\end{bmatrix}
		= 
		\frac{1}{T_{tr}-1}
		\begin{bmatrix} 
			\mbf x_{\nu}^\top\mbf x_{\nu} & \mbf x_{\nu}^\top\mbf y_{\nu} \\ 
			\mbf y_{\nu}^\top\mbf x_{\nu} & \mbf y_{\nu}^\top\mbf y_{\nu} 
		\end{bmatrix}
		$$
	\State Evalutate the log-likelihood of the $t$-copula
$$
\ell(\nu; \mbf u, \mbf v):= 
\sum_{t\in\mathcal T_{tr}}  \log c(u_t, v_t; \nu, \hat \rho(\nu))
\quad 
\t{where}
\quad 
\hat \rho(\nu) = \frac{\hat \sigma_{xy}(\nu)}{\hat \sigma_x(\nu) \hat \sigma_y(\nu)}
$$
\EndFor
\State Set 
	$\displaystyle \nu^{\star} \gets \arg\max_{\nu \in \mathcal V} \ell(\nu; \mbf u, \mbf v)$, and
	%$\widehat{\mbf \Sigma}^{\star}:=\widehat{\mbf \Sigma}(\nu^{\star})$, 
	$\displaystyle \hat \rho^\star \gets  \hat \rho (\nu^\star)$
	\Ensure $(\nu^\star, \hat \rho^\star)$
\end{algorithmic}
\end{algorithm}



%%%%%%%%%%%%%%%%%%%%%%%%%%%%%%%%%%%%%%%%%%%%%%%%%%%%%
%%%%%%%%%%%%%%%%%%%%%%%%%%%%%%%%%%%%%%%%%%%%%%%%%%%%%
%%%%%%%%%%%%%%%% SAMPLING ALGORITHMS %%%%%%%%%%%%%%%%
%%%%%%%%%%%%%%%%%%%%%%%%%%%%%%%%%%%%%%%%%%%%%%%%%%%%%
%%%%%%%%%%%%%%%%%%%%%%%%%%%%%%%%%%%%%%%%%%%%%%%%%%%%%

\subsection{Copula sampling algorithms}\label{subsec:copula_sampling}

%\subsubsection{Conditional sampling}

%==============[	  ALGORITHM  ]==============
\begin{algorithm}[H]
\caption{Conditional sampling}
\label{alg:conditional_sampling}
\begin{algorithmic}[1]
\Require Conditional copula distribution $C_{V|U}(v|u)$ with calibrated $\hat \theta$
\State Generate two independent variates $u,q\sim \mathcal U(0,1)$
\State Keep the first variate $u$ as is
\State Transform the second variate $q$ into $v$ using the inverse of the conditional distribution:
$$v = C^{-1}_{V|U}(q|u)$$
\Ensure The resulting pair $(u,v)$ will follow the desired copula distribution
\end{algorithmic}
\end{algorithm}
%----------------------------------------------------

We applied \cref{alg:conditional_sampling} to sample from the Clayton and Frank copulas.

%\textit{Clayton}. 
%Its conditional distribution is given by:
%\begin{equation}\label{eq:clayton_cond_prob}
%	C_{V|U}(v|u)= \frac{\partial}{\partial u} (u^{-\theta}+v^{-\theta} - 1)^{-1/\theta} = u^{-(1+\theta)}(u^{-\theta}+v^{-\theta} - 1)^{-(1+\theta)/\theta},
%\end{equation}
%and setting \eqref{eq:clayton_cond_prob} equal to the fixed probability $q$ delivers the Clayton copula $q$-quantile curve. Solving for $v$ delivers (\cite{alexander2008market}; p.275):
%$$v = C_{V|U}^{-1} (q|u)= \left[1+u^{-\theta}(q^{-\theta/(1+\theta)} - 1)\right]^{-1/\theta}.$$

\vspace{0.2cm}
%----------------------------------------------------
{\centering
\renewcommand{\arraystretch}{1.8}  
\begin{tabular}{ccp{10cm}}
%\toprule
\textit{Clayton} && 
$\displaystyle 
C_{V|U}^{-1} (q|u)= \left[1+u^{-\theta}(q^{-\theta/(1+\theta)} - 1)\right]^{-1/\theta}
$ 
\\
\textit{Frank} && 
$\displaystyle 
C_{V|U}^{-1} (q|u)= -\frac{1}{\theta} \ln \left( 1 + \frac{q \cdot (1 - e^{-\theta})}{q  (e^{-\theta u} - 1) - e^{-\theta u}} \right)
$
\\
%\bottomrule
\end{tabular}
\par}
%----------------------------------------------------




%\subsubsection{Theorem 4.3.7 in \cite{Nelsen2006}}

%==============[	  ALGORITHM  ]==============
\begin{algorithm}[H]
\caption{Sampling from Archimedean Copulas using Theorem 4.3.7 in \cite{Nelsen2006}}
\label{alg:sampling_theorem_nelsen}
\begin{algorithmic}[1]
\Require Generator function $\varphi$ of the desired Archimedean $C$-copula and its calibrated $\hat \theta$
\State Generate two independent variates $s,t\sim\mathcal U(0,1)$
\State Set $w \gets K_C^{(-1)}(t;\hat \theta)$, where 
$\displaystyle K_C(w;\hat \theta) = t - \frac{\varphi(w;\hat \theta)}{\varphi'(w^+;\hat \theta)}$
\State Set $u \gets \varphi^{[-1]}(s\varphi(w;\hat \theta))$ and $v \gets \varphi^{[-1]}((1-s)\varphi(w;\hat \theta))$
\Ensure The pair $(u,v)$ will follow the desired copula distribution
\end{algorithmic}
\end{algorithm}
%----------------------------------------------------
From Theorem 4.3.7 in \cite{Nelsen2006} (p.129), it can be shown that \cref{alg:sampling_theorem_nelsen} generates random variates $(u,v)$ whose joint distribution function is an Archimedean copula $C$ with generator $\varphi$. %in $\Omega$:
Note that $K_C$ is given by Theorem 4.3.4. (p.127) and it denotes the $C$-measure of the set $\{(u,v)\in [0,1]^2 \mid C(u,v)\leq w\}$
$$
K_C(w; \theta) := t - \frac{\varphi(w; \theta)}{\varphi'(w^+; \theta)},
$$ 
and ${\varphi'(w^+)}$ denotes the right-sided derivative of the generator.

In our case, we use this algorithm to sample from Gumbel, Joe and N14. Their $C$-measures are given  below. Note that there is no analytical solution for their inverse, so we have to resort to numerical inversion.

\vspace{0.2cm}
%----------------------------------------------------
{\centering
\renewcommand{\arraystretch}{1.8}  
\begin{tabular}{ccp{10cm}}
%\toprule
\textit{Gumbel} && $\displaystyle K_C(w; \theta)=w \cdot \5{1- \frac{\log (w)}{\theta}}$ 
\\
\textit{Joe} && $\displaystyle K_C(w; \theta) = w - \frac{1}{\theta} 
\cdot 
\frac{\log[1-(1-w)^{\theta}] \cdot [1-(1-w)^{\theta}]}
{(1-w)^{\theta-1}}$ 
\\
\textit{N14} && $\displaystyle K_C(w; \theta) = -w \cdot (-2 + w^{1/\theta})$ 
\\
%\bottomrule
\end{tabular}
\par}
%----------------------------------------------------


%==============[	  Sampling from Gaussian  ]==============
\begin{algorithm}[H]
\label{alg:sampling_gaussian}
\caption{Sampling from Gaussian Copula}
\begin{algorithmic}[1]
\Require Calibrated $\hat{\mbf \Sigma}$ %from \cref{alg:calibrating_gaussian}
\State Generate correlated Gaussian pairs $(x,y)\sim\mathcal N(\mbf 0, \widehat{\mbf \Sigma})$
\State Transform the Gaussian pairs into uniform variates %using the standard normal CDF:
$u = \Phi(x)$ and $v = \Phi(y)$.
\Ensure The pair $(u,v)$  follows the Gaussian copula distribution.
\end{algorithmic}
\end{algorithm}

%==============[	  Sampling from Student-t copula  ]==============
\begin{algorithm}[H]
\label{alg:sampling_student-t}
\caption{Sampling from Student-$t$ Copula}
\begin{algorithmic}[1]
\Require Calibrated $\nu^\star$ and $\hat{\mbf \Sigma}(\nu^\star)$ %from \cref{alg:calibrating_student-t}
\State Sample from a bivariate normal: $\displaystyle (x_1, x_2) \sim \mathcal{N}(\mbf 0, \widehat{\mbf \Sigma}(\nu^{\star}))$ 
%Generate correlated bivariate normal random variables $(x_1, x_2) \sim \mathcal{N}(\mbf 0, \widehat{\mbf \Sigma}^{\star})$
\State Sample from a chi-square distribution with $\nu^\star$ degrees of freedom: $\displaystyle \chi \sim \chi^2_{\nu^\star}$
%Generate a random variable $\chi$ from a chi-square distribution with $\nu^\star$ degrees of freedom: $\chi \sim \chi^2_{\nu^\star}$
\State Compute the Student-$t$ variates: $y_1 = \fraccc{x_1}{\sqrt{\chi/\nu^{\star}}}$ and $y_2 = \fraccc{x_2}{\sqrt{\chi/\nu^{\star}}}$
\State Transform the Student-$t$ variates into uniform variates %using the Student-$t$ CDF:
$u = t_{\nu}(y_1)$ and $v = t_{\nu}(y_2)$ 
%with parameter $\nu^\star$ and correlation $\rho$
\Ensure The pair $(u,v)$ follows the Student-$t$ copula distribution
\end{algorithmic}
\end{algorithm}
