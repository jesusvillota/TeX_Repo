Let $\tau_{\t{Stock}}$ and $\tau_{\t{ETF}}$ denote the transaction costs on stocks and ETFs respectively. Then, the portfolio returns adjusted for transaction costs are: 



\begin{align*}
r_{d}^{\mathcal{P}} 
:= 
\frac{1}{|\mathcal{P}_{d}|}
\sum_{(i, j), \in \mathcal{P}_{d}}
\red{
T R_{{{L}}, \theta}\angl{(i,j), {d-1}} 
}
\cdot 
AR_{{d}}^{(i,j)}
~
\red{-
(\tau_{\t{Stock}}+\beta^{(i,j)}\tau_{\t{ETF}}) \cd 
\abs{
T R_{{{L}}, \theta}\angl{(i,j), {d}} 
-
T R_{{{L}}, \theta}\angl{(i,j), {d-1}} 
}}
\end{align*}


Formula inspired by my conversation with ChatGPT: 
\href{https://chatgpt.com/share/be9a4ea1-9f14-484a-a4ed-cd45b8532678}{Link}

\bx
Corrections: 
\red{
\begin{enumerate}
  \item \textbf{You need to shif the TR one period back}: Today you obtain the returns that arise by taking a position yesterday and holding it until today. The position yesterday is taken according to TR(d-1)
  \item When computing the beta-neutral positions, you are using \textbf{excess returns, doesn't this imply that in your positions, you need to hold the risk-free asset}? Suggestion: Compute the model without considering the risk-free asset.
  \item In order to avoid \textbf{overlap with the notation of the event study} (where it is more common to refer to ``abnormal returns''), we may want to rename the AR of the beta neutral strategy as ``beta-neutral returns''. For example. $\mathfrak{n}^{(i,j)}_d$
\end{enumerate}
}




