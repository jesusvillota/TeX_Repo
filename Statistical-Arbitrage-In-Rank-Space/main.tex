\documentclass[12pt,article]{memoir}
\usepackage{/Users/jesusvillotamiranda/Documents/LaTeX/$$JVM_Macros}
\Subject{Statistical Arbitrage in Rank Space}
\Arg{}

\begin{document}

\tableofcontents

%\textbf{ChatGPT Prompt:}
%
%\bblue{
%
%In a market consisting of $N$ stocks, we denote the dividend-adjusted return on stock $i$ at trading day $t$ by $r_{i, t}$. We adopt a factor model for stock return,
%\begin{align}\label{eq:1}
%r_t- r_f = \beta_t F_t+ \epsilon_t, \quad t=1,2, \ldots, T 
%\end{align}
%
%
%}
%
%
%
%\Vhrulefill
\newpage
%%%%%%%%%%%%%%%%%%%%%%%%%%%%%%%%%%%%%%%%%%%%%%%%%%%%%
%%%%%%%%%%%%%%%%%%%%%%%%%%%%%%%%%%%%%%%%%%%%%%%%%%%%%
%%%%%%%%%%%%%%%%%%%%%%%%%%%%%%%%%%%%%%%%%%%%%%%%%%%%%
%%%%%%%%%%%%%%%%%%%%%%%%%%%%%%%%%%%%%%%%%%%%%%%%%%%%%
In a market consisting of $N$ stocks, we denote the dividend-adjusted return on stock $i$ at trading day $t$ by $r_{i, t}$. We adopt a factor model for stock return,
\begin{align}\label{eq:1}
r_t- r_f = \beta_t F_t+ \epsilon_t, \quad t=1,2, \ldots, T 
\end{align}
Here, $r_t=\left\{r_{i, t}\right\}_{i=1}^N \in \mathbb{R}^N$ are the dividend-adjusted daily return, $r_f \in \mathbb{R}$ is the risk-free rate, $F_t \in \mathbb{R}^{K \times 1}$ are the underlying factors, $\beta_t \in \mathbb{R}^{N \times K}$ are the corresponding loadings on $K$ factors, and $\epsilon_t \in \mathbb{R}^N$ are the residual returns. Factor candidates varies widely, ranging from economical-driven factors such as the Fama-French factors, to statistically-driven factors derived from PCA. In our approach, factors are selected as the leading eigenvectors in PCA. The number of factors $K$ is chosen based on the eigenvalue spectrum of the empirical correlation of daily returns.


Without loss of generality, these factors can be interpreted as portfolios of stocks,
\begin{equation}\label{eq:2}
	F_t=\omega_t\left(r_t-r_f\right)
\end{equation}
where $\omega_t \in \mathbb{R}^{K \times N}$ contains corresponding portfolio weights. Combining \cref{eq:1} and \cref{eq:2} yields
\begin{equation}\label{eq:3}
	r_t-r_f=\beta_t \omega_t\left(r_t-r_f\right)+\epsilon_t \Rightarrow \epsilon_t=\left(I-\beta_t \omega_t\right)\left(r_t-r_f\right):=\Phi_t\left(r_t-r_f\right)
\end{equation}

Here,
\begin{equation}\label{eq:4}
	\Phi_t:=\left(I-\beta_t \omega_t\right)
\end{equation}
defines a linear transformation from $r_t$ to $\epsilon_t$. More importantly, $\epsilon_{i, t}$ can be viewed as the return of a tradable portfolio with weights specified by the $i$-th row of $\Phi_t$. Consequently, the investing universe spanned by $r_t$ is termed as name equity space, and that spanned by $\epsilon_t$ as name residual space.

%%%%%%%%%%%%%%%%%%%%%%%%%%%%%%%%%%%%%%%%%%%%%%%%%%%%%
%%%%%%%%%%%%%%%%%%%%%%%%%%%%%%%%%%%%%%%%%%%%%%%%%%%%%
%%%%%%%%%%%%%%%%%%%%%%%%%%%%%%%%%%%%%%%%%%%%%%%%%%%%%
%%%%%%%%%%%%%%%%%%%%%%%%%%%%%%%%%%%%%%%%%%%%%%%%%%%%%

We denote the portfolio weights in name equity space as $w_t^{R, \text { name }}$ and portfolio weights in name residual space as $w_t^{\epsilon, \text { name }}$. These weights are related by

\begin{equation}\label{eq:5}
	w_t^{R, \text { name }}=\Phi_t^T w_t^{\epsilon, \text { name }}
\end{equation}
, directly following the equality in portfolio return,

\begin{equation}\label{eq:6}
	\left(w_t^{\epsilon \text { name }}\right)^T \epsilon_t=\left(w_t^{\epsilon, \text { name }}\right)^T \Phi_t\left(r_t-r_f\right)=\left(w_t^{R, \text { name }}\right)^T\left(r_t-r_f\right)
\end{equation}

For factors derived by PCA, we have

\begin{equation}\label{eq:7}
	\Phi_t \beta_t=0 \Longrightarrow\left(w_t^{R, \text { name }}\right)^T \beta_t=\left(w_t^{\epsilon, \text { name }}\right)^T \Phi_t \beta_t=0, \quad \forall w_t^{\epsilon, \text { name }}
\end{equation}
with proof given in the appendix. It means that for any $w_t^{\epsilon \text {,name }}$, the $w_t^{R \text {,name }}$ calculated by \cref{eq:5} satisfy,

\begin{equation}\label{eq:8}
	\left(w_t^{R, \text { name }}\right)^T\left(r_t-r_f\right)=\left(w_t^{\epsilon, \text { name }}\right)^T \Phi_t\left(\beta_t F_t+\epsilon_t\right)=\left(w_t^{\epsilon, \text { name }}\right)^T \Phi_t \epsilon_t=\left(w_t^{R, \text { name }}\right)^T \epsilon_t
\end{equation}

It suggests that the return of our statistical arbitrage portfolios is independent of market factors and relies solely on residual returns, a property usually termed as market neutrality. Ideally, portfolios are also desired to have a zero net value, known as dollar neutrality. Empirical evidence suggests that market-neutral portfolios are also approximately dollar-neutral.


%%%%%%%%%%%%%%%%%%%%%%%%%%%%%%%%%%%%%%%%%%%%%%%%%%%%%
\begin{algorithm}[H]
    \caption{Market decomposition (PCA) [Fig. 5, panel (c1, c2)]}
    \begin{algorithmic}[1]
        \Require $r_t$, $r_{f, t}$, $K$
        \Ensure $\epsilon_t$, $\Phi_t$
        \Function{market\_decomposition}{$r_t, r_{f, t}, K$}
            \State Perform principal component analysis: $r_t - r_{f, t} = U \Sigma V^T$
            \State $F_t \gets (v_1, v_2, \ldots, v_K)$, where $v_k$ is the $k$-th column of $V^T$
            \State Calculate $\omega_t$ by solving $F_t = \omega_t (r_t - r_{f,t})$
            \State Calculate $\beta_t$ as the coefficient of the linear regression $r_t - r_f \sim F_t$
            \State $\Phi_t \gets I - \beta_t \omega_t$
            \State $\epsilon_t \gets \Phi_t (r_t - r_{f, t})$
            \State \Return $\epsilon_t, \Phi_t$
        \EndFunction
    \end{algorithmic}
\end{algorithm}

% Input details:
\textbf{Input:}
\begin{itemize}
    \item $r_t$: return in name space or transformed return in rank space.
    \item $r_{f, t}$: risk-free rate at the end of trading day $t$.
    \item $K$: number of market factors, predetermined by analyzing eigenvalue spectrum of the correlation matrix.
\end{itemize}

% Output details:
\textbf{Output:}
\begin{itemize}
    \item $\epsilon_t$: residual returns in name space or rank space.
    \item $\Phi_t$: transformation between residual space and equity space (Eq. 2.1.1 for name space and Eq. 2.1.10 for rank space).
\end{itemize}

% Notes:
\textbf{Note:}
\begin{itemize}
    \item The algorithm realizes the formulation in section 2.1.
    \item Factors $F_t$ and $\omega_t$ are calculated on a 252-day look-back window.
    \item Loadings $\beta_t$ are calculated on a 60-day look-back window.
    \item $F_t$, $\omega_t$, and $\beta_t$ are updated daily.
    \item $K=5$ for name space and $K=1$ for rank space based on empirical eigenvalue spectrum of the correlation matrix (Fig. 6(c,d)).
\end{itemize}


\textbf{[Appendix]}:
Here, we prove the equality $\Phi_t \beta_t=0$, crucial relationship for market neutrality. 

We denote the return matrix $R_t=\left(r_{t-T+1}, r_{t-T+2}, \ldots, r_t\right) \in \mathbb{R}^{N \times T}$, \bblue{(where $T$ is a window of 252 days)}. Assume singular value decomposition of $R_t$,
$$
R_t-R_t^f=U \Sigma V^T
$$
where $R_t^f \in \mathbb{R}^{1 \times T}$ is the risk-free rate, $U \in \mathbb{R}^{N \times N}, \Sigma \in \mathbb{R}^{N \times T}$, and $V^T \in \mathbb{R}^{T \times T}$. Then, the factors and loadings in Eq. 2.1.1 and $\omega_t$ in Eq. 2.1.2 becomes
$$F_t=\left(\begin{array}{c}v_1^T \\ v_2^T \\ \ldots \\ v_K^T\end{array}\right), \quad \beta_t=\left(u_1, u_2, \ldots, u_K\right)\left(\begin{array}{ccc}\sigma_1 & & \\ & \ldots & \\ & & \sigma_K\end{array}\right), \quad \omega_t=\left(\begin{array}{ccc}\sigma_1^{-1} & & \\ & \ldots & \\ & \ldots & \sigma_K^{-1}\end{array}\right)\left(\begin{array}{c}u_1^T \\ u_2^T \\ \ldots \\ \\ u_K^T\end{array}\right)$$
where $u_i$ and $v_i$ are the $i$-th column of matrix $U$ and $V$. 
Then, because $U$ and $V$ are orthogonal matrix,
$$
\begin{aligned}
\beta_t \omega_t & =I \\
\Longrightarrow \Phi_t \beta_t & =\left(I-\beta_t \omega_t\right) \beta_t=0
\end{aligned}
$$

\Vhrulefill

$$\begin{array}{lllll}
\ub{F_t}{K \times T} 
&= \ub{V_K}{K \times T} 
%&= \ub{w_t}{K \times N} \ub{(R_t-R_t^f)}{N\times T}
\\[2em]
\ub{w_t}{K \times N} 
&= \ub{\Sigma_K^+}{K\times K} \ub{U_K^T}{K\times N}
&= \ub{F_t}{K\times T}\ub{R_t^+}{T\times N}
\\[2em]
\ub{\beta_t}{N\times K} &= \ub{U_K}{N\times K} \ub{\Sigma_K}{K\times K}
&= (\ub{F_t^T F_t}{K\times K})^+ \ub{F_t}{K\times T} \ub{r_t}{T\times 1}
\end{array}$$


%%%%%%%%%%%%%%%%%%%%%%%%%%%%%%%%%%%%%%%%%%%%%%%%%%%%%
\newpage
\red{\textbf{Potential Typo}: 
If $\beta_t \omega_t = I$, then $\Phi_t := (I- \beta_t \omega_t) = I-I = 0$, which doesn't make sense.}

\violet{

Note that:

$$\begin{array}{lllll}
\beta_t 	&= U_K \Sigma_K
\\
\omega_t 	&= \Sigma_K^{-1} U_K\'
\\
\beta_t \omega_t &= U_K \Sigma_K \Sigma_K^{-1} U_K\' &= U_K U_K\' &= \sum_{i=1}^K \vec{u}_i \vec{u}_i^{\top}
\\
(\beta_t \omega_t) \beta_t &= (U_K U_K\')U_K \Sigma_K &= U_K \Sigma_K
\\ 
\Phi_t &= (I-\beta_t \omega_t) &= I - U_K U_K\' &= I_N - (I_N - \sum_{i=K+1}^N \vec{u}_i \vec{u}_i^{\top}) &= \sum_{i=K+1}^N \vec{u}_i \vec{u}_i^{\top}
\\
\Phi_t \beta_t &= (I-\beta_t \omega_t)\beta_t &= \beta_t - \beta_t \omega_t \beta_t &= \beta_t - \beta_t &= 0
%&=  (I - U_K U_K\')U_K \Sigma_K &= U_K \Sigma_K - U_K \red{U_K\'U_K} \Sigma_K &= U_K \Sigma_K - U_K \Sigma_K &=0
\end{array}$$
}


\bblue{

\begin{itemize}

\item Since $U$ and $V$ are orthogonal matrices from the Singular Value Decomposition (SVD), they satisfy $U^T U=I\in\R^{N\times N}$ and $V^T V=I\in\R^{T \times T}$. 

%%%%%%%%%%%%%%%%%%%%%%%%%%%%%%%%%%%%%%%%%%%%%%%%%%%%%
%%%%%%%%%%%%%%%%%%%%%%%%%%%%%%%%%%%%%%%%%%%%%%%%%%%%%
%%%%%%%%%%%%%%%%%%%%%%%%%%%%%%%%%%%%%%%%%%%%%%%%%%%%%
%%%%%%%%%%%%%%%%%%%%%%%%%%%%%%%%%%%%%%%%%%%%%%%%%%%%%

  \item $U \in \mathbb{R}^{N \times N}$ is an orthogonal matrix $\implies$
$
U^{\top} U=U U^{\top}=I_N,
$
where $I_N$ is the $N \times N$ identity matrix. Hence, the columns (and rows) of $U$ are orthonormal vectors in $\mathbb{R}^N$. 

\begin{itemize}
\item $U\'U = I_N$ can be seen from the inner product of the orthonormal vectors $\vec{u}_i$.
$$
U\'U 
=
\left[\begin{array}{c}
\vec{u}_1^{\top} \\
\vec{u}_2^{\top} \\
\vdots \\
\vec{u}_N^{\top}
\end{array}\right]\left[\begin{array}{llll}
\vec{u}_1 & \vec{u}_2 & \ldots & \vec{u}_N
\end{array}\right]=\left[\begin{array}{cccc}
\vec{u}_1^{\top} \vec{u}_1 & \vec{u}_1^{\top} \vec{u}_2 & \ldots & \vec{u}_1^{\top} \vec{u}_N \\
\vec{u}_2^{\top} \vec{u}_1 & \vec{u}_2^{\top} \vec{u}_2 & \ldots & \vec{u}_2^{\top} \vec{u}_N \\
\vdots & \vdots & \ddots & \vdots \\
\vec{u}_N^{\top} \vec{u}_1 & \vec{u}_N^{\top} \vec{u}_2 & \ldots & \vec{u}_N^{\top} \vec{u}_N
\end{array}\right] 
= I_N
$$

\item $UU\' = I_N$ can be seen from the outer product of the orthonormal vectors $\vec{u}_i\vec{u}_i\' \in\R^N$
$$
UU\' 
= 
\left[\begin{array}{llll}
\vec{u}_1 & \vec{u}_2 & \ldots & \vec{u}_N
\end{array}\right]\left[\begin{array}{c}
\vec{u}_1^{\top} \\
\vec{u}_2^{\top} \\
\vdots \\
\vec{u}_N^{\top}
\end{array}\right]
=
\sum_{i=1}^N \vec{u}_i \vec{u}_i^{\top}
= I_N
$$

\end{itemize}

%----------------------------------------------------
\item The matrix $U_K \in \mathbb{R}^{N \times K}$ is formed by taking the first $K$ columns of $U$ ($K \leq N$)
$$
U_K=\left[\begin{array}{llll}
\vec{u}_1 & \vec{u}_2 & \ldots & \vec{u}_K
\end{array}\right],
$$
where $\vec{u}_i\in\R^N$ are the orthonormal columns of $U$. This means:
$$
\vec{u}_i^{\top} \vec{u}_j= \begin{cases}1 & \text { if } i=j \\ 0 & \text { if } i \neq j\end{cases}
$$


\item \textbf{Computing $U_K\' U_K$: }

$U_K^{\top} \in \mathbb{R}^{K \times N}, U_K \in \mathbb{R}^{N \times K} \implies U_K^{\top} U_K \in \mathbb{R}^{K \times K}$. Computation:

$$
U_K^{\top} U_K
=
\left[\begin{array}{c}
\vec{u}_1^{\top} \\
\vec{u}_2^{\top} \\
\vdots \\
\vec{u}_K^{\top}
\end{array}\right]\left[\begin{array}{llll}
\vec{u}_1 & \vec{u}_2 & \ldots & \vec{u}_K
\end{array}\right]=\left[\begin{array}{cccc}
\vec{u}_1^{\top} \vec{u}_1 & \vec{u}_1^{\top} \vec{u}_2 & \ldots & \vec{u}_1^{\top} \vec{u}_K \\
\vec{u}_2^{\top} \vec{u}_1 & \vec{u}_2^{\top} \vec{u}_2 & \ldots & \vec{u}_2^{\top} \vec{u}_K \\
\vdots & \vdots & \ddots & \vdots \\
\vec{u}_K^{\top} \vec{u}_1 & \vec{u}_K^{\top} \vec{u}_2 & \ldots & \vec{u}_K^{\top} \vec{u}_K
\end{array}\right]
= 
I_K
$$
since the vectors $\vec{u}_i$ are orthonormal:


\item \textbf{Computing $U_K U_K\'$ }

$U_K \in \mathbb{R}^{N \times K}, U_K^{\top} \in \mathbb{R}^{K \times N} \implies U_K U_K^{\top} \in \mathbb{R}^{N \times N}$. Computation:
$$
U_K U_K^{\top}=\left[\begin{array}{llll}
\vec{u}_1 & \vec{u}_2 & \ldots & \vec{u}_K
\end{array}\right]\left[\begin{array}{c}
\vec{u}_1^{\top} \\
\vec{u}_2^{\top} \\
\vdots \\
\vec{u}_K^{\top}
\end{array}\right]
=
\sum_{i=1}^K \vec{u}_i \vec{u}_i^{\top}
$$
Hence, $U_K U_K^{\top}$ is not the identity matrix $I_N$ unless $K=N$. Instead, $U_K U_K^{\top}$ is a projection matrix onto the column space of $U_K$. 
\begin{itemize}
\item \textbf{Incomplete Basis}:  The set $\left\{\vec{u}_1, \vec{u}_2, \ldots, \vec{u}_K\right\}$ spans a $K$-dimensional subspace $\mathcal{S}$ of $\mathbb{R}^N$. It does not form a complete basis for $\mathbb{R}^N$ when $K<N$.
\item \textbf{Projection Operator}: $U_K U_K^{\top}$ is a projection matrix onto the subspace $\mathcal{S}$. This projection does not recover $\vec{x}$ unless $\vec{x} \in \mathcal{S}$. For any $\vec{x} \in \mathbb{R}^N$, the projection onto $\mathcal{S}$ is:
$$
U_K U_K^{\top} \vec{x}=\sum_{i=1}^K \vec{u}_i\left(\vec{u}_i^{\top} \vec{x}\right)
$$
\item Note that, since 
$
\sum_{i=1}^N \vec{u}_i \vec{u}_i^{\top} 
= 
(\sum_{i=1}^K \vec{u}_i \vec{u}_i^{\top})+(\sum_{i=K+1}^N \vec{u}_i \vec{u}_i^{\top})
= 
I_N
$, then
$$
\sum_{i=1}^K \vec{u}_i \vec{u}_i^{\top} = I_N - \sum_{i=K+1}^N \vec{u}_i \vec{u}_i^{\top}
$$
\end{itemize}
%----------------------------------------------------
\item \textbf{Question}: Given:
$
\left(\beta_t \omega_t\right) \beta_t=\left(U_K U_K^{\top}\right) U_K \Sigma_K=U_K \Sigma_K
$, does this imply that $U_K U_K^{\top}=I_N$ ?

\textbf{Answer}: No, it does not imply that $U_K U_K^{\top}=I_N$. The equation shows that $U_K U_K^{\top}$ acts as an identity only on the vectors in the column space of $U_K \Sigma_K$.

%----------------------------------------------------
\textbf{Key Points:}
\begin{enumerate}
  \item $U_K U_K^{\top}$ is a Projection Matrix:
It satisfies $\left(U_K U_K^{\top}\right)^2=U_K U_K^{\top}$.
It projects any vector onto the column space of $U_K$.
\item Acting on $U_K \Sigma_K$ :
Since $U_K \Sigma_K\in\t{Col}(U_K)$, projecting it onto the column space throught the projection matrix  $U_K U_K^{\top}$ leaves it unchanged: 
$
(U_K U_K^{\top})(U_K \Sigma_K)=U_K \Sigma_K
$
\item Not Acting as Identity on Entire $\mathbb{R}^N$ :
For $\vec{v}\not\in\t{Col}(U_K)$, the projection matrix $U_K U_K^{\top}$ does not act as the identity. \textit{Example:
Let $\vec{v}$ be orthogonal to the column space of $U_K$ :
$$
U_K^{\top} \vec{v}=0
$$
Then:
$$
U_K U_K^{\top} \vec{v}=U_K\left(U_K^{\top} \vec{v}\right)=U_K(0)=0 \neq \vec{v}
$$
This shows that $U_K U_K^{\top}$ does not act as the identity on $\vec{v}$.
}
\end{enumerate}

%----------------------------------------------------
\textbf{Further Explanation with Mathematical Details}
\begin{itemize}
  \item \textbf{Rank considerations}. $U_K$ has dimensions $N \times K$. Then, $U_K^{\top} U_K=I_K$ because the columns of $U_K$ are orthonormal. However, $U_K U_K^{\top}$ is an $N \times N$ matrix, and $\operatorname{rank}(U_K U_K^{\top})=K$, since it's the product of an $N \times K$ matrix and a $K \times N$ matrix. Since the Identity Matrix $I_N$ has rank $N$ and $\operatorname{rank}(U_K U_K^{\top})=K<N$, it's clear that $U_K U_K^{\top}$ cannot be equal to $I_N$, as a matrix of rank $K$ cannot equal a matrix of rank $N$.

\item \textbf{Projection onto Col($U_K$) considerations}. Any vector $\vec{x}$ in $\mathbb{R}^N$ can be decomposed as 
$
\vec{x}=\vec{x}_{\|}+\vec{x}_{\perp}
$, where 
$\vec{x}_{\|}$ is the component in the column space of $U_K$, and
$\vec{x}_{\perp}$ is the component orthogonal to the column space of $U_K$.
Applying $U_K U_K^{\top}$ to $\vec{x}$ :
$$
U_K U_K^{\top} \vec{x}
=
U_K U_K^{\top}(\vec{x}_{\|}+\vec{x}_{\perp})
=
U_K U_K^{\top} \vec{x}_{\|}+U_K U_K^{\top} \vec{x}_{\perp} 
=
\vec{x}_{\|} 
$$
because, since $U_K U_K^{\top}$ projects onto the column space, 
$U_K U_K^{\top} \vec{x}_{\|}=\vec{x}_{\|}$ and
$U_K U_K^{\top} \vec{x}_{\perp}=\overrightarrow{0}$.
Therefore,
$
U_K U_K^{\top} \vec{x}=\vec{x}_{\|}
$. In general, unless there is no component in $\vec{x}$ that is orthogonal to $\t{Col}(U_K)$; that is, if $\vec{x}_{\perp}=\overrightarrow{0}$ (i.e: $\vec{x}$ is in the column space), $U_K U_K^{\top} \vec{x} \neq \vec{x}$.
\end{itemize}

\end{itemize}
}

%%%%%%%%%%%%%%%%%%%%%%%%%%%%%%%%%%%%%%%%%%%%%%%%%%%%%
%%%%%%%%%%%%%%%%%%%%%%%%%%%%%%%%%%%%%%%%%%%%%%%%%%%%%
%%%%%%%%%%%%%%%%%%%%%%%%%%%%%%%%%%%%%%%%%%%%%%%%%%%%%
%%%%%%%%%%%%%%%%%%%%%%%%%%%%%%%%%%%%%%%%%%%%%%%%%%%%%

\textbf{Why is $\omega_t=\Sigma_K^{-1}U_K\'$ implied by Algorithm 1?}

According to Algorithm 1., $\omega_t$ can be computed by solving $F_t = \omega_t (r_t-r_{f,t})$. 

\begin{itemize}
\item \textbf{Truncated SVD with Top $K$ Components}. We can approximate $R_t-R_t^f$ using the top $K$ components:
$$
R_t-R_t^f \approx U_K \Sigma_K V_K^{\top}
$$

where $U_K \in \mathbb{R}^{N \times K}$ are the first $K$ columns of $U$, $\Sigma_K \in \mathbb{R}^{K \times K}$ are the top $K$ singular values, and $V_K^{\top} \in \mathbb{R}^{K \times T}$ are the first $K$ rows of $V^{\top}$.
Define $F_t = V_K^{\top} \in \mathbb{R}^{K \times T}$.
%$$
%F_t=V_K^{\top}=\left[\begin{array}{c}
%v_1^{\top} \\
%v_2^{\top} \\
%\vdots \\
%v_K^{\top}
%\end{array}\right]
%$$
\end{itemize}


$$\begin{array}{lllll}
\dot R_t &= U\Sigma V\' &\approx U_K\Sigma_K V_K\' 
\\
F_t &:=V_K &= \omega_t \dot R_t &\approx \omega_t U_K\Sigma_K V_K\'
%\\
%\omega_t &= F_t \dot{R}_t^{-1} &= V_K (U_K\Sigma_K V_K\')^{-1}
\end{array}$$

\begin{align*}
F_t &\approx \omega_t U_K\Sigma_K V_K\'
\\
V_K^T &\approx \omega_t U_K\Sigma_K V_K\'
\\
V_K^T V_K &\approx  \omega_t U_K\Sigma_K V_K\'V_K
\\
I &\approx  \omega_t U_K\Sigma_K 
\\
\omega_t &\approx (U_K\Sigma_K)^{-1} 
\\
& \approx \Sigma_K^{-1} U_K\'
\end{align*}

%%%%%%%%%%%%%%%%%%%%%%%%%%%%%%%%%%%%%%%%%%%%%%%%%%%%%
%%%%%%%%%%%%%%%%%%%%%%%%%%%%%%%%%%%%%%%%%%%%%%%%%%%%%
%%%%%%%%%%%%%%%%%%%%%%%%%%%%%%%%%%%%%%%%%%%%%%%%%%%%%
%%%%%%%%%%%%%%%%%%%%%%%%%%%%%%%%%%%%%%%%%%%%%%%%%%%%%

%\newpage
\Vhrulefill \\
My understanding of the algorithm:

for $t$ in $timeline$:
\begin{itemize}
  \item $\dot{\mbf r}_{t-w_{pca}+1:t} - \t{TSmean}(\dot{\mbf r}_{t-w_{pca}+1:t})= \mbf{U \Sigma V\'}\in\R^{w_{pca}\times N}$ ~~~~focus on $\mbf V\in \R^{N \times N}$
\begin{itemize}
  \item $\t{TSmean}(\dot{\mbf r}^{(i)}_{t-w_{pca}+1:t})=\frac{1}{w_{pca}} \sum_{s=t-w_{pca}+1}^t \dot r_{s}^{(i)}$ ~~for $i$ in $firms$
\end{itemize}

  \item $\b \omega_t = \mbf V_{[1:K,:]} \in\R^{K\times N}$ 
  \item $\mbf F_t = \b \omega_t [\dot{\mbf r_t} - \t{mean}(\dot{\mbf r}_{t-w_{pca}+1:t})] \in \R^{K\times 1}$
  \item \green{Solution: Use $\t{TSmean}(\dot{\mbf r}_{t-w_{pca}+1:t})$ to center the returns in the regression}
\begin{itemize}
  \item $[\dot{\mbf r}^{(i)}_{t-w_{reg}+1:t} - \t{TSmean}(\dot{\mbf r}^{(i)}_{t-w_{pca}+1:t})] = \b \beta_t^{(i)} \mbf F_t + \b \epsilon^{(i)}_{t-w_{reg}+1:t}\in \R^{w_{reg}\times 1}$ ~~for each $i$ in $firms$
  \item Extract the last TS element: $[\dot{\mbf r}^{(i)}_{t} - \t{TSmean}(\dot{\mbf r}^{(i)}_{t-w_{pca}+1:t})] = \b \beta_t^{(i)} \mbf F_t + \b \epsilon^{(i)}_{t}\in \R$
  \item Stack them XS: $[\dot{\mbf r}_{t} - \t{TSmean}(\dot{\mbf r}_{t-w_{pca}+1:t})] = \b \beta_t \mbf F_t + \b \epsilon_{t}\in \R^{N\times 1}$

Hence, now:
$$\begin{array}{rl}
\b \eps_t 
&= [\dot{\mbf r}_{t} - \t{TSmean}(\dot{\mbf r}_{t-w_{pca}+1:t})] -  \b \beta_t \mbf F_t
\\
&= [\dot{\mbf r}_{t} - \t{TSmean}(\dot{\mbf r}_{t-w_{pca}+1:t})] -  \b \beta_t \b \omega_t [\dot{\mbf r_t} - \t{mean}(\dot{\mbf r}_{t-w_{pca}+1:t})]
\\ &=
(\mbf I - \b \beta_t \b \omega_t )[\dot{\mbf r}_{t} - \t{TSmean}(\dot{\mbf r}_{t-w_{pca}+1:t})] 
\\ &=
\mbf  \Phi_t [\dot{\mbf r}_{t} - \t{TSmean}(\dot{\mbf r}_{t-w_{pca}+1:t})] 
\end{array}$$

  
\end{itemize}
\end{itemize}




\red{Problem: Inconsistent return centering $\implies$ we cannot compute $\mbf \Phi_t$}
\begin{itemize}
  \item $[\dot{\mbf r}^{(i)}_{t-w_{reg}+1:t} - \t{TSmean}(\dot{\mbf r}^{(i)}_{t-w_{reg}+1:t})] = \b \beta_t^{(i)} \mbf F_t + \b \epsilon^{(i)}_{t-w_{reg}+1:t}\in \R^{w_{reg}\times 1}$ ~~for each $i$ in $firms$
  \item Extract the last TS element: $[\dot{\mbf r}^{(i)}_{t} - \t{TSmean}(\dot{\mbf r}^{(i)}_{t-w_{reg}+1:t})] = \b \beta_t^{(i)} \mbf F_t + \b \epsilon^{(i)}_{t}\in \R$
  \item Stack them XS: $[\dot{\mbf r}_{t} - \t{TSmean}(\dot{\mbf r}_{t-w_{reg}+1:t})] = \b \beta_t \mbf F_t + \b \epsilon_{t}\in \R^{N\times 1}$
\end{itemize}

$$\begin{array}{rl}
\b \eps_t 
&= [\dot{\mbf r}_{t} - \t{TSmean}(\dot{\mbf r}_{t-w_{reg}+1:t})] -  \b \beta_t \mbf F_t
\\
&= [\dot{\mbf r}_{t} - \t{TSmean}(\dot{\mbf r}_{t-w_{reg}+1:t})] -  \b \beta_t \b \omega_t [\dot{\mbf r_t} - \t{mean}(\dot{\mbf r}_{t-w_{pca}+1:t})]
\\ \\  < \t{if }w= w_{pca}=w_{reg}> 
&=
[\dot{\mbf r}_{t} - \t{TSmean}(\dot{\mbf r}_{t-w+1:t})] -  \b \beta_t \b \omega_t [\dot{\mbf r_t} - \t{mean}(\dot{\mbf r}_{t-w+1:t})]
\\ &=
(\mbf I - \b \beta_t \b \omega_t )[\dot{\mbf r}_{t} - \t{TSmean}(\dot{\mbf r}_{t-w+1:t})] 
\\ &=
\mbf  \Phi_t [\dot{\mbf r}_{t} - \t{TSmean}(\dot{\mbf r}_{t-w+1:t})] 
\end{array}$$
 
 
 %%%%%%%%%%%%%%%%%%%%%%%%%%%%%%%%%%%%%%%%%%%%%%%%%%%%%
 %%%%%%%%%%%%%%%%%%%%%%%%%%%%%%%%%%%%%%%%%%%%%%%%%%%%%
 %%%%%%%%%%%%%%%%%%%%%%%%%%%%%%%%%%%%%%%%%%%%%%%%%%%%%
 %%%%%%%%%%%%%%%%%%%%%%%%%%%%%%%%%%%%%%%%%%%%%%%%%%%%%
\Vhrulefill

\newpage
\violet{
\chapter{Fitting an Ornstein-Uhlenbeck (OU) Process}

The objective is to fit an Ornstein-Uhlenbeck (OU) process to the cumulative residual returns $x_t^L$ for each asset. The OU process is a continuous-time stochastic process exhibiting mean-reverting behavior, suitable for modeling financial time series that drift toward a long-term mean.

\subsection{Definition of the Ornstein-Uhlenbeck Process}

The OU process is governed by the stochastic differential equation (SDE):

\begin{equation}
dX_t = \frac{1}{\tau} (\mu - X_t) dt + \sigma dB_t
\end{equation}

where:
\begin{itemize}
    \item $X_t$ is the state variable at time $t$.
    \item $\mu$ is the long-term mean toward which the process reverts.
    \item $\tau$ is the mean-reversion time (the speed of reversion).
    \item $\sigma$ is the volatility parameter.
    \item $dB_t$ is the increment of a standard Brownian motion.
\end{itemize}

\subsection{Discretization of the OU Process}

To fit this continuous-time process to discrete data, we discretize the SDE using the Euler-Maruyama method with a time step $\Delta t = 1$:

\begin{equation}
X_{t+1} = X_t + \frac{1}{\tau} (\mu - X_t) \Delta t + \sigma \epsilon_t
\end{equation}

where $\epsilon_t \sim \mathcal{N}(0, \Delta t)$ is a standard normal random variable.

Simplifying with $\Delta t = 1$:

\begin{equation}
X_{t+1} = X_t + \frac{1}{\tau} (\mu - X_t) + \sigma \epsilon_t
\end{equation}

\subsection{Rewriting the Equation}

Rewriting the equation:

\begin{equation}
X_{t+1} = \left(1 - \frac{1}{\tau}\right) X_t + \frac{\mu}{\tau} + \sigma \epsilon_t
\end{equation}

Define:

\begin{align}
a &= 1 - \frac{1}{\tau} \\
b &= \frac{\mu}{\tau}
\end{align}

The equation becomes:

\begin{equation}
X_{t+1} = a X_t + b + \sigma \epsilon_t
\end{equation}

This is a first-order autoregressive (AR(1)) process with an intercept term.

\subsection{Estimating Parameters Using Linear Regression}

We estimate the parameters $a$ and $b$ by performing linear regression on $X_t$ and $X_{t+1}$:

\begin{equation}
X_{t+1} = a X_t + b + \text{noise}
\end{equation}

\subsubsection{Linear Regression Model}

We set up the linear regression model:

\begin{equation}
y_t^{(i)} = a_i x_t^{(i)} + b_i + \epsilon_t^{(i)}
\end{equation}

where:
\begin{itemize}
    \item $y_t^{(i)} = x_{\text{curr}}^{(i)}$
    \item $x_t^{(i)} = x_{\text{lag}}^{(i)}$
\end{itemize}

\subsection{Estimating Coefficients}

We use the Ordinary Least Squares (OLS) method to estimate $a_i$ and $b_i$ by minimizing the sum of squared residuals:

\begin{equation}
\min_{a_i, b_i} \sum_{t} \left( y_t^{(i)} - a_i x_t^{(i)} - b_i \right)^2
\end{equation}

This can be solved analytically using the normal equations:

\begin{equation}
\beta^{(i)} = \left( X^{(i)\top} X^{(i)} \right)^{-1} X^{(i)\top} y^{(i)}
\end{equation}

where:
\begin{itemize}
	\item $y^{(i)}=[x_0^{(i)}, x_1^{(i)}, ..., x_{T-1}^{(i)}]\'$
	\item $X^{(i)}=[\mathbf{1}_{T-1}, x^{(i)}]$
	\item $x^{(i)} = [x_1^{(i)}, x_2^{(i)}, ..., x_{T}^{(i)}]\'$
    \item $\beta^{(i)} = \begin{bmatrix} a_i \\ b_i \end{bmatrix}$
    \item $X^{(i)}$ is the design matrix for asset $i$, including an intercept term.
\end{itemize}

\subsection{Computing Residuals and Estimating Volatility}

The residuals are computed as:

\begin{equation}
\epsilon_t^{(i)} = y_t^{(i)} - (a_i x_t^{(i)} + b_i)
\end{equation}

The volatility $\sigma_i$ is estimated as the standard deviation of the residuals:

\begin{equation}
\sigma_i = \sqrt{ \frac{1}{T - 2} \sum_{t=1}^{T-1} \left( \epsilon_t^{(i)} \right)^2 }
\end{equation}

\subsection{Recovering OU Process Parameters}

From the estimated $a_i$ and $b_i$, we recover $\tau_i$ and $\mu_i$.

\subsubsection{Mean-Reversion Time}

\begin{equation}
\tau_i = -\frac{1}{\ln(a_i)}
\end{equation}

\subsubsection{Long-Term Mean}

\begin{equation}
\mu_i = \frac{b_i}{1 - a_i}
\end{equation}

\subsection{Constraints on Parameter Estimates}

\subsubsection{Why $a_i \leq 0$ and $a_i \geq 1$ Are Invalid}

For the OU process:

\begin{itemize}
    \item The mean-reversion time $\tau > 0$, implying $a_i = 1 - \frac{1}{\tau} < 1$.
    \item If $a_i \geq 1$, $\ln(a_i) \geq 0$, leading to $\tau_i \leq 0$, which is not meaningful.
    \item If $a_i \leq 0$, $\ln(a_i)$ is undefined (complex), and $a_i$ implies explosive or oscillatory behavior, inconsistent with the OU process.
\end{itemize}

Therefore, valid estimates require $0 < a_i < 1$.

\subsection{Maximum Likelihood Estimation and Log-Likelihood Optimization}

Under the assumption of normally distributed residuals $\epsilon_t^{(i)}$, OLS estimation of the AR(1) model is equivalent to maximizing the log-likelihood function.

\subsubsection{Log-Likelihood Function}

The likelihood function for the AR(1) model is:

\begin{equation}
L(a_i, b_i, \sigma_i) = \prod_{t=1}^{T-1} \frac{1}{\sqrt{2\pi} \sigma_i} \exp\left( -\frac{ \left( y_t^{(i)} - a_i x_t^{(i)} - b_i \right)^2 }{2 \sigma_i^2} \right)
\end{equation}

Taking the natural logarithm:

\begin{equation}
\ln L(a_i, b_i, \sigma_i) = -\frac{(T - 1)}{2} \ln(2\pi \sigma_i^2) - \frac{1}{2 \sigma_i^2} \sum_{t=1}^{T-1} \left( y_t^{(i)} - a_i x_t^{(i)} - b_i \right)^2
\end{equation}

Maximizing the log-likelihood with respect to $a_i$ and $b_i$ is equivalent to minimizing the sum of squared residuals, which is what OLS achieves.
}


%%%%%%%%%%%%%%%%%%%%%%%%%%%%%%%%%%%%%%%%%%%%%%%%%%%%%
%%%%%%%%%%%%%%%%%%%%%%%%%%%%%%%%%%%%%%%%%%%%%%%%%%%%%
%%%%%%%%%%%%%%%%%%%%%%%%%%%%%%%%%%%%%%%%%%%%%%%%%%%%%
%%%%%%%%%%%%%%%%%%%%%%%%%%%%%%%%%%%%%%%%%%%%%%%%%%%%%

\newpage

\textbf{Discretization of an OU Process}

%%%%%%%%%%%%%%%%%%%%%%%%%%%%%%%%%%%%%%%%%%%%%%%%%%%%%
\Vhrulefill 
~~
\href{https://mammouth.ai/app/a/default/c/82166}{[Claude Sonnet 3.5]}
~~
\Vhrulefill 
%%%%%%%%%%%%%%%%%%%%%%%%%%%%%%%%%%%%%%%%%%%%%%%%%%%%%

Let's derive the exact solution for the OU process. Here's the step-by-step derivation:
First, let's rewrite our SDE: $$dX_t = \frac{1}{\tau}(\mu - X_t)dt + \sigma dB_t$$


The key is to use It�'s lemma with a clever choice of function. Let's try: $$Y_t = X_t e^{t/\tau}$$


Applying It�'s product rule: $d(XY) = Yd(X)+Xd(Y) + d(Y)d(X)$ $$dY_t = e^{t/\tau}dX_t + \frac{1}{\tau}X_te^{t/\tau}dt$$
Substitute the original SDE for $dX_t$

$$dY_t = e^{t/\tau}[\frac{1}{\tau}(\mu - X_t)dt + \sigma dB_t] + \frac{1}{\tau}X_te^{t/\tau}dt$$


Simplify: $$dY_t = \frac{\mu}{\tau}e^{t/\tau}dt + \sigma e^{t/\tau}dB_t$$

Integrate from s to t: 
$$Y_t - Y_s = \frac{\mu}{\tau}\int_s^t e^{u/\tau}du + \sigma\int_s^t e^{u/\tau}dB_u$$


Solve the deterministic integral: $$Y_t - Y_s = \mu(e^{t/\tau} - e^{s/\tau}) + \sigma\int_s^t e^{u/\tau}dB_u$$


Substitute back $Y_t = X_t e^{t/\tau}$
: $$X_t e^{t/\tau} - X_s e^{s/\tau} = \mu(e^{t/\tau} - e^{s/\tau}) + \sigma\int_s^t e^{u/\tau}dB_u$$
Solve for $X_t$
$$X_t = X_s e^{-(t-s)/\tau} + \mu(1-e^{-(t-s)/\tau}) + \sigma e^{-t/\tau}\int_s^t e^{u/\tau}dB_u$$


The integral $\int_{s}^{t} e^{u/\tau} dB_u$
 is normally distributed with mean 0 and variance $\frac{\tau}{2}\left(e^{2(t-s) / \tau}-1\right)$

Therefore, setting $\Delta_t = t-s$

$$X_{t+\Delta t} = X_t e^{-\Delta t/\tau} + \mu(1-e^{-\Delta t/\tau}) + \sigma\sqrt{\frac{1-e^{-2\Delta t/\tau}}{2/\tau}}\epsilon_t$$

where $\eps_t$ is a standard normal random variable

The variance of the stochastic integral term can be computed using It� isometry: $E[(\int_s^t e^{u/\tau}dB_u)^2] = \int_s^t e^{2u/\tau}du = \frac{\tau}{2}(e^{2(t-s)/\tau}-1)$

%%%%%%%%%%%%%%%%%%%%%%%%%%%%%%%%%%%%%%%%%%%%%%%%%%%%%
\Vhrulefill 
~~
\href{https://chatgpt.com/share/673478c9-072c-800d-91e0-ce3b3378d284}{[o1]} 
~~
\Vhrulefill
%%%%%%%%%%%%%%%%%%%%%%%%%%%%%%%%%%%%%%%%%%%%%%%%%%%%%

Step 1: Rewrite the SDE in Standard Form
First, rewrite the SDE to group like terms:

$$
d X_t=-\frac{1}{\tau} X_t d t+\frac{\mu}{\tau} d t+\sigma d B_t
$$


This rearranges the terms so that $X_t$ is isolated on one side.

Step 2: Apply an Integrating Factor
To solve this linear SDE, we'll use an integrating factor. The integrating factor $M_t$ is given by:
$$
M_t=\exp \left(-\int \frac{1}{\tau} d t\right)=\exp \left(-\frac{t}{\tau}\right) .
$$

Step 3: Multiply Both Sides by the Integrating Factor
Multiply both sides of the SDE by $M_t$ :

$$
M_t d X_t=-\frac{1}{\tau} M_t X_t d t+\frac{\mu}{\tau} M_t d t+\sigma M_t d B_t .
$$


Substituting $M_t=e^{t / \tau}$, we have:

$$
e^{t / \tau} d X_t=-\frac{1}{\tau} e^{t / \tau} X_t d t+\frac{\mu}{\tau} e^{t / \tau} d t+\sigma e^{t / \tau} d B_t .
$$

Step 4: Simplify the Left-Hand Side
Observe that:
\begin{align*}
d\left(e^{t / \tau} X_t\right)
&= e^{t / \tau}d\1{X_t} + d\1{e^{t / \tau}} X_t + d\1{e^{t / \tau}}d\1{X_t}
\\ &=
e^{t / \tau} d X_t + \1{\frac{1}{\tau} e^{t/\tau} dt}X_t  - \red{ \frac{1}{\tau} e^{t/\tau} \ub{dt d X_t}{=0}}
\\ &=
e^{t / \tau} d X_t+\frac{1}{\tau} e^{t / \tau} X_t d t
\end{align*}


Rewriting:
$$
\violet{e^{t / \tau} d X_t=d\left(e^{t / \tau} X_t\right) -\frac{1}{\tau} e^{t / \tau} X_t d t}
$$

Step 5: Substitute Back into the Equation
Substitute $e^{-t / \tau} d X_t$ from Step 4 back into the equation:
$$
\violet{
d\left(e^{t / \tau} X_t\right) 
-
\frac{1}{\tau} e^{t / \tau} X_t d t
}
=
-\frac{1}{\tau} e^{t / \tau} X_t d t
+
\frac{\mu}{\tau} e^{t / \tau} d t+\sigma e^{t / \tau} d B_t .
$$


Simplify by canceling out the $\1{-\frac{1}{\tau} e^{t / \tau} X_t d t}$ terms on both sides:
$$
d\left(e^{t / \tau} X_t\right)=\frac{\mu}{\tau} e^{t / \tau} d t+\sigma e^{t / \tau} d B_t
$$

Step 6: Integrate both sides from 0 to $t$ :

$$
\int_0^t d\left(e^{s / \tau} X_s\right)=\int_0^t \frac{\mu}{\tau} e^{s / \tau} d s+\int_0^t \sigma e^{s / \tau} d B_s
$$

Compute the left-hand side:
$$
e^{t / \tau} X_t-e^{0 / \tau} X_0=e^{t / \tau} X_t-X_0
$$

Compute the first integral on the right-hand side:
$$
\int_0^t \frac{\mu}{\tau} e^{s / \tau} d s
=
\mu \int_{0}^t \frac{1}{\tau} e^{s/\tau}
=
\mu\left[e^{s / \tau}\right]_0^t
=
\mu\left(e^{t / \tau}-1\right)
$$
Step 8: Simplify the Equation
Now, the equation becomes:
$$
e^{t / \tau} X_t-X_0
=
\mu\left(e^{t / \tau} - 1 \right)
+
\sigma \int_0^t e^{s / \tau} d B_s
$$
Solve for $X_t$
\begin{align*}
X_t 
&= X_0 e^{-t/\tau} +  \mu\left(e^{t / \tau} -1 \right) e^{-t/\tau} + e^{-t/\tau}\sigma \int_0^t e^{s / \tau} d B_s
\\&=
X_0 e^{-t/\tau} +  \mu\left(1-e^{-t / \tau}\right)
+
\sigma \int_0^t e^{-(t-s) / \tau} d B_s
\end{align*}
where $X_0$ is the initial condition at $t=0$



\subsubsection{Discretization Over Time Interval $\Delta t$}

We aim to find the relationship between $X_t$ and $X_{t+\Delta t}$ over a discrete time step $\Delta t$.

\textbf{Exact Discretization}

The exact discretization for the OU process from $t$ to $t+\Delta t$ is:
$$
X_{t+\Delta t}=
X_t e^{-\Delta t / \tau}
+
\mu (1- e^{-\Delta t / \tau})+\epsilon_t,
$$

where:
$e^{-\Delta t / \tau}$ is the decay factor,
$\epsilon_t$ is a Gaussian random variable representing the stochastic component.

\textbf{Variance of the Noise Term}

The variance of $\epsilon_t$ is derived from the integral term in the continuous solution:
$$
\epsilon_t=\sigma \int_t^{t+\Delta t} e^{-(t+\Delta t-s) / \tau} d B_s
$$

Compute the variance $\sigma_\epsilon^2$ :
$$
\begin{aligned}
\sigma_\epsilon^2 & =\operatorname{Var}\left(\epsilon_t\right) \\
& =\sigma^2 \int_t^{t+\Delta t} e^{-2(t+\Delta t-s) / \tau} d s \\
& =\sigma^2 \int_0^{\Delta t} e^{-2 u / \tau} d u \\
& =\frac{\sigma^2 \tau}{2}\left(1-e^{-2 \Delta t / \tau}\right)
\end{aligned}
$$

Combining the deterministic and stochastic components, the discrete-time OU process is:
$$
X_{t+\Delta t}=
X_t  e^{-\Delta t / \tau}
+
\mu (1-e^{-\Delta t / \tau})
+
\epsilon_t,
$$

with $\epsilon_t \sim N\left(0, \sigma_\epsilon^2\right)$ where:
$
\sigma_\epsilon^2=\frac{\sigma^2 \tau}{2}\left(1-e^{-2 \Delta t / \tau}\right) .
$
The discretized OU process resembles an Autoregressive process of order 1 (AR(1)):
$$
X_{t+\Delta t}=a X_t+b+\epsilon_t
$$
where:
$a=e^{-\Delta t / \tau}$ is the autoregressive coefficient,
$b=\mu(1-a)$ is a constant drift term,
$\epsilon_t$ is white noise with variance $\sigma_\epsilon^2$.

\textbf{Modus Operandi: Fit the discretized OU process to discrete data:}

\begin{enumerate}
  \item Collect Data: Ensure your data $\left\{X_{t_i}\right\}$ is sampled at regular intervals $\Delta t$.
  \item Estimate Parameters:
\begin{itemize}
  \item Use statistical methods (e.g., Maximum Likelihood Estimation or Least Squares) to estimate $a, b$, and $\sigma_\epsilon^2$ from the data.
  \item Recover $\tau$ from $a$ :

$$
\tau=-\frac{\Delta t}{\ln a}
$$

\item Compute $\mu$ from $b$ and $a$ :

$$
\mu=\frac{b}{1-a}
$$

\item Estimate $\sigma_{\eps}$ as the empirical standard deviation of residuals


$$\begin{array}{ll}
\t{Residuals for Asset $i$}
&
\epsilon_t^{(i)}=x_{t+1}^{(i)}-(a_i x_t^{(i)}+b_i)
\\
\t{Empirical St.Dev.}
&
\sigma_\eps^{(i)}=\sqrt{\frac{1}{T-2} \sum_{t=1}^{T-1}\left(\epsilon_t^{(i)}\right)^2}
\end{array}$$



\item Estimate $\sigma$ from $\sigma_\epsilon^2$ :

$$
\sigma=\sqrt{\frac{2 \sigma_\epsilon^2}{\tau\left(1-e^{-2 \Delta t / \tau}\right)}}
$$

\end{itemize}

\item Model Validation: Check the residuals $\epsilon_t$ to ensure they are white noise and normally distributed.
\end{enumerate}



\newpage
\chapter{Deep Neural Networks}



\end{document}






