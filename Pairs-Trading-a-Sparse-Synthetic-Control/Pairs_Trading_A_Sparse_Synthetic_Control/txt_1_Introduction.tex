\section{Introduction}


\subsection{Relative pricing}
Asset pricing can be viewed in absolute and relative terms. 
\begin{itemize}
\item Absolute pricing
\begin{itemize}
\item Absolute pricing values securities from fundamentals such as discounted future cash flow. This is a notoriously difficult process with a wide margin for error. 
\item Articles by Bakshi and Chen (1997) and Lee et al. (1997), for
example, are heroic attempts to build quantitative value-investing models.
\end{itemize}
\item Relative pricing
\begin{itemize}
\item Relative pricing is only slightly easier. 
\item Relative pricing means that two
securities that are close substitutes for each other should sell for the same
price-it does not say what that price will be. 
\item Thus, relative pricing allows
for bubbles in the economy, but not necessarily arbitrage or profitable
speculation. 
\item The Law of One Price [LOP] and a ``near-LOP'' are applicable to relative pricing-even if that price is wrong. Ingersoll (1987) defines the LOP as the ``proposition ... that two investments with the same payoff in every state of nature must have the same current value.'' In other words, two securities with the same prices in all
states of the world should sell for the same amount. 
\item Chen and Knez (1995) extend this to argue that ``closely integrated markets should assign to similar payoffs prices that are close.'' They argue that two securities
with similar, but not necessarily, matching payoffs across states should
have similar prices. This is of course a weaker condition and subject to
bounds on prices for unusual states; however, it allows the examination of
``near-efficient'' economies, or in Chen and Knez' case, near integrated
markets. Notice that this theory corresponds to the desire to find two
stocks whose prices move together as long as we can define states of
nature as the time series of observed historical trading days.
\end{itemize}
\end{itemize}


We use an algorithm to choose pairs based on the criterion that they have had the same or nearly the same state prices historically. We then trade pairs whose prices closely match in historical state-space, because the LOP suggests that in an efficient market their prices should be nearly identical. In
this framework, the current study can be viewed as a test of the LOP and near-LOP in the U.S. equity markets, under certain stationarity conditions. We are effectively testing the integration of very local markets-the markets
for specific individual securities. This is similar in spirit to Bossaerts' (1988) test of co-integration of security prices at the portfolio level. We further conjecture that the marginal profits to be had from risk arbitrage of these temporary deviations is crucial to the maintenance of first-order efficiency. We could not have the first effect without the second.
