%%%%%%%%%%%%%%%%%%%%%%%%%%%%%%%%%%%%%%%%%%%%%%%%%%%%%
% SPARSE SYNTHETIC CONTROL
%%%%%%%%%%%%%%%%%%%%%%%%%%%%%%%%%%%%%%%%%%%%%%%%%%%%%
\subsection{Sparse Synthetic Control}

The core component of our pairs trading strategy involves constructing a synthetic asset that replicates the price behavior of a target security (e.g: AAPL) using a sparse combination of assets from a donor pool. 
Let $\mbf y = [y_{t}]_{t=1}^T\in \R^{T}$ denote the log-price time series of a target asset and $\mbf X = [x_{1t}, ..., x_{Nt}]_{t=1}^T\in\R^{T\times N}$ denote the log-price time series of a donor pool of assets. Then, we can build a synthetic asset ${\mbf y}^*$ as
\begin{equation*}
{y}_{t}^* = \sum_{i=1}^N w_i^* x_{it}
\quad \text{for~} t=1,...,T
,
\end{equation*}

where the weights $\mbf w^*=[w_1^*, ..., w_N^*]\'\in \R^N$ are determined via an $\ell_1$-regularized least squares optimization problem
\begin{equation*}
\mathbf{w}^* = \argmin_{\mathbf{w} \in \R^{N}} \left\{\sum_{t=1}^T \left(y_{t} - \sum_{i=1}^N w_i x_{it}\right)^2 + \lambda\|\mathbf{w}\|_1\right\}
\quad \text{s.t.} \quad \mathbf{1}^\top \mathbf{w} = 1
\end{equation*}

where $\|\mathbf{w}\|_1 = \sum_{i=1}^N |w_i|$ denotes the $\ell_1$-norm of the weight vector and $\lambda > 0$ is a regularization parameter that controls the level of sparsity. This formulation promotes sparse solutions through the $\ell_1$ penalty term while maintaining the constraint that the weights sum to unity. The $\ell_1$ regularization, also known as the lasso penalty, induces sparsity by shrinking some weights exactly to zero, effectively performing feature selection among the donor assets. This sparsity-inducing property stems from the non-differentiability of the penalty term at the origin \cite{tibshirani1996regression}.

The optimization problem possesses several key theoretical and practical features that make it particularly suitable for our application. First, the combination of a quadratic loss function with the convex $\ell_1$-penalty and affine constraint guarantees a unique solution under mild regularity conditions \cite{boyd2004convex}. Second, the regularization parameter $\lambda$ (optimally selected through cross-validation) provides direct control over the sparsity level, with larger values yielding solutions with fewer non-zero weights. Third, the simplex constraint $\mbf 1^\top \mbf w = 1$ ensures interpretability of the synthetic control as a weighted portfolio of donor assets. We don't impose a convex hull restriction, which effectively means that we allow for negative ways in the synthetic asset.

The resulting weight vector $\mathbf{w}^*$ will typically have many components exactly equal to zero, with the number of non-zero weights decreasing as $\lambda$ increases. In practice, we identify the support of non-zero weights through thresholding:
\begin{equation*}
\mathcal{I} = \{i \in \{1,...,N\} : |w^*_i| > \epsilon\}
\end{equation*}
where $\epsilon > 0$ is a small tolerance threshold (in our application, $\epsilon \approx 10^{-6}$). The final synthetic asset is then constructed using only assets in $\mathcal{I}$. This approach provides computational advantages over cardinality-constrained formulations while maintaining sparse exposures. Moreover, the convex nature of the problem permits efficient solution via proximal algorithms or quadratic programming techniques, making it suitable for high-dimensional donor pools. %\citep{Beck2009}.

The reader may draw similarities of this process with the Engle-Granger procedure for estimating the cointegration vector associated to the target asset and the donor pool. If we don't impose the $\ell_1$-penalty, it can be shown that, under some conditions, the procedure to finding a trading basket for pairs-trading is equivalent to finding the cointegration vector. For a formal discussion see \cref{sec:cointegration_meets_synthetic_controls}.


%%%%%%%%%%%%%%%%%%%%%%%%%%%%%%%%%%%%%%%%%%%%%%%%%%%%%
% SPARSE SYNTHETIC CONTROL
%%%%%%%%%%%%%%%%%%%%%%%%%%%%%%%%%%%%%%%%%%%%%%%%%%%%%
\subsection{Sparse Synthetic Control}

The core component of our pairs trading strategy involves constructing a synthetic asset that replicates the price behavior of a target security (e.g: AAPL) using a sparse combination of assets from a donor pool. 
Let $\mbf y = [y_{t}]_{t=1}^T\in \R^{T}$ denote the log-price time series of a target asset and $\mbf X = [x_{1t}, ..., x_{Nt}]_{t=1}^T\in\R^{T\times N}$ denote the log-price time series of a donor pool of assets. Then, we can build a synthetic asset ${\mbf y}^*$ as\begin{equation*}
{y}_{t}^* = \sum_{i=1}^N w_i^* x_{it}
,
\end{equation*}

where the weights $\mbf w^*=[w_1^*, ..., w_N^*]\'$ are determined via an $\ell_1$-regularized least squares optimization problem

\begin{equation*}
\mathbf{w}^* = \argmin_{\mathbf{w} \in \R^{N}} \left\{\sum_{t=1}^T \left(y_{t} - \sum_{i=1}^N w_i x_{it}\right)^2 + \lambda\|\mathbf{w}\|_1\right\}
\quad \text{s.t.} \quad \mathbf{1}^\top \mathbf{w} = 1
\end{equation*}
where $\|\mathbf{w}\|_1 = \sum_{i=1}^N |w_i|$ denotes the $\ell_1$-norm of the weight vector and $\lambda > 0$ is a regularization parameter that controls the level of sparsity. This formulation promotes sparse solutions through the $\ell_1$ penalty term while maintaining the constraint that the weights sum to unity. The $\ell_1$ regularization, also known as the lasso penalty, induces sparsity by shrinking some weights exactly to zero, effectively performing feature selection among the donor assets.
\bblue{The $\ell_1$-penalty induces sparsity by shrinking small weights to zero through the non-differentiability of the penalty term at the origin \cite{tibshirani1996regression}}

The optimization problem is convex and can be efficiently solved using standard convex optimization algorithms. The resulting weight vector $\mathbf{w}^*$ will typically have many components exactly equal to zero, with the number of non-zero weights decreasing as $\lambda$ increases. This provides a computationally tractable approach to constructing sparse synthetic controls, where only a subset of the donor assets contribute to replicating the target asset's behavior.


For a discussion of the formal equivalence of this procedure with cointegration analysis, see \cref{sec:cointegration_meets_synthetic_controls}.


\bblue{
The optimization problem possesses several key features:

\begin{itemize}
\item \textbf{Convexity}: The quadratic loss function combined with the convex $\ell_1$-penalty and affine constraint guarantees a unique solution under mild regularity conditions \cite{boyd2004convex}.

\item \textbf{Sparsity Control}: The regularization parameter $\lambda$ governs the number of non-zero weights, with larger values yielding sparser solutions. Optimal $\lambda$ selection can be performed via cross-validation or information criteria.

\item \textbf{Portfolio Interpretation}: The simplex constraint $\mbf 1^\top \mbf w = 1$ ensures interpretability as a weighted portfolio of donor assets. Note that we are not imposing a convex hull restriction (which would restrict donor weights to be nonnegative). 
\end{itemize}

Post-optimization, the support of non-zero weights is identified through thresholding:
\begin{equation*}
\mathcal{I} = \{i \in \{1,...,N\} : |w^*_i| > \epsilon\}
\end{equation*}

where $\epsilon > 0$ is a small tolerance threshold (in our application, $\epsilon \approx 10^{-6}$). The final synthetic asset is then constructed using only assets in $\mathcal{I}$. %, effectively automating donor selection through continuous optimization rather than combinatorial search.
%
This approach provides computational advantages over cardinality-constrained formulations while maintaining sparse exposures. The convex formulation permits efficient solution via proximal algorithms or quadratic programming techniques, making it suitable for high-dimensional donor pools \cite{beck2009fast}.
}

%%%%%%%%%%%%%%%%%%%%%%%%%%%%%%%%%%%%%%%%%%%%%%%%%%%%%%%
%% SPARSE SYNTHETIC CONTROL
%%%%%%%%%%%%%%%%%%%%%%%%%%%%%%%%%%%%%%%%%%%%%%%%%%%%%%
%\subsection{Sparse Synthetic Control}
%
%The core component of our pairs trading strategy involves constructing a synthetic asset that replicates the price behavior of a target security (e.g., AAPL) using a combination of assets from a donor pool.
%Let $\mbf y = [y_{t}]_{t=1}^T\in \R^{T}$ denote the log-price time series of a target asset and $\mbf X = [x_{1t}, ..., x_{Nt}]_{t=1}^T\in\R^{T\times N}$ denote the log-price time series of a donor pool of assets. We construct a synthetic asset ${\mbf y}^*$ through a sparse linear combination
%\begin{equation*}
%{y}_{t}^* = \sum_{i=1}^N w_i^* x_{it}
%.
%\end{equation*}
%%
%The weights $\mbf w^*=[w_1^*, ..., w_N^*]$ are determined via an L1-regularized quadratic program
%%
%\begin{equation*}
%\mathbf{w}^* = \argmin_{\mathbf{w} \in \R^{N}} \sum_{t=1}^T \left(y_{t} - \sum_{i=1}^N w_i x_{it}\right)^2 + \lambda \sum_{i=1}^N |w_i|
%\quad \text{s.t.} \quad
%\mbf 1\' \mbf w = 1
%.
%\end{equation*}
%%
%where $\lambda$ is a regularization parameter that controls the sparsity of the solution. The L1 penalty term $\sum_{i=1}^N |w_i|$ promotes sparsity by encouraging many of the weights to be exactly zero, thereby selecting only a limited number of assets to receive nonzero weights.
%%
%%
%%
%%
%The L1-regularized optimization problem is solved using the following procedure:
%\begin{enumerate}
%\item Define the decision variable $\mbf w \in \R^N$.
%\item Formulate the objective function as the sum of the least squares error and the L1 penalty:
%%
%\begin{equation*}
%\text{Objective} = \sum_{t=1}^T \left(y_{t} - \sum_{i=1}^N w_i x_{it}\right)^2 + \lambda \sum_{i=1}^N |w_i|.
%\end{equation*}
%%
%\item Impose the constraint that the weights sum to 1:
%%
%\begin{equation*}
%\mbf 1\' \mbf w = 1.
%\end{equation*}
%%
%\item Solve the optimization problem using a convex optimization solver to obtain the optimal weights $\mbf w^*$.
%\end{enumerate}
%%
%The resulting weights $\mbf w^*$ are then used to construct the synthetic asset ${\mbf y}^*$ as follows:
%\begin{equation*}
%{y}_{t}^* = \sum_{i=1}^N w_i^* x_{it}.
%\end{equation*}
%%
%This approach ensures that the synthetic asset closely replicates the target asset while maintaining sparsity in the selection of donor assets.



%%%%%%%%%%%%%%%%%%%%%%%%%%%%%%%%%%%%%%%%%%%%%%%%%%%%%
% SPARSE SYNTHETIC CONTROL VIA L1 REGULARIZATION
%%%%%%%%%%%%%%%%%%%%%%%%%%%%%%%%%%%%%%%%%%%%%%%%%%%%%
\subsection{Sparse Synthetic Control via $\ell_1$-Regularization}

The synthetic asset construction employs an $\ell_1$-regularized convex optimization approach to simultaneously achieve portfolio sparsity and tracking accuracy. Let $\mbf y = [y_{t}]_{t=1}^T\in \R^{T}$ denote the log-price trajectory of the target asset and $\mbf X = [x_{1t}, ..., x_{Nt}]_{t=1}^T\in\R^{T\times N}$ represent the log-prices of $N$ assets in the donor pool. We construct a synthetic asset ${y}^*_t$ through a sparse linear combination:

\begin{equation*}
{y}^*_t = \sum_{i=1}^N w^*_i x_{it}
\end{equation*}

The weight vector $\mbf w^* = [w^*_1, ..., w^*_N]^\top$ is determined by solving the following convex optimization problem:

\begin{equation}
\mbf w^* = \argmin_{\mbf w \in \R^N} \left\{ \underbrace{\sum_{t=1}^T \left(y_t - \sum_{i=1}^N w_i x_{it}\right)^2}_{\text{Tracking error}} + \lambda \underbrace{\sum_{i=1}^N |w_i|}_{\ell_1\text{-penalty}} \right\} \quad \text{s.t.} \quad \mbf 1^\top \mbf w = 1
\end{equation}

where $\lambda > 0$ is a regularization parameter controlling the sparsity-accuracy trade-off. The $\ell_1$-penalty induces sparsity by shrinking small weights to zero through the non-differentiability of the penalty term at the origin \cite{tibshirani1996regression}. This convex relaxation avoids the computational complexity of cardinality constraints while maintaining desirable sparsity properties.

The optimization problem possesses several key features:

\begin{itemize}
\item \textbf{Convexity}: The quadratic loss function combined with the convex $\ell_1$-penalty and affine constraint guarantees a unique solution under mild regularity conditions \cite{boyd2004convex}.

\item \textbf{Sparsity Control}: The regularization parameter $\lambda$ governs the number of non-zero weights, with larger values yielding sparser solutions. Optimal $\lambda$ selection can be performed via cross-validation or information criteria.

\item \textbf{Portfolio Interpretation}: The simplex constraint $\mbf 1^\top \mbf w = 1$ ensures interpretability as a weighted portfolio of donor assets, preventing leverage through short positions.
\end{itemize}

Post-optimization, the support of non-zero weights is identified through thresholding:

\begin{equation*}
\mathcal{I} = \{i \in \{1,...,N\} : |w^*_i| > \epsilon\}
\end{equation*}

where $\epsilon > 0$ is a small tolerance threshold (typically $\epsilon \approx 10^{-6}$). The final synthetic asset is then constructed using only assets in $\mathcal{I}$, effectively automating donor selection through continuous optimization rather than combinatorial search.

This approach provides computational advantages over cardinality-constrained formulations while maintaining sparse exposures. The convex formulation permits efficient solution via proximal algorithms or quadratic programming techniques, making it suitable for high-dimensional donor pools \cite{beck2009fast}.


%%%%%%%%%%%%%%%%%%%%%%%%%%%%%%%%%%%%%%%%%%%%%%%%%%%%%
% SPARSE SYNTHETIC CONTROL
%%%%%%%%%%%%%%%%%%%%%%%%%%%%%%%%%%%%%%%%%%%%%%%%%%%%%
\subsection{Sparse Synthetic Control}

The core component of our pairs trading strategy involves constructing a synthetic asset that replicates the price behavior of a target security (e.g., AAPL) using a combination of assets from a donor pool. Let $\mbf y = [y_{t}]_{t=1}^T\in \R^{T}$ denote the log-price time series of a target asset and $\mbf X = [x_{1t}, ..., x_{Nt}]_{t=1}^T\in\R^{T\times N}$ denote the log-price time series of a donor pool of assets. We construct a synthetic asset ${\mbf y}^*$ through a sparse linear combination
\begin{equation*}
{y}_{t}^* = \sum_{i=1}^N w_i^* x_{it},
\end{equation*}
where the weights $\mbf w^*=[w_1^*, ..., w_N^*]$ are determined by solving an L1-regularized optimization problem. This approach promotes sparsity in the weights, ensuring that only a limited number of assets from the donor pool contribute to the synthetic asset.

Specifically, we solve the following convex optimization problem:
\begin{equation*}
\mathbf{w}^* = \argmin_{\mathbf{w} \in \R^{N}} \norm{\mathbf{y} - \mathbf{X}\mathbf{w}}_2^2 + \lambda \norm{\mathbf{w}}_1 
\quad \text{s.t.} \quad 
\mbf 1^\top \mbf w = 1,
\end{equation*}
where $\lambda > 0$ is a regularization parameter that controls the trade-off between sparsity and reconstruction accuracy. The term $\norm{\mathbf{w}}_1 := \sum_{i=1}^N |w_i|$ represents the L1-norm, which induces sparsity in the solution by penalizing nonzero weights.

This formulation has several advantages:
- The optimization problem is convex, ensuring computational efficiency and global optimality.
- The L1 penalty enforces sparsity directly, eliminating the need for heuristic approximations to cardinality constraints.
- The resulting weights are interpretable and robust, as only a subset of donor assets with significant contributions are selected.

The solution is obtained using numerical solvers for convex optimization, and assets with weights below a small threshold (e.g., $10^{-6}$) are effectively treated as having zero contribution. This approach provides a principled and rigorous framework for constructing sparse synthetic controls in pairs trading applications.



%%%%%%%%%%%%%%%%%%%%%%%%%%%%%%%%%%%%%%%%%%%%%%%%%%%%%
% SPARSE SYNTHETIC CONTROL VIA L1 REGULARIZATION
%%%%%%%%%%%%%%%%%%%%%%%%%%%%%%%%%%%%%%%%%%%%%%%%%%%%%
\subsection{Sparse Synthetic Control via \texorpdfstring{$L^1$}{L1} Regularization}

The synthetic asset construction problem is formulated as a convex optimization program with sparsity-inducing regularization. Let $\mbf y = [y_{t}]_{t=1}^T\in \R^{T}$ denote the log-price trajectory of the target asset and $\mbf X = [x_{1t}, ..., x_{Nt}]_{t=1}^T\in\R^{T\times N}$ the log-price matrix of the donor assets. We construct the synthetic asset $\mbf y^*$ through:

\begin{equation*}
{y}_{t}^* = \sum_{i=1}^N w_i^* x_{it}
\end{equation*}

The weights $\mbf w^*\in \R^N$ are determined via the regularized quadratic program:

\begin{equation*}
\mathbf{w}^* = \argmin_{\mathbf{w} \in \R^{N}} \left[ \underbrace{\sum_{t=1}^T \left(y_{t} - \sum_{i=1}^N w_i x_{it}\right)^2}_{\text{Tracking Error}} + \lambda \underbrace{\sum_{i=1}^N |w_i|}_{\text{$L^1$ Regularizer}} \right]
\quad \text{s.t.} \quad 
\left|
\begin{array}{ll}
	\mbf 1^\top \mbf w &= 1 
\end{array}
\right.
\end{equation*}

Where:
\begin{itemize}
\item $\lambda > 0$ is a hyperparameter controlling sparsity-intensity tradeoff
\item The $L^1$ regularizer induces parsimony through the geometric properties of the 1-norm ball
\item The simplex constraint $\mbf 1^\top \mbf w = 1$ ensures interpretable portfolio weights
\end{itemize}

This convex relaxation of the NP-hard cardinality-constrained problem provides:
\begin{itemize}
\item \textbf{Global optimality}: Elimination of local minima through convexity
\item \textbf{Sparsity}: The $L^1$ term shrinks small weights to zero via exact penalization
\item \textbf{Computational efficiency}: Enables use of convex optimization algorithms
\end{itemize}

The optimal weights $\mbf w^*$ satisfy the first-order Karush-Kuhn-Tucker conditions:
\begin{equation*}
\mbf X^\top(\mbf y - \mbf X\mbf w^*) = \frac{\lambda}{2}\mbf s + \mu\mbf 1
\end{equation*}
where $\mbf s \in \{-1,0,1\}^N$ is the subgradient of $\|\mbf w^*\|_1$ and $\mu$ is the Lagrange multiplier for the budget constraint.

The sparse support set $\mathcal{I} := \{i \in [N] : w_i^* \neq 0\}$ emerges endogenously from the optimization, with cardinality controlled by $\lambda$. Larger $\lambda$ values yield sparser solutions through more aggressive weight shrinkage, as shown by the regularization path:

\begin{equation*}
|\mbf X_i^\top(\mbf y - \mbf X\mbf w^*)| \leq \frac{\lambda}{2} + \mu \quad \forall i \notin \mathcal{I}
\end{equation*}

This formulation maintains the synthetic asset's economic interpretation as a statistical arbitrage portfolio while achieving parsimony through modern regularization techniques.


%==============[	  DISCUSSION  ]==============

Many alternative procedures for building a sparse synthetic control can be devised. A natural one would be to directly perform a cardinality constrained quadratic program 
\begin{equation*}
\mathbf{w}^* = \argmin_{\mathbf{w} \in \R^{N}} \sum_{t=1}^T \left(y_{t} - \sum_{i=1}^N w_i x_{it}\right)^2 
\quad \text{s.t.} \quad 
\left|
\begin{array}{ll}
	\mbf 1\' \mbf w &= 1 \\
	\norm{\mathbf{w}}_0 &\leq K
\end{array}
\right
.
\end{equation*}
%
where $\|\mathbf{w}\|_0:=\sum_{i=1}^N \I{w_i\neq 0}$ counts the non-zero elements in $\mbf w$. The goal is to enforce sparsity so that only a limited number of assets receive a nonzero weight. 
%
%
%
%
The NP-hard cardinality constraint can be approximated by many different procedures:

\subsubsection{Mixed integer} % mixed integer stuff here
% explain this in detail

\subsubsection{Procedure 2} % Give a name to this procedure 
\begin{enumerate}
\item Solve the full least squares problem
%
\begin{equation*}
\mathbf{w}^{(1)} = \argmin_{\mathbf{w} \in \mathbb{R}^{N}} \norm{\mathbf{y} - \mathbf{X}\mathbf{w}}_2^2
\quad \text{s.t.} \quad \mathbf{1}^\top \mbf w=1.
\end{equation*}
%
\item Select the $K$ largest weights (in absolute value) from $\mbf w^{(1)}$ into
$$\mathcal I:=\{i : |w_i^{(1)}| \t{~among $K$ largests}\}$$
%
\item Solve the restricted program on support $\mathcal I$
%
\begin{equation*}
	\mbf w^{(2)} = \arg \min_{\mbf w_{\mathcal I}\in \mathbb{R}^K} \norm{\mbf y - \mbf X_{\mathcal I}\mbf w_{\mathcal I}}_{2}^{2}
\quad \text{s.t.} \quad 
\mbf 1\' \mbf w_{\mathcal I} = 1
\end{equation*}
%
where $\mbf X_{\mathcal{I}} \in \mathbb{R}^{T \times K}$ is the resricted donor matrix and $\mbf w_{\mathcal{I}} \in \mathbb{R}^{K}$ is the restricted weight vector for the selected assets.
%
\item Construct the full weight vector $\mbf w^* \in \mathbb{R}^{N}$ by embedding the optimized restricted weights back into the original $N$-dimensional space. 
\begin{equation*}
	w^*_i = 
\mycases{llll}{
w^{(2)}_j & \IF  i = \mathcal I_j
\\
0 & \text{otherwise}
}
\end{equation*}
\end{enumerate}

The problems with these procedure are [insert problems here]
In our application, when implementing this procedure, we would get extremely large weights (positive and negative), yielding a highly leveraged portfolio


%%%%%%%%%%%%%%%%%%%%%%%%%%%%%%%%%%%%%%%%%%%%%%%%%%%%%%
%% SPARSE SYNTHETIC CONTROL
%%%%%%%%%%%%%%%%%%%%%%%%%%%%%%%%%%%%%%%%%%%%%%%%%%%%%%
%\subsection{Sparse Synthetic Control}
%
%The core component of our pairs trading strategy involves constructing a synthetic asset that replicates the price behavior of a target security (e.g: AAPL) using a combination of assets from a donor pool. 
%Let $\mbf y = [y_{t}]_{t=1}^T\in \R^{T}$ denote the log-price time series of a target asset and $\mbf X = [x_{1t}, ..., x_{Nt}]_{t=1}^T\in\R^{T\times N}$ denote the log-price time series of a donor pool of assets. We construct a synthetic asset ${\mbf y}^*$ through a sparse linear combination
%\begin{equation*}
%{y}_{t}^* = \sum_{i=1}^N w_i^* x_{it}
%.
%\end{equation*}
%%
%The weights $\mbf w^*=[w_1^*, ..., w_N^*]$ are determined via a cardinality-constrained quadratic program
%%
%\begin{equation*}
%\mathbf{w}^* = \argmin_{\mathbf{w} \in \R^{N}} \sum_{t=1}^T \left(y_{t} - \sum_{i=1}^N w_i x_{it}\right)^2 
%\quad \text{s.t.} \quad 
%\left|
%\begin{array}{ll}
%	\mbf 1\' \mbf w &= 1 \\
%	\norm{\mathbf{w}}_0 &\leq K
%\end{array}
%\right
%.
%\end{equation*}
%%
%where $\|\mathbf{w}\|_0:=\sum_{i=1}^N \I{w_i\neq 0}$ counts the non-zero elements in $\mbf w$. The goal is to enforce sparsity so that only a limited number of assets receive a nonzero weight. 
%%
%%
%%
%%
%The NP-hard cardinality constraint is approximated by the following procedure: 
%\begin{enumerate}
%\item Solve the full least squares problem
%%
%\begin{equation*}
%\mathbf{w}^{(1)} = \argmin_{\mathbf{w} \in \mathbb{R}^{N}} \norm{\mathbf{y} - \mathbf{X}\mathbf{w}}_2^2
%\quad \text{s.t.} \quad \mathbf{1}^\top \mbf w=1.
%\end{equation*}
%%
%\item Select the $K$ largest weights (in absolute value) from $\mbf w^{(1)}$ into
%$$\mathcal I:=\{i : |w_i^{(1)}| \t{~among $K$ largests}\}$$
%%
%\item Solve the restricted program on support $\mathcal I$
%%
%\begin{equation*}
%	\mbf w^{(2)} = \arg \min_{\mbf w_{\mathcal I}\in \mathbb{R}^K} \norm{\mbf y - \mbf X_{\mathcal I}\mbf w_{\mathcal I}}_{2}^{2}
%\quad \text{s.t.} \quad 
%\mbf 1\' \mbf w_{\mathcal I} = 1
%\end{equation*}
%%
%where $\mbf X_{\mathcal{I}} \in \mathbb{R}^{T \times K}$ is the resricted donor matrix and $\mbf w_{\mathcal{I}} \in \mathbb{R}^{K}$ is the restricted weight vector for the selected assets.
%%
%\item Construct the full weight vector $\mbf w^* \in \mathbb{R}^{N}$ by embedding the optimized restricted weights back into the original $N$-dimensional space. 
%\begin{equation*}
%	w^*_i = 
%\mycases{llll}{
%w^{(2)}_j & \IF  i = \mathcal I_j
%\\
%0 & \text{otherwise}
%}
%\end{equation*}
%\end{enumerate}