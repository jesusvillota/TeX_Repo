\section{Methodology}

\begin{itemize}
	\item  We implement pairs trading in two stages. We form \qquote{pairs} over a 12-month formation period (train) and trade them in the next 6-month period (test). 
\blue{[Gatev et al (2006)]}
%%%%%%%%%%%%%%%%%%%%%%%%%%%%%%%%%%%%%%%%%%%%%%%%%%%%%
\item \textbf{Formation period}
\begin{itemize}
	\item In each pairs formation period, we screen out stocks from CRSP daily files that have one or more days with no trade. This serves to identify relatively liquid stocks as well as to facilitate pairs formation
\blue{[Gatev et al (2006)]}
	\item Next we construct a cumulative total returns index for each stock over the formation period
\blue{[Gatev et al (2006)]}
	\item We then choose a matching partner for the target stock by finding the linear combination of securities that minimize the sum of squared deviations between the two normalized price series. [Unrestricted]
	\begin{itemize}
		\item \brown{We use this approach because it is consistent with the notion of cointegration. In fact, what we are doing is tracking the target asset with a synthetic asset. By construction, the tracking error is centered at 0 and mean reverting. In other words, there is a cointegration relationship between the target and synthetic assets.}
	\end{itemize}
	\item In addition to \qquote{unrestricted} pairs, we will also present results by sector, where we restrict the synthetic asset to contain stocks belonging to the same industry as the target stock. The Industry categories employed are broad, as defined by Standard and Poors: Utilities, Transportation, Financial and Industrials. Each stock is assigned to one of these four groups based on the stock's SIC code. The minimum distance criterion is then used to match stocks within each of the groups. 
\blue{[Gatev et al (2006)]}
\end{itemize}
%%%%%%%%%%%%%%%%%%%%%%%%%%%%%%%%%%%%%%%%%%%%%%%%%%%%%
\item \textbf{Trading period}
\begin{itemize}
	\item Once we have paired up all liquid stocks in the formation period, we study the top 5 and 20 pairs with the smallest historical distance measure (in our case, with the \textit{smallest historical tracking error}), in addition to the 20 pairs after the top 100 (pairs 101-120). 
	\begin{itemize}	
		\item \brown{This last set is valuable because most ot the top pairs share certain characteristics} 
	\end{itemize}
	\item On the day following the last day of the pairs formation period, we begin to trade according to a prespecified rule. 
	\item We base our rules for opening and closing positions on a standard deviation metric. We open a position in a pair when prices diverge by more than 2 historical standard deviations, as esitmated during the pairs formation period. We unwind the position at the next crossing of the prices. If prices do not cross before the end of the trading interval, gais or losses are calculated at the end of the last trading day of the trading interval. If a stock in a pair is delisted from CRSP, we close the position in that pair, using the delising return, or the last available price. 
	\begin{itemize}
		\item A potential problem arises if inaccurate and stale prices exaggerate the excess returns and bias the estimated return of a long position in a plummeting stock. To address this potential concern, we reestimate our results under the extreme assumption that only a long stock experiences a $-100\%$ return when it is delisted. This zero-price extreme includes, among other things, the possibility that of nontrading due to the lack of liquidity. Because selective loss on the long position always harms the pair profit, this extreme assumption biases the results against profitability. 
	\end{itemize}
	\item We report the payoff by going one dollar shot in the higher-priced stock and one dollar long in the lower-priced stock
\end{itemize}
%%%%%%%%%%%%%%%%%%%%%%%%%%%%%%%%%%%%%%%%%%%%%%%%%%%%%
\item \textbf{Excess return computation}
	\begin{itemize}
		\item Pairs that open and converge during the trading interval will have positive cash flows. Because pairs can reopen after initial convergence, they can have multiple positive cash flows during the trading interval.
		\item Pairs that open but do not converge will only have cash flows on the last day of the trading interval when all positions are closed out. 
		\item Therefore, the payoffs to a pairs trading strategies are a set of positive cash flows that are randomly distributed throughout the trading period, and a set of cash flows at the end of the trading interval that can be either positive or negative.
		\item For each pair we can have multiple cash flows during the trading interval, or we may have none in the case when prices never diverge by more than 2 historical standard deviations during the trading interval
		\item Because the trading gains and losses are computed over long-short positions of one dollar, the payoffs have the interpretation of excess returns. 
		\item The excess return on a pair during a trading interval is computed as the reinvested payoffs during the trading interval.
		\begin{itemize}
			\item \brown{This is a conservative approach to computing the excess return, because it implicitly assumes that all cash earns zero interest rate when not invested in an open pair. Because any cash flow during the trading interval is positive by construction, it ignores the fact that these cash flows are received early and understates the computed excess returns.}
		\end{itemize}
		\item In particular, the long and short portfolio positions are marked-to-market daily.
		\item The daily returns on the long and short positions are calculated as value-weighted returns
$$
r_{P,t} = \frac{\sum_{i\in P} w_{i,t} r_{i,t}}{\sum_{i \in P} w_{i,t}}
\qquad
w_{i,t} 
= w_{i,t-1}(1+r_{i,t-1}) 
= (1+r_{i,1})\cdots (1+r_{i,t-1})  
$$
where $r$ refines returns and $w$ defines weights, and the daily returns are compounded to obtain monthly returns. This has the simple interpretation of a buy-and-hold strategy.
	\end{itemize}
\end{itemize}


