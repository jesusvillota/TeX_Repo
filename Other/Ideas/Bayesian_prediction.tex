\documentclass[12pt,article]{memoir}
\usepackage{/Users/jesusvillotamiranda/Documents/LaTeX/$$JVM_Macros}
\Subject{Ideas}
%\Arg{ARG}

\begin{document}
What are the obects of prediction in Portfolio Choice?
\begin{itemize}
  \item Stock returns $\mbf r \in \R$
  \item Quantile $Q$
  \item Portfolio weights $\mbf w$
\end{itemize}

\chapter{Formulation of the prediction problem}

We could think of returns in the context of a state space model. Returns are observable and noisy variables driven by firm characteristics and macro factors. 
\begin{align*}
\mbf r_t &= \mbf D(\b \theta) + \mbf Z(\b \theta)\mbf s_t + \mbf u_t
\\
\mbf s_t &= \mbf T(\b \theta)\mbf s_t + \mbf R(\b \theta) \b \eps_t
\end{align*}

The issue with this formulation is that, in reality returns are realized at the same time as states...

\begin{itemize}
  \item We could allow for non-linearities by defining instead a $\mathcal Z(\b \theta ), \mathcal T(\b \theta)$, which would potentially open the door to neural nets... (and any machine learning)
  \item $\mbf s_t$ are the states containing firm-specific characteristics and macro factors. 
  \item Cool idea: We could include in the states some other variables related to news published about a firm. Say $\mbf n_t^i=(n_t^{i(+)},n_t^{i(-)})$, which could be the measure of mediatic attention for firm $i$, where $n_t^{i(+)}$ denoted positive news coverage, and $n_t^{i(-)}$ denotes negative news coverage.
\end{itemize}



\end{document}




