\section{Theoretical Framework}

This section establishes the theoretical foundations for integrating the Synthetic Control Method (SCM) into asset pricing models. We begin by formalizing the mathematical underpinnings of SCM, followed by its incorporation into asset pricing frameworks. Finally, we outline the key assumptions and conditions required for the validity of our proposed model.

\subsection{Mathematical Foundations of SCM}

\subsubsection{Notation and Preliminary Definitions}

Let us consider a set of $N$ assets indexed by $i = 1, 2, \dots, N$. Each asset is characterized by a vector of observable characteristics $\mathbf{X}_i \in \mathbb{R}^K$, where $K$ denotes the number of covariates or factors influencing asset returns. The return of asset $i$ at time $t$ is denoted by $R_{it} \in \mathbb{R}$.

Define the treated asset as asset $1$ and the control group as the remaining assets $\mathcal{C} = \{2, 3, \dots, N\}$. The goal of SCM is to construct a synthetic counterpart for the treated asset using a weighted combination of control assets.

\subsubsection{Construction of the Synthetic Control}

The synthetic control for asset $1$ is defined as:
\[
R_{1t}^{\text{SCM}} = \sum_{j \in \mathcal{C}} w_j R_{jt}, \quad \text{for } t = 1, 2, \dots, T
\]
where $w_j \geq 0$ are the weights assigned to each control asset $j$, and $\sum_{j \in \mathcal{C}} w_j = 1$. The weights $\mathbf{w} = (w_2, w_3, \dots, w_N)^\top \in \mathbb{R}^{N-1}$ are determined by minimizing the discrepancy between the treated asset and its synthetic control in the pre-treatment period.

\subsubsection{Optimization Problem}

Formally, the weights $\mathbf{w}$ are obtained by solving the following optimization problem:
\[
\mathbf{w}^* = \arg\min_{\mathbf{w} \in \mathbb{R}^{N-1}} \left\| \mathbf{X}_1 - \mathbf{X}_{\mathcal{C}} \mathbf{w} \right\|_2^2 + \lambda \|\mathbf{w}\|_2^2
\]
subject to
\[
w_j \geq 0 \quad \forall j \in \mathcal{C}, \quad \text{and} \quad \sum_{j \in \mathcal{C}} w_j = 1.
\]
Here, $\mathbf{X}_{\mathcal{C}} \in \mathbb{R}^{K \times (N-1)}}$ is the matrix of characteristics for the control assets, and $\lambda \geq 0$ is a regularization parameter to prevent overfitting.

\subsubsection{Properties of the Synthetic Control}

\begin{lemma}
\label{lem:balance}
Under the optimal weights $\mathbf{w}^*$, the synthetic control satisfies:
\[
\mathbf{X}_1 = \mathbf{X}_{\mathcal{C}} \mathbf{w}^* + \boldsymbol{\epsilon},
\]
where $\boldsymbol{\epsilon} \in \mathbb{R}^K$ represents the residual imbalance.
\end{lemma}

\begin{proof}
By the definition of the optimization problem, the weights $\mathbf{w}^*$ minimize the squared norm of the discrepancy $\left\| \mathbf{X}_1 - \mathbf{X}_{\mathcal{C}} \mathbf{w} \right\|_2^2$. Therefore, at the optimum:
\[
\mathbf{X}_1 = \mathbf{X}_{\mathcal{C}} \mathbf{w}^* + \boldsymbol{\epsilon},
\]
where $\boldsymbol{\epsilon}$ is the residual vector capturing the imbalance that cannot be eliminated by the linear combination of control assets.
\end{proof}

\begin{theorem}
\label{thm:consistency}
Assume that the true data-generating process for asset returns is given by:
\[
R_{it} = \mathbf{X}_i^\top \boldsymbol{\beta} + \epsilon_{it}, \quad \forall i \in \{1, 2, \dots, N\}, \quad t = 1, 2, \dots, T,
\]
where $\boldsymbol{\beta} \in \mathbb{R}^K$ is a vector of true coefficients, and $\epsilon_{it}$ are i.i.d. error terms with $\mathbb{E}[\epsilon_{it}] = 0$ and $\text{Var}(\epsilon_{it}) = \sigma^2$.

If the weights $\mathbf{w}^*$ satisfy $\mathbf{X}_1 = \mathbf{X}_{\mathcal{C}} \mathbf{w}^*$, then:
\[
\mathbb{E}[R_{1t} - R_{1t}^{\text{SCM}}] = 0, \quad \forall t \in \{1, 2, \dots, T\}.
\]
\end{theorem}

\begin{proof}
Given the data-generating process:
\[
R_{1t} = \mathbf{X}_1^\top \boldsymbol{\beta} + \epsilon_{1t},
\]
and
\[
R_{1t}^{\text{SCM}} = \sum_{j \in \mathcal{C}} w_j^* R_{jt} = \sum_{j \in \mathcal{C}} w_j^* \left( \mathbf{X}_j^\top \boldsymbol{\beta} + \epsilon_{jt} \right).
\]
Under the condition $\mathbf{X}_1 = \mathbf{X}_{\mathcal{C}} \mathbf{w}^*$, we have:
\[
R_{1t}^{\text{SCM}} = \mathbf{X}_1^\top \boldsymbol{\beta} + \sum_{j \in \mathcal{C}} w_j^* \epsilon_{jt}.
\]
Therefore, the difference is:
\[
R_{1t} - R_{1t}^{\text{SCM}} = \epsilon_{1t} - \sum_{j \in \mathcal{C}} w_j^* \epsilon_{jt}.
\]
Taking expectation:
\[
\mathbb{E}[R_{1t} - R_{1t}^{\text{SCM}}] = \mathbb{E}[\epsilon_{1t}] - \sum_{j \in \mathcal{C}} w_j^* \mathbb{E}[\epsilon_{jt}] = 0 - 0 = 0.
\]
\end{proof}

\subsection{Integration with Asset Pricing}

\subsubsection{Asset Pricing Model Framework}

Consider a linear asset pricing model where the expected return of asset $i$ is a function of its exposure to various risk factors:
\[
E(R_i) = \mathbf{X}_i^\top \boldsymbol{\beta},
\]
where $\mathbf{X}_i \in \mathbb{R}^K$ is the vector of factor loadings for asset $i$, and $\boldsymbol{\beta} \in \mathbb{R}^K$ is the vector of risk premiums associated with each factor.

\subsubsection{Synthetic Control Asset Pricing Model}

Integrating SCM into the asset pricing framework involves constructing a synthetic counterpart for the treated asset and using it to estimate the expected return under the absence of certain interventions or shocks. Formally, the SCM-based expected return for asset $1$ is:
\[
E(R_1^{\text{SCM}}) = \mathbf{X}_1^\top \boldsymbol{\beta} = \sum_{j \in \mathcal{C}} w_j^* \mathbf{X}_j^\top \boldsymbol{\beta} = \mathbf{w}^{*\top} \mathbf{X}_{\mathcal{C}}^\top \boldsymbol{\beta},
\]
where $\mathbf{w}^*$ are the optimal weights obtained from the SCM optimization problem.

\subsubsection{Estimation of Risk Premiums}

To estimate the vector of risk premiums $\boldsymbol{\beta}$, we can set up a system of equations based on the synthetic control framework. Let $\mathbf{R} \in \mathbb{R}^{N \times T}$ be the matrix of asset returns, where the $(i,t)$-th entry is $R_{it}$. Similarly, let $\mathbf{X} \in \mathbb{R}^{N \times K}$ be the matrix of asset characteristics.

Assuming that the synthetic control for each asset provides an unbiased estimator of the expected return, we can write:
\[
\mathbf{R} = \mathbf{X} \boldsymbol{\beta} + \boldsymbol{\epsilon},
\]
where $\boldsymbol{\epsilon} \in \mathbb{R}^{N \times T}$ represents the error terms.

The estimation of $\boldsymbol{\beta}$ can be approached using ordinary least squares (OLS) or other regression techniques. However, incorporating SCM allows for a more refined estimation by leveraging the synthetic control weights to mitigate potential biases arising from omitted variables or model misspecification.

\subsubsection{Integration with Factor Models}

The SCM-based approach can be seamlessly integrated with existing factor models, such as the Fama-French models. For instance, consider the Fama-French three-factor model:
\[
R_{it} - R_{ft} = \alpha_i + \beta_{iM} (R_{Mt} - R_{ft}) + \beta_{iSMB} \text{SMB}_t + \beta_{iHML} \text{HML}_t + \epsilon_{it},
\]
where $R_{ft}$ is the risk-free rate, $R_{Mt}$ is the market return, and SMB and HML are the size and value factors, respectively.

By applying SCM, we can construct a synthetic counterpart for each asset that aligns with its exposure to these factors, potentially enhancing the model's explanatory power and robustness.

\subsection{Assumptions and Conditions}

For the theoretical integration of SCM into asset pricing to hold, several key assumptions and conditions must be satisfied. These are outlined below.

\subsubsection{Assumption 1: Linear Factor Structure}

We assume that asset returns can be expressed as a linear combination of their factor exposures:
\[
R_{it} = \mathbf{X}_i^\top \boldsymbol{\beta} + \epsilon_{it}, \quad \forall i \in \{1, 2, \dots, N\}, \quad t = 1, 2, \dots, T.
\]
This linearity assumption is fundamental to the applicability of both traditional factor models and the SCM-based approach.

\subsubsection{Assumption 2: Exogeneity of Factors}

The risk factors are assumed to be exogenous, meaning that they are not influenced by the asset returns:
\[
\mathbb{E}[\epsilon_{it} \mid \mathbf{X}_i] = 0, \quad \forall i, t.
\]
This ensures that the factor premiums $\boldsymbol{\beta}$ can be consistently estimated.

\subsubsection{Assumption 3: Positive Weights and Convexity}

The weights assigned to the control assets are non-negative and sum to one:
\[
w_j \geq 0 \quad \forall j \in \mathcal{C}, \quad \text{and} \quad \sum_{j \in \mathcal{C}} w_j = 1.
\]
This convexity constraint ensures that the synthetic control is a valid convex combination of the control assets, facilitating interpretability and stability in the estimation process.

\subsubsection{Condition 1: Overlap and Support}

There exists sufficient overlap in the characteristics of the treated and control assets such that:
\[
\mathbf{X}_1 \in \text{conv}(\mathbf{X}_{\mathcal{C}}),
\]
where $\text{conv}(\mathbf{X}_{\mathcal{C}})$ denotes the convex hull of the control assets' characteristics. This condition is necessary for the existence of feasible weights $\mathbf{w}^*$ that can adequately balance the treated asset's characteristics.

\subsubsection{Condition 2: No Perfect Multicollinearity}

The matrix $\mathbf{X}_{\mathcal{C}}$ has full column rank:
\[
\text{rank}(\mathbf{X}_{\mathcal{C}}) = K.
\]
This ensures that the optimization problem for determining the weights $\mathbf{w}^*$ has a unique solution and that the synthetic control can be precisely constructed.

\subsubsection{Condition 3: Bounded Error Terms}

The error terms $\epsilon_{it}$ are bounded with high probability:
\[
|\epsilon_{it}| \leq \overline{\epsilon}, \quad \forall i, t,
\]
for some constant $\overline{\epsilon} > 0$. This condition facilitates the derivation of theoretical properties such as consistency and convergence of the estimator.

\subsubsection{Theoretical Implications of Assumptions and Conditions}

Under the aforementioned assumptions and conditions, the SCM-based asset pricing model inherits desirable theoretical properties:

\begin{theorem}
\label{thm:consistency_beta}
Under Assumptions 1--3 and Conditions 1--3, the estimator $\hat{\boldsymbol{\beta}}$ obtained from the SCM-based asset pricing model is consistent, i.e.,
\[
\hat{\boldsymbol{\beta}} \xrightarrow{p} \boldsymbol{\beta}, \quad \text{as } N, T \rightarrow \infty.
\]
\end{theorem}

\begin{proof}
The proof follows from standard consistency results in linear regression, augmented by the properties of SCM. Given the linear factor structure (Assumption 1) and exogeneity (Assumption 2), the OLS estimator is unbiased. The convexity and positive weights (Assumption 3) ensure that the synthetic control does not extrapolate beyond the support of the control assets (Condition 1). Full rank (Condition 2) guarantees that the design matrix is invertible, and bounded error terms (Condition 3) ensure that the variance of the estimator diminishes as $N, T \rightarrow \infty$. Therefore, by the Law of Large Numbers and Central Limit Theorem, $\hat{\boldsymbol{\beta}}$ converges in probability to $\boldsymbol{\beta}$.
\end{proof}

\begin{corollary}
\label{cor:asymptotic_normality}
Under the conditions of Theorem \ref{thm:consistency_beta}, the estimator $\hat{\boldsymbol{\beta}}$ is asymptotically normally distributed:
\[
\sqrt{NT} (\hat{\boldsymbol{\beta}} - \boldsymbol{\beta}) \xrightarrow{d} \mathcal{N}(\mathbf{0}, \sigma^2 (\mathbf{X}^\top \mathbf{X})^{-1}),
\]
as $N, T \rightarrow \infty$.
\end{corollary}

\begin{proof}
Following Theorem \ref{thm:consistency_beta}, and under the assumption of i.i.d. error terms with finite variance (Assumption 2), the Central Limit Theorem applies. Therefore, the distribution of the scaled estimator converges to a multivariate normal distribution with mean zero and covariance matrix $\sigma^2 (\mathbf{X}^\top \mathbf{X})^{-1}$.
\end{proof}

\subsubsection{Discussion of Assumptions and Conditions}

The assumptions and conditions outlined above are critical for ensuring the theoretical validity of the SCM-based asset pricing model. Assumption 1 establishes the foundational linear relationship between asset returns and risk factors, aligning with conventional asset pricing theories. Assumption 2 ensures that the risk factors are exogenous, a standard requirement for unbiased and consistent estimation of risk premiums.

Assumption 3 and Condition 1 enforce the feasibility and interpretability of the synthetic control by restricting the weights to be non-negative and convex. Condition 2's full rank requirement prevents multicollinearity, ensuring unique and stable solutions for the weights and the risk premium estimates. Condition 3's boundedness of error terms facilitates the application of asymptotic statistical results, underpinning the consistency and normality of the estimators.

These assumptions collectively enable the derivation of key theoretical properties, such as consistency and asymptotic normality, which are essential for establishing the reliability and validity of the SCM-based approach in asset pricing contexts.

\subsection{Extension to Dynamic Settings}

While the above framework addresses static asset pricing models, financial markets are inherently dynamic. Extending SCM to dynamic asset pricing involves accounting for temporal dependencies and evolving risk factors.

\subsubsection{Dynamic Factor Models}

Consider a dynamic factor model where the risk factors themselves evolve over time:
\[
\mathbf{X}_{it} = \mathbf{A} \mathbf{X}_{i,t-1} + \mathbf{u}_{it},
\]
where $\mathbf{A} \in \mathbb{R}^{K \times K}$ is the transition matrix, and $\mathbf{u}_{it}$ are innovation terms.

Integrating SCM into this dynamic framework requires constructing synthetic controls that adapt to the temporal evolution of the factors, potentially involving time-varying weights $\mathbf{w}_t$.

\subsubsection{Time-Varying Weights and Rolling Windows}

To accommodate dynamics, the weights $\mathbf{w}_t$ can be estimated using rolling windows of data:
\[
\mathbf{w}_t^* = \arg\min_{\mathbf{w}} \sum_{s = t - W}^{t} \left\| \mathbf{X}_{1s} - \mathbf{X}_{\mathcal{C}s} \mathbf{w} \right\|_2^2 + \lambda \|\mathbf{w}\|_2^2,
\]
where $W$ is the window size.

This approach allows the synthetic control to capture time-varying relationships between the treated asset and control assets, enhancing the model's adaptability to changing market conditions.

\subsubsection{Theoretical Properties in Dynamic Settings}

Extending Theorem \ref{thm:consistency_beta} to dynamic settings involves ensuring that the time-varying weights $\mathbf{w}_t^*$ adequately track the evolving factor exposures. Under appropriate conditions on the transition matrix $\mathbf{A}$ and the window size $W$, similar consistency and asymptotic normality results can be established for the dynamic SCM-based estimator.

\begin{theorem}
\label{thm:consistency_dynamic}
Under Assumptions 1--3, Conditions 1--3, and additional regularity conditions on the dynamic factor model, the dynamic SCM-based estimator $\hat{\boldsymbol{\beta}}_t$ satisfies:
\[
\hat{\boldsymbol{\beta}}_t \xrightarrow{p} \boldsymbol{\beta}, \quad \text{as } N, T \rightarrow \infty,
\]
uniformly over time.
\end{theorem}

\begin{proof}
The proof extends Theorem \ref{thm:consistency_beta} by incorporating the temporal dependencies and ensuring that the rolling window estimation consistently tracks the true risk premiums. Under the additional regularity conditions on the transition matrix $\mathbf{A}$ and the window size $W$, the convergence properties hold uniformly over time.
\end{proof}

\begin{corollary}
\label{cor:asymptotic_normality_dynamic}
Under the conditions of Theorem \ref{thm:consistency_dynamic}, the dynamic SCM-based estimator $\hat{\boldsymbol{\beta}}_t$ is asymptotically normally distributed:
\[
\sqrt{NT} (\hat{\boldsymbol{\beta}}_t - \boldsymbol{\beta}) \xrightarrow{d} \mathcal{N}(\mathbf{0}, \sigma^2 (\mathbf{X}_t^\top \mathbf{X}_t)^{-1}),
\]
uniformly over time, as $N, T \rightarrow \infty$.
\end{corollary}

\begin{proof}
Following Theorem \ref{thm:consistency_dynamic}, and under the dynamic assumptions, the Central Limit Theorem applies uniformly over time. The dynamic estimator inherits the asymptotic normality from the static case, ensuring reliable inference in a time-evolving context.
\end{proof}

\subsection{Algorithmic Implementation}

To operationalize the theoretical framework, we outline the algorithmic steps for constructing the SCM-based asset pricing model.

\begin{algorithm}[H]
\caption{SCM-Based Asset Pricing Estimation}
\label{alg:SCM_Ap}
\begin{algorithmic}[1]
\REQUIRE
\begin{itemize}
    \item Dataset of asset returns $\mathbf{R} \in \mathbb{R}^{N \times T}$
    \item Matrix of asset characteristics $\mathbf{X} \in \mathbb{R}^{N \times K}$
    \item Regularization parameter $\lambda$
    \item Window size $W$ (for dynamic settings)
\end{itemize}
\ENSURE Estimated risk premiums $\hat{\boldsymbol{\beta}}$

\FOR{each treated asset $i = 1$}
    \FOR{each time period $t = W+1$ to $T$}
        \STATE Define the rolling window data: $s = t-W$ to $t-1$
        \STATE Extract characteristics $\mathbf{X}_{is}$ and $\mathbf{X}_{\mathcal{C}s}$ for $s = t-W$ to $t-1$
        \STATE Solve the optimization problem:
        \[
        \mathbf{w}_t^* = \arg\min_{\mathbf{w}} \sum_{s = t-W}^{t-1} \left\| \mathbf{X}_{is} - \mathbf{X}_{\mathcal{C}s} \mathbf{w} \right\|_2^2 + \lambda \|\mathbf{w}\|_2^2
        \]
        \STATE Construct the synthetic control return:
        \[
        R_{it}^{\text{SCM}} = \sum_{j \in \mathcal{C}} w_{jt}^* R_{jt}
        \]
        \STATE Estimate the risk premiums $\hat{\boldsymbol{\beta}}$ using:
        \[
        \hat{\boldsymbol{\beta}}_t = \left( \mathbf{X}^\top \mathbf{X} \right)^{-1} \mathbf{X}^\top \mathbf{R}^{\text{SCM}}
        \]
    \ENDFOR
\ENDFOR

\RETURN $\hat{\boldsymbol{\beta}}$
\end{algorithmic}
\end{algorithm}

\subsubsection{Computational Considerations}

Implementing the SCM-based asset pricing model involves solving a quadratic optimization problem for each treated asset and each time period within the rolling window. Efficient numerical optimization techniques, such as quadratic programming solvers, are essential for computational feasibility, especially in high-dimensional settings.

Moreover, regularization parameters (e.g., $\lambda$) must be carefully selected, potentially through cross-validation or information criteria, to balance the trade-off between bias and variance in the estimator.

\subsubsection{Extensions and Generalizations}

The theoretical framework can be extended in various ways to accommodate more complex asset pricing scenarios:

\begin{itemize}
    \item \textbf{Nonlinear Factor Models:} Incorporating nonlinear relationships between asset returns and risk factors, potentially using kernel methods or neural networks within the SCM framework.
    \item \textbf{Time-Varying Risk Premia:} Allowing the risk premiums $\boldsymbol{\beta}_t$ to evolve over time, capturing dynamic shifts in market conditions and investor behavior.
    \item \textbf{High-Dimensional Factor Spaces:} Leveraging dimensionality reduction techniques, such as principal component analysis, to manage high-dimensional characteristic vectors $\mathbf{X}_i$.
\end{itemize}

These extensions can enhance the model's flexibility and applicability to a broader range of asset pricing problems, addressing the complexities inherent in financial markets.

\section*{Summary}

In this section, we have developed a rigorous theoretical framework for integrating the Synthetic Control Method into asset pricing models. We formalized the SCM's mathematical foundations, embedded it within a linear factor model framework, and outlined the necessary assumptions and conditions for the model's validity. Furthermore, we extended the framework to dynamic settings and provided an algorithmic approach for practical implementation. This theoretical groundwork sets the stage for subsequent sections, where we will explore the model's properties, compare it with traditional asset pricing approaches, and discuss its potential applications and implications in financial economics.