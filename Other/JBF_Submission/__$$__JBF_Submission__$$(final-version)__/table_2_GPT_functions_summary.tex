\inserthere{tab:function_calling_structure}

\begin{table}[H]
\centering
\begin{threeparttable}
\caption{Function calling schema}
%{\footnotesize
%\renewcommand{\arraystretch}{1}
\begin{tabular}{lL{4.1cm}L{6.8cm}L{5cm}}
%%%%%%%%%%%%%%%%%%%%%%%%%%%%%%%%%%%%%%%%%%%%%%%%%%%%%
%%%%%%%%%%%%%%%%%%%%%%%%%%%%%%%%%%%%%%%%%%%%%%%%%%%%%
\hline \Xhline{2\arrayrulewidth}
%\rowcolor{gray!10}
\multicolumn{2}{l}{\textbf{Function}} & \textbf{Prompt} & \textbf{Options} \tabularnewline
\hline \Xhline{2\arrayrulewidth} 
\multicolumn{2}{l}{1. \texttt{firms}} & \qquote{List all the firms affected by the events narrated in the article} & \texttt{array} \tabularnewline
\hline
 & 1.1. \texttt{firm} & \qquote{Iterate over each \textnormal{\texttt{firm}} in \textnormal{\texttt{firms}}} & \texttt{string}
 \tabularnewline
\cline{2-4} \cline{3-4} \cline{4-4} 
 & 1.2. \texttt{ticker} & \qquote{State the stock market ticker of \textnormal{\texttt{firm}} } & \texttt{string}
 \tabularnewline
\cline{2-4} \cline{3-4} \cline{4-4} 
 & 1.3. \texttt{shock\_type} & \qquote{What type of shock does this article imply on \textnormal{\texttt{firm}} ?} & \{demand, supply, financial, \newline technology, policy\}\tabularnewline
\cline{2-4} \cline{3-4} \cline{4-4} 
 & 1.4. \texttt{shock\_magnitude} & \qquote{How much impact is this shock expected to have on \textnormal{\texttt{firm}}?} & \{minor, major\}\tabularnewline
\cline{2-4} \cline{3-4} \cline{4-4} 
 & 1.5. \texttt{shock\_direction} & \qquote{In what direction is this shock expected to impact \textnormal{\texttt{firm}}?} & \{positive, negative\}\tabularnewline 
%\cline{2-4} \cline{3-4} \cline{4-4} 
\hline \Xhline{2\arrayrulewidth}
\end{tabular}
%}
%\begin{tablenotes}
%\footnotesize
%\mx
%\item \textit{Note: 
%For clarity of exposition, the actual prompts passed to LlaMA are avoided here but can be found in the code. 
%The ``Options'' column imposes the asnwer format that the LLM must follow. For example, in \texttt{firms}, the option \texttt{array} indicates that the answer must be an enumeration of firms, while the option \texttt{string} in the subfunctions \texttt{firm} and \texttt{ticker} indicates that the answer must be a single name. Finally, the \texttt{shock\_} subfunctions ask the LLM to choose from a predefined set of options.
%}
%\end{tablenotes}
\label{tab:function_calling_structure}
\end{threeparttable}
\mx 
\subcaption*{\textit{
This table outlines the structure of the function calling schema we designed to guide the LLM through the analysis of news-implied firm-specific economic shocks. The \qquote{Function} column specifices the name of the tool passed to the LLM. We can understand the umbrella function \texttt{firms} as running a loop over each of its arguments, with the indented subfunctions being referred to the specific argument passed to them. The \qquote{Prompt} column provides an example of the simplified instructions given to the LLM (the actual prompts are longer as the LLM needs clear and detailed instructions, with useful examples for context).  Finally, the ``Options'' column imposes the answer format that the LLM must follow. For example, in \texttt{firms}, the ``\texttt{array}'' option indicates that the answer must be an enumeration of firms, while the  ``\texttt{string}'' option in the subfunctions \texttt{firm} and \texttt{ticker} indicates that the answer must be a single string. Finally, the \texttt{shock\_} subfunctions ask the LLM to choose from a predefined set of possible responses.
}}
\end{table}



%%%%%%%%%%%%%%%%%%%%%%%%%%%%%%%%%%%%%%%%%%%%%%%%%%%%%
%%%%%%%%%%%%%%%%%%%%%%%%%%%%%%%%%%%%%%%%%%%%%%%%%%%%%

%\begin{table}[H]
%\centering
%\begin{threeparttable}
%\caption{Function calling schema}
%{\small
%\begin{tabular}{l||L{4.1cm}|L{6.8cm}|L{5cm}|}
%%%%%%%%%%%%%%%%%%%%%%%%%%%%%%%%%%%%%%%%%%%%%%%%%%%%%%
%%%%%%%%%%%%%%%%%%%%%%%%%%%%%%%%%%%%%%%%%%%%%%%%%%%%%%
%\hline 
%\multicolumn{2}{|l|}{Function} & Description & Options \tabularnewline
%\hline 
%\hline 
%%\multicolumn{2}{|l|}{1. \texttt{publication\_time}} & Date and time of publication & ``''\tabularnewline
%%\hline 
%%\multicolumn{2}{|l|}{2. \texttt{scope}} & Scope or focus of the news article's impact. & \{Firm, Industry, Economy\}\tabularnewline
%%\hline 
%%\multicolumn{2}{|l|}{3. \texttt{news\_category}} & Type of information provided in the article & \{New, Historical, \newline Analysis/Comments\}\tabularnewline
%%\hline 
%\multicolumn{2}{|l|}{1. \texttt{firms}} & List all the firms affected by the events narrated in the article & \texttt{array} \tabularnewline
%\hline
% & 1.1. \texttt{firm} & Iterate over each firm in \texttt{firms} & \texttt{string}
% \tabularnewline
%\cline{2-4} \cline{3-4} \cline{4-4} 
% & 1.2. \texttt{ticker} & State the stock market ticker of this firm & \texttt{string}
% \tabularnewline
%\cline{2-4} \cline{3-4} \cline{4-4} 
% & 1.3. \texttt{shock\_type} & What type of shock does the article imply on this firm? & \{demand, supply, financial, \newline technology, policy\}\tabularnewline
%\cline{2-4} \cline{3-4} \cline{4-4} 
%% & 4.4. \texttt{shock\_duration} & Expected duration of the shock on this firm & \{Short term, Mid term, Long term\}\tabularnewline
%\cline{2-4} \cline{3-4} \cline{4-4} 
% & 4.5. \texttt{shock\_magnitude} & How much imapct is the shock expected to have on this firm? & \{minor, major\}\tabularnewline
%\cline{2-4} \cline{3-4} \cline{4-4} 
% & 4.6. \texttt{shock\_direction} & In what direction is the shock expected to impact this firm? & \{positive, negative\}\tabularnewline 
%\cline{2-4} \cline{3-4} \cline{4-4} 
%% & 4.6. \texttt{trading\_signal} & Trading decision on this firm's stock & \{Long, Not trade, Short\}\tabularnewline
%%\cline{2-4} \cline{3-4} \cline{4-4} 
%% & 4.7. \texttt{market\_timing} & Time of incorporation of the shock on this firm's stock price (\textit{today=publication date}). & \{Before last week, Last week, \newline Yesterday, Today, Tomorrow, \newline Next week, After next week\}\tabularnewline
%%\cline{2-4} \cline{3-4} \cline{4-4} 
%%%%%%%%%%%%%%%%%%%%%%%%%%%%%%%%%%%%%%%%%%%%%%%%%%%%%%
%%%%%%%%%%%%%%%%%%%%%%%%%%%%%%%%%%%%%%%%%%%%%%%%%%%%%%
%\end{tabular}
%}
%\begin{tablenotes}
%\footnotesize
%\mx
%%\item \textit{Note}: A justification of the category choice is asked for all the parameters with an asterisk . 
%\item \textit{Note: For clarity of exposition, the actual prompts passed to Llama are avoided here but can be found in the Appendix. The ``Options'' column imposes the asnwer format that the LLM must follow. For example, in \texttt{firms}, the option \texttt{array} indicates that the answer must be an enumeration of firms}, while the option \texttt{string} in the subfunctions \texttt{firm} and \texttt{ticker} indicates that the answer must be a single name. Finally, the shock subfunctions ask the LLM to choose an option from a predefined set of options.
%\end{tablenotes}
%\end{threeparttable}
%\end{table}