%----------------------------------------------------
\inserthere{fig:pairs_decay}
\begin{figure}[H]
	\centering
	\caption{Decay of pairs trading excess returns}
	\includegraphics[scale=0.5]{/Users/jesusvillotamiranda/Library/CloudStorage/OneDrive-UniversidaddeLaRioja/GitHub/Repository/pairs_trading_sparse_synthetic_replica/__OUTPUT__/figures/comparison.pdf}
  \label{fig:pairs_decay}
  \subcaption*{\textit{Note:} 40-month moving average of monthly excess returns of traditional pairs trading versus generalized pairs trading. As we can see, restricting the coefficient of the replicating portfolio to the canonical basis added economic until 1990, there on, as 1-to-1 spreads became more heavily traded, the profitability of this type of spread trading decreaased substantially. Hoewever, the generalized approach that allows more flexibility in the coefficient vector of the replicating portfolio maintains profitability even in the last decade, where markets have become heavily efficient (note that excess returns to traditional pairs trading are 0 in the last decade).
  }
\end{figure}
%----------------------------------------------------

\section{Empirical Application}


\paragraph{Data}
\begin{itemize}
	\item We work with CRSP daily files from January 1st 1962 to December 31rd 2024. 
	\item Exchanges NYSE, AMEX, and NASDAQ (EXCHCD 1, 2, and 3)
	\item Share codes 10 and 11
\end{itemize}



Our empirical design retains the two-stage rolling-window approach of \cite{Gatev2006}. For every month in the sample we first estimate the strategy over a 12-month \emph{formation window} and then trade the resulting spreads during the subsequent 6-month \emph{trading window}.

\paragraph{Formation period}
During each formation period, we define our universe of stocks by filtering out from CRSP any security that exhibits at least one non-trading day in that period. 

Then, for every stock in this universe we construct its associated cumulative-return index, which that tracks the value of one dollar invested in that stock at the start of the window and continuously reinvests cash dividends.

For each stock in our selected universe, we treat it as a potential target and seek to construct its sparse synthetic replica. As outlined previously, this is achieved by regressing the target's normalized price series against the series of all other liquid stocks in the donor pool, using a LASSO penalty to enforce sparsity. 

For our main analysis, we select a penalization parameter of $\lambda=0.001$, which effectively shrinks the coefficients of irrelevant assets to zero and yields a parsimonious replicating portfolio, typically composed of around 20 constituent securities. The spread is the time series of residuals from this regression--the difference between the target stock's price and the price of its synthetic replica. Following the formation period, we carry forward the top 20 target-replica strategies with the lowest sum of squared residuals for trading.

\paragraph{Trading period}

The trading phase commences on the first day following the formation period. Our trading rules are designed to capitalize on the expected mean reversion of the spread. We initiate a dollar-neutral position when a target stock's price diverges significantly from that of its synthetic replica. Specifically, a position is opened when the spread widens beyond two of its historical standard deviations, as estimated during the 12-month formation period. This involves shorting one dollar of the relatively overvalued asset and simultaneously investing one dollar in the undervalued asset. The position is held until the spread reverts and crosses its mean (i.e., the prices cross), at which point it is closed. If a position is still open at the end of the 6-month trading window, it is liquidated at the prevailing market prices. Similarly, if a constituent stock is delisted, the position is closed out using the delisting return or the last available price. Positions may open, close, and reopen multiple times within the six-month window.


\paragraph{Payoffs}
The payoffs from this strategy are equivalent to excess returns, as they are derived from self-financing, dollar-neutral positions. The total excess return for each target-replica strategy over the trading period is the sum of its reinvested payoffs. Positions are marked-to-market daily, and the daily returns on the long and short portfolios are compounded to calculate monthly returns, which mirrors a buy-and-hold approach for the underlying assets.

The realized payoffs from this strategy are characterized by a sequence of cash flows: positive flows accrue each time a spread opens and subsequently converges within the trading window, while unresolved spreads generate a final cash flow at the end of the period. If no divergence occurs, no trade is initiated. Returns are computed as excess returns, with all positions marked to market daily and returns compounded to the monthly level. Specifically, the daily return on a portfolio $P$ is calculated as
%----------------------------------------------------
$$
r_{P, t} = \frac{\sum_{i \in P} w_{i, t} r_{i, t}}{\sum_{i \in P} w_{i, t}}
$$
%----------------------------------------------------
$$
w_{i, t} = w_{i, t-1}(1 + r_{i, t-1}) = \prod_{s=1}^{t-1} (1 + r_{i, s})
$$
%----------------------------------------------------
where $r_{i, t}$ denotes the return and $w_{i, t}$ the evolving weight of asset $i$ at time $t$. This approach is equivalent to a buy-and-hold strategy, with daily returns compounded to obtain monthly performance.

\paragraph{Excess-return metrics}
Following \cite{Gatev2006}, we report two measures of excess return: the return on committed capital (which assumes a dollar is allocated to every pair selected for trading, regardless of whether a trade is triggered). That is, it divides aggregate pay-offs by the number of spreads selected for trading, charging the strategy for capital tied up even if a spread never opens.The fully invested return scales payoffs by the actual capital deployed in open trades. The former is a conservative metric, reflecting the opportunity cost of capital commitment, while the latter provides a more realistic gauge of trading profitability for flexible investors, reflecting the viewpoint of a flexible trading desk that is able to redeploy idle funds

\paragraph{Overlapping portfolios}
To generate a continuous time series of returns, we initiate this entire process at the beginning of every month in our sample (excluding the initial 12 months required for the first formation period). This creates a series of overlapping 6-month trading periods. To correct for the serial correlation induced by this overlap, we follow the standard procedure of averaging the monthly returns across all strategies that are concurrently active, as in Jegadeesh (1993). The resulting series can be interpreted as the net payoff to a trading desk managing six distinct portfolios, with each portfolio staggered by one month.

\input{
/Users/jesusvillotamiranda/Library/CloudStorage/OneDrive-UniversidaddeLaRioja/GitHub/Repository/pairs_trading_sparse_synthetic_replica/__OUTPUT__/tables/Table1.tex
}


\input{/Users/jesusvillotamiranda/Library/CloudStorage/OneDrive-UniversidaddeLaRioja/GitHub/Repository/pairs_trading_sparse_synthetic_replica/__OUTPUT__/tables/Table4_modified.tex}

%On the first trading day after the formation window closes, each selected spread is monitored for six months ($\approx$126 trading days).  A position is opened whenever the residual $\varepsilon_{i,t}$ deviates from zero by more than two of its own historical standard deviations, $\sigma_{i}^{\mathcal{F}}$.  The trade is closed the moment the residual crosses zero; if convergence fails to occur before the window ends--or if either leg delists--the position is liquidated at the prevailing prices.  Positions may open, close, and reopen multiple times within the six-month window.

%For every remaining stock we build a ``normalized price index`` as a cumulative-return series that reinvests dividends and is normalised to \$1 on the first day of the window

%
%
%Our empirical strategy closely mirrors the approach of \cite{Gatev2006}, with adaptations to accommodate the generalized, sparse synthetic replica framework described above. The implementation unfolds in two main phases: a 12-month formation window, during which replicating portfolios are constructed, followed by a 6-month trading window in which the strategy is executed.
%
%
%Our empirical implementation follows the foundational framework established by \cite{Gatev2006}, adapted to accommodate our sparse synthetic replica methodology. The trading strategy operates through two distinct phases: a 12-month formation period during which we construct replicating portfolios, followed by a 6-month trading period where we execute the actual trades based on spread deviations.

%----------------------------------------------------
%The procedure begins each month. First, we source our universe of securities from the CRSP daily files. To ensure we are working with sufficiently liquid assets and to facilitate the construction of replicating portfolios, we filter out any stock that has experienced one or more non-trading days during the formation period. For all remaining stocks, we then construct a cumulative total return index, normalized to a starting value of one dollar. This creates the normalized "price" series used for analysis, which inherently includes reinvested dividends.

%Following industry convention, we exclude any stock that records a zero trading volume on at least one day of the 12-month formation window, thereby restricting attention to reasonably liquid names and avoiding spurious price staleness.  For every remaining stock we construct a cumulative-return index, $p_{i,t}$, that tracks the value of one dollar invested at the start of the window and continuously reinvests cash dividends.

%1. Data and pre-processing  
%   -- Common shares (CRSP share codes 10 and 11) are drawn from the CRSP daily files.  
%   -- Within each formation window we drop any stock that records a zero trading volume on at least one day, thereby guaranteeing uninterrupted price series and a minimum degree of liquidity.  
%   -- For every remaining stock we build a cumulative-return index that reinvests dividends and is normalised to \$1 on the first day of the window.

%To ensure liquidity and facilitate robust portfolio construction, we begin each formation period by filtering the CRSP daily stock universe, excluding any security that exhibits at least one non-trading day. For each eligible stock, we compute a cumulative total return index over the formation window, which serves as the basis for subsequent analysis. The replicating portfolio for each target stock is then identified by solving a penalized regression problem: we seek the linear combination of donor stock price series that most closely tracks the normalized price path of the target, subject to a lasso penalty ($\lambda=0.001$) that encourages sparsity. This typically results in replicating portfolios composed of approximately 20 stocks, balancing fidelity of replication with practical considerations around trading costs.





%The core innovation lies in our approach to replica construction. Rather than seeking a single matching partner for each target stock, we employ LASSO regularization with a penalization parameter of $\lambda$ = 0.001. This parameter choice deliberately promotes sparsity, typically yielding replicating portfolios composed of approximately 20 constituent stocks--sufficient for robust replication while remaining practically tradable. The LASSO optimization simultaneously performs variable selection and weight estimation, identifying the most economically relevant securities from the donor pool while determining their optimal portfolio weights.



%----------------------------------------------------


%
%3. Trade initiation and liquidation  
%   Starting on the first day after the formation window, we monitor each chosen spread.  
%   -- A position is opened when \(|s_{i,t}|\) exceeds two historical standard deviations (calculated inside the formation window).  
%   -- We short the overpriced leg (positive spread) for \$1 and go long the under-priced leg for \$1.  
%   -- The trade is closed at the first subsequent zero crossing of the spread, at delisting (using the CRSP delisting return), or on the final day of the trading window--whichever comes first.
   

%----------------------------------------------------
%Position Management Rules
%
%Our trading rules adapt the mean-reversion principle to the synthetic replica context. We initiate positions when the spread (regression residual) exceeds two historical standard deviations, as estimated during the formation period. Specifically, when the target security trades above its synthetic replica by more than this threshold, we establish a short position in the target and corresponding long positions in the replica constituents. Conversely, when the target trades below its synthetic replica beyond the threshold, we reverse these positions.
%
%Position unwinding occurs upon spread convergence--defined as the next crossing of the target and replica prices. If convergence fails to materialize before the trading period expires, we close all positions at prevailing market prices on the final trading day. In cases where constituent securities are delisted during active trades, we immediately close the affected positions using available delisting returns or final quoted prices.
%

%----------------------------------------------------


%Following \cite{Do2010}, we focus our analysis on the top 20 pairs, ranked by the quality of their replication during the formation period (measured by regression fit). The trading phase commences immediately after the formation period concludes, implementing a systematic approach to spread-based position entry and exit.
%
%Consistent with the focus of \cite{Do2010}, we concentrate our analysis on the top 20 target-replica pairs, as ranked by historical tracking error during the formation period. Trading commences on the first day following the end of the formation window, governed by a simple mean-reversion rule: a long-short position is initiated whenever the spread between the target and its replica widens beyond two standard deviations of its historical value (as estimated during the formation period). Positions are closed at the next crossing of the spread, or, if no crossing occurs, at the conclusion of the trading window. In the event of a delisting, the position is unwound using the delisting return or the last available price. All trades are executed in dollar-neutral fashion, with one dollar allocated to each leg of the trade.

%----------------------------------------------------

%4. Mark-to-market accounting  
%   Let \(r_{k,t}\) denote the daily return on leg \(k\) (long or short).  Portfolio weights evolve as  
%   \[
%     w_{k,t}=w_{k,t-1}\,(1+r_{k,t-1}),
%   \qquad
%     r_{P,t}= \frac{\sum_k w_{k,t}\,r_{k,t}}{\sum_k w_{k,t}},
%   \]
%   so pay-offs compound exactly as in a buy-and-hold position that is re-balanced only when trades open or close.
%
%
%

%Return Calculation Framework
%
%We employ a dollar-neutral approach, investing equal dollar amounts long and short to maintain market neutrality. The daily returns for both long and short portfolio components are calculated as value-weighted averages of constituent returns:
%
%$$r_{P, t}=\frac{\sum_{i \in P} w_{i, t} r_{i, t}}{\sum_{i \in P} w_{i, t}}$$
%
%where weights evolve dynamically according to:
%
%$$w_{i, t}=w_{i, t-1}(1+r_{i, t-1})$$
%
%This formulation represents a buy-and-hold approach within each position, with daily returns compounded to generate monthly performance metrics.

%----------------------------------------------------
%Following \cite{Gatev2006}, we assess performance using two distinct measures of excess return. The first, "return on committed capital," is a conservative metric that scales the total payoffs by the number of strategies selected for trading (i.e., 20), regardless of whether a trade was triggered. This accounts for the opportunity cost of capital that must be allocated to a strategy. The second, "return on fully invested capital," divides the payoffs only by the number of strategies that actually opened positions. This arguably provides a more direct measure of profitability on capital that was actively employed.
%
%Following \cite{Gatev2006}, we calculate two distinct return measures to capture different aspects of capital efficiency. The "committed capital" return scales portfolio payoffs by the total number of selected pairs, regardless of whether they generate trading signals during the period. This conservative measure accounts for the opportunity cost of capital allocation to strategies that may remain inactive. Alternatively, the "fully invested" return divides payoffs only by pairs that actually trade, providing insight into the strategy's performance when capital is actively deployed.

%\subparagraph{6. Excess-return metrics}  
%In line with \cite{Gatev2006} we report:  
%(i) \textit{Return on committed capital}, obtained by dividing aggregate pay-offs by the full set of 20 candidate spreads, and  
%(ii) \textit{Return on employed capital}, which scales gains and losses by the number of spreads that actually opened during the month.  The first measure is more conservative as it recognises the opportunity cost of idle capital; the second reflects the viewpoint of a flexible trading desk that reallocates unused funds elsewhere.
%
%
%   Following \cite{Gatev2006} we report two excess-return metrics.  
%   -- \emph{Return on committed capital} divides aggregate pay-offs by the number of spreads selected for trading, charging the strategy for capital tied up even if a spread never opens.  
%   -- \emph{Return on employed capital} scales by the number of spreads that actually open, better reflecting a manager able to redeploy idle funds.
%   
%   reflects the viewpoint of a flexible trading desk that is able to redeploy idle funds
%
%Following \cite{Gatev2006}, we report two measures of excess return: the return on committed capital (which assumes a dollar is allocated to every pair selected for trading, regardless of whether a trade is triggered) and the fully invested return (which scales payoffs by the actual capital deployed in open trades). The former is a conservative metric, reflecting the opportunity cost of capital commitment, while the latter provides a more realistic gauge of trading profitability for flexible investors.


%%----------------------------------------------------
%To generate a continuous time series of returns, we initiate this entire process at the beginning of every month in our sample (excluding the initial 12 months required for the first formation period). This creates a series of overlapping 6-month trading periods. To correct for the serial correlation induced by this overlap, we follow the standard procedure of averaging the monthly returns across all strategies that are concurrently active, as in \cite{Jegadeesh1993}. The resulting series can be interpreted as the net payoff to a trading desk managing six distinct portfolios, with each portfolio staggered by one month.
%
%
%To generate a continuous time series of strategy returns, we initiate a new set of trades at the start of each month, except for the initial 12 months required for formation. This rolling procedure produces overlapping six-month trading windows, and we address the resulting serial correlation by averaging monthly returns across strategies initiated in different months, following the methodology of Jegadeesh and Titman (1993). The resulting return series can be interpreted as the performance of a proprietary trading desk managing multiple staggered portfolios in parallel, each overseen by a different trader.
%
%
%\subparagraph{7. Overlapping portfolios}  
%To generate a continuous excess-return series we launch a new 12/6 formation-trading cycle at the beginning of every calendar month (after an initial 12-month burn-in).  The procedure yields six staggered portfolios whose returns are averaged each month, following the overlap-adjustment of \cite{Jegadeesh1993}.  The resulting time series represents the performance of a multi-manager desk that rolls capital smoothly across successive cohorts.
%

%We implement a rolling strategy structure, initiating new formation periods monthly throughout our sample. This approach generates overlapping six-month trading periods, creating a time series of returns that we adjust for induced correlation following the methodology of Jegadeesh and Titman (1993). The resulting performance series reflects the returns available to a practical trading desk operating multiple staggered strategies simultaneously--a realistic representation of institutional implementation.

%   A new set of synthetic replicas is formed at the start of every month once the first year of data has elapsed, producing six overlapping trading portfolios at any point in time.  Monthly excess returns are obtained by equally weighting these six staggered portfolios, the overlap adjustment of \cite{Jegadeesh1993}.  The resulting time series can be viewed as the consolidated P\&L of a proprietary desk that assigns each of the six books to a different trader, one month apart in start dates.



%%%%%%%%%%%%%%%%%%%%%%%%%%%%%%%%%%%%%%%%%%%%%%%%%%%%%
%%%%%%%%%%%%%%%%%%%%%%%%%%%%%%%%%%%%%%%%%%%%%%%%%%%%%
%%%%%%%%%%%%%%%%%%%  PROMPT   %%%%%%%%%%%%%%%%%%%%%%%
%%%%%%%%%%%%%%%%%%%%%%%%%%%%%%%%%%%%%%%%%%%%%%%%%%%%%
%%%%%%%%%%%%%%%%%%%%%%%%%%%%%%%%%%%%%%%%%%%%%%%%%%%%%

%The text below "\paragraph{Empirical Application}" presents the details about the empirical application of the strategy. 

%Note that text enclosed in \bblue{} are literally extracted from the Gatev et al paper, so using it literally constitutes plagiarism: instead, either you should aim to rephrase the ideas inferred from the \bblue{} text and rephrase them / apply them in a differnt style, or you could cite them literally between quotes.

% Also, replace mentions to references by citations: 
% Gatev et al: \cite{Gatev2006}
% Do and Faff: \cite{Do2010}

% Comments made in the text below "\paragraph{Empirical Application}" is to be ignored.

% [REWRITE ONLY THE SECTION BELOW] %
%----------------------------------------------------

%\paragraph{Empirical Application}
%
%The methodology of this paper follows \cite{Gatev2006}. \bblue{We implement pairs trading in two stages. We form pairs over a 12-month period (formation period) and trade them in the next 6-month period (trading period).}
%
%\bblue{In each pairs formation period, we screen out all stocks from the CRSP daily files that have one or more days with no trade. This serves to identify relatively liquid stocks as well as to facilitate pairs formation. Next, we construct a cumulative total returns index for each stock over the formation period.} We choose a lasso penalization parameter of $\lambda=0.001$ to promote some sparsity (the number of constituents of the replicating portfolio is around 20 stocks). 
%% Here we could add a table where we compare the lasso parameter against its implied average cardinality on the set of donor stocks
%
%
%We then find the replicating portfolio for each target stock by finding the linear combination of security prices that minimizes the sum of squared deviations between the two normalized price series. \bblue{Pairs are thus formed by exhaustive matching in in normalized daily ``price'' space, where price includes reinvested dividends.}
%
%%\bblue{Once we have paired up all liquid stocks in the formation period, we study the top 5 and 20 pairs with the smallest historical distance measure, in addition to the 20 pairs after the top 100 (pairs 101-120).}
%
%As in \cite{Do2010} we focus on the top 20 pairs. 
%\bblue{On the day following the last day of the pairs formation period, we begin to trade according to a prespecified rule.}
%\bblue{We select trading rules based on the proposition that we open a long-
%short position when the pair prices have diverged by a certain amount and
%close the position when the prices have reverted}
%\bblue{webase our rules for opening and closing positions on a standard deviation metric. We open a position in a pair when prices diverge by more than two historical standard deviations, as estimated during the pairs formation period. We unwind the position at the next crossing of the prices. If prices do not cross before the end of the trading interval, gains or losses are calculated at the end of the last trading day of the trading interval. If a stock in a pair is delisted from CRSP, we close the position in that pair, using the delisting return, or the last available price}
%\bblue{We report the payoffs by going one dollar short in the higher-priced stock and one dollar long in the lower-priced stock}
%
%\bblue{Pairs that open and converge during the trading
%interval will have positive cash flows. Because pairs can reopen after initial convergence, they can have multiple positive cash flows during the trading interval. Pairs that open but do not converge will only have cash flows on the last day of the trading interval when all positions are closed out. Therefore, the payoffs to pairs trading strategies are a set of positive cash flows that are randomly distributed throughout the trading period, and a set of cash flows at the end of the trading interval that can be either positive or negative. For each pair we can have multiple cash flows during the trading interval, or we may have none in the case when prices never diverge by more than two standard deviations during the trading interval. Because the trading gains and losses are computed over long-short positions of one dollar, the payoffs have the interpretation of
%excess returns. The excess return on a pair during a trading interval is computed as the reinvested payoffs during the trading interval.5 In particular, the long and short portfolio positions are marked-to-market daily. The daily returns on the long and short positions are calculated as value-weighted returns in the following way:}
%
%\bblue{
%$$
%r_{P, t}=\frac{\sum_{i \in P} w_{i, t} r_{i, t}}{\sum_{i \in \mathrm{P} w_{i, t}}} 
%$$
%$$
%w_{i, t}=w_{i, t-1}\left(1+r_{i, t-1}\right)=\left(1+r_{i, 1}\right) \cdots\left(1+r_{i, t-1}\right)
%$$
%}
%
%\bblue{where $r$ defines returns and $w$ defines weights, and the daily returns are compounded to obtain monthly returns. This has the simple interpretation of a buy-and-hold strategy.}
%
%As in \cite{Gatev2006}, 
%\bblue{we consider two measures of excess return on a portfolio of pairs: the return on committed capital and the fully invested return, that is, the return on actual employed capital. The former scales the portfolio payoffs by the number of pairs that are selected for trading, the latter divides the payoffs by the number of pairs that open during the trading period. The
%former measure of excess return is clearly more conservative: if a pair does not trade for the whole of the trading period, we still include a dollar of committed capital as the cumulative return in our calculation of excess
%return. It takes into account the opportunity cost of hedge funds of having to commit capital to a strategy even if the strategy does not trade. To the extent that hedge funds are flexible in their sources and uses of funds, computing excess return relative to the actual capital employed may give a more realistic measure of the trading profits.}
%
%\bblue{We initiate the pairs strategy by trading the pairs at the beginning of every month in the sample period, with the exception of the first 12 months, which are needed to estimate pairs for the strategy starting in the first month. The result is a time series of overlapping six-month
%trading period excess returns. We correct for the correlation induced by overlap by averaging monthly returns across trading strategies that start one month apart as in Jegadeesh and Titman (1993). The resulting time series has the interpretation of the payoffs to a proprietary trading
%desk, which delegates the management of the six portfolios to six different traders whose formation and trading periods are staggered by one month.}
%


