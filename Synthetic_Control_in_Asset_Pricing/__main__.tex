\documentclass[12pt,a4paper]{article}
\usepackage{/Users/jesusvillotamiranda/Documents/LaTeX/JVM_Macros}
\usepackage[nomarkers, nolists, tablesfirst]{endfloat} % puts tables and figures at the end of the document
\usepackage{authblk} % to write the title with the authors
\usepackage{booktabs} % for better tables
\usepackage{datetime}
\newdateformat{mydate}{\ordinal{DAY} \monthname[\THEMONTH] \THEYEAR}


%----------------------------------------------------
\title{ 
%\textsc{
Synthetic Control Method in Asset Pricing
%}
}

\author[1]{
{ 
Jesus Villota Miranda$^{\dagger}$
%\footnote{
%\scriptsize{
%I am deeply grateful to 
%I gratefully acknowledge financial support from ...
%}}
}

\bx 
{\small
$\big{<}$
\noindent $^{\dagger}$CEMFI, Calle Casado del Alisal, 5, 28014 Madrid, Spain 
$\big{>}$

$\big{<}$
Email: \href{mailto:jesus.villota@cemfi.edu.es}{\texttt{jesus.villota@cemfi.edu.es}}
$\big{>}$

\textbf{This version: \mydate\today}
}
}
\date{}
%----------------------------------------------------

%%%%%%%%%%%%%%%%%%%%%%%%%%%%%%%%%%%%%%%%%%%%%%%%%%%%%
%%%%%%%%%%%%%%%%%%%%%%%%%%%%%%%%%%%%%%%%%%%%%%%%%%%%%
\begin{document}
\maketitle
\thispagestyle{empty}
%----------------------------------------------------
\begin{abstract}
This paper develops a novel framework for applying the Synthetic Control Method (SCM) to asset pricing and portfolio management. We extend the traditional SCM methodology by incorporating financial market features, including short-selling capabilities and regularization techniques to manage transaction costs and portfolio concentration. The framework is further enhanced through machine learning approaches that capture nonlinear relationships between financial instruments. We demonstrate the method's versatility through various applications, including statistical arbitrage, hedging of complex securities, ETF replication, and risk analysis. Our methodology provides practitioners with a systematic approach to construct synthetic portfolios that closely track target instruments while maintaining implementation feasibility through controlled trading costs and portfolio turnover. 
%The empirical results suggest that our enhanced SCM framework outperforms traditional linear methods, particularly in markets characterized by complex, nonlinear relationships.

\vspace{1cm}

\noindent\textbf{JEL Codes:} C14, C45, G11, G12, G13

\noindent\textbf{Keywords:} Synthetic Control Method; Machine Learning; Asset Pricing; Portfolio Management; Statistical Arbitrage; Risk Management; ETF Replication; Nonlinear Methods
\end{abstract}
%----------------------------------------------------

\newpage
\setcounter{page}{1}

%%%%%%%%%%%%% INTRODUCTION %%%%%%%%%%%%%%%%%%
%%----------------------------------------------------
\section{Introduction}
%----------------------------------------------------
\hspace{0.5cm} In financial markets, news play a pivotal role in shaping stock prices. Every day, market participants respond to a broad spectrum of news ranging from firm-specific announcements, such as earnings releases, to macroeconomic events, such as central bank interest rate announcements, or geopolitical developments, like international trade conflicts or political elections. The Efficient Market Hypothesis (EMH), introduced by \cite{fama1970efficient}, posits that markets efficiently incorporate new information almost instantaneously. Both theoretical perspectives and empirical observations indicate that markets do not always exhibit such efficiency, particularly when the information is complex or ambiguous. This discrepancy between theory and reality suggests significant room for improvement in understanding how news is processed by market participants and how it influences asset prices.

A substantial body of literature has endeavored to predict market reactions to news, yet significant gaps persist. 
%----------------------------------------------------
First, there is a lack of granularity in the analysis of information. Traditional approaches frequently rely on sentiment analysis, reducing the richness of news content to binary classifications of positive or negative sentiment. Despite their reductionist nature, sentiment analysis remains popular due to the ease of implementation and interpretability
%%%%%%%%%%%%%%%%%%%%%%%%%%%%%%%%%%%%%%%%%%%%%%%%%%%%%
%%%%%%%%%%%%%%%%%%%%%%%%%%%%%%%%%%%%%%%%%%%%%%%%%%%%%
(\cite{tetlock2007giving}, \cite{tetlock2008more}, \cite{bollen2011twitter}, \cite{hanley2010information}, \cite{loughran2011liability}, \cite{garcia2013sentiment}, \cite{jegadeesh2013word},\cite{wei2018stock}, \cite{ke2019predicting}, \cite{lee2020bert}, 
\cite{lopez2023can}).
%%%%%%%%%%%%%%%%%%%%%%%%%%%%%%%%%%%%%%%%%%%%%%%%%%%%%
%%%%%%%%%%%%%%%%%%%%%%%%%%%%%%%%%%%%%%%%%%%%%%%%%%%%%
Sentiment analysis often misses the intricacy inherent in news and is based on linguistic patterns, rather than on economically relevant considerations.
%--such as the interplay between multiple factors--
% or subtle shifts in tone. 
 Other studies have sought to enhance this granularity through topic modeling, which categorizes text into broad themes 
%%%%%%%%%%%%%%%%%%%%%%%%%%%%%%%%%%%%%%%%%%%%%%%%%%%%%
%%%%%%%%%%%%%%%%%%%%%%%%%%%%%%%%%%%%%%%%%%%%%%%%%%%%%
(\cite{antweiler2006us}, \cite{hansen2018transparency}, \cite{bybee2021business}, \cite{bybee2023narrative}).
%%%%%%%%%%%%%%%%%%%%%%%%%%%%%%%%%%%%%%%%%%%%%%%%%%%%%
%%%%%%%%%%%%%%%%%%%%%%%%%%%%%%%%%%%%%%%%%%%%%%%%%%%%%
However, these models are limited in adapting to new and evolving information and lack the specificity needed to assess the precise impact of news on individual firms or sectors. Topic models can identify broad themes, but they struggle to capture the changing context of financial news, particularly when new narratives emerge, such as unexpected geopolitical events or technological disruptions. 
Concurrently, other branches of literature experimented with vector-based models 
%%%%%%%%%%%%%%%%%%%%%%%%%%%%%%%%%%%%%%%%%%%%%%%%%%%%%
%%%%%%%%%%%%%%%%%%%%%%%%%%%%%%%%%%%%%%%%%%%%%%%%%%%%%
(\cite{hoberg2016text}, \cite{chen2021stock}, \cite{jha2022does}, \cite{benincasa2022different}, \cite{zhang2023feel}, \cite{gabaix2023asset}).
%%%%%%%%%%%%%%%%%%%%%%%%%%%%%%%%%%%%%%%%%%%%%%%%%%%%%
%%%%%%%%%%%%%%%%%%%%%%%%%%%%%%%%%%%%%%%%%%%%%%%%%%%%%
Traditional approaches like Word2Vec and GloVe, which map words to continuous vector spaces based on their co-occurrence patterns, revolutionized natural language processing by enabling mathematical operations on words and capturing basic semantic relationships. The advent of transformer architectures marked a significant advancement, leading to more sophisticated models such as \texttt{BERT}, \texttt{RoBERTa} or \texttt{GPT}. These transformer-based models process text through multiple attention layers, allowing them to generate context-aware embeddings by considering the relationships between all words in a sequence simultaneously. However, even when fine-tuned with domain-specific training data (e.g., \texttt{FinBERT}), these methods cannot inherently incorporate economic structure, limiting their ability to comprehend the economic implications of news articles.

%----------------------------------------------------
Second, there is an insufficient focus on firm-specific analysis in the existing literature. Many studies examine the impact of news on broader market indices, such as the S\&P500 or DJIA, rather than on individual firms. While research by 
%%%%%%%%%%%%%%%%%%%%%%%%%%%%%%%%%%%%%%%%%%%%%%%%%%%%%
%%%%%%%%%%%%%%%%%%%%%%%%%%%%%%%%%%%%%%%%%%%%%%%%%%%%%
\cite{cutler1988moves}, \cite{mitchell1994impact}, \cite{bollen2011twitter}, \cite{garcia2013sentiment}, \cite{baker2016measuring}, \cite{manela2017news}, \cite{baker2021triggers} 
%%%%%%%%%%%%%%%%%%%%%%%%%%%%%%%%%%%%%%%%%%%%%%%%%%%%%
%%%%%%%%%%%%%%%%%%%%%%%%%%%%%%%%%%%%%%%%%%%%%%%%%%%%%
and others provides valuable insights into market-wide reactions, these studies fall short in elucidating how specific firms are affected by news events. Firm-specific impacts are often masked when aggregated at the index level, leading to a loss of critical information about how particular entities are influenced by specific news. 
During the COVID-19 pandemic, while overall market indices were impacted significantly, firm-specific effects varied widely, with some sectors like technology and healthcare experiencing positive returns while others, such as hospitality, travel, and retail, experiencing significant negative impacts due to widespread lockdowns and reduced consumer spending. 
Such nuanced differences are often obscured when focusing solely on market indices. Tools like Named Entity Recognition (NER), which could help identify firms impacted by particular events, remain underutilized in financial research, further contributing to the lack of firm-level granularity.

Third, there is an over-reliance on headlines as the basis for news analysis. Headlines are often used due to their availability and the simplicity of extracting sentiment from them, making them convenient but insufficient for comprehensive analysis. Numerous studies, including 
%%%%%%%%%%%%%%%%%%%%%%%%%%%%%%%%%%%%%%%%%%%%%%%%%%%%%
%%%%%%%%%%%%%%%%%%%%%%%%%%%%%%%%%%%%%%%%%%%%%%%%%%%%%
\cite{chan2003stock}, \cite{oncharoen2018deep}, \cite{wei2018stock}, \cite{lopez2023can}, \cite{chen2022expected} 
%%%%%%%%%%%%%%%%%%%%%%%%%%%%%%%%%%%%%%%%%%%%%%%%%%%%%
%%%%%%%%%%%%%%%%%%%%%%%%%%%%%%%%%%%%%%%%%%%%%%%%%%%%%
utilize headlines to gauge market sentiment.  Headlines are designed to capture attention, not to provide a comprehensive summary of all relevant details. Consequently, relying solely on headlines can lead to overly simplistic analyses that fail to capture critical contextual details necessary for accurately predicting market reactions.

This paper seeks to address these three limitations by leveraging Large Language Models (LLMs) to facilitate a more granular, firm-specific analysis of complete news articles. LLMs, such as GPT and LLaMA, offer a sophisticated mechanism for processing news due to their ability to handle large contexts, understand intricate language patterns, and recognize implicit relationships. For example, LLMs could simulate human analysis of news articles, understanding the economic shocks that a news article describes upon a specific firm --such as supply chain disruptions affecting manufacturing, shifts in consumer demand impacting retail, or policy changes influencing energy sectors-- and quantifying both the magnitude and direction of these impacts on specific firms. Unlike traditional sentiment scores, LLMs are capable of capturing the full context of articles, thereby enriching our understanding of the specific economic effects conveyed by news. The ability of LLMs to understand nuanced language, recognize implicit relationships, and integrate contextual information makes them particularly well-suited for analyzing financial news. By using LLMs, we can move beyond simplistic sentiment measures and towards a more holistic understanding of how news influences firm behavior and market outcomes.

In this study, I apply LLMs to a dataset of Spanish business news articles from DowJones Newswires, spanning June 2020 to September 2021, a particularly unstable period marked by economic disruptions due to the COVID-19 pandemic. This period was purposefully chosen because it presents a challenging environment for traditional
% sentiment and topic modeling 
 methods, which often fail under rapidly evolving conditions. 
% By testing the proposed methodology during such an unstable period, we aim to evaluate its robustness. It is relatively easy for traditional methods to perform well in stable times, but during periods of heightened uncertainty, their limitations become apparent. 
 This is precisely the scenario in which we seek to determine whether LLM-based analysis can provide superior insights. As a benchmark, we will compare our LLM clustering method with KMeans clustering of embeddings. That is, one could simply represent a news article as a  vector in high-dimensional space and then apply an unsupervised clustering method such as KMeans. This is more sophisticated than a sentiment or topic based method, but still does not rely on a comprehensive parsing of the news article content. 

Our methodology consists on defining a schema with which we will guide an LLM to detect article-implied firm-specific shocks from business news and to further classify them by their type (demand, supply, technological, policy, financial), magnitude (minor, major) and direction (positive, negative). By categorizing and comprehending the economic implications of news, LLMs provide insights that extend beyond mere sentiment analysis, helping to elucidate the underlying mechanisms driving market behavior. 
%The approach taken in this paper involves not only identifying the firms mentioned in each article but also determining the type, magnitude, and direction of the shocks implied by the news. 
This allows for a more detailed assessment of how specific pieces of information influence particular firms, providing a richer and more precise picture of market dynamics.

The objective of this paper is not to parse the largest dataset available or to develop a realistic trading strategy with commercial application. Rather, it aims to introduce a novel methodology for analyzing news articles in a granular and firm-specific manner, demonstrating its utility through a reduced dataset. By focusing on a smaller, high-quality dataset, the study emphasizes methodological rigor and interpretability. The findings are intended to contribute to a more nuanced understanding of how market participants process news, using a simple trading strategy to illustrate the potential of this approach in capturing the complexities of information processing in financial markets. This methodological contribution lays the groundwork for future research that could extend these techniques to larger datasets and more complex trading applications, ultimately enhancing our ability to understand and predict market behavior in response to news.

The remainder of this paper is organized as follows: Section 2 presents the dataset and preprocessing steps. Section 3 provides a mathematical framework to treat news articles, Section 4 focuses on clustering news articles: first we present our benchmark framework (KMeans clustering of vector embeddings) and then, our novel LLM-based methodology. In Section 5 we construct a simple trading strategy: first we launch market-beta-neutral positions for each firm-article pair, then we extract the cluster-average Sharpe Ratios, and finally we select the optimal clusters for trading based on two algorithms that we propose. Lastly, we form a portfolio and evaluate the trading strategy out of sample. Section 6 performs robustness checks by looking at the dependence of our trading strategy results to hyperparameter variability. Finally, Section 7 concludes and discusses the implications of the findings.

%%%%%%%%%%%%%%%%%%%% DATA %%%%%%%%%%%%%%%%%%%%%%%%
%\hspace{0.5cm} This paper employs a dataset of Spanish business news articles sourced from Dow Jones Newswires, covering the period from June 24, 2020, to September 30, 2021. The selection of this timeframe is deliberate, driven by two key considerations. First, we aim to test a novel methodology on a relatively small dataset to ensure feasibility and demonstrate its utility in understanding market reactions to news. Our goal is not to conduct a big data analysis, but rather to showcase the methodology's potential on a manageable dataset. Second, to evaluate the extrapolative power of the proposed methodology, it is important that the data span a period characterized by instability and volatility, which is why we focus on the Covid-19 era.

The dataset consists of high-quality articles that have been filtered to include only those mentioning Spanish publicly traded firms listed on the IBEX-35 index. These 35 companies represent the largest firms in Spain by market capitalization and are typically the most liquid and actively traded Spanish stocks. Moreover, these companies tend to receive the most consistent media coverage, making them ideal for the scope of our analysis.

The use of Dow Jones Newswires as our news source is also intentional. Dow Jones has a standard practice of including the stock market ticker of firms directly affected by the article in parentheses, which significantly facilitates the extraction of named entities (i.e., Named Entity Recognition, or NER). The tickers used by Dow Jones align with those employed by Yahoo Finance, enabling seamless integration between our NER algorithm and subsequent firm-specific trading operations via the Yahoo Finance API.

Specifically, we employ a pattern recognition algorithm (using the \texttt{regex} library in Python) to identify firm-specific mentions of IBEX-35 companies by searching for patterns of the form \qquote{(<WORD>.MC)} for any \qquote{<WORD>}. For instance, consider the following example article (translated into English for convenience):

\begin{news}
    [A news article about Telef�nica and Cellnex]    % Caption
    [news:cellnex-article]                            % Reference
    {Cellnex will face more competition in Europe} Telef�nica's \red{(TEF.MC)} subsidiary, Telxius Telecom, has agreed to sell its telecommunications tower division in Europe and Latin America to American Tower (AMT), which will expand the latter's presence in Europe and increase competition for the Spanish wireless telecommunications group Cellnex Telecom \red{(CLNX.MC)}, according to Equita Sim. The transaction "represents the entry of a new independent tower operator into the Spanish market and potentially more competition for future growth in the European market as well," says the brokerage firm.
\end{news}

Our NER algorithm successfully identifies \texttt{TEF.MC} (Telef�nica) and \texttt{CLNX.MC} (Cellnex Telecom) as the entities directly impacted by the news article. To enhance the reliability of firm identification, we further validate the extracted entities using a Large Language Model (LLM).
In particular, we feed the articles to the LLM, which parses them according to a predefined schema. The first task in this schema is to identify all the Spanish firms listed on the IBEX-35 that are directly affected by the shocks described in the news article. 
Due to the high quality of the dataset, the correlation between entities identified by the LLM and those extracted via pattern recognition is almost exact.

For subsequent analysis, we partition the dataset into three splits: \textit{Train}, \textit{Validation}, and \textit{Test}. Each split serves a distinct purpose that will be explained in detail as we progress through the paper. Summary statistics for each data split are provided in \cref{tab:Articles_Summary_Statistics}.

%----------------------------------------------------
\input{/Users/jesusvillotamiranda/Library/CloudStorage/OneDrive-UniversidaddeLaRioja/CEMFI/__MSc__/__Second_year__/6th_Term/MasterThesis/__Output/EDA_Articles_by_SplitSummary_Statistics.tex}
%----------------------------------------------------

The most frequently used words in the entire dataset are depicted in \cref{fig:WordCloud} by means of a WordCloud. As shown, the most prominent words include \qquote{empresa} (firm), \qquote{compa��a} (company), and \qquote{espa�a} (Spain), reinforcing that the dataset primarily comprises Spanish business news, with a prevalence of technical terms such as \qquote{beneficio neto} (net profit), \qquote{precio objetivo} (target price), \qquote{proyecto} (project), and \qquote{operaci�n} (operation).

%----------------------------------------------------
\inserthere{fig:WordCloud}
\begin{figure}[H]
  \centering
  \caption{Word Cloud of the Entire Dataset}
  \includegraphics[scale=0.496]{/Users/jesusvillotamiranda/Library/CloudStorage/OneDrive-UniversidaddeLaRioja/CEMFI/__MSc__/__Second_year__/6th_Term/MasterThesis/__Output/EDA_WordCloud.pdf}
  \label{fig:WordCloud}
\end{figure}
%----------------------------------------------------

The distribution of the number of articles published per day is illustrated in \cref{fig:hist_1}, showing that the most frequent publication rate is between 5 and 10 articles per day, though some days exhibit unusually high publication counts. \cref{fig:hist_2} shows the distribution of the number of words per article, with the majority of articles containing between 70 and 280 words. This indicates that the articles are relatively succinct, providing direct information. However, a small subset of articles exceeds 500 words, indicating more in-depth coverage.

%----------------------------------------------------
\inserthere{fig:histograms}
\begin{figure}[H]
  \caption{Histogram of \# News Articles per Day and \# Words per Article}
  \centering
  \begin{subfigure}[b]{0.46\textwidth}
    \centering
    \includegraphics[width=\textwidth]{/Users/jesusvillotamiranda/Library/CloudStorage/OneDrive-UniversidaddeLaRioja/CEMFI/__MSc__/__Second_year__/6th_Term/MasterThesis/__Output/EDA_Histogram_of_Number_of_News_Articles_per_day.pdf}
    \caption{Number of News Articles per Day}
    \label{fig:hist_1}
  \end{subfigure}
  \hspace{0.05\textwidth} % Add horizontal space between the subfigures
  \begin{subfigure}[b]{0.46\textwidth}
    \centering
    \includegraphics[width=\textwidth]{/Users/jesusvillotamiranda/Library/CloudStorage/OneDrive-UniversidaddeLaRioja/CEMFI/__MSc__/__Second_year__/6th_Term/MasterThesis/__Output/EDA_Number_of_Words_per_Article.pdf}
    \caption{Number of Words per Article}
    \label{fig:hist_2}
  \end{subfigure}
  \label{fig:histograms}
\end{figure}
%----------------------------------------------------

The time series of the number of articles published per day throughout the sample period is shown in \Cref{fig:ts_articles}. The series exhibits considerable variability, with frequent fluctuations from fewer than 5 articles per day to sudden spikes exceeding 20 articles. The 30-day moving average smooths the series, confirming the previous observation that, on average, between 5 and 10 articles are published daily.

%----------------------------------------------------
\inserthere{fig:ts_articles}
\begin{figure}[H]
  \centering
  \caption{Time Series of Number of Articles per Day and 30-Period Moving Average}
  \includegraphics[scale=0.445]{/Users/jesusvillotamiranda/Library/CloudStorage/OneDrive-UniversidaddeLaRioja/CEMFI/__MSc__/__Second_year__/6th_Term/MasterThesis/__Output/EDA_Time_Series_of_Articles.pdf}
  \label{fig:ts_articles}
\end{figure}
%----------------------------------------------------

%%%%%%%%%%%%%%%%%% METHODOLOGY %%%%%%%%%%%%%%%%%%%%%%

Different from \cite{hu2021networks}: 
\begin{itemize}
  \item instead of considering that news articles embed a leader-follower relationship, we don't impose any structure in the news articles
  \item we perform NER in a more realistic way, by having an LLM parse the news articles and extracting the firms that it considers as \qquote{directly affected by the news articles}. The problen with \cite{hu2021networks}'s NER is that they need to assume that every firm mentioned in a news article is relevant to the news article. This is actually not the case in most news articles, where many firms are mentioned contextually, or even more extreme, sometimes there is no relationship going on between the firms mentioned in the article. For example, we could have a news article like this: \qquote{Moodys lowers the credit rating of Banco Santander}. It's clear that this article is not talking about the existence of a relationship between Moodys and Banco Santander, however, in \cite{hu2021networks}'s logic, these article defines a connection between those two firms.
\end{itemize}

Our methodology is less restrictive and imposes no structure on the treatment of business news articles. 



%%%%%%%%%%%%%%%%%%%%%%%%%%%%%%%%%%%%%%%%%%%%%%%%%%%%%
\begin{quote}
%%%%%%%%%%%%%%%%%%%%%%%%%%%%%%%%%%%%%%%%%%%%%%%%%%%%%

Given a set of textual news articles aggregated in period $T, \D_T:=\left\{m_1, \ldots, m_d\right\}$ and a universe of $n$ firms $\F :=\{1, \ldots, n\}$, we identify the set of news linkage pairs $l^{(i,j)}_T$ between firms $i, j \in \F $ as:
$$
l^{(i,j)}_T
%\stackrel{\text { def }}{=}
:=
 \bigcup_{m_d \in \D_T}\9{ (i, j) ~\bigg{|}~ 
m_d \t{ describes a relationship between }i, j \in \F ~\t{}
% j \text { in } m_d \text { title, } i \text { in } m_d \text { headline, } i \neq j
}
$$


With the well-defined news linkage pairs, we then define the \qquote{News-implied Firm Network} at time period $T$ by the adjacency matrix 
%Given a weighted direct graph $\mathcal{G}=(\mathcal{V}, \mathcal{E})$ 
$$
\mathcal{W}_T := 
\2{
\begin{array}{ccc}
	|l^{(1,1)}_T| 		& \cdots  	& |l^{(1,n)}_T|
	\\
	\vdots				& \ddots 	& \vdots 
	\\
	|l^{(n,1)}_T|		& \cdots  	& |l^{(n,n)}_T|
\end{array}
}
$$
So far we have not imposed any hierarchy, so $\mathcal{W}_T$ is an asymmetric matrix with all zero diagonal values. However, we could further ask the LLM to give us the structure of the relationship between $i,j$. Depending on this relationship, we can define the following type of relationships: 
\begin{itemize}
  \item $i \sim j \iff i$ and $j$ are competitors within the same industry 
  \item $i \succ j \iff i $ is the supplier of $j$
  \item $i \bowtie j \iff $ in the rest of the cases
\end{itemize}

%%%%%%%%%%%%%%%%%%%%%%%%%%%%%%%%%%%%%%%%%%%%%%%%%%%%%
%%%%%%%%%%%%%%%%%%%%%%%%%%%%%%%%%%%%%%%%%%%%%%%%%%%%%

\Vhrulefill
{\center \href{https://chatgpt.com/share/66f342fc-07f0-800d-9210-0506f9c26169}{Conversation w/ ChatGPT}
\par}
\Vhrulefill

%%%%%%%%%%%%%%%%%%%%%%%%%%%%%%%%%%%%%%%%%%%%%%%%%%%%%
%%%%%%%%%%%%%%%%%%%%%%%%%%%%%%%%%%%%%%%%%%%%%%%%%%%%%

\subsection{Introduction}

The increasing availability of textual data from business news articles provides a rich source of information for studying the relationships between firms. Traditionally, firm networks inferred from such data rely on simple co-occurrence models, where firms are assumed to be connected if they are mentioned together in an article. 

%----------------------------------------------------
\begin{quote}
In particular, 
\begin{itemize}
  \item Let $\mathcal{F}=\left\{F_1, F_2, \ldots, F_n\right\}$ represent the set of $n$ firms you are analyzing.
  \item Let $\mathcal{A}=\left\{A_1, A_2, \ldots, A_m\right\}$ represent the set of $m$ news articles that mention these firms.
\end{itemize}
We assume that the news articles can be mapped to specific dates, allowing for a time dimension if needed, i.e., $A_i(t)$, where $t$ represents the publication date of article $A_i$.

\textit{Firm-Article Matrix (Incidence Matrix)}

\begin{itemize}
  \item Define an incidence matrix $M \in\{0,1\}^{n \times m}$, where the entry $M_{i j}=1$ if firm $F_i$ is mentioned in article $A_j$, and $M_{i j}=0$ otherwise.
  \item This matrix allows us to encode which firms are co-mentioned in the same articles.
\end{itemize}

\textit{Co-occurrence Matrix}

From the incidence matrix $M$, we can construct a co-occurrence matrix $C \in \mathbb{R}^{n \times n}$, where each entry $C^{i j}$ captures the number of articles in which firms $F_i$ and $F_j$ are co-mentioned.
Mathematically, this can be expressed as:
$$
C=M M^T
$$
Here, $C^{i j}$ counts the number of articles that mention both firm $F_i$ and firm $F_j$.


\textit{Weighted Network Representation}

The co-occurrence matrix $C$ can be used to define a weighted undirected graph $G=$ $(\mathcal{F}, \mathcal{E}, w)$, where:
\begin{itemize}
  \item $\mathcal{F}$ is the set of firms (nodes).
  \item $\mathcal{E} \subseteq \mathcal{F} \times \mathcal{F}$ is the set of edges between firms, where an edge exists between firms $F i$ and $F^j$ if $C^{i j}>0$ (i.e., they have been co-mentioned in at least one article).
  \item $w: \mathcal{E} \rightarrow \mathbb{R}+$ is the weight function, where the weight of the edge between firms $F_i$ and $F j$ is given by $w\left(F_i, F^j\right)=C^{i j}$. This weight represents the strength of the connection between the two firms, based on the number of co-occurrences in news articles.
\end{itemize}

\end{quote}
%----------------------------------------------------




However, this approach does not account for the nature or type of relationships between firms, nor does it consider the directionality or complexity of those relationships.

In this paper, I propose a novel methodology for constructing firm networks using Large Language Models (LLMs) to analyze the textual content of news articles. The LLM is tasked with two goals: (1) determining whether a substantive relationship exists between a pair of firms based on the context provided in the article, and (2) classifying the type of relationship (e.g., supplier-customer, competitor, partnership). Additionally, I incorporate the \textbf{directionality} of certain types of relationships, such as supplier-customer or mergers and acquisitions (M\&A), where relationships are inherently asymmetric. In particular, directionality refers to relationships between different firms where $F_i \rightarrow F_j$ but $F_j \not \rightarrow F_i$. I also consider \textbf{reflexivity}, where firms can have self-relations, such as internal restructuring or stock buybacks. In this case $F_i \leftrightarrow F_i$
The methodology yields a nuanced and comprehensive firm network that captures both the strength and type of relationships between firms, providing a powerful tool for analyzing firm interactions, market dynamics, and the effects of external shocks.

\subsection{Mathematical Framework for Firm Networks Using LLMs}

\subsection{Firm Set and News Articles}

Let $ \F = \{F_1, F_2, \dots, F_n\} $ represent the set of $ n $ firms under consideration. Each firm is potentially mentioned in a set of news articles $ \mathcal{A} = \{A_1, A_2, \dots, A_m\} $, where $ m $ denotes the number of articles. Each article $ A_i \in \mathcal{A} $ contains textual content $ T(A_i) $ and is published on a specific date $ t(A_i) $.

\subsection{LLM-based Relationship Detection and Classification}

For each article $ A_i $, the LLM processes the textual content $ T(A_i) $ and performs two key tasks:
\begin{enumerate}
    \item \textbf{Relationship Detection}: The LLM determines whether there is a substantive relationship between a pair of firms $(F_i,F_j)\in\F\times\F$ based on the context provided in article $A_k$.
    \item \textbf{Relationship Classification}: If a relationship exists, the LLM classifies the relationship between $(F_i,F_j)$ described in article $A_k$ into a relationship type $ r_{ij}(A_k) \in \mathcal{T} $, where $\T=\{\t{Supplier, Competitor, Partnership, M\&A, Legal, Other}\}$ is the set of relationship types. Note that some relationships are \qquote{directional}, while others are not. In particular:
%\begin{itemize}
%  \item Supplier: One firm supplies goods or services to another.
%  \item Competitor: Firms operate in the same industry and compete for market share.
%  \item Partnership: Firms collaborate on a joint project or initiative.
%  \item Legal Dispute: Firms are involved in a legal battle.
%  \item Mergers \& Acquisitions: Firms are involved in a merger or acquisition event.
%\end{itemize}
    \begin{itemize}
        \item \textit{Supplier-Customer}: $ F_i \to F_j $, where $ F_i $ is the supplier and $ F_j $ is the customer.
        \item \textit{Competitor}: $ F_i \leftrightarrow F_j $, where both firms compete for market share.
        \item \textit{Partnership}: $ F_i \leftrightarrow F_j $, where the firms collaborate on a project or initiative.
        \item \textit{Mergers \& Acquisitions (M\&A)}: $ F_i \to F_j $, where firm $ F_i $ absorbs or acquires firm $ F_j $.
        \item \textit{Legal Dispute}: $ F_i \to F_j $, where firm $ F_i $ sues or takes legal action against firm $ F_j $.
        \item \textit{Other}: contains the rest of relationships that the LLM was unable to clasify in the previous categories. For simplicity, we make this an undirected relationship, so $F_i\leftrightarrow F_j$. 
    \end{itemize}
\end{enumerate}

Note that in these defitions, order matters, as we are always considering that $F_i \to F_j$ in any directional relationship between any $(F_i, F_j)\in\F\times\F$. 

The LLM also assigns a relationship score $ \text{LLM\_score}(A_k, F_i, F_j, r) $, reflecting the confidence in the existence and strength of the relationship $ r_{ij}(A_k) $ between firms $ F_i $ and $ F_j $ based on article $ A_k $.

\subsection{General Relationship Matrix}

Let $\mathcal{R}\left(A_i\right) \subseteq \F \times \F$ represent the set of firm pairs $\left(F_i, F_j\right)$ that the LLM determines to be related based on the content of article $A_i$. The task of the LLM is to analyze the article $T\left(A_i\right)$ and determine when a meaningful relationship or event connects the firms.

For each article $A_i$, the LLM processes the text $T\left(A_i\right)$ and returns a set of firm relationships: 
$$
\mathcal{R}(A_i)
=
\9{
\left(F_i, F_j\right) \in \F \times \F
\mid 
\t{LLM concludes $F_i$ and $F_j$ are economically tied in $A_i$}
}
$$

The key here is that $\mathcal{R}\left(A_i\right)$ is determined by the LLM's understanding of the text, identifying cases where firms are tied by contracts, joint ventures, lawsuits, partnerships, or other significant business events, rather than simple co-mentioning.

From the set of firm relationships across all articles, we can construct a relationship matrix $R \in \mathbb{R}^{n \times n}$, where each entry $R_{i j}$ quantifies the strength of the relationship between firm $F_i$ and firm $F_j$. The entry $R_{i j}$ is computed as:

$$
R_{i j}
=
\sum_{k=1}^m \I{(F_i, F_j) \in \mathcal{R}(A_k) }
\cdot 
\mathrm{LLM} \_\operatorname{score}\left(A_k, F_i, F_j\right)
$$


\subsection{Type-Specific Relationship Matrix}
Since we have richer information about the relationship type, we can define a relationship matrix that is specific to each type of relationship $r\in\T$. 
For each article $A_i$, we now have: 
$$
\mathcal{R}(A_i)
=
\9{
\4{F_i, F_j, r_{ij}(A_k)} \in \F \times \F \times \T
~\bigg{|}~
\t{LLM detects and classifies a relationship}
}
$$
For each relationship type $ r \in \mathcal{T} $, we define a relationship matrix $ R^r \in \mathbb{R}^{n \times n} $, where the entry $ R^r_{ij} $ quantifies the strength of the relationship $ r $ between firms $ F_i $ and $ F_j $.

$$
R^r_{ij} = \sum_{k=1}^{m} 
\I{(F_i, F_j, r_{ij}(A_k)) \in \mathcal{R}(A_k)} 
\cdot
\text{LLM\_score}(A_k, F_i, F_j, r_{ij}(A_k)),
$$

where:
\begin{itemize}
    \item $\I{(F_i, F_j, r_{ij}(A_k)) \in \mathcal{R}(A_k)}$ is an indicator function that equals 1 if article $ A_k $ identifies relationship $ r $ between firms $ F_i $ and $ F_j $, and 0 otherwise.
    \item $ \text{LLM\_score}(A_k, F_i, F_j, r_{ij}(A_k)) $ is the score provided by the LLM that quantifies the strength of the relationship.
\end{itemize}

For some relationship types, such as \textit{supplier-customer}, the matrix $ R^r $ is \textbf{asymmetric}, meaning $ R^r_{ij} \neq R^r_{ji} $. In contrast, for \textit{competitor} or \textit{partnership} relationships, the matrix $ R^r $ is \textbf{symmetric}, meaning $ R^r_{ij} = R^r_{ji} $.

\subsection{Reflexivity in Firm Relationships}

In addition to relationships between firms, we also consider \textbf{reflexive relationships}, where a firm $ F_i $ has a relationship with itself, denoted $ F_i \leftrightarrow F_i $. Reflexivity can capture internal actions such as:
\begin{itemize}
    \item \textit{Firm Restructuring}: Internal reorganization or governance changes.
    \item \textit{Stock Buybacks}: Financial actions where a firm repurchases its own shares.
    \item \textit{Internal Legal Actions}: Actions that affect a firm's own internal compliance or governance.
\end{itemize}

The reflexive relationships are represented in the diagonal elements $ R^r_{ii} $ of the relationship matrix for each type $ r $:

$$
R^r_{ii} = \sum_{k=1}^{m} 
\I{(F_i, F_j, r_{ij}(A_k)) \in \mathcal{R}(A_k)}
\cdot \text{LLM\_score}(A_k, F_i, F_i, r_{ij}(A_k)),
$$

where the diagonal element $ R^r_{ii} $ quantifies the strength of firm $ F_i $'s reflexive relationship under relationship type $r$.

%%%%%%%%%%%%%%%%%%%%%%%%%%%%%%%%%%%%%%%%%%%%%%%%%%%%%
%Mathematically, reflexivity would correspond to the diagonal elements of the relationship matrix $R^r$. 
Specifically:
\begin{itemize}
  \item If $R_{i i}^r>0$, firm $F_i$ has a reflexive relationship in the context of relationship type $r$ (e.g., self-influence or self-reference).
  \item If $R_{i i}^r=0$, firm $F_i$ does not have a reflexive relationship.

\end{itemize}

%%%%%%%%%%%%%%%%%%%%%%%%%%%%%%%%%%%%%%%%%%%%%%%%%%%%%

\subsection{Network Representation with Directionality and Reflexivity}

The firm network is constructed as a \textbf{multi-layered graph} $G=\{G^r\}_{r\in\T}$
%, where each $G^r=\left(\F, \mathcal{E}^r, w^r\right)$ is a relationship-specific layer 
composed of relationship-specific layers $G^r=\left(\F, \mathcal{E}^r, w^r\right)$, where:
\begin{itemize}
  \item $\F$ is the set of firms (nodes).
  \item $\mathcal{E}^r \subseteq \F \times \F$ is the set of edges representing relationships of type $r$, where an edge exists between $F_i$ and $F_j$ if $R_{i j}^r>0$. Depending on the type of relationship $r \in \mathcal{T}$, the edges may be \textbf{directed}, \textbf{undirected} or \textbf{looping}:
\begin{itemize}
    \item \textit{Directed edges}: For relationships such as supplier-customer, M\&A, and legal disputes, the edges are directed, representing asymmetric relationships where $ F_i \to F_j $ but not necessarily $ F_j \to F_i $.
    \item \textit{Undirected edges}: For symmetric relationships like competition and partnerships, the edges are undirected, meaning $ F_i \leftrightarrow F_j $.
    \item \textit{Looping edges}: For reflexive relationships where $F_i\leftrightarrow F_i$. The weight of the self-loop reflects the strength of the firm's internal actions.
\end{itemize}
  \item $w^r: \mathcal{E}^r \rightarrow \mathbb{R}_{+}$ is the weight function, where the weight of the edge between $F_i$ and $F_j$ is given by $w^r\left(F_i, F_j\right)=R_{i j}^r$, representing the strength of relationship type $r$ between the two firms.
\end{itemize}

This creates multiple layers of networks, each representing a different type of firm interaction.

%The firm network is constructed as a \textbf{multi-layered graph} $ G = (\F, \mathcal{E}, w) $, where:
%\begin{itemize}
%    \item $ \F $ represents the set of firms (nodes).
%    \item $ \mathcal{E} \subseteq \F \times \F $ represents the set of edges, with an edge $ (F_i, F_j) $ indicating a relationship between firms $ F_i $ and $ F_j $.
%\begin{itemize}
%  \item Directed edges $\mathcal{E}_D \subseteq \F \times \F$ represent directional relationships. For example, if firm $F_i$ supplies firm $F_j$, then $\left(F_i, F_j\right) \in \mathcal{E}_D$.
%  \item Undirected edges $\mathcal{E}_U \subseteq \F \times \F$ represent symmetric relationships. For example, if firms $F_i$ and $F_j$ are competitors, then $\left(F_i, F_j\right) \in \mathcal{E}_U$.
%\end{itemize}
%    \item $ w: \mathcal{E} \to \mathbb{R}_+ $ is a weight function that assigns a weight $ w(F_i, F_j) = R^r_{ij} $, reflecting the strength of the relationship.
%\end{itemize}


\subsection{Dynamic Networks and Temporal Analysis}

To capture how relationships between firms evolve over time, we introduce a \textbf{time-varying relationship matrix} for each relationship type $ r $:

$$
R^r_{ij}(t) = \sum_{k=1}^{m} \I{(F_i, F_j, r_{ij}(A_k)) \in \mathcal{R}(A_k)} \cdot \text{LLM\_score}(A_k, F_i, F_j, r_{ij}(A_k)) \cdot \I{t(A_k) = t}.
$$

This matrix captures the relationships at a specific time $ t $ based on the articles published during that time period. The resulting \textbf{dynamic network} $ G(t) $ allows for the study of how firm interactions and network structures evolve in response to economic events, mergers, or external shocks.

\subsection{Analytical Methods and Potential Insights}

With the constructed network, several analyses can be performed to extract valuable insights:
\begin{itemize}
    \item \textbf{Centrality Measures}: Compute various centrality measures (degree, eigenvector, betweenness) to identify key firms within specific relationship networks. For example, firms central in the "supplier" network might play crucial roles in supply chains, while those central in the "competitor" network might dominate their industries.
    \item \textbf{Community Detection}: Apply community detection algorithms to identify clusters of firms that are closely related. Different clusters may emerge in different layers, such as a supply chain ecosystem or a competitive industry cluster.
    \item \textbf{Temporal Analysis}: Track the evolution of firm relationships over time, analyzing how major economic events (e.g., financial crises, regulatory changes) impact the structure and strength of firm networks.
    \item \textbf{Impact of Reflexivity}: Study firms with significant self-relations (high diagonal values in the relationship matrix) to understand how internal actions, such as restructuring or stock buybacks, affect firm performance and market position.
\end{itemize}

\subsection{Conclusion}

This paper introduces a comprehensive and nuanced approach to constructing firm networks from business news articles by leveraging LLMs to detect and classify firm relationships. By incorporating relationship types, directionality, and reflexivity, the proposed methodology provides a rich framework for analyzing firm interactions and market dynamics. The resulting multi-layered, directed- and undirected network allows for advanced analysis of firm relationships, providing insights into how firms navigate competitive environments, form partnerships, and manage internal actions.

This approach opens the door to further research into how firm networks evolve, how firms' internal and external actions affect their market position, and how different types of firm interactions impact the broader economic environment.






%%%%%%%%%%%%%%%%%%%%%%%%%%%%%%%%%%%%%%%%%%%%%%%%%%%%%
\end{quote}
%%%%%%%%%%%%%%%%%%%%%%%%%%%%%%%%%%%%%%%%%%%%%%%%%%%%%



%%%%%%%%%%%%%%%%%% CONCLUSION %%%%%%%%%%%%%%%%%%%%%%
%%----------------------------------------------------
\section{Conclusion}
%----------------------------------------------------
\hspace{0.5cm} 
This paper investigates how information from business news affects stock market prices. We analyze a dataset of Spanish business articles during a particularly volatile period-the COVID-19 pandemic-and examine firm-specific stock market reactions to news. We show that transforming text into vector embeddings and clustering them using KMeans yields clusters that are firm-specific and industry-specific. However, the distribution of articles across clusters is unstable over sequential data splits, indicating temporal instability. When we implement a cluster-based trading strategy-similar to portfolio sorts-on the KMeans clusters, we observe an over-reliance on the past performance of a cluster. That is, signals are short-lived due to temporal instability. Consequently, the out-of-sample profitability of the trading strategy is negligible, evidencing the method's poor temporal generalizability. Therefore, a model based on embeddings is superficial and is not able to anticipate market trends.

%----------------------------------------------------
Alternatively, we develop a novel approach by guiding a Large Language Model (LLM) through a structured news-parsing schema, enabling it to analyze news-implied firm-specific economic shocks. The schema involves identifying the firms affected by the articles and classifying the implied shocks on such firms by their type, magnitude, and direction. This LLM-based methodology demonstrates several advantages over the traditional clustering approach. Even in a volatile period, it produces stable distributions of articles across clusters in sequential splits, demonstrating robust temporal stability. Moreover, the resulting trading signals are both long-lasting and economically relevant, as they are based on fundamental economic shocks rather than statistical patterns. The results show that the LLM-based trading strategy effectively identifies winners and losers, illustrating the parser's ability to anticipate market trends by comprehending the economic implications of firm-specific shocks. This approach generates a consistent profile of earnings in the test set, with results robust to the choice of hyperparameters-the holding period length of the trading strategy and the number of selected clusters for trading. Our findings demonstrate a promising avenue: LLMs, when guided by appropriate economic frameworks, can help predict market reactions to news through systematic classification of economic shocks embedded in financial narratives.


%%%%%%%%%%%%%%%%%% BIBLIOGRAPHY %%%%%%%%%%%%%%%%%%%%%%
%\bibliography{bib_references_DOI.bib}
%\bibliographystyle{plain}

%----------------------------------------------------
%\newpage
%% To make sure all the tables/figures appear before the appendix
%\processdelayedfloats 
%% Reset the numbering for appendix figures and tables
%\renewcommand{\thefigure}{A\arabic{figure}} 
%\renewcommand{\thetable}{A\arabic{table}}
%----------------------------------------------------

%%%%%%%%%%%%%%%%%%% APPENDIX %%%%%%%%%%%%%%%%%%%%%%
%\appendix
%\section{Appendix}
%%----------------------------------------------------
\subsection{Cointegration Meets Synthetic Controls: A Formal Equivalence}
\label{sec:cointegration_meets_synthetic_controls}

In this appendix section, we develop a formal argument showing how, under some stringent assumptions, our notion of \emph{synthetic control} can be viewed as a special case of \emph{cointegration}. This connection underlies the intuition that, when one normalizes the first variable of a cointegrated system to 1, the remaining cointegration relationships effectively produce the \emph{synthetic} version of the first variable when the cointegration vector satisfies a specific restriction. 
%Here, we adopt a rigorous perspective aimed at bridging the econometric concept of cointegration with methodologies employed in the synthetic control literature (e.g., Abadie and Gardeazabal).

%\subsection{Cointegration}

Let $\{y_{i,t} \}_{t=1}^{T}$ denote the time series sequence of log-prices for each asset $i\in\{1,\ldots,N\}$.
%
Throughout, we assume each $y_{i,t}$ is an $I(1)$ process (integrated of order 1). 
%
Formally, an $I(1)$ process is one that becomes \emph{stationary} (and typically ergodic) upon differencing once:
$\Delta y_{i,t} := y_{i,t} - y_{i,t-1} \sim I(0).$
%
The notion of cointegration, due to Engle and Granger, is central in analyzing potentially long-run equilibria among these variables.

%----------------------------------------------------
\begin{definition}[Engle and Granger (1987)]
The components of $\mbf{y}_t:=[y_{1t}, ..., y_{Nt}]$ are said to be cointegrated of order $d$, $b$, denoted $\mbf{y}_t \sim CI(d,b)$, if (a) all components of $\mbf{y}_t$ are $I(d)$ and (b) a vector $\b{\beta}\neq 0$ exists so that $\b{\beta}'\mbf{y}_t \sim I(d-b)$, $b > 0$. The vector $\b{\beta}$ is called the cointegrating vector.
\end{definition}
%----------------------------------------------------
%\begin{definition}[Campbell and Perron (1991)]
%An $(n \times 1)$ vector of variables $\mbf{y}_t$ is said to be cointegrated if at least one nonzero $n$-element vector $\b{\beta}_i$ exists such that $\b{\beta}'_i\mbf{y}_t$ is trend-stationary. $\b{\beta}_i$ is called a cointegrating vector. If $r$ such linearly independent vectors $\b{\beta}_i(i = 1,\ldots,r)$ exist, we say that $\{\mbf{y}_t\}$ is cointegrated with cointegrating rank $r$. We then define the $(n \times r)$ matrix of cointegrating vectors $\b{\beta} = (\b{\beta}_1,\ldots,\b{\beta}_r)$. The $r$ elements of the vector $\b{\beta}'\mbf{y}_t$ are trend-stationary, and $\b{\beta}$ is called the cointegrating matrix.
%\end{definition}
%----------------------------------------------------

%\subsection{Synthetic Control}
%\begin{definition}
%In a synthetic control problem we have a target element $y_1$ that we seek to mimick through a linear combination of elements in a donor pool $\mbf y_{2:n}=(y_2,...,y_n)$ with weigths $\mbf w=\arg\underset{w\in\W}{\min} \sum_{t=1}^T (y_1-\mbf w'\mbf y_{2:n})^2$ where $\mathcal{W} := \{\mbf{w}\in\mathbb{R}_{+}^{n-1}: \sum_{j=2}^n w_j=1\}$.
%\end{definition}

\begin{definition}[Synthetic Control]\label{def:synthetic_control}
Let $\{y_1, y_2, \dots, y_n\}$ be a collection of random variables, where $y_1$ is the ``target'' variable and 
$\mathbf{y}_{2:n} = (y_2,\dots,y_n)$ constitute the ``donor pool''. A \emph{synthetic control} for 
$y_1$ is constructed by choosing weights $\mathbf{w}$ in the $(n-1)$-dimensional space
$\mathcal{W} := \{\mbf{w}\in\mathbb{R}_{+}^{n-1}: \sum_{j=2}^n w_j=1\}$
that satisfy
%to minimize the sum of squared deviations over $T$ observations:
$\mbf w=\arg\underset{w\in\W}{\min} 
\sum_{t=1}^T 
(y_{1,t}-\mbf w'\mbf y_{2:n,t})^2$.
%Because $\mathcal{W}$ is precisely the convex hull of the standard basis vectors in $\mathbb{R}^{n-1}$, the resulting synthetic control $\mathbf{w}'\,\mathbf{y}_{2:n}$ is a convex combination of the donor pool elements $(y_2,\dots,y_n)$.
\end{definition}


%\subsection{Equivalence}
Given that cointegration relationships prevail up to scale and sign changes, then, under suitable conditions on the cointegration vector, there exists a nontrivial constant $\kappa$ that allows us to reinterpret the cointegration relationship as one of a synthetic control. In particular,
%----------------------------------------------------
\begin{proposition} 
For a cointegrated vector $\mbf y$  with rank $r$, if (at least) one of the cointegrating vectors $\b \beta$ satisfies the restriction
$\mathcal R=
\{
%\begin{array}{ll}
%\b \beta > 0, ~
\mbf 1' \b \beta  = 0
%\\
%\beta_j \geq 0, j\neq i
%\end{array}
\}$,
%( i.e, that its components 
%are nonnegative and 
%sum to 1
%, and at least one of them is strictly positive). 
 then we can scale the cointegration vector by $\kappa=1/\beta_i$ such that $\kappa \b \beta ' \mbf y$ is stationary and describes a \qquote{synthetic control} relationship (as per \cref{def:synthetic_control}) between $y_i$ and $\mbf y_{-i}$. 
\end{proposition}
%----------------------------------------------------

\begin{proof}
The proof is straightforward. For a cointegration vector $\b \beta$ where $\mathcal R$ holds, we have that $\mbf 1'\b \beta = \sum_{j=1}^n \beta_j= 0$, which trivially implies $\beta_i = -\sum_{j\neq i}\beta _j$. For the sake of the proof, set that $\beta_i$ to the first component ($\beta_1$). Then
$\beta_1 = -\sum_{j=2}^n \beta_j$ and $\kappa = (\beta_1)^{-1} = -(\sum_{j=2}^n \beta_j)^{-1}$
$$
\kappa \b \beta' \mbf y 
= 
\frac{1}{\beta_1} [\beta_1 ~~ \b \beta_{2:n}] \mbf y_t 
=
\2{1 ~~ \frac{-\b \beta'_{2:n}}{\sum_{j=2}^n\beta_j}}
\2{\v{y_1 \\ \mbf y_{2:n}}}
=
y_1 - \frac{\beta_2}{\sum_{j=2}^n \beta_j}y_2 - \cdots - 
\frac{\beta_n}{\sum_{j=2}^n \beta_j} y_n
\sim I(0)
$$
describes a stationary cointegration relationship in $\mbf y$, and since
\begin{align*}
y_1 
&= \frac{\beta_2}{\sum_{j=2}^n \beta_j}y_2 + \cdots + 
\frac{\beta_n}{\sum_{j=2}^n \beta_j} y_n + \eps 
\\&= \mbf w' \mbf y_{2:n} + \eps 
%, \t{~~where~} \eps\sim I(0)
\end{align*}
with $\eps\sim I(0)$ and $\mbf w:=\1{\frac{\beta_2}{\sum_{j=2}^n \beta_j}, ..., \frac{\beta_n}{\sum_{j=2}^n \beta_j}}'\in \W$, then this relationship is endowed with a synthetic control structure. A similar reasoning applies to any other $\beta_i$ different from $\beta_1$.
\end{proof}
%----------------------------------------------------

%----------------------------------------------------
%%%%%%%%%%%%%%%%%%%%%%%%%%%%%%%%%%%%%%%%%%%%%%%%%%%%%
% DISCUSSION: CARDINALITY-CONSTRAINED SYNTHETIC CONTROL
%%%%%%%%%%%%%%%%%%%%%%%%%%%%%%%%%%%%%%%%%%%%%%%%%%%%%
\subsection{Why not use a cardinality-constrained Synthetic Control?}
\label{sec:discussion_card_constr}

While the $\ell_1$-regularized approach provides a computationally efficient and convex framework for constructing sparse synthetic controls, it is worth considering alternative methods that directly impose sparsity through cardinality constraints. A natural alternative is to solve a cardinality-constrained quadratic program, which explicitly limits the number of non-zero weights in the synthetic asset. Formally, this can be expressed as:
\begin{equation*}
\mathbf{w}^* = \argmin_{\mathbf{w} \in \R^{N}} \sum_{t=1}^T \left(y_{t} - \sum_{i=1}^N w_i x_{it}\right)^2 
\quad \text{s.t.} \quad 
\left|
\begin{array}{ll}
	\mbf 1^\top \mbf w &= 1 \\
	\norm{\mathbf{w}}_0 &\leq K
\end{array}
\right.
\end{equation*}

where $\|\mathbf{w}\|_0 := \sum_{i=1}^N \mathbb{I}\{w_i \neq 0\}$ counts the number of non-zero elements in $\mathbf{w}$, and $K$ is a user-defined sparsity level. This formulation directly enforces sparsity by restricting the synthetic asset to be constructed from at most $K$ donor assets. However, the cardinality constraint introduces significant computational challenges, as the problem becomes NP-hard due to its combinatorial nature. Below, we discuss two approaches to approximate this problem and their limitations.

\subsubsection{Mixed-Integer Programming Approach}
One way to tackle the cardinality-constrained problem is to reformulate it as a mixed-integer quadratic program (MIQP). This involves introducing binary variables $z_i \in \{0, 1\}$ for $i = 1, \dots, N$, where $z_i = 1$ indicates that the $i$-th asset is included in the synthetic control, and $z_i = 0$ otherwise. The problem can then be rewritten as:
\begin{equation*}
\mathbf{w}^*, \mathbf{z}^* 
= 
\left[
\begin{array}{rlll}
\underset{\mathbf{w} \in \R^{N},~\mathbf{z} \in \{0, 1\}^N}{\arg\min}
&
\sum_{t=1}^T \left(y_{t} - \sum_{i=1}^N w_i x_{it}\right)^2
%\norm{\mathbf{y} - \mathbf{X}\mathbf{w}}_2^2
\\
\text{s.t.}  &
\left|
\begin{array}{ll}
\mathbf{1}^\top \mathbf{w} &= 1, \\
\sum_{i=1}^N z_i &\leq K, \\
|w_i| &\leq M z_i \quad \text{for } i = 1, \dots, N,
\end{array}
\right.
\end{array}
\right]
\end{equation*}

where $M$ is a sufficiently large constant that bounds the magnitude of the weights. The constraint $|w_i| \leq M z_i$ ensures that $w_i$ can only be non-zero if $z_i = 1$. While this formulation is exact, it is computationally intensive, especially for large donor pools, as it requires solving a mixed-integer program. The computational complexity grows exponentially with the number of assets, making it impractical for high-dimensional settings.

\subsubsection{Two-Step Heuristic Procedure}
An alternative approach is to use a two-step heuristic procedure that approximates the cardinality-constrained solution without requiring mixed-integer programming. This procedure proceeds as follows:

\begin{enumerate}
\item \textbf{Solve the full least squares problem:} First, solve the unconstrained least squares problem to obtain an initial weight vector:
\begin{equation*}
\mathbf{w}^{(1)} 
= 
\argmin_{\mathbf{w} \in \mathbb{R}^{N}} 
\norm{\mathbf{y} - \mathbf{X}\mathbf{w}}_2^2
\quad \text{s.t.} \quad \mathbf{1}^\top \mbf w = 1.
\end{equation*}

\item \textbf{Select the $K$ largest weights:} Identify the $K$ largest weights (in absolute value) from $\mathbf{w}^{(1)}$ and define the support set:
\begin{equation*}
\mathcal{I} := \{i : |w_i^{(1)}| \text{ is among the $K$ largest}\}.
\end{equation*}

\item \textbf{Solve the restricted program:} Solve the least squares problem restricted to the support set $\mathcal{I}$:
\begin{equation*}
	\mbf w^{(2)} = \arg \min_{\mbf w_{\mathcal I} \in \mathbb{R}^K} \norm{\mbf y - \mbf X_{\mathcal I}\mbf w_{\mathcal I}}_{2}^{2}
\quad \text{s.t.} \quad 
\mbf 1^\top \mbf w_{\mathcal I} = 1,
\end{equation*}
where $\mbf X_{\mathcal{I}} \in \mathbb{R}^{T \times K}$ is the restricted donor matrix and $\mbf w_{\mathcal{I}} \in \mathbb{R}^{K}$ is the restricted weight vector.

\item \textbf{Construct the full weight vector:} Embed the optimized restricted weights back into the original $N$-dimensional space:
\begin{equation*}
	w^*_i = 
\begin{cases}
w^{(2)}_j & \text{if } i = \mathcal{I}_j, \\
0 & \text{otherwise}.
\end{cases}
\end{equation*}
\end{enumerate}

While this heuristic is computationally efficient, it has several drawbacks. First, the initial least squares solution $\mathbf{w}^{(1)}$ may not provide a good indication of which assets are most relevant, especially in the presence of multicollinearity or noise. Second, the procedure can lead to extreme weights (both positive and negative) in the final solution, resulting in a highly leveraged portfolio that may not be practical for trading. This is because the restricted optimization step does not impose any bounds on the magnitude of the weights, allowing for large positive and negative values that cancel each other out to satisfy the unit sum constraint.

\subsubsection{Comparison with $\ell_1$-Regularized Approach}
In contrast to the cardinality-constrained approaches, the $\ell_1$-regularized method provides a more balanced trade-off between sparsity and computational efficiency. By shrinking some weights exactly to zero, the $\ell_1$ penalty achieves sparsity without requiring explicit cardinality constraints. Moreover, the convex nature of the problem ensures that it can be solved efficiently using proximal algorithms or quadratic programming techniques, even for high-dimensional donor pools. Additionally, the regularization parameter $\lambda$ provides fine-grained control over the sparsity level, allowing the user to tune the solution based on their specific requirements.

In practice, we found that the $\ell_1$-regularized approach yields more stable and interpretable synthetic controls compared to the cardinality-constrained methods. The latter often produce highly leveraged portfolios with extreme weights, which are undesirable in a trading context. Furthermore, the computational advantages of the $\ell_1$-regularized approach make it more suitable for real-world applications, where scalability and robustness are critical.

In conclusion, while cardinality-constrained formulations offer a conceptually appealing way to enforce sparsity, their practical limitations make them less attractive for constructing synthetic controls in pairs trading. The $\ell_1$-regularized approach strikes a better balance between sparsity, interpretability, and computational efficiency, making it the preferred choice for our application.



%\subsection{Cardinality-Constrained Programming}
%\subsubsection{Mixed Integer Quadratic Programming}
%The cardinality-constrained problem can be reformulated as a mixed-integer quadratic program (MIQP) by introducing binary variables $z_i \in \{0,1\}$ that indicate whether asset $i$ is included in the synthetic control:
%
%\begin{equation*}
%\begin{array}{ll}
%\min_{\mathbf{w}, \mathbf{z}} & \sum_{t=1}^T \left(y_{t} - \sum_{i=1}^N w_i x_{it}\right)^2 \\
%\text{subject to} & \mathbf{1}^\top \mathbf{w} = 1 \\
%& -Mz_i \leq w_i \leq Mz_i, \quad i = 1,\ldots,N \\
%& \sum_{i=1}^N z_i \leq K \\
%& z_i \in \{0,1\}, \quad i = 1,\ldots,N
%\end{array}
%\end{equation*}
%
%where $M$ is a sufficiently large constant that bounds the absolute values of the weights. The binary constraints $z_i \in \{0,1\}$ make this problem NP-hard, requiring branch-and-bound techniques for its solution. While modern MIQP solvers (e.g., Gurobi, CPLEX) can handle problems of moderate size, the computational burden becomes prohibitive for large donor pools or when the optimization needs to be performed repeatedly, as in our rolling-window implementation of the pairs trading strategy.
%
%\subsubsection{Iterative Weight Thresholding}
%An alternative approach, which we refer to as Iterative Weight Thresholding (IWT), follows a sequential procedure that approximates the cardinality-constrained solution. The method consists of solving a sequence of unconstrained problems, progressively focusing on the most relevant assets:
%
%\begin{enumerate}
%\item Solve the full least squares problem
%\begin{equation*}
%\mathbf{w}^{(1)} = \argmin_{\mathbf{w} \in \mathbb{R}^{N}} \norm{\mathbf{y} - \mathbf{X}\mathbf{w}}_2^2
%\quad \text{s.t.} \quad \mathbf{1}^\top \mbf w=1
%\end{equation*}
%
%\item Select the $K$ largest weights (in absolute value) from $\mbf w^{(1)}$ to form the support set
%$$\mathcal I:=\{i : |w_i^{(1)}| \text{~among~} K \text{~largest}\}$$
%
%\item Solve the restricted program on support $\mathcal I$
%\begin{equation*}
%\mbf w^{(2)} = \arg \min_{\mbf w_{\mathcal I}\in \mathbb{R}^K} \norm{\mbf y - \mbf X_{\mathcal I}\mbf w_{\mathcal I}}_{2}^{2}
%\quad \text{s.t.} \quad 
%\mbf 1\' \mbf w_{\mathcal I} = 1
%\end{equation*}
%where $\mbf X_{\mathcal{I}} \in \mathbb{R}^{T \times K}$ is the restricted donor matrix and $\mbf w_{\mathcal{I}} \in \mathbb{R}^{K}$ is the restricted weight vector.
%
%\item Construct the full weight vector $\mbf w^* \in \mathbb{R}^{N}$ by embedding the optimized restricted weights:
%\begin{equation*}
%w^*_i = 
%\begin{cases}
%w^{(2)}_j & \text{if}~ i = \mathcal I_j \\
%0 & \text{otherwise}
%\end{cases}
%\end{equation*}
%\end{enumerate}
%
%While both approaches-MIQP and IWT-can theoretically achieve the desired cardinality constraint, they present significant practical challenges in our pairs trading context. The MIQP formulation, although exact, becomes computationally intractable for large donor pools and high-frequency rebalancing. The IWT approach, while computationally efficient, tends to produce extreme weights in our empirical implementation. Specifically, when applying IWT to our dataset, we observed weights often exceeding $\pm 500\%$ of the portfolio value, resulting in highly leveraged positions that would be impractical to implement due to transaction costs and risk management constraints.
%
%These limitations motivate our choice of the $\ell_1$-regularized approach presented in the main text. The lasso penalty provides a convex relaxation of the cardinality constraint that is both computationally efficient and tends to produce more reasonable weight allocations. Moreover, the continuous nature of the $\ell_1$ penalty allows for smoother transitions in portfolio weights over time, reducing turnover compared to the discrete selection methods discussed above.
%----------------------------------------------------

%----------------------------------------------------
%%%%%%%%%%%%%%%%%%%%%%%%%%%%%%%%%%%%%%%%%%%%%%%%%%%%%
\subsection{Algorithms}
%%%%%%%%%%%%%%%%%%%%%%%%%%%%%%%%%%%%%%%%%%%%%%%%%%%%%




%\begin{algorithm}[H]
\caption{Mispricing Detective}
\label{alg:mispricing_detective}
\begin{algorithmic}[1]
%----------------------------------------------------
\Require
price time series of the target asset $\{p^{\t{trgt}}_{t}\}_{t \in \T}$;
price time series of the synthetic pool $\{p^{\t{synth}}_{t}\}_{i\in\D, t \in \T}$

%----------------------------------------------------
\mx 
\Ensure cumulative mispricing indices 
$\{
CMI^{\t{trgt}|\t{synth}}
\}_{t \in \T}
, 
\{
CMI^{\t{synth}|\t{trgt}}
\}_{t \in \T}$
%----------------------------------------------------
\mx 
\Function{SyntheticControlBuilder}{$\{p^{\t{trgt}}_{t}\}_{t \in \T}, \{p^{i}_{t}\}_{i\in\D, t \in \T}$}
$$
~~\{ \hat w^{i} \}_{i\in \D}
\gets  \arg \min_
% {\mbf w} 
{\{ \hat w^i \}_{i\in\D}}
\bigg{\{}
\sum_{t\in \mathcal T^{tr}}
\left(
\log p_{t}^\t{trgt}
- 
\sum_{i\in \mathcal D}
w^i \log p_{t}^i
\right)^2
+ 
\lambda \sum_{i\in\mathcal D} |w^i|
\bigg{\}}
\quad 
\t{s.t.}
\quad 
\sum_{i\in\mathcal D} w^i=1
%\\[1em]
%\mbf w_{\t{sc}} &= \{w_i^* : w_i^* \geq 10^{-1}\}
.
$$

$
\{ p^{\t{synth}}_t \}_{t\in\T} \gets  
\{\sum_{i\in \mathcal D} \hat w^i p_{t}^i \}_{t\in\T}
$
%----------------------------------------------------
\mx 
    \State \Return $\{ p^{\t{synth}}_t \}_{t\in\T}$
\EndFunction
\end{algorithmic}
\end{algorithm}
%----------------------------------------------------


%----------------------------------------------------
\subsection{Barra model}


Before proceeding with the formulation of our pairs trading approach, it is essential to understand the economic drivers behind the relative performance between our target and synthetic assets. While our sparse synthetic control methodology creates a close replicating portfolio, any pairs trading strategy fundamentally relies on exploiting temporary divergences in pricing relationships. Therefore, a rigorous factor-based analysis provides critical insights into the structural sources of these divergences.

To this end, we employ the Barra factor model framework, which decomposes the returns of both target and synthetic assets into systematic components (fundamental and industry factors) and idiosyncratic components. This decomposition serves multiple purposes in our research:

First, it provides economic interpretation of the spread returns by quantifying how much of the relative performance is driven by different factor exposures versus pure alpha. Second, it validates the quality of our synthetic control construction by measuring how closely the factor exposures match between target and synthetic assets. Third, it identifies specific factor tilts that persist even after optimization, potentially representing structural drivers of spread returns that our trading strategy can exploit.

By understanding these factor relationships before developing trading signals, we can distinguish between transient mispricings (which represent opportunities) and permanent structural differences (which represent risks). This factor-based foundation also helps explain why certain statistical relationships identified by our copula models may exist, providing theoretical underpinning to the empirical patterns we observe in subsequent sections.

%The Barra model thus serves as a critical bridge between our synthetic asset construction and trading signal generation, ensuring that our overall approach is grounded in economic intuition rather than relying solely on statistical patterns.

The Barra model for our target and synthetic asset may be written as
\begin{align*}
\2{\v{r_t \\ r_t^*}} =
\2{\v{\alpha \\ \alpha^*}} 
+ 
\2{\v{\b \beta\' \\ \b \beta^{*\'}}} 
\mbf f_t
+
\2{\v{\b \gamma\' \\ \b \gamma^{*\'}}} 
\mbf i_t
+
\2{\v{\eps_t \\ \eps_t^*}} 
\end{align*}

where we consider $K=8$ fundamental factors $\mbf f_t$ (i.e.: $\b \beta, \b \beta^*, \mbf f_t \in \mathbb R^K$) and $M=17$ industry factors $\mbf i_t$ (i.e.: $\b \gamma, \b \gamma^*, \mbf i_t \in \mathbb R^M$).
%
The \qquote{active return} between the target and synthetic asset is given by:
$$
\dot{r}_t := r_t - r_t^* = (\alpha - \alpha^*) + (\b \beta - \b \beta^*)\'\mbf f_t +  (\b \gamma - \b \gamma^*)\'\mbf i_t + (\eps_t - \eps_t^*)
.
$$
Now defining the \textit{relative alpha, beta and gamma}, respectively, as
$
\dot{\alpha}:= (\alpha - \alpha^*),
\dot{\b \beta} := (\b \beta - \b \beta^*),
\dot{\b \gamma} := (\b \gamma - \b \gamma^*)
$
and setting $\dot{\eps}_t := (\eps_t - \eps_t^*)$, we may write the model in terms of the portfolio's active return
\begin{equation}\label{eq:barra}
\dot{r}_t = \dot{\alpha} + \dot{\b \beta}\' \mbf f_t + \dot{\b \gamma}\' \mbf i_t + \dot{\eps}_t
.
\end{equation}

In \cref{fig:factor_corr_matrix} we show the factor correlation matrix $\t{Corr}(\mbf X)\in\mathbb{R}^{J\times J}$ of all the factors
$$
\mbf{X} 
= \2{\v{\mbf f_1\', \\ \vdots \\ \mbf f_T\', } 
~
% \c 
\v{\mbf i_1\' \\ \vdots \\ \mbf i_T\' }}
\in\mathbb R^{T\times J}
,
$$
where $J=K+M$. 
In our application we are using 
[\texttt{MKT\_RF}, \texttt{SMB}, \texttt{HML}, \texttt{RMW}, \texttt{CMA}, \texttt{MOM}, \texttt{ST\_REV}, \texttt{LT\_REV}] as the fundamental factors, and [\texttt{Food}, \texttt{Mines}, \texttt{Oil}, \texttt{Clths}, \texttt{Durbl}, \texttt{Chems}, \texttt{Cnsum}, \texttt{Cnstr}, \texttt{Steel}, \texttt{FabPr}, \texttt{Machn}, \texttt{Cars}, \texttt{Trans}, \texttt{Utils}, \texttt{Rtail}, \texttt{Finan}, \texttt{Other}] as the industry factors. 
As we can see, correlations are very high among factors, specially among industry factors, which means that regular OLS estimation of \cref{eq:barra} will deliver highly unstable coefficients due to multicollinearity.



%==============[	  FACTOR CORRELATION MATRIX  ]==============
\inserthere{fig:factor_corr_matrix}
\begin{figure}[H]
  \centering
  \caption{Factor Correlation Matrix}
  %----------------------------------------------------
  \begin{subfigure}{\textwidth}
  \centering	
  \caption{Train}
  \includegraphics[scale=0.5]{/Users/jesusvillotamiranda/Library/CloudStorage/OneDrive-UniversidaddeLaRioja/GitHub/Repository/arbitragelab-master/__OUTPUT_TeX__/figures/Factor_Correlation_Matrix_(17_Ind)_Train.pdf}
  \label{subfig:factor_corr_matrix_train}
  \end{subfigure}

	\vspace{0.5cm} % Adjust the spacing as needed

  \begin{subfigure}{\textwidth}
  \centering
  \caption{Test}
  \includegraphics[scale=0.5]{/Users/jesusvillotamiranda/Library/CloudStorage/OneDrive-UniversidaddeLaRioja/GitHub/Repository/arbitragelab-master/__OUTPUT_TeX__/figures/Factor_Correlation_Matrix_(17_Ind)_Test.pdf}
  \label{subfig:factor_corr_matrix_test}
  \end{subfigure}
%----------------------------------------------------
\label{fig:factor_corr_matrix}
\end{figure}


Hence, to properly estimate the model parameters, we employ an orthogonal regression approach based on Principal Component Analysis (PCA), which will allow us to obtain more stable estimates of the factor exposures. The implementation follows these steps. 

%==============[	  Standardization  ]==============
%First, we standardize all factors (both fundamental and industry factors) to have zero mean and unit variance, which yields
%$
%%\tilde{\mbf{x}}_t = (\mbf{x}_t - \bar{\mbf{x}}) / \sigma_{\mbf{x}}.
%\tilde{\mbf X}\in \mathbb{R}^{T\times J}.
%$
%
%==============[	  Principal Component Analysis  ]==============
First, we compute the covariance matrix of the factors $\mbf \Sigma := \text{Cov}({\mbf{X}})\in \mathbb R^{J\times J}$ and obtain its eigendecomposition
$
\mbf \Sigma \mbf V = \mbf \Lambda \mbf V,
$
where $\mbf \Lambda:=\diag(\lambda_1,...,\lambda_J)\in \mathbb R^{J\times J}$ are the eigenvalues and $\mbf{V} := [\mbf{v}_1,...,\mbf{v}_J]\in \mathbb R^{J\times J}$ are the corresponding eigenvectors, both sorted in descending order of the $\lambda$'s. 
The principal components are given by:
$
%\mbf{p}_t = \mbf{V}\' \tilde{\mbf{x}}_t.
\mbf P = {\mbf X} \mbf V \in \mathbb{R}^{T\times J}.
$

%========[	  Principal Component Regression (Unrestricted)  ]=========
Second, we regress the active returns onto the principal components
$$
\dot{r}_t = a^{(u)} + \sum_{i=1}^J b^{(u)}_i p_{t,i} + \nu_t^{(u)}
%\dot{r}_t = a^{(u)} + \mbf p_t\' \b b^{(u)} + \nu_t^{(u)}
$$
where $a^{(u)}$ is the \qquote{unrestricted} intercept, $\b b^{(u)}:=[b^{(u)}_1,...,b^{(u)}_J]$ are the \qquote{unrestricted} coefficients for each principal component, and $\nu_t$ is the error term.
%
%==============[	  Selection of Significant Components  ]==============
We keep only the statistically significant principal components at the $0.05$ significance level:
$
\mathcal{S} := \{i : p\text{-value}(b_i^{(u)}) < 0.05\}.
$
%
%========[	  Principal Component Regression (Restricted)  ]=========
Then, we estimate a restricted model using only the significant principal components
$$
\dot{r}_t = a^{(r)} + \sum_{i \in \mathcal{S}} b_i^{(r)} p_{t,i} + \nu_t^{(r)}.
$$
    
%==============[	  Transformation to Factor Space  ]==============
Finally, we transform the coefficients back to original factor space. Let $\b b^{(r)}\in \mathbb R^J$ denote the vector filled with $b_i^{(r)}$ if $i\in \mathcal S$ and 0 otherwise. Then, we can write
$
\dot r_t 
= a^{(r)} + \mbf p_t\' \b b^{(r)}+\nu_t^{(r)} 
= a^{(r)} + {\mbf x}_t\' \mbf V \b b^{(r)}+\nu_t^{(r)}
,
$
where $\mbf p_t$ and $\mbf x_t$ are rows of $\mbf P$ and $\mbf X$, respectively (given as column vectors).
Thus, by setting $\dot \alpha = a^{(r)}$ and
$
% 2{\v{\dot{\b \beta} \\ \dot{\b \gamma}}} = \mbf V \b b^{(r)}
[\v{\dot{\b \beta} ~ \dot{\b \gamma}}]\' = \mbf V \b b^{(r)}
$
we recover alpha and the factor betas and gammas while avoiding the instability due to multicollinearity.
%==============[	  HAC Standard Errors  ]==============
Both the unrestricted and restricted models are estimated with Heteroskedasticity and Autocorrelation Consistent (HAC) standard errors using a maximum lag of 5 periods to account for potential serial correlation and heteroskedasticity in the residuals. 

%==============[	  Advantages of this procedure  ]==============
%This approach offers several advantages. First, by using orthogonal principal components, we eliminate multicollinearity concerns. Second, by selecting only significant components, we reduce dimensionality and potential overfitting. Finally, the transformation back to the original factor space allows for direct interpretation of the factor exposures $\dot{\b  \beta}$ and $\dot{\b \gamma}$ in our active return decomposition model.

%For robustness, we repeat this procedure with various industry factor classifications, ranging from 10 to 49 industries, to assess the sensitivity of our results to the industry granularity. If no principal components are found to be statistically significant at the 5\% level, we default to an intercept-only model, effectively attributing the entire active return to the $\dot{\alpha}$ term.

%----------------------------------------------------
\input{/Users/jesusvillotamiranda/Library/CloudStorage/OneDrive-UniversidaddeLaRioja/GitHub/Repository/arbitragelab-master/__OUTPUT__/barra_table.tex}
%----------------------------------------------------


\cref{tab:barra_model} presents the results of our Barra model decomposition using the Principal Component Regression (PCR) approach described above. This table provides insights into the factor exposures that drive the relative performance between our target and synthetic assets.

The relative alpha ($\dot \alpha$) represents the portion of active returns not explained by factor exposures. In both the training and testing periods, the relative alpha is not statistically significant ($p$-values of 0.6766 and 0.1419, respectively), suggesting that factor exposures, rather than idiosyncratic effects, are the primary drivers of the spread between target and synthetic assets. The alpha changes from -0.0002 in the training period to 0.0009 in the testing period, though this difference remains statistically insignificant.

Among the fundamental factors, several notable patterns emerge. The profitability factor (\texttt{RMW}) exhibits the largest positive exposure in the training period (0.6985), indicating that the target asset tends to outperform the synthetic asset when profitability is rewarded in the market. In the testing period, the most substantial exposure shifts to the long-term reversal factor (\texttt{LT\_REV}) with a coefficient of 0.4929. Interestingly, the value factor (\texttt{HML}) shows a consistent negative exposure across both periods, becoming more pronounced in the testing period (-0.4563). This suggests that the target asset tends to underperform the synthetic asset during periods when value stocks outperform growth stocks.

The industry factor exposures reveal significant sector-based drivers of the spread returns. The machinery sector (\texttt{Machn}) shows the strongest positive exposure in both periods, dramatically increasing from 0.4337 in training to 1.0429 in testing. This indicates that the target asset has significantly higher exposure to the machinery sector compared to the synthetic asset. Conversely, the retail sector (\texttt{Rtail}) demonstrates the most substantial negative exposure in both periods (-0.5772 and -0.6114), suggesting that the synthetic asset has greater retail exposure than the target. Other sectors with notable negative exposures include \texttt{Steel} (-0.3551 and -0.4512) and \texttt{Durables} (-0.2761 and -0.1056).

%Some industry exposures exhibit interesting shifts between periods. The transportation sector (\texttt{Trans}) reverses from -0.1163 in training to 0.2022 in testing, while the construction sector (\texttt{Cnstr}) shifts from a substantial positive exposure (0.2890) to a slight negative one (-0.0143). These changes reflect the dynamic nature of sector relationships between the target and synthetic assets over time.

The model's explanatory power improves markedly from the training to testing period, with the adjusted $R^2$ increasing from 0.0687 to 0.2525. Both models are statistically significant based on their $F$-statistics (10.5893 and 19.7223) with $p$-values below 0.0001. This improvement in fit suggests that the factor structure identified during the training period becomes more pronounced during the testing period, potentially enhancing the strategy's effectiveness.

These results demonstrate that while our synthetic control methodology successfully creates a close match to the target asset, systematic differences in factor exposures remain that can be exploited by our trading strategy. In particular, the significant exposures to fundamental factors like profitability and long-term reversal, alongside pronounced industry tilts, represent potential sources of alpha that our pairs trading approach can capitalize on.
%----------------------------------------------------

\end{document}