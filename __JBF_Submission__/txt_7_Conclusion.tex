%----------------------------------------------------
\section{Conclusion}
%----------------------------------------------------
This paper investigates how information from business news affects stock market prices. We analyze a dataset of Spanish business articles during a particularly volatile period-the COVID-19 pandemic-and examine firm-specific stock market reactions to news. We show that transforming text into vector embeddings and clustering them using KMeans yields clusters that are firm-specific and industry-specific. However, the distribution of articles across clusters is unstable over sequential data splits, indicating temporal instability. When we implement a cluster-based trading strategy-similar to portfolio sorts-on the KMeans clusters, we observe an over-reliance on the past performance of a cluster. That is, signals are short-lived due to temporal instability. Consequently, the out-of-sample profitability of the trading strategy is negligible, evidencing the method's poor temporal generalizability. Therefore, a model based on embeddings is superficial and is not able to anticipate market trends.

%----------------------------------------------------
Alternatively, we develop a novel approach by guiding a Large Language Model (LLM) through a structured news-parsing schema, enabling it to analyze news-implied firm-specific economic shocks. The schema involves identifying the firms affected by the articles and classifying the implied shocks on such firms by their type, magnitude, and direction. This LLM-based methodology demonstrates several advantages over the traditional clustering approach. Even in a volatile period, it produces stable distributions of articles across clusters in sequential splits, demonstrating robust temporal stability. Moreover, the resulting trading signals are both long-lasting and economically relevant, as they are based on fundamental economic shocks rather than statistical patterns. The results show that the LLM-based trading strategy effectively identifies winners and losers, illustrating the parser's ability to anticipate market trends by comprehending the economic implications of firm-specific shocks. This approach generates a consistent profile of earnings in the test set, with results robust to the choice of hyperparameters-the holding period length of the trading strategy and the number of selected clusters for trading. Our findings demonstrate a promising avenue: LLMs, when guided by appropriate economic frameworks, can help predict market reactions to news through systematic classification of economic shocks embedded in financial narratives.
