\documentclass[12pt,article]{memoir}
\usepackage{/Users/jesusvillotamiranda/Documents/LaTeX/$$JVM_Macros}
\Subject{A mathematical model for Business News Articles}
%\Arg{Notes}


\begin{document}

\href{https://chatgpt.com/share/56b67d2e-9d0e-4cb3-b553-81f40b76fa03}{ChatGPT's conversation}

\section*{Modeling Price Dynamics Influenced by News Using SDEs}

\subsection*{1. Defining the Framework}

\subsubsection*{1.1. Price Process \(P(t)\)}

Let \(P(t)\) denote the price of an asset at time \(t\). The price process \(P(t)\) is typically modeled as a continuous-time stochastic process, which evolves according to a Stochastic Differential Equation (SDE). In our case, we want to model this process under the influence of news events.

\subsubsection*{1.2. News Event Process \(E(t)\)}

Let \(E(t)\) represent the process governing news events. The news process could include discrete events occurring at specific times or a continuous flow of information. Each news event \(E(t)\) can be characterized by:
\begin{itemize}
    \item \textbf{Content Vector \(C(t)\)}: A vector representing the characteristics of the news (e.g., sentiment, relevance).
    \item \textbf{Intensity \(I(t)\)}: A scalar representing the impact or intensity of the news.
    \item \textbf{Timing \(t_i\)}: The time at which the news event occurs.
\end{itemize}

\subsubsection*{1.3. Market Impact Function \(f(E(t))\)}

The market impact function \(f(E(t))\) represents the effect of the news event on the asset price. It can be thought of as a mapping from the news event characteristics to the drift and volatility terms in the price process.

\subsection*{2. Stochastic Differential Equation (SDE) for Price Dynamics}

We model the asset price \(P(t)\) using an SDE, which incorporates the effects of the news events:

\[
dP(t) = \mu(t, P(t), E(t)) \, dt + \sigma(t, P(t), E(t)) \, dW(t)
\]

where:
\begin{itemize}
    \item \(\mu(t, P(t), E(t))\) is the drift term, which represents the expected rate of change of the price due to both market factors and the news.
    \item \(\sigma(t, P(t), E(t))\) is the diffusion term, representing the volatility of the price, which can also be influenced by news.
    \item \(W(t)\) is a Wiener process (standard Brownian motion), representing the stochastic component of the price dynamics.
\end{itemize}

\subsection*{3. Modeling the Drift Term \(\mu(t, P(t), E(t))\)}

The drift term \(\mu(t, P(t), E(t))\) captures the deterministic trend in the price evolution, which is affected by the news:

\[
\mu(t, P(t), E(t)) = \mu_0(t, P(t)) + f_{\mu}(E(t))
\]

where:
\begin{itemize}
    \item \(\mu_0(t, P(t))\) is the baseline drift term, representing the natural price evolution in the absence of news. This might be driven by factors such as interest rates, dividends, or other market forces.
    \item \(f_{\mu}(E(t))\) is the news impact component of the drift, which models how the news event \(E(t)\) influences the expected return. For example, positive news might increase the drift, leading to an upward price movement.
\end{itemize}

\subsection*{4. Modeling the Volatility Term \(\sigma(t, P(t), E(t))\)}

The diffusion term \(\sigma(t, P(t), E(t))\) represents the uncertainty or volatility in the price process:

\[
\sigma(t, P(t), E(t)) = \sigma_0(t, P(t)) + f_{\sigma}(E(t))
\]

where:
\begin{itemize}
    \item \(\sigma_0(t, P(t))\) is the baseline volatility, representing the natural market volatility without news.
    \item \(f_{\sigma}(E(t))\) is the news impact component of the volatility, which models how the news event affects the uncertainty of future price movements. For instance, breaking news might increase volatility temporarily.
\end{itemize}

\subsection*{5. Incorporating Discrete News Events}

If news events occur at discrete times \(t_i\), we can modify the SDE to include jump terms:

\[
dP(t) = \mu(t, P(t), E(t)) \, dt + \sigma(t, P(t), E(t)) \, dW(t) + \sum_{t_i \leq t} J_i \, \delta(t - t_i)
\]

where:
\begin{itemize}
    \item \(J_i\) represents the jump magnitude at the time \(t_i\), which can be modeled as a function of the news event \(E(t_i)\).
    \item \(\delta(t - t_i)\) is the Dirac delta function, ensuring that the jump occurs precisely at time \(t_i\).
\end{itemize}

This jump-diffusion model allows us to capture sudden price changes due to significant news events.

\subsection*{6. Modeling the News Impact Function \(f(E(t))\)}

The functions \(f_{\mu}(E(t))\) and \(f_{\sigma}(E(t))\) represent how the characteristics of the news event influence the drift and volatility. These can be modeled as:

\[
f_{\mu}(E(t)) = \beta_{\mu}^T C(t) I(t)
\]

\[
f_{\sigma}(E(t)) = \beta_{\sigma}^T C(t) I(t)
\]

where:
\begin{itemize}
    \item \(\beta_{\mu}\) and \(\beta_{\sigma}\) are vectors of coefficients that quantify the sensitivity of the drift and volatility to the news content \(C(t)\) and intensity \(I(t)\).
    \item The dot product \(\beta_{\mu}^T C(t)\) represents the weighted influence of the various aspects of the news on the drift.
\end{itemize}

\subsection*{7. Calibration and Estimation}

To apply this model, we need to estimate the parameters \(\mu_0(t, P(t))\), \(\sigma_0(t, P(t))\), \(\beta_{\mu}\), and \(\beta_{\sigma}\) using historical 

\end{document}












