\section{Conclusions} \label{sec:conclusion}

This paper introduces a novel pairs trading framework that integrates sparse synthetic control methods with copula-based dependence modeling. Our methodology addresses several key limitations of traditional pairs trading approaches by combining adaptive synthetic asset construction with flexible dependence modeling. The empirical results demonstrate the effectiveness of this integrated approach across multiple dimensions.

The sparse synthetic control methodology successfully constructs parsimonious tracking portfolios from a broad donor pool, automatically identifying the most influential assets while maintaining interpretability. By employing $\ell_1$-regularization rather than explicit cardinality constraints, our approach achieves computational efficiency without sacrificing robustness. The empirical application to S\&P 500 constituents shows that relatively few donor assets (27 in our case) are sufficient to create effective synthetic controls.

The copula-based dependence modeling framework proves particularly valuable in capturing complex relationships between target and synthetic assets. Our analysis reveals that elliptical copulas, particularly the Student-t specification, provide the best fit for modeling return dependencies. This finding suggests that while return relationships are predominantly symmetric, they exhibit heavier tails than implied by Gaussian dependence. The superior performance of the Student-t copula over simpler specifications validates the importance of accommodating tail dependence in pairs trading strategies.

The trading strategy's performance metrics across different copula specifications demonstrate the framework's robustness. The N14 mixed copula achieves the highest annualized return (17.26\%) and Sharpe ratio (3.97), while maintaining moderate volatility (4.35\%). Notably, all specifications deliver positive risk-adjusted returns, with Sharpe ratios ranging from 3.14 to 3.97, suggesting that the strategy's profitability is not overly dependent on the specific choice of copula.

Several directions for future research emerge from this work. First, exploring time-varying copulas could help capture evolving dependence structures in dynamic market conditions. Second, extending the framework to handle multiple target assets simultaneously could enhance portfolio diversification. Finally, incorporating transaction costs and market impact models would provide more realistic performance estimates for practical implementation.