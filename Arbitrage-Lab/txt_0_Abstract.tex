Financial markets frequently exhibit transient price divergences between economically linked assets, yet traditional pairs trading strategies struggle to adapt to structural breaks and complex dependencies, limiting their robustness in dynamic regimes. 
%
This paper addresses these challenges by developing a novel framework that integrates sparse synthetic control with copula-based dependence modeling to enhance adaptability and risk management. 
%
Economically, our approach responds to the need for strategies that systematically identify latent linkages while mitigating overfitting in high-dimensional asset pools. 
%
The sparse synthetic control methodology constructs a parsimonious synthetic asset via a constrained linear combination of candidates from a broad donor pool, automating pair selection while prioritizing interpretability and computational efficiency. 
%
By embedding this within a copula-based dependence framework, we capture non-linear and tail dependencies between target and synthetic assets. 
%
Trading signals, grounded in the relative mispricing between these assets, employ a cumulative index that resets after position closures to isolate episodic opportunities, with disciplined entry rules requiring concurrent misalignment signals to filter noise. 
%
Empirical analysis demonstrates the superior performance of our approach across diverse market conditions. 