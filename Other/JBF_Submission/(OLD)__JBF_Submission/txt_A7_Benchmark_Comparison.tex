\subsection{Why not using a different benchmark?} \label{sec:A7}

In evaluating our novel Large Language Model (LLM) methodology for classifying news-implied firm-specific shocks, it is imperative to establish a robust and relevant benchmark. Our chosen benchmark involves transforming news articles into high-dimensional vector embeddings followed by clustering these embeddings using the KMeans algorithm. This section delineates the rationale behind selecting KMeans clustering of vector embeddings over other potential benchmarks such as sentiment analysis and topic modeling.

%%%%%%%%%%%%%%%%%%%%%%%%%%%%%%%%%%%%%%%%%%%%%%%%%%%%%
%%%%%%%%%%%%%%%%%%%%%%%%%%%%%%%%%%%%%%%%%%%%%%%%%%%%%
%%%%%%%%%%%%%%%%%%%%%%%%%%%%%%%%%%%%%%%%%%%%%%%%%%%%%
%%%%%%%%%%%%%%%%%%%%%%%%%%%%%%%%%%%%%%%%%%%%%%%%%%%%%

\subsubsection*{Why not Sentiment Analysis as a benchmark?}

Sentiment analysis is a widely recognized technique in natural language processing that aims to determine the emotional tone conveyed in a text, typically categorizing content as positive, negative, or neutral. While sentiment analysis provides a straightforward approach to gauging the general tone of news articles, it falls short in several critical aspects when juxtaposed with our objectives.

%\paragraph{Lack of Granularity}

First, sentiment analysis is not sufficiently granular. Our LLM methodology classifies news articles into 20 distinct categories of economic shocks while sentiment analysis classifies articles in a coarse manner, typically into positive, negative, or neutral categories, which is inadequate for benchmarking a detailed classification model. 

%\paragraph{Economic Irrelevance}

Second, sentiment analysis predominantly focuses on the linguistic and emotional aspects of the text, which do not necessarily correlate with the economic impact on firms. For instance, a neutral-toned article could describe a significant economic event, while a positive sentiment might not always translate to favorable economic outcomes. Consequently, the sentiment does not provide direct insights into the economic consequences, making it an economically irrelevant benchmark for our purposes.

%\paragraph{Sensitivity to Linguistic Nuances}

Third, sentiment analysis algorithms are often sensitive to linguistic subtleties, leading to inconsistent results across different languages and contexts. For example, sarcasm or idiomatic expressions can distort sentiment scores, undermining the reliability of sentiment analysis as a benchmark. 
This variability poses a challenge for standardization, especially in a multilingual context. For instance, the sentiment derived from analyzing the text in English may significantly differ from the sentiment in Spanish. 

%\paragraph{Lack of Robustness}

Fourth, sentiment analysis is not robust in the sense that different sentiment analysis tools yield divergent assessments of the same text. As shown below, we observe considerable differences in the identified sentiment when applying multiple sentiment analysis providers to a specific article. This lack of consistency undermines the reliability of sentiment analysis as a benchmark, making it unsuitable for our purposes.

\begin{quote}
\textit{
Sentiment analysis is highly sensitive to the specific tool
or model employed. Here, we demonstrate this by analyzing a piece of
business news using various popular sentiment analysis tools:
\texttt{TextBlob}, \texttt{text2data}, \texttt{VADER}, and
\texttt{FinBERT}. The methods vary significantly in both their approach
to sentiment determination and the output they provide, as illustrated
below.}%------------------- BEGIN FOOTNOTE -------------------------
\footnote{Note that applying Loughran-Macdonald is not recommended in for short texts as it yields sparse results. For example, in the example we are considering, it outputs a category distribution that only loads on \qquote{Strong Modal}, which is not a really useful analysis.
 
\texttt{
LM\_Scores = \{'Negative': 0, 'Positive': 0, 'Uncertainty': 0, 'Litigious': 0, 
'Strong\_Modal': 2, 'Weak\_Modal': 0, 'Constraining': 0, 'Complexity': 0\}}
}
\end{quote}
%--------------------- END FOOTNOTE -----------------------------

%%%%%%%%%%%%%%%%%%%%%%%%%%%%%%%%%%%%%%%%%%%%%%%%%%%%%

\begin{news}
    [A news article about Telef�nica and Cellnex | Sentiment: \texttt{TextBlob}]  
    [news:cellnex-article]                            
    {\green{Cellnex will face more competition in Europe} \resubp[dark_green]{\text{~\textnormal{\textbf{Score: 0.50}}~}}} 
    \bblue{Telef�nica's (TEF.MC) subsidiary, Telxius Telecom, has agreed to sell its telecommunications tower division in Europe and Latin America to American Tower (AMT), which will expand the latter's presence in Europe and increase competition for the Spanish wireless telecommunications group Cellnex Telecom (CLNX.MC), according to Equita Sim.}
    \resubp[blue]{\text{~\textnormal{\textbf{Score: 0.00}}~}}\bluegreen{The transaction "represents the entry of a new independent tower operator into the Spanish market and potentially more competition for future growth in the European market as well," says the brokerage firm.} \resubp[bluegreen]{\text{~\textnormal{\textbf{Score: 0.06}}~}} 


\vspace{0.5cm}
{\centering  
\textnormal{\textsc{Overall}} 
 \resubp[bluegreen]{\text{~\textnormal{\textbf{Score: 0.085}}~}}
\par}
\end{news}

\begin{quote}
\textit{Note: \texttt{TextBlob} is a general-purpose sentiment analysis tool that relies on a pre-built lexicon to assess the polarity of the text. It computes a sentiment score ranging from -1 to 1, where -1 signifies a negative sentiment, 1 indicates a positive sentiment, and 0 represents a neutral sentiment. The methodology behind \texttt{TextBlob} focuses on tokenizing the input into words and phrases, which are compared against its built-in polarity dictionary.}
\end{quote}

%%%%%%%%%%%%%%%%%%%%%%%%%%%%%%%%%%%%%%%%%%%%%%%%%%%%%

\begin{news}
    [A news article about Telef�nica and Cellnex | Sentiment: \texttt{text2data}]                         
    {\bblue{Cellnex will face more competition in Europe} \resubp[blue]{\text{~\textnormal{\textbf{Score: 0.145}}~}}
    }
    \red{Telef�nica's (TEF.MC) subsidiary, Telxius Telecom, has agreed to sell its telecommunications tower division in Europe and Latin America to American Tower (AMT), which will expand the latter's presence in Europe and increase competition for the Spanish wireless telecommunications group Cellnex Telecom (CLNX.MC), according to Equita Sim.} \resubp[red]{\text{~\textnormal{\textbf{Score: -0.512}}~}} \red{The transaction "represents the entry of a new independent tower operator into the Spanish market and potentially more competition for future growth in the European market as well," says the brokerage firm. }	\resubp[red]{\text{~\textnormal{\textbf{Score: -0.560}}~}}
    
\vspace{0.5cm}
{\centering  
\textnormal{\textsc{Overall}} 
 \resubp[red]{\text{~\textnormal{\textbf{Score: -0.61}}~}}
\par}
\end{news}
\begin{quote}
\textit{Note: \texttt{text2data} employs scientific deep learning NLP methods to analyze sentiment. Every sentence is split into smaller chunks and represented as a tree structure, capturing the syntactic relationships between words and phrases. To determine the final sentiment score, \texttt{text2data} uses probabilistic methods based on a pre-trained data model, providing an output score between -1 and 1, where -1 is negative and 1 is positive. 
}\end{quote}

%%%%%%%%%%%%%%%%%%%%%%%%%%%%%%%%%%%%%%%%%%%%%%%%%%%%%

\begin{news}
    [A news article about Telef�nica and Cellnex | Sentiment: \texttt{VADER}]  
    [news:cellnex-article]                            
    {\bblue{Cellnex will face more competition in Europe} \resubp[blue]{\text{~\textnormal{\textbf{Score: 0.00}}~}}} 
    \green{Telef�nica's (TEF.MC) subsidiary, Telxius Telecom, has agreed to sell its telecommunications tower division in Europe and Latin America to American Tower (AMT), which will expand the latter's presence in Europe and increase competition for the Spanish wireless telecommunications group Cellnex Telecom (CLNX.MC), according to Equita Sim.}
    \resubp[dark_green]{\text{~\textnormal{\textbf{Score: 0.69}}~}}\green{The transaction "represents the entry of a new independent tower operator into the Spanish market and potentially more competition for future growth in the European market as well," says the brokerage firm.} \resubp[dark_green]{\text{~\textnormal{\textbf{Score: 0.57}}~}} 


\vspace{0.5cm}
{\centering  
\textnormal{\textsc{Overall}} 
 \resubp[dark_green]{\text{~\textnormal{\textbf{Score: 0.81}}~}}
\par}
\end{news}

\begin{quote}
\textit{Note: 
\texttt{VADER} (Valence Aware Dictionary and sEntiment Reasoner) is a lexicon and rule-based sentiment analysis tool 
%that is particularly effective for analyzing social media and other forms of short text. It 
uses a combination of lexical features (i.e., words) that are generally classified as having positive, negative, or neutral valence. \texttt{VADER} produces four sentiment metrics: positive, negative, neutral, and a compound score. The compound score is a normalized, weighted composite score that ranges from -1 to 1, indicating the overall sentiment of the text. In this example, we provide the compound measure sentence by sentence and for the whole text.
}
\end{quote}

%%%%%%%%%%%%%%%%%%%%%%%%%%%%%%%%%%%%%%%%%%%%%%%%%%%%%

\newpage
\begin{news}
    [A news article about Telef�nica and Cellnex | Sentiment: \texttt{FinBERT}]  
    [news:cellnex-article]                            
    {\red{Cellnex will face more competition in Europe} \resubp[red]{\text{~\textnormal{\textbf{Negative, 0.75}}~}} } 
    \bblue{Telef�nica's (TEF.MC) subsidiary, Telxius Telecom, has agreed to sell its telecommunications tower division in Europe and Latin America to American Tower (AMT), which will expand the latter's presence in Europe and increase competition for the Spanish wireless telecommunications group Cellnex Telecom (CLNX.MC), according to Equita Sim.} \resubp[blue]{\text{~\textnormal{\textbf{Neutral, 0.98}}~}} \red{The transaction "represents the entry of a new independent tower operator into the Spanish market and potentially more competition for future growth in the European market as well," says the brokerage firm.}	\resubp[red]{\text{~\textnormal{\textbf{Negative, 0.81}}~}}

\vspace{0.5cm}
{\centering  
\textnormal{\textsc{Overall}} 
 \resubp[red]{\text{~\textnormal{\textbf{Negative, 0.94}}~}}
\par}
\end{news}
\begin{quote}
\textit{Note: \texttt{FinBERT} is a domain-specific transformer-based model trained on financial texts. Unlike the previous models, \texttt{FinBERT} provides both a sentiment classification (Positive, Negative, Neutral) and a confidence score ranging from 0 to 1, representing the model's certainty about the sentiment classification.}
\end{quote}



%%%%%%%%%%%%%%%%%%%%%%%%%%%%%%%%%%%%%%%%%%%%%%%%%%%%%
%%%%%%%%%%%%%%%%%%%%%%%%%%%%%%%%%%%%%%%%%%%%%%%%%%%%%
%%%%%%%%%%%%%%%%%%%%%%%%%%%%%%%%%%%%%%%%%%%%%%%%%%%%%
%%%%%%%%%%%%%%%%%%%%%%%%%%%%%%%%%%%%%%%%%%%%%%%%%%%%%

\subsubsection*{Why not Topic Modeling as a benchmark?}

Topic modeling, particularly techniques like Latent Dirichlet Allocation (LDA), decomposes text into a set of latent topics based on word co-occurrences. Topic modelling  offer a more granular approach compared to sentiment analysis and could potentially offer a valid benchmark for our purpose. However, we argue that transforming news articles into vector embeddings and subsequently clustering them using KMeans offers a more balanced approach than topic modeling.

%\paragraph{Enhanced Semantic Representation}
Topic models rely on bag-of-words representations, which disregard the order and context of words. This limitation hampers the model's ability to capture complex semantic relationships and contextual nuances essential for accurately identifying economic shocks. Consequently, topic models may overlook subtle but economically significant information present in the text. On the other hand, vector embeddings encapsulate rich semantic information by capturing the relationships between words in a continuous vector space. Unlike topic models, which are confined to word co-occurrences, embedding models, particularly transformer-based, generate context-dependent representations, allowing for a nuanced understanding of polysemy and context. This means that the same word can have different embeddings depending on the context of the sentence, such as "Apple" in "Apple is a leading tech company" versus "Apple is a type of fruit." 

%\paragraph{Scalability and Flexibility}

An important advantage of vector embeddings is that they scale efficiently with large corpora and can be generated at various granularities, including word, sentence, or document levels. This scalability makes embeddings highly adaptable for diverse downstream tasks such as clustering, classification, and similarity detection. In contrast, topic models often require extensive manual tuning and become computationally expensive with larger datasets, limiting their practicality for extensive analyses. This makes embeddings a superior choice for grouping news articles and analyzing their economic implications, as compared to the relatively rigid and broad classifications produced by topic models.

%\paragraph{Disadvantages of embeddings against Topic Models...}

It is true, however, that topic models excel at grouping articles based on shared themes, offering a straightforward way to identify and interpret these themes by examining the common content of the grouped articles. This interpretability is a key advantage of topic models, as it allows for clear labeling of themes. In contrast, vector embeddings lack inherent interpretability at the dimension level. The individual dimensions of an embedding do not have an intuitive meaning, making it challenging to directly understand the relationships they capture. However, this limitation can be mitigated by clustering the embeddings to then apply a similar interpretive process as with topic models: analyzing the articles within each cluster to infer the common patterns. As demonstrated in our analysis, these clusters often correspond to firm-specific or industry-specific topics, offering valuable insights into economic relationships and forming a valuable benchmark for our LLM's classification of firm-specific shocks.

%\paragraph{Alignment with LLM Architecture}
Lastly, using embeddings as a benchmark is particularly compelling because they represent the foundational layer of an LLM. The first step an LLM's processing pipeline is to transform the text that it is fed into high-dimensional embeddings for further processing. By benchmarking against embeddings, we ensure a direct and relevant comparison between the foundational representations used by LLMs and our specialized classification methodology. This comparison highlights the added value of the LLM's capacity to convert these semantic representations (i.e: the vector embeddings) into economically meaningful classifications. (i.e: our news-implied firm-specific shock classifications).

%\subsubsection*{Conclusion}
In summary, KMeans clustering of vector embeddings offers a robust and economically relevant benchmark for our LLM-based methodology. It provides a rich semantic representation, context-dependent flexibility, and scalability that surpass sentiment analysis and topic modeling. Additionally, its alignment with the underlying architecture of LLMs ensures a meaningful comparison. As demonstrated in our analysis, the clusters derived through this approach are predominantly firm or industry-specific, thereby offering a suitable and superior benchmark against which to measure the effectiveness of our granular classification of news-implied firm-specific shocks.
