\section{Paper Overview}

\begin{frame}{Paper Overview}
    \begin{itemize}
      \item The paper investigates how \blue{selective access to financial information} affects the way investors learn and form beliefs.
      
      \bigskip
      \item It compares two environments:
        \begin{itemize}
          \item \textbf{Selective feedback}: Investors only observe the outcome of an investment (the stock) if they choose it.
          \item \textbf{Full feedback}: Investors observe the outcome of the stock regardless of their choice.
        \end{itemize}
        
        \bigskip
      \item The study is motivated by real-world settings where information is sometimes only available through active participation (e.g., entrepreneurship, hiring, investment projects).
    \end{itemize}
  \end{frame}

\begin{frame}{Experimental Design}
    \begin{enumerate}
        \item Adult participants (via MTurk) repeatedly chose between a bond and a stock.
        
        \medskip
        \item Random assignment to either selective or full feedback conditions.
        
        \medskip
        \item After each choice, participants estimated the probability that the stock was "good" (i.e., paid from a favorable distribution).
        
        \medskip
        \item The experiment measured both \blue{belief formation} and \blue{choice behavior}, as well as risk preferences and financial literacy.
      \end{enumerate}

      \bigskip
      The study aims to disentangle the effects of \blue{information acquisition} (how much and what kind of data is gathered) and \blue{information processing} (how well that data is used) on investor learning.
\end{frame}
 

\begin{frame}{Main Findings}
    \begin{itemize}
      \item \blue{Selective feedback} leads investors to process information more accurately: their beliefs are, on average, 5\% closer to the Bayesian benchmark than those in the full feedback condition.
      
      \medskip
      \item However, selective feedback also results in smaller, less representative samples of information, introducing a \red{sampling error} of similar magnitude.
      
      \medskip
      \item The two effects—better processing but more sampling error—\textbf{offset each other}, so overall learning outcomes are similar across environments.
      
      \medskip
      \item The study reveals a dynamic process: selective feedback triggers more adaptive learning, especially after negative outcomes, and helps explain why previous literature found mixed results.
    %   \item The paper's approach—focusing on beliefs, not just choices—provides new insights into how investors learn in different informational settings.
    \end{itemize}
\end{frame}