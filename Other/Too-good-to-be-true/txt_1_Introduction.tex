%\section{Introduction}
%
%Over the past few decades, a growing body of research in finance has proposed various trading strategies, many of which show strong profitability in out-of-sample tests. However, as these strategies become widely known and implemented, their profitability tends to decay, suggesting potential overstatements in their initial evaluations. The profitability observed in out-of-sample periods can be overstated for two primary reasons.
%
%First, the design of trading strategies often relies on specific sample data, leading to what is known as in-sample overfitting. Researchers might tune hyperparameters, technical indicators, or other features of a strategy to optimize performance on the test data available to them, resulting in performance that is not replicable when tested on new data. Such practices lead to the overstatement of a strategy's true profitability.
%
%Second, even if a trading strategy is genuinely profitable out-of-sample, the publication of the strategy and its wide dissemination in the financial markets leads to its eventual demise. Market participants act on the strategy, incorporating the knowledge into prices, thus arbitraging away the profits as dictated by the efficient market hypothesis (EMH).
%
%This paper aims to provide both a theoretical and empirical investigation into why the profitability of trading strategies in academic finance literature tends to be overstated. Specifically, we explore how the out-of-sample profitability of these strategies decays due to either overfitting or market efficiency. To illustrate this, we propose testing the profitability of trading strategies "out-of-sample" on past historical data, replicating the conditions where such strategies were not widely known. This provides a cleaner estimate of whether the strategies were truly profitable before they became known to the market.
%
%The rest of the paper is structured as follows. Section 2 reviews the relevant literature on overfitting in finance and market efficiency. Section 3 presents the theoretical framework that explains the mechanisms of profitability decay. Section 4 details the empirical methodology used to test trading strategies on past data. Section 5 discusses the results, and Section 6 concludes the paper.

%%%%%%%%%%%%%%%%%%%%%%%%%%%%%%%%%%%%%%%%%%%%%%%%%%%%%
%%%%%%%%%%%%%%%%%%%%%%%%%%%%%%%%%%%%%%%%%%%%%%%%%%%%%
%%%%%%%%%%%%%%%%%%%%%%%%%%%%%%%%%%%%%%%%%%%%%%%%%%%%%
%%%%%%%%%%%%%%%%%%%%%%%%%%%%%%%%%%%%%%%%%%%%%%%%%%%%%

\section{Introduction}

Many trading strategies proposed in finance papers claim to exhibit significant out-of-sample profitability. These strategies, often developed using sophisticated models like machine learning algorithms, are designed and tested on historical data to demonstrate their ability to generate returns beyond simple benchmarks. However, upon closer inspection, the out-of-sample profitability of these strategies may be overstated due to two key factors: "out-of-sample overfitting" and the inevitable consequences of market efficiency.

First, the issue of "out-of-sample overfitting" arises when researchers, knowingly or unknowingly, design their strategies in a way that incorporates information from the test data. Traditionally, a machine learning model is trained on a training set, hyperparameters are optimized using a validation set, and the model is evaluated on an untouched test set. However, in practice, researchers may be tempted to peek at the test results during model design-particularly when working with complex models such as neural networks. In these cases, researchers might adjust hyperparameters (e.g., the number of layers, nodes per layer, or learning rates) based on how well the model performs on the test data. This subtle form of data snooping leads to a model that is overly tuned to the specific test data, producing impressive results that cannot be replicated in a \qquote{true} out-of-sample dataset. Thus, the profitability presented in the final evaluation is misleading, as it reflects optimization on both the training and test data, rather than genuine out-of-sample performance.

Second, even when a trading strategy is genuinely profitable out-of-sample, the dissemination of the strategy through academic publication often leads to its rapid demise due to market efficiency. According to the efficient market hypothesis (EMH), once a trading strategy is published and becomes widely known, market participants will quickly incorporate this information into prices, thereby arbitraging away any excess profits. In this sense, the life cycle of a trading strategy follows a predictable path: initial discovery and profitability, followed by increased adoption, and eventually the disappearance of profits as the strategy becomes crowded and its edge is neutralized.

This paper aims to address both of these issues-out-of-sample overfitting and post-publication arbitrage-by evaluating the profitability of trading strategies across time using a rolling-window framework. Specifically, we propose testing these strategies over the longest possible historical timeline, applying the strategy to different time periods to generate multiple out-of-sample results. This approach has several advantages. First, it allows us to compute "asymptotic" statistics by repeatedly observing the strategy's out-of-sample performance across different time periods. This provides a more robust estimate of the strategy's true profitability, reducing the risk of overstatement from a single backtest. Second, by evaluating the strategy's performance at different points in time, we can determine whether the strategy was ever genuinely profitable. If the strategy consistently produced profits in earlier time periods but not in recent years, this would suggest that it was once valuable but has since been arbitraged away. Conversely, if the strategy never generated sustained profitability across time, this would imply that its apparent success in one sample was merely an artifact of overfitting or data snooping.

Thus, this paper contributes to the literature by providing both a theoretical and empirical framework to explain why the profitability of trading strategies proposed in finance papers tends to decay over time. In Section 2, we review the relevant literature on overfitting and the efficient market hypothesis. In Section 3, we develop a theoretical model that explains the mechanisms of profitability decay. Section 4 presents our empirical methodology, detailing the rolling-window framework used to test trading strategies over historical data. Section 5 discusses the results, and Section 6 concludes the paper.
