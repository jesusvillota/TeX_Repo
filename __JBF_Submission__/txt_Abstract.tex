In financial markets, news impact stock prices. Despite the postulated \qquote{Efficient Market Hypothesis}, evidence shows inefficiencies, especially with complex information. 
Research attempting to explain such inefficiencies has often relied on dictionary-based methods, sentiment analysis, topic modeling, and more recently, vector-based models,
which still lack a comprehensive understanding of the economic implications of information. 
Additionally, many studies disregard firm-specific news-implied shocks and overly depend on headlines for analysis. 
This paper addresses these limitations by leveraging Large Language Models (LLMs) to provide a comprehensive, firm-specific analysis of full news articles. 
Using a dataset of Spanish business news from DowJones Newswires during a period of high uncertainty we apply LLMs to understand economic shocks affecting firms, categorizing them by type, magnitude, and direction. 
The findings show that LLM-based analysis provides superior insights during volatile periods compared to a benchmark model (KMeans clustering of vector embeddings). 
By using LLMs to parse news in a human-like manner, we gain clearer understanding of market reactions to firm-specific information, as evidenced by the profitability of our simple trading strategy.
