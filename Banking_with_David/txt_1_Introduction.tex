\section{Introduction}

The dissemination of public information about firms through news articles plays a significant role in financial markets. News articles can act as public signals that inform market participants, including investors, creditors, and banks, about the prospects of firms. When negative news is published about a firm, such as poor financial results, legal troubles, or operational difficulties, one would expect a corresponding reaction from the firm's creditors, including banks. For example, banks might adjust their credit terms, increase interest rates, or reduce the availability of new loans. However, the reaction of banks to such public signals can be mitigated by their access to private information. Banks that have established relationships with firms and possess private information about their creditworthiness may react differently to public news compared to banks without such relationships. This paper examines how banks adjust their lending behavior in response to news articles, and how their reactions differ based on their level of private information and specialization in certain sectors.

A core assumption in this paper is that the impact of a news article on a firm's credit relationships depends not only on the content of the article but also on the existing relationship between the firm and its creditors. When a bank has an established lending relationship with a firm, it is likely to possess private information about the firm's financial health, operational stability, and credit risk. This private information enables the bank to form a more accurate assessment of the firm's long-term viability, potentially making the bank less sensitive to negative public signals, such as adverse news articles. For example, if a bank believes that the shock affecting the firm is transitory in nature, it may continue to extend credit to the firm, despite negative news. On the other hand, if the bank perceives the shock as permanent, it may reduce or cease lending. This dynamic is particularly relevant for banks that specialize in lending to firms in specific sectors. Sector-specialized banks may possess a deeper understanding of industry dynamics, allowing them to distinguish between transitory and permanent shocks more effectively.

To analyze the credit implications of news articles on firms, this paper employs Large Language Models (LLMs) to systematically evaluate the content of news articles and quantify the projected credit risk associated with each firm mentioned. LLMs, such as GPT-based models, have demonstrated exceptional capabilities in understanding and interpreting unstructured text, making them ideal for analyzing business news. By querying LLMs in a structured manner, we obtain a measure of the impact that each news article is expected to have on a firm's creditworthiness. Specifically, we use LLMs to extract insights about the nature of the shock (transitory or permanent), its expected financial consequences, and the potential response of creditors. The results of this analysis are then used to forecast the expected adjustments in loan terms, such as interest rates, loan amounts, and collateral requirements.

This paper contributes to the literature in several ways. First, it extends the relationship lending theory by exploring how public signals, in the form of news articles, interact with banks' private information. Previous studies have examined how banks adjust their lending behavior based on private information (Boot, 2000; Petersen and Rajan, 1994), but fewer studies have focused on how public signals, such as news, influence lending decisions. Second, this paper introduces the use of LLMs in financial research, offering a novel approach to quantify the impact of qualitative information, such as news, on firm-level outcomes. By integrating LLM-generated insights into credit risk modeling, we provide a new perspective on how creditors interpret and react to public information. Finally, we contribute to the growing body of research on bank specialization by examining how sector-focused banks may better assess the implications of negative news for firms in their specialized sectors.

The remainder of the paper is organized as follows. Section \ref{sec:litreview} reviews the existing literature on relationship lending, public signals, and the use of LLMs in financial markets. Section \ref{sec:theoretical} presents a theoretical framework for how banks adjust their lending behavior in response to public and private information. Section \ref{sec:data} describes the data and methodology, including the process of analyzing news articles using LLMs and linking them to firm-level borrowing data. Section \ref{sec:results} discusses the empirical results and their implications for banks' lending behavior. Finally, Section \ref{sec:conclusion} offers concluding remarks and suggestions for future research.

