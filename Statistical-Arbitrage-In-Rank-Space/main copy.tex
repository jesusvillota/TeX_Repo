\documentclass[12pt,article]{memoir}
\usepackage{/Users/jesusvillotamiranda/Documents/LaTeX/$$JVM_Macros}
\Subject{Statistical Arbitrage in Rank Space}
\Arg{}

\begin{document}

In a market consisting of $N$ stocks, we denote the dividend-adjusted return on stock $i$ at trading day $t$ by $r_{i, t}$. We adopt a factor model for stock return,
\begin{align}\label{eq:1}
\mbf r_t- r_t^f \mbf 1_N = \mbf B_t \mbf f_t+ \b \epsilon_t, \quad t=1,2, \ldots, T 
\end{align}
Here, $\mbf r_t=\left\{r_{i, t}\right\}_{i=1}^N \in \mathbb{R}^N$ are the dividend-adjusted daily return, $r_t^f \in \mathbb{R}$ is the risk-free rate, $\mbf f_t \in \mathbb{R}^{K \times 1}$ are the underlying factors, $\mbf B_t \in \mathbb{R}^{N \times K}$ are the corresponding loadings on $K$ factors, and $\b \epsilon_t \in \mathbb{R}^N$ are the residual returns. Factor candidates varies widely, ranging from economical-driven factors such as the Fama-French factors, to statistically-driven factors derived from PCA. In our approach, factors are selected as the leading eigenvectors in PCA. The number of factors $K$ is chosen based on the eigenvalue spectrum of the empirical correlation of daily returns.


Without loss of generality, these factors can be interpreted as portfolios of stocks,
\begin{equation}\label{eq:2}
	F_t=\omega_t\left(r_t-r_f\right)
\end{equation}
where $\omega_t \in \mathbb{R}^{K \times N}$ contains corresponding portfolio weights. Combining \cref{eq:1} and \cref{eq:2} yields
\begin{equation}\label{eq:3}
	r_t-r_f=\beta_t \omega_t\left(r_t-r_f\right)+\epsilon_t \Rightarrow \epsilon_t=\left(I-\beta_t \omega_t\right)\left(r_t-r_f\right):=\Phi_t\left(r_t-r_f\right)
\end{equation}

Here,
\begin{equation}\label{eq:4}
	\Phi_t:=\left(I-\beta_t \omega_t\right)
\end{equation}
defines a linear transformation from $r_t$ to $\epsilon_t$. More importantly, $\epsilon_{i, t}$ can be viewed as the return of a tradable portfolio with weights specified by the $i$-th row of $\Phi_t$. Consequently, the investing universe spanned by $r_t$ is termed as name equity space, and that spanned by $\epsilon_t$ as name residual space.

We denote the portfolio weights in name equity space as $w_t^{R, \text { name }}$ and portfolio weights in name residual space as $w_t^{\epsilon, \text { name }}$. These weights are related by

\begin{equation}\label{eq:5}
	w_t^{R, \text { name }}=\Phi_t^T w_t^{\epsilon, \text { name }}
\end{equation}
, directly following the equality in portfolio return,

\begin{equation}\label{eq:6}
	\left(w_t^{\epsilon \text { name }}\right)^T \epsilon_t=\left(w_t^{\epsilon, \text { name }}\right)^T \Phi_t\left(r_t-r_f\right)=\left(w_t^{R, \text { name }}\right)^T\left(r_t-r_f\right)
\end{equation}

For factors derived by PCA, we have

\begin{equation}\label{eq:7}
	\Phi_t \beta_t=0 \Longrightarrow\left(w_t^{R, \text { name }}\right)^T \beta_t=\left(w_t^{\epsilon, \text { name }}\right)^T \Phi_t \beta_t=0, \quad \forall w_t^{\epsilon, \text { name }}
\end{equation}
with proof given in the appendix. It means that for any $w_t^{\epsilon \text {,name }}$, the $w_t^{R \text {,name }}$ calculated by \cref{eq:5} satisfy,

\begin{equation}\label{eq:8}
	\left(w_t^{R, \text { name }}\right)^T\left(r_t-r_f\right)=\left(w_t^{\epsilon, \text { name }}\right)^T \Phi_t\left(\beta_t F_t+\epsilon_t\right)=\left(w_t^{\epsilon, \text { name }}\right)^T \Phi_t \epsilon_t=\left(w_t^{R, \text { name }}\right)^T \epsilon_t
\end{equation}

It suggests that the return of our statistical arbitrage portfolios is independent of market factors and relies solely on residual returns, a property usually termed as market neutrality. Ideally, portfolios are also desired to have a zero net value, known as dollar neutrality. Empirical evidence suggests that market-neutral portfolios are also approximately dollar-neutral.

\Vhrulefill

We are given a factor model for stock returns:
$$
r_t-r_f=\beta_t F_t+\epsilon_t, \quad t=1,2, \ldots, T
$$

where:
\begin{itemize}
  \item $r_t=\left\{r_{i, t}\right\}_{i=1}^N \in \mathbb{R}^N$ represents the vector of dividend-adjusted daily returns of $N$ stocks at time $t$,
  \item $r_f \in \mathbb{R}$ is the risk-free rate,
  \item $F_t \in \mathbb{R}^K$ is the vector of $K$ factors at time $t$,
  \item $\beta_t \in \mathbb{R}^{N \times K}$ is the matrix of factor loadings,
  \item $\epsilon_t \in \mathbb{R}^N$ represents the residual returns (unexplained component).
\end{itemize}


Our goal is to extract factors $F_t$ statistically using PCA from the returns data, $r_t-r_f$, and to select the number $K$ based on the eigenvalue spectrum of the empirical correlation matrix.


MODUS OPERANDI: 

\underline{Step 1: Center the Returns Data}

1. Compute the excess returns:

$$
\tilde{r}_t=r_t-r_f, \quad \text { for } t=1,2, \ldots, T
$$

2. Construct the returns matrix $\mathbf{R} \in \mathbb{R}^{T \times N}$ :

$$
\mathbf{R}=\left(\begin{array}{c}
\tilde{r}_1^T \\
\tilde{r}_2^T \\
\vdots \\
\tilde{r}_T^T
\end{array}\right)
$$

where each row $\tilde{r}_t^T \in \mathbb{R}^N$ represents the excess returns of all stocks on day $t$.


\underline{Step 2: Compute the Empirical Correlation Matrix}

1. Standardize $\mathbf{R}$ (if necessary) so that each column has mean 0 and standard deviation 1. Let's denote the standardized version as $\tilde{\mathbf{R}}$.

2. Compute the empirical covariance (or correlation) matrix $\mathbf{C} \in \mathbb{R}^{N \times N}$ :
$$
\mathbf{C}=\frac{1}{T-1} \tilde{\mathbf{R}}^T \tilde{\mathbf{R}}
$$
This matrix $\mathbf{C}$ captures the co-movement of the excess returns across the $N$ stocks.


\underline{Step 3: Perform PCA on the Empirical Correlation Matrix}

1. Perform an eigenvalue decomposition on $\mathbf{C}$ :

$$
\mathbf{C V}=\mathbf{V} \Lambda
$$

where:
\begin{itemize}
  \item $\mathbf{V} \in \mathbb{R}^{N \times N}$ is the matrix of eigenvectors,
  \item $\Lambda=\operatorname{diag}\left(\lambda_1, \ldots, \lambda_N\right)$ is the diagonal matrix of eigenvalues, sorted such that $\lambda_1 \geq$ $\lambda_2 \geq \cdots \geq \lambda_N$.
\end{itemize}

2. Select the Number of Factors $K$ : Choose $K$ based on the eigenvalue spectrum. For example, you might select $K$ such that the cumulative proportion of variance explained by the first $K$ eigenvalues exceeds a certain threshold (e.g., $80 \%$ or $90 \%$ ).

\underline{Step 4: Construct the Factors}

1. Let $\mathbf{V}_K \in \mathbb{R}^{N \times K}$ be the matrix containing the first $K$ eigenvectors.

2. Compute the factors $F_t$ as:
$$
F_t=\mathbf{V}_K^T \tilde{r}_t, \quad \text { for } t=1,2, \ldots, T
$$


Here, $F_t \in \mathbb{R}^K$ represents the $K$ principal components at time $t$.



\underline{Step 5: Interpretation in Terms of Portfolio Weights}

From Equation (2), we interpret the factors $F_t$ as portfolios of stocks:

$$
F_t=\omega_t\left(r_t-r_f\right)
$$

where $\omega_t \in \mathbb{R}^{K \times N}$ contains the portfolio weights. In the PCA setup:
- $\quad \omega_t$ corresponds to $\mathbf{V}_K^T$, which gives the linear combination of stocks (or loadings) used to construct the factors.

%%%%%%%%%%%%%%%%%%%%%%%%%%%%%%%%%%%%%%%%%%%%%%%%%%%%%
\end{document}






