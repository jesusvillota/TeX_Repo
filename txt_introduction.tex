

%----------------------------------------------------
% 18 June 2024 | Raw text from my own writing
%----------------------------------------------------

\hspace{0.5cm} This paper aims to provide a novel and universal approach to analyzing the impact of business news on stock prices. Our approach is novel in that it is the first time that Large Language Models are employed to \textit{comprehensively} analyze the shocks described in business news articles for return prediction purposes, and it is universal in the sense that it does not rely on access to structured metadata from paid news portals, which, in general, are not widely accessible to the common researcher.

\mx 
Our database consists of a set of Spanish business news articles from DowJones spanning June 2020 to September 2021. Such articles are filtered in a way that allows us to extract the firms directly involved in them. In the first exercise, we convert the wording of the articles from text to high-dimensional vector embeddings by using a transformer model. Such representation captures the general contextual and semantic meaning of the text but is not able to capture the subtle nuances of the shocks described there on the affected firms. 

\mx 
We then cluster our news articles by applying the KMeans algorithm to the associated vector embeddings. This procedure delivers 26 clusters of business news, where each cluster usually pools together articles about a firm or set of firms in the same sector. 

\mx 
For each firm affected by an article, a market model is constructed on some window previous to the day where the article's information got incorporated into the market. Such model is then used to construct a beta-neutral strategy that extracts the abnormal returns of the firm after controlling for the market. 
%\mx 
By obtaining the metrics of this strategy and comparing them across clusters, we can obtain a measure of the profitability of each cluster. We then propose two algorithms that exploit this information to select the optimal clusters.
%to build a trading rule. 

\mx 
Finally, a trading rule is constructed by launching trades on the selected clusters over a specific holding period. By projecting the trading rule onto the test set, we obtain a measure of the profitability of the whole procedure out-of-sample. 

\mx 
The strategy based on KMeans clustering of vector embeddings is not able to generate a consistent earnings profile in the test set, which occurs due to the instability of the clustering method. Namely, the distribution of articles through clusters across data splits shows a very inconsistent pattern, which already hints at the fact that signals generated by the trading rule will not be generalizable out-of-sample.

\mx 
In the second part of the paper, we feed the news articles to a Large Language Model (LLM) and ask it to manually parse them. In particular, we ask the LLM to extract the affected firms and to individually classify the shock implied by the article in each affected firm based on a predefined schema. Such schema consists of a classification of news articles' shocks based on three categories: type (demand, supply, financial, technology, policy), magnitude (minor, major), and direction (positive, negative). We can then cluster the articles based on this classification. 

\mx
In this case, the distribution of articles through clusters is very stable across data splits, which indicates that the trading rule will generate signals that will be generalizable across data splits. Indeed, this is confirmed by the out-of-sample performance of the trading strategy, which shows a consistent earnings profile. These results are robust to hyperparameter variability. In particular, we show that the distribution of Sharpe Ratios in the test set for different choices of holding period length and maximum number of traded clusters is right-skewed and centered at positive values.

%----------------------------------------------------
% 29 April 2024 | Raw text from my own writing
%----------------------------------------------------

%This paper aims to provide a novel and universal approach to analyzing the impact of business news on stock prices. Our approach is novel in that it is the first time that Large Language Models are employed to comprehensively analyze the shocks described in business news articles for return prediction purposes, and it is universal in the sense that it does not rely on access to structured metadata from paid news portals, which, in general, are not widely accessible to the common researcher.


%
%\mx 
%We start with an unstructured corpus of textual data, namely Spanish business news articles, which undergoes preprocessing before being fed into the GPT API. Subsequently, we guide GPT in parsing these news articles and generating structured responses using "function calling", i.e., a predefined set of functions and parameters of our writing to rigorously instruct GPT on how to respond. 
%
%\mx
%Such functions prompt GPT to identify various attributes of the article, including its publication datetime, the type of information provided (new information, historical, analysis/comments, marketing) and its scope (firm-level, industry, global). If the scope of the article is at the firm level, we further ask GPT to list the firms that are directly affected by the events narrated therein. For each identified firm, we request GPT to provide its associated stock market ticker (if publicly traded; otherwise, it returns "NaN"), and further, to classify several aspects regarding how the shock pertains to the firm. Specifically, we prompt GPT to classify the type of shock (e.g., demand, supply, regulatory), the expected duration (short-term, long-term, permanent), the magnitude or relevance (minor, mild, major), and the expected impact on the firm's performance (positive, neutral, negative). Lastly, we obtain GPT's own trading signal by putting it in the shoes of a financial advisor tasked with deciding whether to \textit{Long} or \textit{Short} the stock associated with the firm affected by the news article.
%
%\mx 
%The methodological framework outlined above not only facilitates the identification of pertinent metadata but also generates a structured and comparable array of responses, enabling us to progress the analysis to a supervised stage, which consists of two parts.
%
%\mx 
%In the first part, we examine the predictability of stock returns in response to the shocks delineated in the news articles. To achieve this, we will conduct regressions of the stock returns of affected firms on the dummified shock classifications provided by GPT. This analysis will elucidate the direction and significance of shock predictors for return prediction.
%
%\mx 
%In the second part, we will assess the market timing capabilities of GPT through a market timing test. This involves contrasting GPT's decisions with the decisions that should have been made based on realized returns. In other words, we will juxtapose the stock return predictions of GPT, which inform decisions to long or short the stock, with the actual stock return performance of the respective stock.
%
%\mx 
%Finally, we will construct a set of Long-Short portfolios. One of these portfolios will trade shock signals, determining whether to go long or short based on the shock classifications provided by GPT. Another portfolio will be created using GPT's raw market timing capabilities, disregarding the deeper understanding of the news article implied by the shock analysis and categorization.
%
%\mx 
%All in all, our approach allows us to transition from an unsupervised learning procedure with unstructured data to a supervised learning procedure with structured data that enables us to study the stock return predictability of the shocks described by business news articles.
%
%
%%%%%%%%%%%%%%%%%%%%%%%%%%%%%%%%%%%%%%%%%%%%%%%%%%%%%%%%%%%%%%%%%%%%%%%%%%
%%%%%%%%%%%%%%%%%%%%%%%%%%%%%%%%%%%%%%%%%%%%%%%%%%%%%%%%%%%%%%%%%%%%%%%%%%
%%%%%%%%%%%%%%%%%%%%%%%%%%%%%%%%%%%%%%%%%%%%%%%%%%%%%%%%%%%%%%%%%%%%%%%%%%
%%%%%%%%%%%%%%%%%%%%%%%%%%%%%%%%%%%%%%%%%%%%%%%%%%%%%%%%%%%%%%%%%%%%%%%%%%
%
%%\mx 
%%The methodological framework described above not only facilitates the identification of relevant metadata but also produces a structured and comparable set of responses that allows us to advance the analysis further to a supervised stage,
%
%
%%In the first part, we analyze the stock return predictability of the shocks described in the news articles. For this purpose, we will regress the stock returns of the affected firms on the dummified shock classifications made by GPT. This will allow us to shed light on the direction and significance of shock predictors for return prediction. 
%%
%%For the second part, we will analyze the market timing capabilities of GPT by performing a market timing test. Here we will contrast the decision made by GPT with the decision that should have been taken based on the realized returns. In n other words, we will compare the stock return predictions of GPT (which underlie in the decision made to long or short the stock) and the actual stock return performance of the stock in question. 
%
%%Finally, we will construct a set of Long-Short portfolios. One of such portfolio will trade shock signals; that is, it will decide whether to long/short based on the shock classifications made by GPT. Another portfolio will be constructed using GPT's raw market timing abilities (that is, by simply asking GPT to long or short based on the news and without regard to the deeper understanding on the news article implied by the shock analysis and categorization)
%
%
%%in which we study the stock return predictability of the shock analysis and market timing signals generated by GPT.
%
%%Namely, the output from GPT's response consists of a classification of the shocks implied by the news articles. We launch a set of queries to GPT: we ask it to identify the publication datetime of the article, the type of article (set of possible answers: new information, historical, analysis or comments, marketing) and its scope (firm-level, industry, global). If the scope of the article is at the firm-level, then, we ask GPT to list the firms that are primarily and directly affected by the events narrated in the article. Then, for each firm in that list of firms, we prompt GPT to give us its associated stock market ticker (if the company is publicly traded, otherwise, it spits ``NaN'') and we further ask GPT to classify a set of aspects of how that shock relates to the firm in question. In particular, we ask it about the shock type (demand, supply, regulation,...), the expected duration of that shock (set of possible answers: short-term, long-term, permanent), the magnitude or relevance (set of possible answers: minor, mild, or major), and the direction in which that shock is expected to affect the firm's performance (set of possible answers: positive, neutral negative). Further, we ask GPT to provide a trading signal based on the news article; namely, we put GPT in the shoes of a financial advisor having to decide on whether to Long or Short the stock associated to the affected firm based on the events described in the news article. Once we have obtained a structured answer from GPT we can obtain the metadata of the identified firms to analyze the stock return predictability of the shock analysis and market timing signals generated by GPT.
%%
%
%
%
%%In particular, our approach departs from an unstructured dataset of textual data (a corpus of Spanish news articles) that is preprocessed and fed to GPT. At this stage, we direct GPT on how to parse these news articles and generate a structured response through ``function calling'' (i.e., writing a set of functions and parameters to precisely instruct GPT on to output its completions). This methodology allows us to not only identify the relevant metadata but also to obtain a structured and comparable output that places us in the right place to take the analysis to a supervised stage.
%%---
%%
%%At this stage, a set of functions is defined to instruct GPT on how to parse and analyze the news articles. 
%%
%%At this stage, we task GPT to produce a set of answers according to a function schema that we predefined in advance. 
%%
%%
%%---
%
%%
%%. Different from the previous literature, our procedures don't rely on having access to structured metadata from payable news portals, which are not widely accessible to the common researcher. 
%%
%%In other words, our approach is universal in that it doesn't require metadata from news portals that require subscriptions. 
%%
%%In this sense, we obtain all the information by asking GPT. This is an unsupervised learning approach. 
%%
%%How do we get around this? By delegating the identification of all the relevant metadata to GPT. However, this delegation only works if the right textual preprocessing is performed and the right questions are asked to GPT.  
%%
%%The modus operandi consists of starting from an unsupervised learning approach with unstructured textual data (news articles), which we then preprocess and feed to GPT. 
%%
%%Departing from unstructured news article data, we parse those articles by passing them through the GPT API. 
%%
%%
%%This leads GPT to produce a structured output that can then be employed to analyze the stock return predictability of GPT. 
%%
%
%
%
%%Namely, the output from GPT's response consists of a classification of the shocks implied by the news articles. We launch a set of queries to GPT: we ask it to identify the publication datetime of the article, the type of article (set of possible answers: new information, historical, analysis or comments, marketing) and its scope (firm-level, industry, global). If the scope of the article is at the firm-level, then, we ask GPT to list the firms that are primarily and directly affected by the events narrated in the article. Then, for each firm in that list of firms, we prompt GPT to give us its associated stock market ticker (if the company is publicly traded, otherwise, it spits ``NaN'') and we further ask GPT to classify a set of aspects of how that shock relates to the firm in question. In particular, we ask it about the shock type (demand, supply, regulation,...), the expected duration of that shock (set of possible answers: short-term, long-term, permanent), the magnitude or relevance (set of possible answers: minor, mild, or major), and the direction in which that shock is expected to affect the firm's performance (set of possible answers: positive, neutral negative). Further, we ask GPT to provide a trading signal based on the news article; namely, we put GPT in the shoes of a financial advisor having to decide on whether to Long or Short the stock associated to the affected firm based on the events described in the news article. Once we have obtained a structured answer from GPT we can obtain the metadata of the identified firms to analyze the stock return predictability of the shock analysis and market timing signals generated by GPT.
%
%
%% That is, our approach allows us to transition from an unsupervised learning with unstructured data to a supervised learning procedure with structured data. This latter procedure consists of two parts.
%
%
%%%%%%%%%%%%%%%%%%%%%%%%%%%%%%%%%%%%%%%%%%%%%%%%%%%%%%
%%----------------------------------------------------
%% 29 April 2024 | Parsed using GPT (asked it to simply rephrase my text)
%%----------------------------------------------------
%%%%%%%%%%%%%%%%%%%%%%%%%%%%%%%%%%%%%%%%%%%%%%%%%%%%%%
%
%%This paper endeavors to establish a methodical framework for the analysis of business news articles, aiming to transform unstructured news data into structured insights through the utilization of the GPT API.
%%
%%By instructing GPT through a series of predefined functions, we orchestrate the parsing of news articles to yield structured responses. This approach facilitates the extraction of comparable insights, primarily focusing on classifying the shocks conveyed within the articles. Our methodology involves querying GPT to ascertain key attributes of the articles, including publication datetime, article type (such as new information, historical context, analysis, or commentary), and scope (ranging from firm-level to industry or global perspectives).
%%
%%When the scope pertains to specific firms, we prompt GPT to identify the principal firms directly impacted by the events narrated in the article. Subsequently, for each identified firm, we solicit GPT to classify various aspects of the shock, encompassing its type (such as demand, supply, or regulatory), expected duration (ranging from short-term to long-term or permanent), magnitude or relevance (ranging from minor to major), and anticipated direction of impact on the firm's performance (positive, neutral, or negative). Additionally, we task GPT with generating trading signals based on the news articles, simulating the role of a financial advisor deciding whether to Long or Short the stock associated with the affected firm based on the article's content.
%%
%%The initial phase of our study focuses on analyzing the predictability of stock returns based on these shock classifications. Employing regression analysis, we explore the relationship between stock returns of affected firms and the identified shock predictors, examining both the direction and significance of these predictors.
%%
%%Subsequently, we delve into assessing the market timing capabilities of GPT through a rigorous market timing test. This involves contrasting GPT's decisions regarding stock returns with the ideal decisions that should have been made, thereby evaluating the effectiveness of GPT's predictions in real-world market scenarios.
%%
%%In the final phase, we construct a series of Long-Short portfolios to further explore the practical implications of GPT's insights. One such portfolio will trade based on shock signals, leveraging GPT's shock classifications to inform Long or Short decisions. Another portfolio will utilize GPT's raw market timing abilities, disregarding deeper shock analysis, and solely relying on GPT's instantaneous market predictions derived from news articles.
%
%
%
%%%%%%%%%%%%%%%%%%%%%%%%%%%%%%%%%%%%%%%%%%%%%%%%%%%%%%
%%----------------------------------------------------
%% 29 April 2024 | Parsed using GPT (asked it to do whatever it want with my text)
%%----------------------------------------------------
%%%%%%%%%%%%%%%%%%%%%%%%%%%%%%%%%%%%%%%%%%%%%%%%%%%%%%
%%
%%This paper presents a systematic framework for analyzing business news articles using the GPT API. Departing from unstructured data, we employ a structured approach by instructing GPT through predefined functions, enabling the extraction of comparable and analyzable responses.
%%
%%Our methodology begins by parsing articles through the GPT API, directing it to identify key attributes such as publication datetime, article type, and scope. For articles focused on individual firms, GPT identifies the primary affected firms and classifies aspects of the associated shock, including type, duration, magnitude, and expected impact direction.
%%
%%Furthermore, we task GPT with providing trading signals based on the news content, simulating the decision-making process of a financial advisor. This approach lays the foundation for our subsequent analyses.
%%
%%In the first part of the paper, we investigate the predictability of stock returns based on GPT's shock classifications. Through regression analysis, we examine the significance and directionality of these predictors.
%%
%%Subsequently, we assess GPT's market timing capabilities by comparing its stock return predictions with actual performance. This market timing test evaluates the efficacy of GPT's decision-making process in real-world scenarios.
%%
%%Finally, we construct Long-Short portfolios to further explore GPT's capabilities. One portfolio trades based on GPT's shock signals, while another relies solely on GPT's raw market timing abilities, providing insights into the impact of deeper news analysis.
%%
%%By employing this structured approach, we aim to enhance our understanding of the relationship between business news content and stock market movements, leveraging the capabilities of advanced language models like GPT. 
%%
%%
%%
%%
