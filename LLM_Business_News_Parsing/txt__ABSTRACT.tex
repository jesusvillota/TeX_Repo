%\begin{abstract}
%%%%%%%%%%%%%%%%% THIS VERSION: 192 WORDS %%%%%%%%%%%%%%%%%%%%%
In this paper, we explore the incorporation of information in the stock market by building a trading strategy that trades clusters of news. First, we construct the clusters by applying KMeans to the vector embedding representations of the articles. We then propose two algorithms that select \textit{in-sample} the most profitable clusters for trading and project them \textit{out-of-sample} to evaluate the performance of the trading strategy. 
%\mx 
In the second part, we perform the clustering by feeding the news articles to a Large Language Model and asking it to classify the type, magnitude, and direction of the shocks implied by each article on the affected firms according to a predefined schema. We then apply the same algorithms for cluster selection as before and check the performance of our trading strategy.
%
The results show an inconsistent earnings profile in the test set for the KMeans clustering, while the LLM clustering is able to exploit the information of the articles in a profitable way, creating a consistent earnings profile. These results are robust to hyperparameter variability, such as the number of periods in the holding window and the upper bound on the number of traded clusters. 
%%%%%%%%%%%%%%%%%%%%%%%%%%%%%%%%%%%%%%%%%%%%%%%%%%%%%

%\end{abstract}