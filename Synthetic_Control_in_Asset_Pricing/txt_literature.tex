\section{Literature Review}

\subsection{Asset Pricing Models}

Asset pricing is a fundamental area in financial economics, concerned with determining the fair value of financial assets based on their risk and expected return. Traditional asset pricing models have laid the groundwork for understanding the intricate relationships between risk factors and asset returns.

\subsubsection{Capital Asset Pricing Model (CAPM)}
The Capital Asset Pricing Model (CAPM) \cite{Sharpe1964, Lintner1965, Mossin1966} is one of the earliest and most influential models in asset pricing. CAPM posits that the expected return of an asset is linearly related to its systematic risk, measured by beta ($\beta$), which reflects the asset's sensitivity to market movements. The model is expressed as:
\[
E(R_i) = R_f + \beta_i (E(R_m) - R_f)
\]
where $E(R_i)$ is the expected return of asset $i$, $R_f$ is the risk-free rate, and $E(R_m)$ is the expected return of the market portfolio. Despite its simplicity and intuitive appeal, CAPM has been criticized for its strong assumptions, such as investors holding diversified portfolios and markets being frictionless.

\subsubsection{Arbitrage Pricing Theory (APT)}
Developed by Ross \cite{Ross1976}, the Arbitrage Pricing Theory (APT) offers a multi-factor approach to asset pricing. Unlike CAPM, which relies on a single market factor, APT allows for multiple macroeconomic factors to influence asset returns. The APT model is given by:
\[
E(R_i) = R_f + \beta_{i1}F_1 + \beta_{i2}F_2 + \dots + \beta_{ik}F_k
\]
where $F_1, F_2, \dots, F_k$ are the systematic factors affecting returns. APT provides greater flexibility in capturing various sources of risk but requires the identification of relevant factors, which can be challenging in practice.

\subsubsection{Fama-French Three-Factor Model}
Fama and French \cite{Fama1993} extended the CAPM by introducing two additional factors: size (SMB, small minus big) and value (HML, high minus low). The three-factor model is expressed as:
\[
E(R_i) = R_f + \beta_i (E(R_m) - R_f) + s_i \text{SMB} + h_i \text{HML}
\]
This model significantly improves the explanatory power for asset returns by accounting for the size and value effects observed in empirical data. Subsequent research has further expanded the factor models to include momentum, profitability, and investment factors, leading to more comprehensive frameworks like the Fama-French five-factor model \cite{Fama2015}.

\subsubsection{Recent Advancements and Challenges}
Recent advancements in asset pricing have focused on incorporating behavioral factors, machine learning techniques, and high-dimensional data to better capture the complexities of financial markets. Models such as the Consumption-based Asset Pricing Model (CAPM with consumption factors) \cite{Ludvigson2004} and various extensions using robust statistical methods \cite{Ang2014} have been proposed to address the limitations of traditional models.

However, challenges remain, including model misspecification, the difficulty of identifying relevant risk factors, and the dynamic nature of financial markets that may render static models inadequate. These challenges underscore the need for innovative approaches, such as the Synthetic Control Method, to enhance asset pricing models' flexibility and robustness.

\subsection{Synthetic Control Method}

The Synthetic Control Method (SCM) \cite{Abadie2010, Abadie2015} is a data-driven approach initially developed for causal inference in comparative case studies. SCM constructs a synthetic version of the treated unit by optimally weighting a combination of control units, enabling the estimation of counterfactual outcomes in the absence of treatment.

\subsubsection{Methodological Underpinnings}
SCM is grounded in the idea of creating a weighted average of potential control units that closely resembles the treated unit in terms of pre-intervention characteristics. The method involves selecting weights that minimize the discrepancy between the treated and synthetic control units across multiple covariates and time periods. Mathematically, the synthetic control $\mathbf{W}$ is determined by solving:
\[
\mathbf{W} = \arg\min_{\mathbf{w}} \left\| \mathbf{X}_1 - \mathbf{X}_0 \mathbf{w} \right\|_2^2
\]
subject to $\mathbf{w} \geq 0$ and $\sum w_j = 1$, where $\mathbf{X}_1$ represents the treated unit's characteristics and $\mathbf{X}_0$ represents the characteristics of the donor pool (control units).

\subsubsection{Applications in Economics and Beyond}
Since its inception, SCM has been widely applied in various fields beyond its original use in policy evaluation. Notable applications include:
\begin{itemize}
    \item \textbf{Policy Impact Analysis:} Assessing the effects of policy interventions, such as the economic impact of California's Tobacco Control Program \cite{Abadie2010}.
    \item \textbf{Macroeconomic Studies:} Evaluating the impact of economic crises, trade agreements, and other macroeconomic events \cite{Abadie2015}.
    \item \textbf{Healthcare Economics:} Analyzing the effects of healthcare policies and interventions on health outcomes \cite{Doudchenko2016}.
    \item \textbf{Marketing and Business Strategy:} Measuring the impact of marketing campaigns and strategic business decisions \cite{Galiani2017}.
\end{itemize}

These applications demonstrate SCM's versatility and effectiveness in providing credible counterfactuals, particularly in situations where randomized controlled trials are infeasible.

\subsubsection{Advantages and Limitations}
SCM offers several advantages, including its non-parametric nature, flexibility in handling multiple covariates, and ability to provide transparent and interpretable results. However, it also has limitations, such as sensitivity to the choice of donor pool, potential overfitting in high-dimensional settings, and challenges in inference, particularly regarding uncertainty quantification \cite{Chernozhukov2021}.

\subsection{Intersection of SCM and Asset Pricing}

While the Synthetic Control Method has been extensively utilized in various economic and social science applications, its integration into asset pricing remains relatively unexplored. The intersection of SCM and asset pricing presents an innovative avenue for enhancing traditional models by leveraging SCM's strengths in constructing synthetic counterparts and capturing complex, nonlinear relationships.

\subsubsection{Existing Studies and Theoretical Work}
To date, there is limited literature directly applying SCM to asset pricing. However, some studies have begun to explore related methodologies that share conceptual similarities with SCM. For instance, \cite{Chernozhukov2018} discusses the use of synthetic controls in high-dimensional settings, which is pertinent to asset pricing models that often involve numerous risk factors. Additionally, research on portfolio optimization and the construction of synthetic portfolios using alternative weighting schemes \cite{Jensen1968, Merton1969} provides a foundational basis for integrating SCM into asset pricing.

\subsubsection{Identified Gaps in the Literature}
The primary gaps in the existing literature include:
\begin{itemize}
    \item \textbf{Lack of Theoretical Frameworks:} There is a paucity of theoretical models that formally incorporate SCM into asset pricing, leaving room for the development of robust mathematical foundations.
    \item \textbf{Empirical Validation:} Few empirical studies have tested SCM-based asset pricing models, limiting the understanding of their practical applicability and performance compared to traditional models.
    \item \textbf{Handling High-Dimensional Data:} Traditional SCM may struggle with the high-dimensional nature of asset pricing data, necessitating methodological advancements to effectively apply SCM in this context.
    \item \textbf{Dynamic Market Conditions:} Existing applications of SCM primarily focus on static or quasi-static scenarios, whereas financial markets are inherently dynamic, requiring extensions of SCM to accommodate temporal changes and evolving risk factors.
\end{itemize}

Addressing these gaps is crucial for advancing the field of asset pricing. By developing a theoretical framework that integrates SCM with asset pricing models, this paper aims to provide a foundation for future empirical studies and methodological innovations.

\subsubsection{Potential Contributions of This Paper}
This paper seeks to bridge the identified gaps by:
\begin{itemize}
    \item \textbf{Establishing Theoretical Foundations:} Developing a rigorous mathematical framework that seamlessly integrates SCM into asset pricing models.
    \item \textbf{Enhancing Model Flexibility:} Demonstrating how SCM can capture nonlinearities and structural changes in asset returns, thereby addressing limitations of traditional linear models.
    \item \textbf{Proposing Methodological Advancements:} Introducing modifications to the standard SCM to better handle high-dimensional financial data and dynamic market conditions.
    \item \textbf{Setting the Stage for Empirical Research:} Providing a comprehensive theoretical basis that can be empirically tested and validated in future studies.
\end{itemize}

By tackling these aspects, the paper aims to contribute significantly to both the theoretical and practical aspects of asset pricing, offering novel tools and insights for financial economists and practitioners.
