\section{Theoretical Framework}

This section develops a theoretical framework to explain the two main mechanisms by which the profitability of trading strategies proposed in academic finance papers decays over time. First, we consider the role of out-of-sample overfitting, where researchers inadvertently incorporate information from test data into their model development process, leading to inflated estimates of profitability. Second, we examine how market efficiency erodes the profitability of even genuinely profitable strategies, as rational traders arbitrage away the excess returns once the strategy becomes widely known.

\subsection{Overfitting and Out-of-Sample Profitability Decay}

Consider a researcher developing a trading strategy based on a machine learning model, such as a neural network. Let $\mathcal{D}_{train}$, $\mathcal{D}_{val}$, and $\mathcal{D}_{test}$ represent the training, validation, and test data sets, respectively. In an ideal setting, the researcher should train the model on $\mathcal{D}_{train}$, tune hyperparameters using $\mathcal{D}_{val}$, and evaluate the final model on $\mathcal{D}_{test}$, which has been held out throughout the development process.

Let $f_{\theta}$ represent the trading strategy, parameterized by $\theta$, which is chosen based on the optimization of some objective function, $\mathcal{L}(f_{\theta})$, on the validation data $\mathcal{D}_{val}$. In a properly conducted test, $\theta$ is fixed before applying $f_{\theta}$ to the test data $\mathcal{D}_{test}$. However, if the researcher peeks at the test data during the hyperparameter tuning process, they may choose $\theta$ such that $f_{\theta}$ maximizes profitability on $\mathcal{D}_{test}$ rather than $\mathcal{D}_{val}$.

This can be formally represented as:
\[
\hat{\theta} = \arg \max_{\theta} \mathcal{L}(f_{\theta}; \mathcal{D}_{test}),
\]
where $\hat{\theta}$ is the hyperparameter set chosen to optimize performance on the test data. This leads to overfitting, as the model is tuned to the specific characteristics of $\mathcal{D}_{test}$, resulting in overstated out-of-sample profitability. When the strategy is applied to genuinely unseen data (e.g., data from a future time period), its performance decays because the model is overly tailored to the peculiarities of the test data.

Let $\mathcal{P}_{test}$ denote the profitability on the test set, and $\mathcal{P}_{future}$ denote the profitability on a new, genuinely out-of-sample set of data. If the researcher has overfitted the model, we would expect:
\[
\mathbb{E}[\mathcal{P}_{future}] < \mathbb{E}[\mathcal{P}_{test}],
\]
indicating that the observed profitability in the test period was inflated due to overfitting, and the true profitability on future data is lower.

\subsection{Market Efficiency and Profitability Decay}

Even when a trading strategy is genuinely profitable, its profitability tends to erode over time as it becomes widely known and adopted by market participants. This phenomenon is predicted by the efficient market hypothesis (EMH), which states that all available information is quickly incorporated into asset prices, leaving no room for excess returns.

Let $\alpha_t$ represent the excess return generated by a trading strategy at time $t$. Under the EMH, any trading strategy that generates $\alpha_t > 0$ will eventually be adopted by other traders, leading to increased competition and the elimination of mispricing. This process can be formalized as a dynamic adjustment where the alpha decays over time:
\[
\alpha_{t+1} = \alpha_t - \lambda \cdot N_t,
\]
where $\lambda$ is a parameter representing the rate at which excess returns are arbitraged away, and $N_t$ is the number of market participants adopting the strategy at time $t$. As $N_t$ increases, the alpha converges toward zero:
\[
\lim_{t \to \infty} \alpha_t = 0.
\]

This implies that even a strategy with legitimate out-of-sample profitability will see its returns diminish over time as it becomes widely known. Empirical evidence for this phenomenon is well-documented in the literature, with studies showing that trading strategies tend to lose their profitability soon after publication (McLean and Pontiff, 2016).

Thus, we expect the profitability of any widely known trading strategy to decay to zero over time, as arbitrageurs exploit the mispricing until it is fully corrected. This theoretical framework provides the foundation for the empirical analysis in Section 4, where we test the profitability of trading strategies over time using a rolling-window framework.
