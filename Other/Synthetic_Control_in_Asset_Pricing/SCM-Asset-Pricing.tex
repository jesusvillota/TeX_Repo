\documentclass[12pt,article]{memoir}
\usepackage{/Users/jesusvillotamiranda/Documents/LaTeX/$$JVM_Macros}
\Subject{SCM Asset Pricing}
\Arg{}

\begin{document}

\section{Theoretical Framework}

\subsection{Traditional Asset Pricing Framework}

We begin by establishing the traditional factor-based asset pricing framework. The standard approach models excess returns of asset $i$ at time $t$ as:

\begin{equation}
R_{it} = \alpha_i + \sum_{k=1}^K \beta_{ik}f_{kt} + \epsilon_{it}
\label{eq:traditional}
\end{equation}

where $f_{kt}$ represents the $k$-th risk factor at time $t$, $\beta_{ik}$ captures the sensitivity of asset $i$ to factor $k$, and $\epsilon_{it}$ represents idiosyncratic risk.

\subsection{Synthetic Control Method in Asset Pricing}

We propose an alternative framework based on the Synthetic Control Method (SCM). Let $R_{it}$ be the return of target asset $i$ at time $t$. Instead of decomposing returns into factor exposures, we construct a synthetic portfolio from a donor pool of $J$ assets that mimics the behavior of the target asset:

\begin{equation}
R_{it}^S = \sum_{j=1}^J w_{ij}R_{jt}
\label{eq:synthetic}
\end{equation}

where $R_{it}^S$ is the synthetic return and $w_{ij}$ represents the weight of donor asset $j$ in the synthetic portfolio for target asset $i$, subject to:

\begin{align}
w_{ij} &\geq 0 \quad \forall j \in \{1,\ldots,J\} \\
\sum_{j=1}^J w_{ij} &= 1
\end{align}

\subsection{Theoretical Properties}

\subsubsection{Relationship to Factor Models}

To establish the connection between our SCM approach and traditional factor models, we substitute the factor representation of each donor asset into Equation \ref{eq:synthetic}:

\begin{equation}
R_{it}^S = \sum_{j=1}^J w_{ij}(\alpha_j + \sum_{k=1}^K \beta_{jk}f_{kt} + \epsilon_{jt})
\end{equation}

This yields:

\begin{equation}
R_{it}^S = \underbrace{\sum_{j=1}^J w_{ij}\alpha_j}_{\text{synthetic alpha}} + \sum_{k=1}^K \underbrace{(\sum_{j=1}^J w_{ij}\beta_{jk})}_{\text{synthetic beta}_k}f_{kt} + \underbrace{\sum_{j=1}^J w_{ij}\epsilon_{jt}}_{\text{synthetic noise}}
\label{eq:decomposition}
\end{equation}

\subsubsection{Advantages over Traditional Factor Models}

The SCM framework offers several theoretical advantages:

1. \textit{Time-varying Risk Exposures}: The synthetic portfolio weights can be re-estimated over rolling windows, allowing for natural evolution of risk exposures:

\begin{equation}
\beta_{ik,t}^S = \sum_{j=1}^J w_{ij,t}\beta_{jk}
\end{equation}

2. \textit{Endogenous Factor Selection}: Instead of pre-specifying factors, the method endogenously selects relevant assets, implicitly determining the important risk factors through the weight optimization process.

3. \textit{Non-linear Risk Exposure}: The synthetic portfolio can capture non-linear relationships between the target asset and risk factors through the combination of donor assets with different factor exposures.

\subsection{Weight Optimization}

The optimal weights are determined by minimizing the tracking error during a pre-event window $[1,T_0]$:

\begin{equation}
\min_{w} \sum_{t=1}^{T_0} (R_{it} - \sum_{j=1}^J w_{ij}R_{jt})^2
\label{eq:optimization}
\end{equation}

subject to the previously stated constraints. Additionally, we introduce a regularization term to prevent overfitting:

\begin{equation}
\min_{w} \sum_{t=1}^{T_0} (R_{it} - \sum_{j=1}^J w_{ij}R_{jt})^2 + \lambda\sum_{j=1}^J w_{ij}^2
\end{equation}

where $\lambda$ is a regularization parameter.

\subsection{Donor Pool Selection}

The donor pool is constructed based on the following criteria:

1. \textit{Market Capitalization}: Donors should have similar size characteristics:
\begin{equation}
|\log(MC_i) - \log(MC_j)| \leq \delta_{MC}
\end{equation}

2. \textit{Industry Classification}: Donors should share similar industry exposure:
\begin{equation}
IC_i \cap IC_j \neq \emptyset
\end{equation}

3. \textit{Trading Volume}: Donors should meet minimum liquidity requirements:
\begin{equation}
\text{ADTV}_j \geq \theta
\end{equation}

where ADTV represents average daily trading volume.

\subsection{Inference and Statistical Properties}

Under suitable regularity conditions, we can establish the following properties:

\begin{theorem}[Consistency]
As $T_0 \to \infty$, if the data generating process is stationary and ergodic, then:
\begin{equation}
\text{plim}_{T_0 \to \infty} \hat{w}_{ij} = w_{ij}^*
\end{equation}
where $w_{ij}^*$ represents the true optimal weights.
\end{theorem}

\begin{theorem}[Asymptotic Normality]
Under appropriate regularity conditions:
\begin{equation}
\sqrt{T_0}(\hat{w}_{ij} - w_{ij}^*) \xrightarrow{d} N(0, \Sigma)
\end{equation}
where $\Sigma$ is the asymptotic variance-covariance matrix.
\end{theorem}

These theoretical results provide the foundation for statistical inference and hypothesis testing in our framework.



\Vhrulefill


\section{Theoretical Framework}

\subsection{From Policy Evaluation to Asset Pricing: A Theoretical Bridge}

The Synthetic Control Method (SCM), originally developed for policy evaluation in economics \citep{abadie2003economic, abadie2010synthetic}, provides a systematic approach to constructing counterfactuals for treated units using a weighted combination of control units. We establish a formal bridge between this methodology and asset pricing theory.

In the traditional SCM setting, for a treated unit $i$ and control units $j = 1,\ldots,J$, the outcome variable $Y_{it}$ is modeled as:

\begin{equation}
    Y_{it} = Y_{it}^N + \alpha_{it}D_{it}
\end{equation}

where $Y_{it}^N$ is the counterfactual outcome in absence of treatment, $\alpha_{it}$ is the treatment effect, and $D_{it}$ is the treatment indicator.

In our asset pricing context, we reframe this as:

\begin{equation}
    R_{it} = R_{it}^S + \epsilon_{it}
\end{equation}

where $R_{it}$ is the observed return of asset $i$ at time $t$, $R_{it}^S$ is the synthetic return constructed from comparable assets, and $\epsilon_{it}$ represents the idiosyncratic component unique to asset $i$.

\subsection{The Donor Pool Concept in Financial Markets}

\subsubsection{Theoretical Foundation}

The donor pool in our framework consists of assets that share similar risk-return characteristics with the target asset. Formally, we define the synthetic return as:

\begin{equation}
    R_{it}^S = \sum_{j=1}^J w_j R_{jt}
\end{equation}

where $w_j$ are weights satisfying:

\begin{equation}
    \sum_{j=1}^J w_j = 1, \quad w_j \geq 0
\end{equation}

The optimal weights are determined by minimizing the distance between the target asset and its synthetic counterpart over a pre-specified period $[T_0, T_1]$:

\begin{equation}
    \min_{w} \sum_{t=T_0}^{T_1} (R_{it} - \sum_{j=1}^J w_j R_{jt})^2
\end{equation}

subject to the above constraints.

\subsubsection{Economic Interpretation}

Unlike traditional factor models where factors are predetermined (e.g., market, size, value), our approach allows the ``factors'' (donor assets) to be selected based on their ability to replicate the target asset's behavior. This introduces a crucial distinction: the synthetic control approach implicitly captures both observable and unobservable common factors affecting asset returns.

\subsection{Addressing Limitations of Factor Models}

\subsubsection{Time-Varying Risk Exposures}

Traditional factor models typically assume constant factor loadings:

\begin{equation}
    R_{it} = \alpha_i + \sum_{k=1}^K \beta_{ik}F_{kt} + \epsilon_{it}
\end{equation}

Our SCM framework naturally accommodates time-varying risk exposures through rolling estimation windows:

\begin{equation}
    R_{it} = \sum_{j=1}^J w_{jt}R_{jt} + \epsilon_{it}
\end{equation}

where $w_{jt}$ are time-varying weights estimated over $[t-\tau, t]$.

\subsubsection{Non-linear Factor Relationships}

While factor models assume linear relationships between returns and risk factors, SCM can capture non-linear relationships through the weighted combination of actual returns. The synthetic return can be expressed as:

\begin{equation}
    R_{it}^S = g(\mathbf{F_t}; \mathbf{w_t})
\end{equation}

where $g(\cdot)$ is a potentially non-linear function of factors $F_t$, implicitly determined by the weights $w_t$.

\subsubsection{Endogenous Factor Selection}

Our framework addresses the joint-hypothesis problem inherent in factor models by letting the data determine the relevant comparison assets. The optimization problem becomes:

\begin{equation}
    \min_{w,J} \{\sum_{t=T_0}^{T_1} (R_{it} - \sum_{j=1}^J w_j R_{jt})^2 + \lambda \cdot |J|\}
\end{equation}

where $\lambda$ is a penalty parameter controlling the complexity of the donor pool.

\subsection{Theoretical Properties}

Under suitable regularity conditions, we establish the following properties:

\begin{enumerate}
    \item Consistency: As $T \to \infty$, the synthetic control estimator converges to the true return-generating process.
    
    \begin{equation}
        \plim_{T \to \infty} (R_{it}^S - R_{it}^*) = 0
    \end{equation}
    
    \item Efficiency: The synthetic control estimator achieves the minimum variance among all weighted combinations of control units that satisfy the constraints.
    
    \item Robustness: The estimator remains consistent under misspecification of the factor structure, provided the donor pool contains assets affected by the same set of factors as the target asset.
\end{enumerate}

These properties establish the theoretical foundation for using SCM as an alternative to traditional factor models in asset pricing.

% Add to your bibliography file:
% @article{abadie2003economic,
%   title={The economic costs of conflict: A case study of the Basque Country},
%   author={Abadie, Alberto and Gardeazabal, Javier},
%   journal={American Economic Review},
%   volume={93},
%   number={1},
%   pages={113--132},
%   year={2003}
% }
%
% @article{abadie2010synthetic,
%   title={Synthetic control methods for comparative case studies: Estimating the effect of California's tobacco control program},
%   author={Abadie, Alberto and Diamond, Alexis and Hainmueller, Jens},
%   journal={Journal of the American Statistical Association},
%   volume={105},
%   number={490},
%   pages={493--505},
%   year={2010}
% }


\section{Methodology}

\subsection{The SCM Model in Asset Pricing}

\subsubsection{Model Specification}

We adapt the synthetic control methodology to the asset pricing context through a three-stage procedure. For any target asset $i$, we construct its synthetic counterpart as follows:

\begin{equation}
    R_{it}^S = \sum_{j \in \mathcal{J}} w_{jt}R_{jt}
\end{equation}

where $\mathcal{J}$ represents the optimally selected donor pool, and weights $w_{jt}$ are estimated to minimize the tracking error:

\begin{equation}
    \min_{w_{jt}} \sum_{s=t-\tau}^{t-1} (R_{is} - \sum_{j \in \mathcal{J}} w_{jt}R_{js})^2
\end{equation}

subject to:
\begin{align}
    \sum_{j \in \mathcal{J}} w_{jt} &= 1 \\
    w_{jt} &\geq 0 \quad \forall j \in \mathcal{J}
\end{align}

where $\tau$ defines the estimation window length.

\subsection{Donor Pool Selection}

\subsubsection{Multi-criteria Optimization}

We formalize the donor pool selection through a hierarchical screening process:

\begin{equation}
    \mathcal{J} = \{j: S_j \geq \bar{S} \cap d(X_i, X_j) \leq \delta \cap \rho_{ij} \geq \bar{\rho}\}
\end{equation}

where:
\begin{itemize}
    \item $S_j$ represents firm size (market capitalization), with $\bar{S}$ as a minimum threshold
    \item $d(X_i, X_j)$ is a distance metric between firm characteristics vectors
    \item $\rho_{ij}$ is the historical return correlation, with $\bar{\rho}$ as a minimum threshold
\end{itemize}

The characteristics vector $X_i$ includes:
\begin{equation}
    X_i = [\text{Size}_i, \text{B/M}_i, \text{Momentum}_i, \text{Volatility}_i, \text{Industry}_i]
\end{equation}

\subsubsection{Dynamic Pool Updates}

To maintain pool relevance, we implement a rolling update mechanism:

\begin{equation}
    \mathcal{J}_t = \mathcal{J}_{t-1} \cup \mathcal{A}_t \setminus \mathcal{D}_t
\end{equation}

where $\mathcal{A}_t$ and $\mathcal{D}_t$ represent additions and deletions based on the screening criteria at time $t$.

\subsection{Treatment Period Definition}

\subsubsection{Event-based Framework}

We define treatment periods around significant corporate events:

\begin{equation}
    D_{it} = \begin{cases}
        1 & \text{if } t \in [t_e, t_e + h] \\
        0 & \text{otherwise}
    \end{cases}
\end{equation}

where $t_e$ is the event time and $h$ is the event window length.

Key events considered include:
\begin{itemize}
    \item Earnings announcements: $\mathcal{E}_{EA}$
    \item M\&A announcements: $\mathcal{E}_{MA}$
    \item Management changes: $\mathcal{E}_{MC}$
    \item Product launches: $\mathcal{E}_{PL}$
\end{itemize}

\subsection{Weight Estimation and Optimization}

\subsubsection{Rolling Window Estimation}

We implement a time-varying weight estimation procedure:

\begin{equation}
    w_{jt}^* = \arg\min_{w_{jt}} \sum_{s=t-\tau}^{t-1} v_s(R_{is} - \sum_{j \in \mathcal{J}} w_{jt}R_{js})^2
\end{equation}

where $v_s$ represents time-varying importance weights:

\begin{equation}
    v_s = \exp(-\kappa(t-1-s)/\tau)
\end{equation}

with $\kappa$ controlling the decay rate of historical observations.

\subsubsection{Regularization and Stability}

To ensure stability and prevent overfitting, we introduce an elastic net regularization term:

\begin{equation}
    L(w_{jt}) = \sum_{s=t-\tau}^{t-1} v_s(R_{is} - \sum_{j \in \mathcal{J}} w_{jt}R_{js})^2 + \lambda_1 \sum_{j \in \mathcal{J}} |w_{jt}| + \lambda_2 \sum_{j \in \mathcal{J}} w_{jt}^2
\end{equation}

\subsection{High-Frequency Considerations}

\subsubsection{Intraday Dynamics}

For high-frequency applications, we modify the estimation procedure to account for intraday patterns:

\begin{equation}
    R_{it,k} = R_{it,k}^S + \eta_{t,k} + \epsilon_{it,k}
\end{equation}

where $k$ indexes intraday periods and $\eta_{t,k}$ captures time-of-day effects.

\subsubsection{Microstructure Adjustments}

To address microstructure noise, we implement:

\begin{enumerate}
    \item Returns sampling at optimal frequencies determined by volatility signature plots
    \item Bid-ask bounce correction through effective spread estimation:
        \begin{equation}
            s_{it} = 2\sqrt{-\text{Cov}(R_{it}, R_{it-1})}
        \end{equation}
    \item Volume-weighted price averaging within sampling intervals
\end{enumerate}

\subsection{Weight Stability and Rebalancing}

\subsubsection{Turnover Control}

We introduce a turnover penalty in the optimization:

\begin{equation}
    \Delta w_t = \sum_{j \in \mathcal{J}} |w_{jt} - w_{j,t-1}|
\end{equation}

with the modified objective:

\begin{equation}
    L_{total}(w_{jt}) = L(w_{jt}) + \gamma \Delta w_t
\end{equation}

\subsubsection{Adaptive Rebalancing}

Rebalancing triggers are defined by:

\begin{equation}
    \text{Rebalance if: } |\hat{\epsilon}_{it}| > k\hat{\sigma}_{\epsilon} \text{ or } \Delta w_t > \bar{\Delta}
\end{equation}

where $\hat{\epsilon}_{it}$ is the tracking error and $\hat{\sigma}_{\epsilon}$ its estimated standard deviation.

\Vhrulefill

\section{Econometric Properties}

\subsection{Asymptotic Framework}

\subsubsection{Assumptions}

We begin by establishing the necessary assumptions for our asymptotic analysis:

\begin{assumption}[Data Generating Process]
For each asset $i$ and time $t$, returns follow:
\begin{equation}
    R_{it} = \mu_i(F_t) + \epsilon_{it}
\end{equation}
where $F_t$ is a vector of common factors, $\mu_i(\cdot)$ is a continuous function, and $\epsilon_{it}$ satisfies:
\begin{enumerate}[label=(\alph*)]
    \item $E[\epsilon_{it}|F_t] = 0$
    \item $E[\epsilon_{it}\epsilon_{jt}|F_t] = 0$ for $i \neq j$
    \item $E[\epsilon_{it}\epsilon_{is}|F_t,F_s] = 0$ for $t \neq s$
\end{enumerate}
\end{assumption}

\begin{assumption}[Mixing Conditions]
The sequence $\{(R_{it}, F_t)\}_{t=1}^T$ is strictly stationary and $\alpha$-mixing with mixing coefficients $\alpha(h)$ satisfying:
\begin{equation}
    \sum_{h=1}^{\infty} h^2\alpha(h)^{\delta/(2+\delta)} < \infty
\end{equation}
for some $\delta > 0$.
\end{assumption}

\begin{assumption}[Moment Conditions]
For all $i$ and $t$:
\begin{enumerate}[label=(\alph*)]
    \item $E|R_{it}|^{4+\delta} < \infty$
    \item $E|\epsilon_{it}|^{4+\delta} < \infty$
    \item $\sup_t E\|F_t\|^{4+\delta} < \infty$
\end{enumerate}
\end{assumption}

\subsection{Consistency Theory}

\subsubsection{Weight Convergence}

Let $w_T^*$ denote the optimal weights estimated using $T$ observations. We establish:

\begin{theorem}[Weight Consistency]
Under Assumptions 1-3, as $T \to \infty$:
\begin{equation}
    \|w_T^* - w^0\| \xrightarrow{p} 0
\end{equation}
where $w^0$ represents the population optimal weights.
\end{theorem}

\begin{proof}
The proof proceeds in three steps:

1. First, we show that the objective function converges uniformly:
\begin{equation}
    \sup_{w \in \mathcal{W}} |Q_T(w) - Q(w)| \xrightarrow{p} 0
\end{equation}
where
\begin{align}
    Q_T(w) &= \frac{1}{T}\sum_{t=1}^T (R_{it} - \sum_{j=1}^J w_jR_{jt})^2 \\
    Q(w) &= E[(R_{it} - \sum_{j=1}^J w_jR_{jt})^2]
\end{align}

2. Using the mixing conditions, we apply a uniform law of large numbers:
\begin{equation}
    \|Q_T(w) - Q(w)\|_{\infty} = O_p(T^{-1/2}\log T)
\end{equation}

3. Finally, we establish identification of $w^0$ through the positive definiteness of the second moment matrix of returns.
\end{proof}

\subsubsection{Rate of Convergence}

We establish the convergence rate under additional regularity conditions:

\begin{theorem}[Convergence Rate]
Under Assumptions 1-3 and suitable regularity conditions:
\begin{equation}
    \sqrt{T}(w_T^* - w^0) \xrightarrow{d} N(0, V)
\end{equation}
where
\begin{equation}
    V = H^{-1}\Sigma H^{-1}
\end{equation}
with
\begin{align}
    H &= E[\nabla^2 Q(w^0)] \\
    \Sigma &= \lim_{T \to \infty} \text{Var}(\sqrt{T}\nabla Q_T(w^0))
\end{align}
\end{theorem}

\subsection{Inference Procedures}

\subsubsection{Asymptotic Distribution}

For the synthetic return estimator:

\begin{theorem}[Asymptotic Normality]
Under stated assumptions:
\begin{equation}
    \sqrt{T}(R_{it}^S - R_{it}^*) \xrightarrow{d} N(0, \Omega)
\end{equation}
where
\begin{equation}
    \Omega = R_{jt}'VR_{jt} + \sigma_{\epsilon}^2
\end{equation}
\end{theorem}

\subsubsection{Hypothesis Testing Framework}

For testing the accuracy of synthetic control matches:

\begin{equation}
    H_0: R_{it}^S = R_{it}^* \quad \text{vs} \quad H_1: R_{it}^S \neq R_{it}^*
\end{equation}

Test statistic:
\begin{equation}
    \tau_T = T(R_{it}^S - R_{it}^*)'\hat{\Omega}^{-1}(R_{it}^S - R_{it}^*)
\end{equation}

Under $H_0$:
\begin{equation}
    \tau_T \xrightarrow{d} \chi^2(k)
\end{equation}

\subsection{Finite Sample Properties}

\subsubsection{Bias Analysis}

The finite sample bias of the synthetic control estimator is:

\begin{equation}
    E[R_{it}^S - R_{it}^*] = B_T + O(T^{-1})
\end{equation}

where
\begin{equation}
    B_T = -\frac{1}{2T}tr(H^{-1}\Sigma) + O(T^{-3/2})
\end{equation}

\subsubsection{Higher-Order Properties}

We derive the Edgeworth expansion:

\begin{equation}
    P(\sqrt{T}(R_{it}^S - R_{it}^*) \leq x) = \Phi(x) + T^{-1/2}p_1(x)\phi(x) + O(T^{-1})
\end{equation}

where $p_1(x)$ is a polynomial depending on the third and fourth moments.

\subsection{Robust Inference}

\subsubsection{HAC Estimation}

For robust variance estimation:

\begin{equation}
    \hat{\Omega} = \sum_{|h| < m_T} k(h/m_T)\hat{\Gamma}(h)
\end{equation}

where
\begin{equation}
    \hat{\Gamma}(h) = \frac{1}{T}\sum_{t=|h|+1}^T \hat{u}_t\hat{u}_{t-|h|}'
\end{equation}

with $k(\cdot)$ being a kernel function and $m_T$ the bandwidth parameter.

\subsubsection{Bootstrap Procedures}

We establish the validity of the following bootstrap procedure:

1. Generate bootstrap samples:
\begin{equation}
    R_{it}^{*b} = R_{it}^S + \epsilon_{it}^{*b}
\end{equation}

2. Compute bootstrap weights:
\begin{equation}
    w_T^{*b} = \arg\min_{w \in \mathcal{W}} \sum_{t=1}^T (R_{it}^{*b} - \sum_{j=1}^J w_jR_{jt})^2
\end{equation}

3. Bootstrap distribution:
\begin{equation}
    \sqrt{T}(w_T^{*b} - w_T^*) \xrightarrow{d} N(0, V)
\end{equation}

\subsection{Time-Varying Parameter Extensions}

\subsubsection{Local Asymptotic Framework}

For time-varying parameters:

\begin{equation}
    w_t = w_0 + h_T\beta(t/T)
\end{equation}

where $h_T \to 0$ as $T \to \infty$.

Local linear estimator:
\begin{equation}
    \hat{w}_t = \arg\min_{w,\beta} \sum_{s=1}^T K_h(t-s)(R_{is} - \sum_{j=1}^J (w_j + \beta_j(t-s))R_{js})^2
\end{equation}

\subsubsection{Uniform Inference}

We establish uniform confidence bands:

\begin{equation}
    P(\sup_{t \in [0,1]} |\hat{w}_t - w_t| \leq c_{\alpha}\sqrt{\log T/Th}) \to 1-\alpha
\end{equation}

where $c_{\alpha}$ is the critical value obtained from the distribution of the supremum of a Gaussian process.

%\end{antml:parameter>
%</invok


%%%%%%%%%%%%%%%%%%%%%%%%%%%%%%%%%%%%%%%%%%%%%%%%%%%%%
%%%%%%%%%%%%%%%%%%%%%%%%%%%%%%%%%%%%%%%%%%%%%%%%%%%%%
%%%%%%%%%%%%%%%%%%%%%%%%%%%%%%%%%%%%%%%%%%%%%%%%%%%%%
%%%%%%%%%%%%%%%%%%%%%%%%%%%%%%%%%%%%%%%%%%%%%%%%%%%%%
%%%%%%%%%%%%%%%%%%%%%%%%%%%%%%%%%%%%%%%%%%%%%%%%%%%%%
%%%%%%%%%%%%%%%%%%%%%%%%%%%%%%%%%%%%%%%%%%%%%%%%%%%%%
%%%%%%%%%%%%%%%%%%%%%%%%%%%%%%%%%%%%%%%%%%%%%%%%%%%%%
%%%%%%%%%%%%%%%%%%%%%%%%%%%%%%%%%%%%%%%%%%%%%%%%%%%%%
\Vhrulefill
\Vhrulefill
%%%%%%%%%%%%%%%%%%%%%%%%%%%%%%%%%%%%%%%%%%%%%%%%%%%%%
%%%%%%%%%%%%%%%%%%%%%%%%%%%%%%%%%%%%%%%%%%%%%%%%%%%%%
%%%%%%%%%%%%%%%%%%%%%%%%%%%%%%%%%%%%%%%%%%%%%%%%%%%%%
%%%%%%%%%%%%%%%%%%%%%%%%%%%%%%%%%%%%%%%%%%%%%%%%%%%%%
%%%%%%%%%%%%%%%%%%%%%%%%%%%%%%%%%%%%%%%%%%%%%%%%%%%%%
%%%%%%%%%%%%%%%%%%%%%%%%%%%%%%%%%%%%%%%%%%%%%%%%%%%%%
%%%%%%%%%%%%%%%%%%%%%%%%%%%%%%%%%%%%%%%%%%%%%%%%%%%%%

\title{Synthetic Control Method (SCM) in Asset Pricing: A Rigorous Mathematical Guide with Weighted Optimization, Regularization, and Linear Regression Interpretation}

\author{Jesus Villota Miranda\\
cemfi}

\date{}

\maketitle

\tableofcontents

\newpage

\section*{Introduction}

The Synthetic Control Method (SCM) is a powerful statistical technique initially developed for causal inference in comparative case studies. In asset pricing, SCM can be adapted to construct a synthetic portfolio that closely replicates the return dynamics of a target asset by optimally weighting a combination of other assets. This guide provides a rigorous mathematical framework for SCM in asset pricing, incorporating a weighting matrix in the optimization, regularization techniques, and an interpretation of SCM within a linear regression context.

\section{Mathematical Framework for SCM with Weighted Optimization}

\subsection{Notation and Setup}

\begin{itemize}
    \item \textbf{Time Horizon}: Discrete time periods \( t = 1, 2, \dots, T \).
    \item \textbf{Assets}:
    \begin{itemize}
        \item \textbf{Target Asset}: Return series \( Y_{0t} \) for \( t = 1, 2, \dots, T \).
        \item \textbf{Donor Pool Assets}: Return series \( Y_{it} \) for assets \( i = 1, 2, \dots, N \).
    \end{itemize}
    \item \textbf{Calibration (Training) Period}: A subset of periods \( t = t_1, t_2, \dots, t_k \) used for model estimation.
    \item \textbf{Vectors and Matrices}:
    \begin{itemize}
        \item \( \mathbf{Y}_0 = [Y_{0t_1}, Y_{0t_2}, \dots, Y_{0t_k}]' \) (dimension \( k \times 1 \)).
        \item \( \mathbf{Y} = [\mathbf{Y}_1, \mathbf{Y}_2, \dots, \mathbf{Y}_N] \), where \( \mathbf{Y}_i = [Y_{it_1}, Y_{it_2}, \dots, Y_{it_k}]' \) (dimension \( k \times N \)).
    \end{itemize}
\end{itemize}

\subsection{Weighted Optimization Objective}

We introduce a weighting matrix \( \mathbf{W} \) to weight the discrepancies in the objective function, allowing differential importance to different observations.

\subsubsection{Objective Function with Weighting Matrix}

The weighted optimization problem is formulated as:

\[
\min_{\mathbf{w}} \; \left( \mathbf{Y}_0 - \mathbf{Y} \mathbf{w} \right)' \mathbf{W} \left( \mathbf{Y}_0 - \mathbf{Y} \mathbf{w} \right)
\]

\begin{itemize}
    \item \( \mathbf{w} = [w_1, w_2, \dots, w_N]' \) is the weight vector assigned to donor pool assets.
    \item \( \mathbf{W} \) is a \( k \times k \) symmetric positive definite weighting matrix.
\end{itemize}

\subsubsection{Interpretation of the Weighting Matrix}

\begin{itemize}
    \item \textbf{Purpose}: The weighting matrix \( \mathbf{W} \) allows us to place more emphasis on certain time periods or to account for heteroskedasticity in the errors.
    \item \textbf{Examples}:
    \begin{itemize}
        \item \textbf{Diagonal Matrix}: If \( \mathbf{W} \) is diagonal with elements \( w_{tt} \), it weights each time period \( t \) individually.
        \item \textbf{Inverse Covariance Matrix}: Setting \( \mathbf{W} = \mathbf{\Sigma}^{-1} \), where \( \mathbf{\Sigma} \) is the covariance matrix of the errors, leads to the Generalized Least Squares (GLS) estimator.
    \end{itemize}
\end{itemize}

\subsection{Constraints}

We impose the following constraints on the weights:

\begin{enumerate}
    \item \textbf{Sum-to-One Constraint}:
    \[
    \sum_{i=1}^{N} w_i = 1
    \]
    \item \textbf{Optional Non-Negativity Constraint}:
    \[
    w_i \geq 0 \quad \text{for all } i = 1, 2, \dots, N
    \]
\end{enumerate}

\subsection{Optimization Problem}

The optimization problem becomes:

\[
\begin{aligned}
\min_{\mathbf{w}} & \quad \left( \mathbf{Y}_0 - \mathbf{Y} \mathbf{w} \right)' \mathbf{W} \left( \mathbf{Y}_0 - \mathbf{Y} \mathbf{w} \right) + \lambda \mathcal{R}(\mathbf{w}) \\
\text{subject to} & \quad \sum_{i=1}^{N} w_i = 1 \\
& \quad w_i \geq 0 \quad \text{(if non-negativity is imposed)}
\end{aligned}
\]

\begin{itemize}
    \item \( \lambda \geq 0 \) is a regularization parameter.
    \item \( \mathcal{R}(\mathbf{w}) \) is a regularization term to prevent overfitting.
\end{itemize}

\subsection{Regularization Techniques}

\subsubsection{Purpose of Regularization}

Regularization introduces a penalty for large or complex weight vectors, enhancing the model's generalization ability by preventing overfitting to the calibration data.

\subsubsection{Types of Regularization}

\begin{enumerate}
    \item \textbf{Lasso Regularization (L1 Penalty)}:

    \[
    \mathcal{R}(\mathbf{w}) = \| \mathbf{w} \|_1 = \sum_{i=1}^{N} |w_i|
    \]

    \begin{itemize}
        \item Promotes sparsity in \( \mathbf{w} \), potentially setting some weights to zero.
        \item Useful when the number of donor assets \( N \) is large.
    \end{itemize}

    \item \textbf{Ridge Regularization (L2 Penalty)}:

    \[
    \mathcal{R}(\mathbf{w}) = \| \mathbf{w} \|_2^2 = \sum_{i=1}^{N} w_i^2
    \]

    \begin{itemize}
        \item Shrinks weights towards zero but does not enforce exact zeros.
        \item Stabilizes the solution when predictors are highly correlated.
    \end{itemize}

    \item \textbf{Elastic Net Regularization}:

    Combines L1 and L2 penalties:

    \[
    \mathcal{R}(\mathbf{w}) = \alpha \| \mathbf{w} \|_1 + (1 - \alpha) \| \mathbf{w} \|_2^2
    \]

    \begin{itemize}
        \item \( \alpha \in [0, 1] \) balances between Lasso and Ridge penalties.
    \end{itemize}
\end{enumerate}

\subsubsection{Recommendation}

\begin{itemize}
    \item \textbf{Elastic Net Regularization} is recommended because it combines the benefits of Lasso and Ridge:
    \begin{itemize}
        \item Handles multicollinearity among donor assets.
        \item Allows for variable selection and coefficient shrinkage.
    \end{itemize}
    \item The choice of \( \lambda \) and \( \alpha \) can be determined via cross-validation.
\end{itemize}

\subsection{Solution to the Regularized Weighted Optimization}

\subsubsection{Lagrangian Function}

Define the Lagrangian (assuming non-negativity constraints are not imposed for simplicity):

\[
\mathcal{L}(\mathbf{w}, \mu) = \left( \mathbf{Y}_0 - \mathbf{Y} \mathbf{w} \right)' \mathbf{W} \left( \mathbf{Y}_0 - \mathbf{Y} \mathbf{w} \right) + \lambda \mathcal{R}(\mathbf{w}) - \mu \left( \sum_{i=1}^{N} w_i - 1 \right)
\]

\begin{itemize}
    \item \( \mu \) is the Lagrange multiplier for the equality constraint.
\end{itemize}

\subsubsection{First-Order Conditions}

For differentiable regularization terms (e.g., Ridge):

\begin{enumerate}
    \item \textbf{Gradient with respect to \( \mathbf{w} \)}:

    \[
    -2 \mathbf{Y}' \mathbf{W} \left( \mathbf{Y}_0 - \mathbf{Y} \mathbf{w} \right) + \lambda \nabla \mathcal{R}(\mathbf{w}) - \mu \mathbf{1} = \mathbf{0}
    \]

    \begin{itemize}
        \item For Ridge regularization, \( \nabla \mathcal{R}(\mathbf{w}) = 2 \mathbf{w} \).
    \end{itemize}

    \item \textbf{Derivative with respect to \( \mu \)}:

    \[
    \sum_{i=1}^{N} w_i - 1 = 0
    \]
\end{enumerate}

\subsubsection{Solution for Ridge Regularization}

With Ridge regularization (\( \mathcal{R}(\mathbf{w}) = \| \mathbf{w} \|_2^2 \)):

\[
-2 \mathbf{Y}' \mathbf{W} \left( \mathbf{Y}_0 - \mathbf{Y} \mathbf{w} \right) + 2 \lambda \mathbf{w} - \mu \mathbf{1} = \mathbf{0}
\]

Simplify:

\[
2 \left( \mathbf{Y}' \mathbf{W} \mathbf{Y} + \lambda \mathbf{I} \right) \mathbf{w} - 2 \mathbf{Y}' \mathbf{W} \mathbf{Y}_0 - \mu \mathbf{1} = \mathbf{0}
\]

Let \( \mathbf{M} = \mathbf{Y}' \mathbf{W} \mathbf{Y} + \lambda \mathbf{I} \) and \( \mathbf{b} = \mathbf{Y}' \mathbf{W} \mathbf{Y}_0 \). Then:

\[
\mathbf{M} \mathbf{w} - \mathbf{b} + \frac{\mu}{2} \mathbf{1} = \mathbf{0}
\]

From the constraint:

\[
\mathbf{1}' \mathbf{w} = 1
\]

Stacking the equations:

\[
\begin{bmatrix}
\mathbf{M} & \frac{1}{2} \mathbf{1} \\
\mathbf{1}' & 0
\end{bmatrix}
\begin{bmatrix}
\mathbf{w} \\
\mu
\end{bmatrix}
=
\begin{bmatrix}
\mathbf{b} \\
1
\end{bmatrix}
\]

This linear system can be solved for \( \mathbf{w} \) and \( \mu \).

\subsubsection{Solution for Lasso Regularization}

With Lasso regularization, the optimization problem becomes a Quadratic Programming (QP) problem with an L1 penalty, which is not differentiable.

\begin{itemize}
    \item \textbf{Approach}: Use specialized algorithms such as the Coordinate Descent method or Least Angle Regression (LARS).
    \item \textbf{Constraints}: The sum-to-one constraint is maintained, and non-negativity constraints can be incorporated.
\end{itemize}

\subsection{Interpretation as a Portfolio of Traded Assets}

The synthetic asset is constructed as:

\[
\hat{Y}_{0t} = \sum_{i=1}^{N} w_i Y_{it}
\]

\begin{itemize}
    \item The weights \( w_i \) represent the proportion of the investment in each donor asset.
    \item Incorporating regularization may result in some weights being exactly zero (sparse portfolio) or shrunk towards zero (diversified portfolio).
\end{itemize}

\section{Interpretation of SCM in a Linear Regression Context}

\subsection{SCM as a Constrained Regression}

The SCM optimization problem can be interpreted as a constrained linear regression model.

\subsubsection{Model Specification}

Consider the linear model:

\[
Y_{0t} = \sum_{i=1}^{N} w_i Y_{it} + \varepsilon_t
\]

\begin{itemize}
    \item \( Y_{0t} \) is the dependent variable (target asset return).
    \item \( Y_{it} \) are the independent variables (donor assets' returns).
    \item \( w_i \) are the regression coefficients (weights).
    \item \( \varepsilon_t \) is the error term.
\end{itemize}

\subsubsection{Constraints}

\begin{itemize}
    \item The regression coefficients \( w_i \) are subject to the sum-to-one constraint and possibly non-negativity constraints.
\end{itemize}

\subsubsection{Estimation}

The estimation of \( w_i \) involves minimizing the residual sum of squares under the specified constraints, which is equivalent to the SCM optimization problem.

\subsection{Connection to Generalized Least Squares (GLS)}

When the weighting matrix \( \mathbf{W} \) is the inverse of the error covariance matrix, the estimation corresponds to GLS under constraints.

\subsection{Econometric Implications}

\begin{itemize}
    \item \textbf{Standard Errors}: The covariance matrix of the estimated weights can be derived from the regression framework.
    \item \textbf{Inference}: Standard regression inference techniques can be applied, taking into account the constraints.
\end{itemize}

\section{Econometric Properties for Inference with Regularization and Weighting}

\subsection{Estimator Properties}

\subsubsection{Consistency and Bias}

\begin{itemize}
    \item \textbf{Regularization Bias}: Regularization introduces bias into the estimates in exchange for reduced variance.
    \item \textbf{Consistency}:
    \begin{itemize}
        \item With appropriate choice of regularization parameters (e.g., \( \lambda \rightarrow 0 \) as \( k \rightarrow \infty \)), the estimator can remain consistent.
        \item The weighting matrix \( \mathbf{W} \) does not affect consistency if correctly specified.
    \end{itemize}
\end{itemize}

\subsubsection{Asymptotic Distribution}

\begin{itemize}
    \item The asymptotic distribution of the estimator is more complex due to regularization and constraints.
    \item Under certain conditions, the estimator is asymptotically normal.
\end{itemize}

\subsection{Statistical Inference}

\subsubsection{Variance Estimation}

\begin{itemize}
    \item \textbf{Bootstrap Methods}: Non-parametric bootstrap can be used to estimate the variance of \( \hat{\mathbf{w}} \) since analytic expressions may be intractable due to regularization.
    \item \textbf{Analytic Approximations}: For Ridge regression, the variance can be approximated analytically.
\end{itemize}

\subsubsection{Hypothesis Testing}

\begin{itemize}
    \item \textbf{Penalty-adjusted Tests}: Adjust standard tests to account for the bias introduced by regularization.
    \item \textbf{Cross-Validation}: Use cross-validation techniques to select regularization parameters that balance bias and variance.
\end{itemize}

\subsection{Model Selection and Regularization Parameter Tuning}

\begin{itemize}
    \item \textbf{Cross-Validation}: Divide the data into training and validation sets to evaluate the performance for different \( \lambda \) and \( \alpha \) values.
    \item \textbf{Information Criteria}: Use criteria such as AIC or BIC adjusted for the penalty term.
\end{itemize}

\subsection{Robustness and Sensitivity Analysis}

\begin{itemize}
    \item Assess how sensitive the results are to the choice of weighting matrix \( \mathbf{W} \) and regularization parameters.
    \item Perform robustness checks by varying \( \mathbf{W} \) (e.g., using different weighting schemes) and observing the impact on the estimated weights.
\end{itemize}

\section{Conclusion}

Incorporating a weighting matrix and regularization into the SCM framework enhances its flexibility and robustness in asset pricing applications. The weighting matrix allows differential importance of observations, while regularization prevents overfitting and handles multicollinearity among donor assets. Interpreting SCM within a linear regression context bridges the gap between SCM and traditional econometric methods, facilitating the application of standard inference techniques.

\section*{References}

\begin{enumerate}
    \item Abadie, A., \& Gardeazabal, J. (2003). The Economic Costs of Conflict: A Case Study of the Basque Country. \textit{American Economic Review}, 93(1), 113-132.
    \item Abadie, A., Diamond, A., \& Hainmueller, J. (2010). Synthetic Control Methods for Comparative Case Studies: Estimating the Effect of California's Tobacco Control Program. \textit{Journal of the American Statistical Association}, 105(490), 493-505.
    \item Tibshirani, R. (1996). Regression Shrinkage and Selection via the Lasso. \textit{Journal of the Royal Statistical Society: Series B (Methodological)}, 58(1), 267-288.
    \item Zou, H., \& Hastie, T. (2005). Regularization and Variable Selection via the Elastic Net. \textit{Journal of the Royal Statistical Society: Series B (Statistical Methodology)}, 67(2), 301-320.
    \item Hastie, T., Tibshirani, R., \& Friedman, J. (2009). \textit{The Elements of Statistical Learning}. Springer.
\end{enumerate}



%%%%%%%%%%%%%%%%%%%%%%%%%%%%%%%%%%%%%%%%%%%%%%%%%%%%%
%%%%%%%%%%%%%%%%%%%%%%%%%%%%%%%%%%%%%%%%%%%%%%%%%%%%%
%%%%%%%%%%%%%%%%%%%%%%%%%%%%%%%%%%%%%%%%%%%%%%%%%%%%%
%%%%%%%%%%%%%%%%%%%%%%%%%%%%%%%%%%%%%%%%%%%%%%%%%%%%%
%%%%%%%%%%%%%%%%%%%%%%%%%%%%%%%%%%%%%%%%%%%%%%%%%%%%%
%%%%%%%%%%%%%%%%%%%%%%%%%%%%%%%%%%%%%%%%%%%%%%%%%%%%%
%%%%%%%%%%%%%%%%%%%%%%%%%%%%%%%%%%%%%%%%%%%%%%%%%%%%%
%%%%%%%%%%%%%%%%%%%%%%%%%%%%%%%%%%%%%%%%%%%%%%%%%%%%%
%%%%%%%%%%%%%%%%%%%%%%%%%%%%%%%%%%%%%%%%%%%%%%%%%%%%%

\section*{Introduction}

In this analysis, we aim to derive the asymptotic properties of the estimator used in constructing the synthetic control for asset pricing, as specified in your preliminary document. We will rigorously analyze the estimator's consistency and asymptotic distribution under appropriate econometric assumptions.

\section{Setup and Notation}

Consider:

\begin{itemize}
    \item \textbf{Universe of Stocks}: $\mathcal{S} = \{0, 1, \ldots, N\}$.
    \item \textbf{Time Periods}: $\mathcal{T} = \{1, 2, \ldots, T\}$.
    \item \textbf{Target Asset}: Stock $0$.
    \item \textbf{Donor Pool}: $\mathcal{J} = \{1, 2, \ldots, J\}$, where $J \leq N$.
    \item \textbf{Returns}:
    \begin{itemize}
        \item $R_{it}$: Return of stock $i$ at time $t$.
    \end{itemize}
\end{itemize}

We define the \textbf{synthetic return} for the target asset as:

\[
R_{0t}^* = \sum_{j \in \mathcal{J}} w_j R_{jt}.
\]

\textbf{Objective}: Estimate weights $\{w_j\}_{j \in \mathcal{J}}$ by minimizing the tracking error over the training period $\mathcal{T}_{\text{tr}} \subseteq \mathcal{T}$:

\[
\begin{aligned}
\min_{\{w_j\}} & \quad \sum_{t \in \mathcal{T}_{\text{tr}}} \left( R_{0t} - \sum_{j \in \mathcal{J}} w_j R_{jt} \right)^2 \\
\text{subject to} & \quad \sum_{j \in \mathcal{J}} w_j = 1.
\end{aligned}
\]

\textbf{Note}: We allow weights to be negative (short selling is permitted).

\section{Estimator Derivation}

\subsection{Vector and Matrix Notation}

\begin{itemize}
    \item Let $T_{\text{tr}} = |\mathcal{T}_{\text{tr}}|$ be the number of periods in the training sample.
    \item \textbf{Target Return Vector}:
    \[
    \mathbf{R}_0 = \begin{bmatrix} R_{0, t_1} \\ R_{0, t_2} \\ \vdots \\ R_{0, t_{T_{\text{tr}}}} \end{bmatrix} \in \mathbb{R}^{T_{\text{tr}} \times 1}.
    \]
    \item \textbf{Donor Returns Matrix}:
    \[
    \mathbf{R} = \begin{bmatrix}
    R_{1, t_1} & R_{2, t_1} & \cdots & R_{J, t_1} \\
    R_{1, t_2} & R_{2, t_2} & \cdots & R_{J, t_2} \\
    \vdots & \vdots & \ddots & \vdots \\
    R_{1, t_{T_{\text{tr}}}} & R_{2, t_{T_{\text{tr}}}} & \cdots & R_{J, t_{T_{\text{tr}}}}
    \end{bmatrix} \in \mathbb{R}^{T_{\text{tr}} \times J}.
    \]
    \item \textbf{Weights Vector}:
    \[
    \mathbf{w} = \begin{bmatrix} w_1 \\ w_2 \\ \vdots \\ w_J \end{bmatrix} \in \mathbb{R}^{J \times 1}.
    \]
    \item \textbf{Constraint Vector}:
    \[
    \mathbf{1}' \mathbf{w} = \sum_{j=1}^J w_j = 1,
    \]
    where $\mathbf{1} = \begin{bmatrix} 1 & 1 & \cdots & 1 \end{bmatrix}' \in \mathbb{R}^{J \times 1}$.
\end{itemize}

\subsection{Optimization Problem in Matrix Form}

The optimization problem becomes:

\[
\begin{aligned}
\min_{\mathbf{w}} & \quad (\mathbf{R}_0 - \mathbf{R} \mathbf{w})' (\mathbf{R}_0 - \mathbf{R} \mathbf{w}) \\
\text{subject to} & \quad \mathbf{1}' \mathbf{w} = 1.
\end{aligned}
\]

\subsection{Lagrangian Formulation}

Introduce a Lagrange multiplier $\lambda$ for the equality constraint:

\[
\mathcal{L}(\mathbf{w}, \lambda) = (\mathbf{R}_0 - \mathbf{R} \mathbf{w})' (\mathbf{R}_0 - \mathbf{R} \mathbf{w}) - \lambda (\mathbf{1}' \mathbf{w} - 1).
\]

\subsection{First-Order Conditions (FOC)}

Compute the gradient of the Lagrangian with respect to $\mathbf{w}$ and $\lambda$:

\begin{enumerate}
    \item \textbf{Derivative with respect to $\mathbf{w}$}:
    \[
    \frac{\partial \mathcal{L}}{\partial \mathbf{w}} = -2 \mathbf{R}' (\mathbf{R}_0 - \mathbf{R} \mathbf{w}) - \lambda \mathbf{1} = \mathbf{0}.
    \]
    \item \textbf{Derivative with respect to $\lambda$}:
    \[
    \frac{\partial \mathcal{L}}{\partial \lambda} = - (\mathbf{1}' \mathbf{w} - 1) = 0.
    \]
\end{enumerate}

\subsection{Solving the FOC}

\subsubsection{Equation (1):}

\[
2 \mathbf{R}' \mathbf{R} \mathbf{w} - 2 \mathbf{R}' \mathbf{R}_0 + \lambda \mathbf{1} = \mathbf{0}.
\]

Simplify:

\[
\mathbf{R}' \mathbf{R} \mathbf{w} - \mathbf{R}' \mathbf{R}_0 + \frac{\lambda}{2} \mathbf{1} = \mathbf{0}.
\]

\subsubsection{Equation (2):}

\[
\mathbf{1}' \mathbf{w} = 1.
\]

\subsection{System of Equations}

Stack the equations:

\[
\begin{cases}
\mathbf{R}' \mathbf{R} \mathbf{w} + \dfrac{\lambda}{2} \mathbf{1} = \mathbf{R}' \mathbf{R}_0 \\
\mathbf{1}' \mathbf{w} = 1.
\end{cases}
\]

Let:

\begin{itemize}
    \item $\mathbf{M} = \mathbf{R}' \mathbf{R}$ (a $J \times J$ matrix).
    \item $\mathbf{b} = \mathbf{R}' \mathbf{R}_0$ (a $J \times 1$ vector).
\end{itemize}

Then the system is:

\[
\begin{bmatrix}
\mathbf{M} & \dfrac{1}{2} \mathbf{1} \\
\mathbf{1}' & 0
\end{bmatrix}
\begin{bmatrix}
\mathbf{w} \\
\lambda
\end{bmatrix}
=
\begin{bmatrix}
\mathbf{b} \\
1
\end{bmatrix}.
\]

\section{Asymptotic Properties of the Estimator}

To analyze the asymptotic properties, we consider the behavior of the estimator $\hat{\mathbf{w}}$ as $T_{\text{tr}} \to \infty$.

\subsection{Assumptions}

\begin{enumerate}
    \item \textbf{Stationarity and Ergodicity}: The return series $\{R_{it}\}$ for $i = 0, 1, \dots, N$ are stationary and ergodic processes.
    \item \textbf{Finite Moments}: The returns have finite second moments:
    \[
    \mathbb{E}[R_{it}^2] < \infty.
    \]
    \item \textbf{Non-singularity}: The matrix $\lim_{T_{\text{tr}} \to \infty} \dfrac{1}{T_{\text{tr}}} \mathbf{R}' \mathbf{R}$ is positive definite.
    \item \textbf{No Perfect Multicollinearity}: The columns of $\mathbf{R}$ are linearly independent in the limit.
    \item \textbf{Exogeneity}: The error term is uncorrelated with the regressors in large samples:
    \[
    \lim_{T_{\text{tr}} \to \infty} \dfrac{1}{T_{\text{tr}}} \mathbf{R}' \boldsymbol{\varepsilon} = \mathbf{0},
    \]
    where $\boldsymbol{\varepsilon} = \mathbf{R}_0 - \mathbf{R} \mathbf{w}_0$, and $\mathbf{w}_0$ is the true weight vector.
\end{enumerate}

\subsection{Consistency of the Estimator}

We aim to show that $\hat{\mathbf{w}} \xrightarrow{p} \mathbf{w}_0$ as $T_{\text{tr}} \to \infty$.

\subsubsection{Proof Outline}

\begin{enumerate}
    \item \textbf{Law of Large Numbers (LLN)}:
    \begin{itemize}
        \item By the LLN, we have:
        \[
        \dfrac{1}{T_{\text{tr}}} \mathbf{R}' \mathbf{R} \xrightarrow{p} \mathbf{\Sigma}_{RR},
        \]
        \[
        \dfrac{1}{T_{\text{tr}}} \mathbf{R}' \mathbf{R}_0 \xrightarrow{p} \mathbf{\Sigma}_{R R_0},
        \]
        where $\mathbf{\Sigma}_{RR}$ is the population covariance matrix of the donor assets, and $\mathbf{\Sigma}_{R R_0}$ is the population covariance vector between the donor assets and the target asset.
    \end{itemize}
    \item \textbf{Convergence of the Estimator}:
    \begin{itemize}
        \item The sample moments converge to their population counterparts.
        \item The estimator satisfies:
        \[
        \mathbf{M} \hat{\mathbf{w}} + \dfrac{\hat{\lambda}}{2} \mathbf{1} = \mathbf{b}.
        \]
        \item In large samples:
        \[
        \mathbf{\Sigma}_{RR} \hat{\mathbf{w}} + \dfrac{\hat{\lambda}}{2T_{\text{tr}}} \mathbf{1} \approx \mathbf{\Sigma}_{R R_0}.
        \]
        \item Since $\mathbf{1}' \hat{\mathbf{w}} = 1$, the solution converges to the true weights $\mathbf{w}_0$.
    \end{itemize}
\end{enumerate}

\subsubsection{Conclusion}

Under the given assumptions, the estimator $\hat{\mathbf{w}}$ is \textbf{consistent}.

\subsection{Asymptotic Distribution of the Estimator}

We derive the asymptotic distribution of $\hat{\mathbf{w}}$ to establish asymptotic normality.

\subsubsection{Linearization}

Let the true model be:

\[
\mathbf{R}_0 = \mathbf{R} \mathbf{w}_0 + \boldsymbol{\varepsilon},
\]

where $\boldsymbol{\varepsilon}$ is a vector of error terms with mean zero and covariance matrix $\sigma^2 \mathbf{I}$.

\subsubsection{First-Order Taylor Expansion}

Consider the estimator $\hat{\mathbf{w}}$ as a function of sample moments:

\[
\hat{\mathbf{w}} = f\left( \mathbf{R}' \mathbf{R}, \mathbf{R}' \mathbf{R}_0 \right).
\]

We can linearize $f$ around the population moments.

\subsubsection{Asymptotic Distribution}

\begin{enumerate}
    \item \textbf{Define the Estimation Error}:

    \[
    \hat{\mathbf{w}} - \mathbf{w}_0 = \left( \mathbf{M}^{-1} - \dfrac{\mathbf{M}^{-1} \mathbf{1} \mathbf{1}' \mathbf{M}^{-1}}{\mathbf{1}' \mathbf{M}^{-1} \mathbf{1}} \right) (\mathbf{R}' \boldsymbol{\varepsilon}) + o_p(T_{\text{tr}}^{-1/2}).
    \]

    \item \textbf{Variance-Covariance Matrix}:

    The asymptotic variance of $\hat{\mathbf{w}}$ is given by:

    \[
    \text{Var}(\hat{\mathbf{w}}) = \sigma^2 \left( \mathbf{M}^{-1} - \dfrac{\mathbf{M}^{-1} \mathbf{1} \mathbf{1}' \mathbf{M}^{-1}}{\mathbf{1}' \mathbf{M}^{-1} \mathbf{1}} \right).
    \]

    \item \textbf{Asymptotic Normality}:

    Under the Central Limit Theorem (CLT):

    \[
    \sqrt{T_{\text{tr}}} (\hat{\mathbf{w}} - \mathbf{w}_0) \xrightarrow{d} \mathcal{N}\left( \mathbf{0}, \sigma^2 \left( \mathbf{\Sigma}_{RR}^{-1} - \dfrac{\mathbf{\Sigma}_{RR}^{-1} \mathbf{1} \mathbf{1}' \mathbf{\Sigma}_{RR}^{-1}}{\mathbf{1}' \mathbf{\Sigma}_{RR}^{-1} \mathbf{1}} \right) \right).
    \]
\end{enumerate}

\subsection{Derivation Details}

\subsubsection{1. Expansion of the Estimator}

From the FOC:

\[
\mathbf{M} \hat{\mathbf{w}} + \dfrac{\hat{\lambda}}{2} \mathbf{1} = \mathbf{b}.
\]

Subtract the population counterpart:

\[
\mathbf{M} (\hat{\mathbf{w}} - \mathbf{w}_0) + \dfrac{\hat{\lambda} - \lambda_0}{2} \mathbf{1} = \mathbf{b} - \mathbf{M} \mathbf{w}_0 - \dfrac{\lambda_0}{2} \mathbf{1}.
\]

But since $\mathbf{b} = \mathbf{R}' \mathbf{R}_0$ and $\mathbf{M} \mathbf{w}_0 = \mathbf{R}' \mathbf{R} \mathbf{w}_0$, and using $\mathbf{R}_0 = \mathbf{R} \mathbf{w}_0 + \boldsymbol{\varepsilon}$:

\[
\mathbf{b} - \mathbf{M} \mathbf{w}_0 = \mathbf{R}' \boldsymbol{\varepsilon}.
\]

Thus:

\[
\mathbf{M} (\hat{\mathbf{w}} - \mathbf{w}_0) + \dfrac{\hat{\lambda} - \lambda_0}{2} \mathbf{1} = \mathbf{R}' \boldsymbol{\varepsilon}.
\]

\subsubsection{2. Incorporating the Constraint}

From the constraint:

\[
\mathbf{1}' (\hat{\mathbf{w}} - \mathbf{w}_0) = 0,
\]

since $\mathbf{1}' \hat{\mathbf{w}} = \mathbf{1}' \mathbf{w}_0 = 1$.

\subsubsection{3. Solving for $\hat{\mathbf{w}} - \mathbf{w}_0$}

Let $ \mathbf{C} = \mathbf{M} $, $ \mathbf{d} = \mathbf{R}' \boldsymbol{\varepsilon} $, and $ \mathbf{A} = \mathbf{1}' $.

We have:

\[
\begin{cases}
\mathbf{C} (\hat{\mathbf{w}} - \mathbf{w}_0) + \dfrac{\hat{\lambda} - \lambda_0}{2} \mathbf{1} = \mathbf{d} \\
\mathbf{A} (\hat{\mathbf{w}} - \mathbf{w}_0) = 0.
\end{cases}
\]

This is a system of linear equations with constraints.

\subsubsection{4. Applying the Frisch-Waugh-Lovell Theorem}

We can project $\mathbf{d}$ onto the orthogonal complement of $\mathbf{1}$:

\begin{itemize}
    \item Let $ \mathbf{P} = \mathbf{I} - \dfrac{\mathbf{1} \mathbf{1}'}{\mathbf{1}' \mathbf{1}} $ be the projection matrix onto the orthogonal complement of $\mathbf{1}$.
    \item The adjusted estimation error:

    \[
    \hat{\mathbf{w}} - \mathbf{w}_0 = \mathbf{C}^{-1} \mathbf{P} \mathbf{d}.
    \]

    However, since $\mathbf{1}' (\hat{\mathbf{w}} - \mathbf{w}_0) = 0$, the component along $\mathbf{1}$ is zero.
\end{itemize}

\subsubsection{5. Asymptotic Variance}

Compute the variance of $\hat{\mathbf{w}} - \mathbf{w}_0$:

\[
\text{Var}(\hat{\mathbf{w}} - \mathbf{w}_0) = \mathbf{C}^{-1} \mathbf{P} \text{Var}(\mathbf{d}) \mathbf{P}' \mathbf{C}^{-1}.
\]

Since $\mathbf{d} = \mathbf{R}' \boldsymbol{\varepsilon}$, and assuming $\boldsymbol{\varepsilon}$ is homoskedastic with variance $\sigma^2$:

\[
\text{Var}(\mathbf{d}) = \sigma^2 \mathbf{R}' \mathbf{R}.
\]

Therefore:

\[
\text{Var}(\hat{\mathbf{w}} - \mathbf{w}_0) = \sigma^2 \mathbf{C}^{-1} \mathbf{P} \mathbf{C} \mathbf{P}' \mathbf{C}^{-1} = \sigma^2 \left( \mathbf{C}^{-1} - \dfrac{\mathbf{C}^{-1} \mathbf{1} \mathbf{1}' \mathbf{C}^{-1}}{\mathbf{1}' \mathbf{C}^{-1} \mathbf{1}} \right).
\]

This matches the expression derived earlier.

\section{Implications for Inference}

\subsection{Standard Errors}

The asymptotic covariance matrix can be estimated using sample counterparts:

\[
\hat{\text{Var}}(\hat{\mathbf{w}}) = \hat{\sigma}^2 \left( \mathbf{M}^{-1} - \dfrac{\mathbf{M}^{-1} \mathbf{1} \mathbf{1}' \mathbf{M}^{-1}}{\mathbf{1}' \mathbf{M}^{-1} \mathbf{1}} \right),
\]

where:

\[
\hat{\sigma}^2 = \dfrac{1}{T_{\text{tr}} - J} (\mathbf{R}_0 - \mathbf{R} \hat{\mathbf{w}})' (\mathbf{R}_0 - \mathbf{R} \hat{\mathbf{w}}).
\]

\subsection{Hypothesis Testing}

\begin{itemize}
    \item \textbf{Individual Weights}: Test $ H_0: w_j = w_{j0} $ using:

    \[
    t_j = \dfrac{\hat{w}_j - w_{j0}}{\sqrt{\hat{\text{Var}}(\hat{w}_j)}} \sim t_{T_{\text{tr}} - J},
    \]

    where $\hat{\text{Var}}(\hat{w}_j)$ is the $j$-th diagonal element of $\hat{\text{Var}}(\hat{\mathbf{w}})$.

    \item \textbf{Joint Hypotheses}: Use Wald tests for testing linear restrictions on $\hat{\mathbf{w}}$.
\end{itemize}

\subsection{Confidence Intervals}

Construct $(1 - \alpha)$ confidence intervals for each $w_j$:

\[
\hat{w}_j \pm z_{\alpha/2} \sqrt{\hat{\text{Var}}(\hat{w}_j)},
\]

where $ z_{\alpha/2} $ is the critical value from the standard normal distribution.

\section{Conclusion}

We have rigorously derived the asymptotic properties of the synthetic control estimator in the asset pricing context:

\begin{itemize}
    \item \textbf{Consistency}: The estimator $\hat{\mathbf{w}}$ is consistent under standard econometric assumptions.
    \item \textbf{Asymptotic Normality}: The estimator is asymptotically normally distributed, with a covariance matrix that accounts for the equality constraint.
    \item \textbf{Inference}: Standard errors and confidence intervals can be constructed using the derived asymptotic variance, facilitating hypothesis testing and inference.
\end{itemize}

This analysis provides a solid econometric foundation for the use of synthetic control methods in financial applications, particularly in constructing portfolios that replicate the returns of a target asset.

\section*{Remarks}

\begin{itemize}
    \item \textbf{Relaxation of Assumptions}: If the error terms are heteroskedastic or autocorrelated, adjustments to the variance estimator (e.g., using robust standard errors) are necessary.
    \item \textbf{Time Series Considerations}: In financial data, returns may exhibit autocorrelation and conditional heteroskedasticity (e.g., GARCH effects). These need to be accounted for in a more refined analysis.
    \item \textbf{Finite Sample Properties}: While the asymptotic analysis provides insights for large samples, finite sample performance should be assessed via simulation studies or bootstrap methods.
\end{itemize}

\section*{References}

\begin{enumerate}
    \item Hayashi, F. (2000). \textit{Econometrics}. Princeton University Press.
    \item White, H. (2001). \textit{Asymptotic Theory for Econometricians}. Academic Press.
    \item Davidson, R., \& MacKinnon, J. G. (1993). \textit{Estimation and Inference in Econometrics}. Oxford University Press.
\end{enumerate}



\end{document}






