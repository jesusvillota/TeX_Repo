\paragraph{Generalizing Pairs Trading: From Pairs to Sparse Synthetic Replicas}

The decay in profitability of the classic pairs trade, however, does not necessarily invalidate the underlying principle of relative-value arbitrage. Rather, it suggests that the traditional one-to-one pairing constraint may be too restrictive for modern, complex markets where simple, stable relationships between two individual stocks have become scarce. The core insight of pairs trading --that a security's value can be assessed relative to a close substitute-- remains powerful. The challenge lies in identifying or creating a more robust substitute.

Traditional pairs trading imposes a severe cardinality constraint by limiting the replication of a target asset to a single substitute security. This restriction, while conceptually elegant, may be unnecessarily limiting in practice. Consider General Motors: rather than searching for a single stock whose normalized price series closely tracks GM's movements, we could construct a synthetic replica using a linear combination of multiple securities. This approach maintains the fundamental economic intuition of pairs trading--exploiting temporary deviations between economically related assets--while providing greater flexibility in replica construction.

The theoretical foundation remains rooted in relative pricing theory, where securities serving as close economic substitutes should exhibit similar price dynamics. However, our framework relaxes the stringent requirement that such substitution be achieved through a single asset. Instead, we allow for the construction of synthetic substitutes through sparse linear combinations, potentially improving the replication of the target asset.

This paper proposes a generalization of the pairs trading framework that relaxes this rigid cardinality constraint. Instead of searching for a single, naturally occurring substitute for a target asset, we propose to construct a ``synthetic replica'' of it. Methodologically, this is achieved by regressing the normalized price series of the target asset against a broad ``donor pool'' of potential substitutes. The fitted values from this regression form the price series of the synthetic replica --a portfolio of assets weighted to best track the target. The spread in this generalized framework is therefore the regression error: the price difference between the target asset and its synthetic replica. 

\paragraph{Making the replicating portfolio sparse}
A crucial challenge, however, is that an unconstrained Ordinary Least Squares (OLS) regression using a large donor pool would yield a dense synthetic replica, composed of hundreds of assets. Such a portfolio would be prohibitively expensive to trade due to transaction costs and slippage, rendering the strategy impractical and suffering from the curse of dimensionality. To overcome this, we introduce a penalized regression approach using the Least Absolute Shrinkage and Selection Operator (LASSO). The LASSO penalty is ideally suited for this context as it promotes sparsity by forcing the coefficients of less relevant assets in the donor pool to exactly zero. This technique simultaneously performs automated stock selection and estimates the weights for the synthetic replica, effectively identifying the most important constituents for creating a parsimonious, and therefore tradable, replicating portfolio. 

\paragraph{Law of One Price}
By the Law of One Price, the target stock should have the same (normalized) price as its replicating portfolio, hence, any deviation in their difference (i.e.:, in the regression error or spread) should eventually close. Our trading strategy capitalizes on this by betting on the mean reversion of the spread, shorting the target asset when the residual is positive (i.e., the target is overpriced relative to its replica) and going long when it is negative. In this context, the cointegrating vector is 

\paragraph{Cointegration}
Gatev et al argue that pairs trading is related to cointegration (in the sense of Engle and Granger), as the spread of pairs is expected to mean revert (i.e: it is stationary). Formally, for a particular pair of stocks $i$ and $j$, their \textit{assumed} cointegrating vector is the difference between their canonical vectors $\boldsymbol \alpha = \mbf e_i - \mbf e_j$. \cref{fig:pairs_decay} shows that finding cointegrating vectors of that shape used to be an easy task in the past, as pair spreads where highly mean-reverting, and therefore, it was highly profitable to bet on this characteristic. Howevever, this property has recently become harder to trade, as markets are now more efficient. The methodology proposed in this paper relaxes the structure imposed on the cointegrating vector to $\boldsymbol \alpha = \mbf e_i - \boldsymbol \beta$, where $i$ denotes the target stock and $\boldsymbol \beta$ contains a 0 in the target stock's position and the regression coefficients of the target stock on a donor pool of assets, and a 

In this context, there is a branch of literature in statistical arbitrage called "mean reverting portfolios", whose main concern is to build portfolios of stocks that are cointegrated. Both, pairs trading and our generalized approach based on pairs trading a replicating portfolio can be embedded withing this umbrella. Note that the weights of these mean-reverting portfolios are precisely the cointegrating vectors specified in the previous paragraph.

Our methodology adapts the core formation and trading period structure of \cite{Gatev2006} to this sparse, synthetic framework. We re-examine whether this more flexible and robust form of pairs trading--trading an asset against its sparse synthetic counterpart--can restore profitability in the modern era.

%----------------------------------------------------

%However, the essence of the idea --trading the spread between two assets that are related-- can be generalized. We could take a stock, say, General Motors, and instead of looking for another stock whose \textit{normalized} price series behaves similarly to that of General Motors, we could simply try to reconstruct the \textit{normalized} price series of General Motors as a linear combination of those of other stocks. That is, a simple linear regression where we regress the price of General Motors onto the prices of a selection of assets would allow us to closely replicate the price of the first. 
%
%Similar to traditional pairs trading, this idea is also rooted in asst pricing theory. In particular, in the framework of relative pricing, where we consider that two securities are close substitutes to each other. In the case of pairs trading, we try to find one stock which could deem as "substitute" of the target stock, whereas my idea is more flexible, ins the sense that it allows to construct a synthetic substitute from a linear combination of other stocks. In either case, the mission of finding a substitute can be relaxed by not imposing such a severe cardinality constrain! Note that in the case of pairs trading, we impose a severe cardinality constraint where we limit the reconstruction of the substitue of a target asset to a unique other asset. However, if we relax this constraint and allow ourselved to find such a substitute using more assets, we can create a synthetic replica of the original stock, such that the spread between the original asset and its replica satisfies some nice and desirable properties for profitable trading. In particular, we know that the residual in an OLS regression has 0 mean by construction in the formation sample, if we exploit the estimates of the OLS regression out of sample for a reduced time period, we can hope that those residuals maintain those nice properties, and hence, we could trade the regression residual by betting on it mean reversion to 0.
%
%
%Hence, this paper proposes a novel framework rooted in asset pricing theory to pairs trade a target asset against a synthetic replica built as a linear combination of other stocks. Because trading a large pool of assets is costly, we obviously want to limit the cardinality of the donor pool of assets that constitute the replica. Pairs trading is the extreme case, where such cardinality constraint is set equal to 1. We propose a more flexible framework where we impose a lasso penalty on the OLS regression to promote sparsity in our solution, without restricting the solution with hard cardinality constraints. Lasso has some nice properties and is ideal in this context, where we aim to select the stocks that contribute the most to the replication of the target asset without the curse of dimensionality of using too many assets, which would turn our trading strategy unfeasible. 
%
%Our methodology replicates the core application from Gatev et al, and adapts it to the case where the cardinality constraint in the lookup for a substitute of the target asset is relaxed.
