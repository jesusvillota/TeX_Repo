\documentclass[12pt,article]{memoir}
\usepackage{/Users/jesusvillotamiranda/Documents/LaTeX/$$JVM_Macros}
\Subject{SUBJECT}
\Arg{ARG}

\begin{document}

In a market consisting of $N$ stocks, we denote the dividend-adjusted return on stock $i$ at trading day $t$ by $r_{i, t}$. We adopt a factor model for stock return,
\begin{align}\label{eq:1}
r_t-r_f=\beta_t F_t+\epsilon_t, \quad t=1,2, \ldots, T 
\end{align}
Here, $r_t=\left\{r_{i, t}\right\}_{i=1}^N \in \mathbb{R}^N$ are the dividend-adjusted daily return, $r_f \in \mathbb{R}$ is the risk-free rate, $F_t \in \mathbb{R}^{K \times 1}$ are the underlying factors, $\beta_t \in \mathbb{R}^{N \times K}$ are the corresponding loadings on $K$ factors, and $\epsilon_t \in \mathbb{R}^N$ are the residual returns. Factor candidates varies widely, ranging from economical-driven factors such as the Fama-French factors, to statistically-driven factors derived from PCA. In our approach, factors are selected as the leading eigenvectors in PCA. The number of factors $K$ is chosen based on the eigenvalue spectrum of the empirical correlation of daily returns.


Without loss of generality, these factors can be interpreted as portfolios of stocks,
\begin{equation}\label{eq:2}
	F_t=\omega_t\left(r_t-r_f\right)
\end{equation}
where $\omega_t \in \mathbb{R}^{K \times N}$ contains corresponding portfolio weights. Combining \cref{eq:1} and \cref{eq:2} yields
\begin{equation}\label{eq:3}
	r_t-r_f=\beta_t \omega_t\left(r_t-r_f\right)+\epsilon_t \Rightarrow \epsilon_t=\left(I-\beta_t \omega_t\right)\left(r_t-r_f\right):=\Phi_t\left(r_t-r_f\right)
\end{equation}

Here,
\begin{equation}\label{eq:4}
	\Phi_t:=\left(I-\beta_t \omega_t\right)
\end{equation}
defines a linear transformation from $r_t$ to $\epsilon_t$. More importantly, $\epsilon_{i, t}$ can be viewed as the return of a tradable portfolio with weights specified by the $i$-th row of $\Phi_t$. Consequently, the investing universe spanned by $r_t$ is termed as name equity space, and that spanned by $\epsilon_t$ as name residual space.

We denote the portfolio weights in name equity space as $w_t^{R, \text { name }}$ and portfolio weights in name residual space as $w_t^{\epsilon, \text { name }}$. These weights are related by

\begin{equation}\label{eq:5}
	w_t^{R, \text { name }}=\Phi_t^T w_t^{\epsilon, \text { name }}
\end{equation}
, directly following the equality in portfolio return,

\begin{equation}\label{eq:6}
	\left(w_t^{\epsilon \text { name }}\right)^T \epsilon_t=\left(w_t^{\epsilon, \text { name }}\right)^T \Phi_t\left(r_t-r_f\right)=\left(w_t^{R, \text { name }}\right)^T\left(r_t-r_f\right)
\end{equation}

For factors derived by PCA, we have

\begin{equation}\label{eq:7}
	\Phi_t \beta_t=0 \Longrightarrow\left(w_t^{R, \text { name }}\right)^T \beta_t=\left(w_t^{\epsilon, \text { name }}\right)^T \Phi_t \beta_t=0, \quad \forall w_t^{\epsilon, \text { name }}
\end{equation}
with proof given in the appendix. It means that for any $w_t^{\epsilon \text {,name }}$, the $w_t^{R \text {,name }}$ calculated by \cref{eq:5} satisfy,

\begin{equation}\label{eq:8}
	\left(w_t^{R, \text { name }}\right)^T\left(r_t-r_f\right)=\left(w_t^{\epsilon, \text { name }}\right)^T \Phi_t\left(\beta_t F_t+\epsilon_t\right)=\left(w_t^{\epsilon, \text { name }}\right)^T \Phi_t \epsilon_t=\left(w_t^{R, \text { name }}\right)^T \epsilon_t
\end{equation}

It suggests that the return of our statistical arbitrage portfolios is independent of market factors and relies solely on residual returns, a property usually termed as market neutrality. Ideally, portfolios are also desired to have a zero net value, known as dollar neutrality. Empirical evidence suggests that market-neutral portfolios are also approximately dollar-neutral.
\end{document}




