\section{Trading Intensity}
The extraordinary performance of our proposed LLM-based methodology warrants a careful examination of its implementation costs and practical viability. While our primary objective has been to develop a framework that better captures the economic content of news articles and their subsequent market impact, the practical implementation of such strategies necessarily involves trading frictions that could affect their real-world efficacy. In this section, we analyze the trading intensity patterns of both methodologies to provide a more complete assessment of their relative merits and to understand how transaction costs might influence their comparative advantages.
We begin by examining the temporal evolution of open positions for both approaches, which provides insights into their underlying trading dynamics and stability characteristics. This analysis is followed by detailed trading intensity metrics and concludes with a reassessment of portfolio statistics after accounting for transaction costs.

%The results from the trading strategy are really good, almost too good to be true. 
%
%As we already said, this is a paper whose goal is to better understand the incorporation of information into the market, so our focus was preeminently in developing a methodology that is able to anticipate the markets. In the process, we ignored the implications of the trading intensity of the strategy and focused only on trying to see if providing economic structure when parsing news articles did really provide improved insights into predicting market reactionsto news. 
%
%In this section, we shed light into the trading intensity of the trading strategy to see how our conclusions change as we consider realistic implications such as trading costs into the analysis. First, we will plot the number of open positions per day implied by each algorithm. 

%----------------------------------------------------
\inserthere{fig:open_positions_comparison}

\begin{figure}[htbp]
\caption{Evolution of Open Positions: KMeans vs LLM Clustering}
\label{fig:open_positions_comparison}

% Panel A: KMeans
\begin{subfigure}{\textwidth}
\caption{Panel A: KMeans Clustering}
\centering
\includegraphics[scale=0.45]{fig_KMeans_Open_Positions.pdf}
\end{subfigure}

\vspace{0.7cm}

% Panel B: LLM
\begin{subfigure}{\textwidth}
\caption{Panel B: LLM Clustering}
\centering
\includegraphics[scale=0.45]{fig_LLAMA_Open_Positions.pdf}
\end{subfigure}

\vspace{0.2cm}
\begin{minipage}{\textwidth}
\setlength{\parindent}{0pt}
{\footnotesize\textit{Note: 
This figure shows the daily evolution of the number of open positions for both Greedy (blue) and Stable (green) algorithms across different data splits (Train, Validation, Test) using KMeans clustering (Panel A) and LLM clustering (Panel B). The time period spans from July 2020 to September 2021. Vertical dashed lines separate the different data splits. The Greedy algorithm selects clusters that maximize (minimize) the cluster-average-$SR$ for long (short) positions, while the Stable algorithm minimizes the rank difference between training and validation rankings. The number of traded clusters is $\theta = 0.5k=13$ for KMeans ($k^*=26$ clusters) and $\theta = 0.5k=10$ for LLM ($k^*=20$ clusters).
}}
\end{minipage}
\end{figure}
%----------------------------------------------------

The temporal evolution of open positions reveals fundamental differences in the stability and reliability of trading signals generated by KMeans versus LLM-based clustering approaches. The KMeans implementation exhibits pronounced volatility in position management, particularly evident in the Greedy algorithm's behavior, which shows extreme fluctuations ranging from 6 to 105 positions. This erratic pattern suggests that KMeans-detected clusters are highly sensitive to market noise and potentially capture transient correlations rather than fundamental relationships. The substantial divergence between Greedy and Stable algorithms under KMeans further underscores the method's instability, as even minor variations in cluster selection criteria lead to dramatically different trading decisions.
In stark contrast, the LLM-based approach demonstrates remarkably more coherent and stable position management. Both Greedy and Stable algorithms maintain more closely aligned position counts, typically ranging between 20 and 75 positions, with highly correlated temporal movements. This convergence in behavior between algorithms suggests that LLM-identified clusters capture more fundamental and persistent market relationships. Particularly telling is the test period performance, where KMeans exhibits increased position volatility and extreme spikes, while the LLM approach maintains consistent position patterns across both algorithms. This stability in the out-of-sample period provides strong evidence that LLM-derived signals, grounded in economic analysis of firm-specific shocks, generalize more effectively to unseen data.

%----------------------------------------------------
\input{tab_Trading_Intensity_Comparison.tex}
%----------------------------------------------------

% Analysis of Trading Intensity Metrics (Table1) %
The trading intensity metrics provide quantitative validation of the structural differences between KMeans and LLM clustering approaches. Under KMeans, the dramatic disparity between Greedy and Stable algorithms (averaging 40.1 versus 10.77 positions, with standard deviations of 18.59 and 6.41 respectively) reflects the method's fundamental instability. More concerning is the Stable algorithm's exceptionally high Changes/Position ratio (3.228 versus 0.798 for Greedy), indicating frequent position adjustments necessitated by the transient nature of KMeans-identified clusters.
The LLM implementation demonstrates substantially more balanced and stable metrics across both algorithms. Average position counts converge (31.8 for Greedy, 26.61 for Stable) with more moderate standard deviations (14.84 and 12.16), suggesting that both aggressive and conservative cluster selection approaches identify similar, fundamentally-driven trading opportunities. The more balanced Changes/Position ratios (1.234 and 1.473) and consistent turnover rates (approximately 39\% for both algorithms) indicate that LLM-identified clusters require less frequent rebalancing, supporting the hypothesis that they capture more persistent market relationships.
These patterns become particularly pronounced in the test period, where KMeans shows increased turnover (reaching 39.30\% for Stable) and position volatility, while the LLM approach maintains more stable trading activity (37.56\% and 37.85\% turnover for Greedy and Stable). This superior out-of-sample stability provides compelling evidence that LLM's economic approach to cluster identification produces more robust and generalizable trading signals compared to the purely statistical approach of KMeans.


%% OPTION 1 %
%The evolution of open positions reveals striking differences in trading patterns between the KMeans and LLM-based approaches. In Panel A, the KMeans-based strategy exhibits high volatility in the number of open positions, particularly for the Greedy algorithm, which fluctuates dramatically between 6 and 105 positions. The Stable algorithm maintains a consistently lower and more controlled position count (ranging from 0 to 30), suggesting a more conservative approach to cluster selection. This substantial disparity between algorithms points to potential instability in the KMeans clustering structure, as the Greedy algorithm appears highly sensitive to local market conditions.
%
%In contrast, Panel B demonstrates that the LLM-based approach achieves a more balanced and coordinated trading pattern between the Greedy and Stable algorithms. Both algorithms maintain similar position ranges (typically between 20 and 75 positions) and exhibit highly correlated movements, suggesting that the LLM clustering produces more robust and consistent trading signals. This convergence in behavior between Greedy and Stable algorithms indicates that the LLM clusters capture more fundamental and persistent market patterns, rather than spurious relationships that might be detected by the KMeans approach.
%
%Notably, the test period (shaded area) reveals particularly telling differences. The KMeans strategy shows increased volatility in position counts during this period, with the Greedy algorithm experiencing extreme spikes above 100 positions, while the Stable algorithm remains notably subdued. This divergence in behavior might explain the strategy's poor out-of-sample performance. Conversely, the LLM strategy maintains more stable and coordinated position counts during the test period, with both algorithms showing similar patterns and ranges. This stability in trading intensity aligns with the strategy's superior out-of-sample performance and suggests that the LLM clusters provide more reliable trading signals that generalize better to unseen data.
%
%The temporal stability of position counts in the LLM approach also has practical implications for transaction costs and portfolio management. The more consistent position counts and coordinated behavior between algorithms suggest lower turnover and more manageable trading costs compared to the erratic position changes observed in the KMeans strategy.
%
%
%
%% OPTION 2 %
%The evolution of open positions reveals striking differences in trading patterns between the KMeans and LLM-based approaches, with important implications for trading costs and portfolio performance. In Panel A, the KMeans-based strategy exhibits high volatility in position counts, particularly for the Greedy algorithm, which averages 40.1 positions with a substantial standard deviation of 18.59 positions, ranging dramatically from 6 to 105 positions. The Stable algorithm maintains a markedly lower and more controlled position count (averaging 10.77 positions with a standard deviation of 6.41), suggesting a more conservative approach to cluster selection.
%
%In contrast, Panel B demonstrates that the LLM-based approach achieves more balanced trading patterns between the Greedy and Stable algorithms. The Greedy algorithm averages 31.8 positions (std. 14.84) while the Stable algorithm maintains a similar 26.61 positions (std. 12.16), with their ranges largely overlapping. This convergence in behavior between algorithms suggests that the LLM clustering produces more robust and consistent trading signals.
%
%These differences in trading intensity have direct implications for trading costs and portfolio efficiency. Despite maintaining fewer positions, the KMeans Stable algorithm exhibits a higher Changes/Position ratio (3.228 vs 0.798 for Greedy), indicating more frequent position adjustments that lead to higher turnover. The LLM approach shows more balanced Changes/Position ratios between Greedy and Stable algorithms (1.234 and 1.473 respectively), with turnover rates that remain relatively stable across all periods.
%
%Particularly notable is the test period performance, where the divergence in trading patterns becomes most pronounced. While the KMeans strategy shows increased position volatility and trading costs (reaching a turnover of 39.30\% for Stable), the LLM strategy maintains more consistent position counts and lower trading costs (37.56\% and 37.85\% turnover for Greedy and Stable respectively). This stability in trading activity helps explain the superior out-of-sample performance of the LLM approach, as evidenced by its better net returns and risk-adjusted metrics in the test period.

%%----------------------------------------------------
%\begin{figure}[htbp]
%\caption{Evolution of Open Positions: KMeans vs LLM Clustering}
%\label{fig:open_positions_comparison}
%
%\begin{subfigure}[t]{0.49\textwidth}
%\caption{Panel A: KMeans Clustering}
%\centering
%\includegraphics[width=\textwidth]{fig_KMeans_Open_Positions.pdf}
%\end{subfigure}
%\hfill
%\begin{subfigure}[t]{0.49\textwidth}
%\caption{Panel B: LLM Clustering}
%\centering
%\includegraphics[width=\textwidth]{fig_LLAMA_Open_Positions.pdf}
%\end{subfigure}
%
%\vspace{0.5cm}
%\begin{minipage}{\textwidth}
%\setlength{\parindent}{0pt}
%{\footnotesize\textit{Note:} 
%This figure shows the daily evolution of the number of open positions for both Greedy (blue) and Stable (green) algorithms across different data splits (Train, Validation, Test) using KMeans clustering (Panel A) and LLM clustering (Panel B). The time period spans from July 2020 to September 2021. Vertical dashed lines separate the different data splits. The Greedy algorithm selects clusters that maximize (minimize) the cluster-average-$SR$ for long (short) positions, while the Stable algorithm minimizes the rank difference between training and validation rankings. The number of traded clusters is $\theta = 0.5k=13$ for KMeans ($k^*=26$ clusters) and $\theta = 0.5k=10$ for LLM ($k^*=20$ clusters).
%}
%\end{minipage}
%\end{figure}
%%----------------------------------------------------

%----------------------------------------------------
\input{tab_Portfolio_Statistics_Comparison_NET.tex}
%----------------------------------------------------
Finally, the introduction of trading costs significantly impacts the performance metrics of both clustering approaches, though with notably different implications for their practical viability. The KMeans-based strategy exhibits substantial performance degradation, particularly evident in the test period where both algorithms generate significant losses (Greedy: -20.0\%, Stable: -23.6\% average annual returns). This deterioration is accompanied by elevated risk metrics, with the Stable algorithm showing particularly concerning characteristics including high standard deviation (14.2\%) and extreme kurtosis (14.59) in the test period, suggesting frequent occurrence of extreme returns.
In contrast, the LLM-based approach demonstrates superior resilience to trading costs, maintaining more stable performance characteristics across all periods. Most notably, in the test period, the strategy achieves near-neutral to positive performance (Greedy: -1.5\%, Stable: +3.1\% annual returns) with substantially lower risk metrics (standard deviations of 6.2\% and 7.0\% respectively). The LLM approach's more moderate VaR and CVaR measures (around -8.2\% to -12.4\% in the test period) compared to KMeans (-8.9\% to -28.5\%) further underscore its superior risk management characteristics under transaction costs.
This stark contrast in net performance can be attributed to the fundamentally different nature of the signals generated by each approach. While KMeans' statistically-driven clusters require frequent rebalancing that amplifies transaction costs, the LLM's economically-motivated clusters appear to identify more persistent price patterns that remain profitable even after accounting for trading frictions. However, it is worth noting that neither approach was explicitly optimized for transaction cost efficiency, suggesting potential for further improvement through cost-aware portfolio construction. These results highlight that while our LLM-based news parser successfully captures predictable market reactions to news articles, practitioners implementing such strategies would benefit from incorporating transaction costs into their optimization framework.

%As we can see, given that the trading intensity is really high, the profitability of the trading strategy out of sample is reduced substantially. The trading strategy has not been optimized to account for trading costs and is therefore sensitive to them. The objective of the trading strategy was to identify winners and losers by reading news articles with the sole purpose of anticipating market trends. The exercise we did was about discovering the predictability of market reactions to news articles by designing an LLM-based news parser. A trader interested in this application would prefer to optimize the trading strategy to make it robust to trading costs. 