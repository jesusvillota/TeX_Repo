\section{Copula-Based Pairs Trading Strategy}

\subsection{Mathematical Framework}

Let $\{X_t\}_{t=1}^T$ and $\{Y_t\}_{t=1}^T$ denote the log-returns of two financial assets (in our implementation, AAPL and MSFT respectively) over time horizon $T$. We assume these time series are strictly stationary and their joint distribution can be modeled using copulas.

\subsubsection{Copula Specification}

By Sklar's theorem, there exists a copula function $C: [0,1]^2 \rightarrow [0,1]$ such that the joint distribution $F$ of $(X_t, Y_t)$ can be written as:
\begin{equation*}
F(x,y) = C(F_X(x), F_Y(y))
\end{equation*}

where $F_X$ and $F_Y$ are the marginal cumulative distribution functions of $X_t$ and $Y_t$ respectively.

We consider five different copula specifications:

\begin{itemize}
\item \textsc{Elliptical Copulas}

\begin{itemize}
    \item \textbf{Gaussian Copula:}
\begin{equation*}
	C_{\rho}(u,v) = \Phi_{\rho}(\Phi^{-1}(u), \Phi^{-1}(v))
\end{equation*}
    where $\Phi_{\rho}$ is the bivariate standard normal CDF with correlation $\rho$ and $\Phi^{-1}$ is the inverse standard normal CDF.
    \item \textbf{Student-t Copula:}
\begin{equation*}
	C_{\rho,\nu}(u,v) = t_{\rho,\nu}(t_{\nu}^{-1}(u), t_{\nu}^{-1}(v))
\end{equation*}
    where $t_{\rho,\nu}$ is the bivariate Student-t CDF with correlation $\rho$ and $\nu$ degrees of freedom.
\end{itemize}
\end{itemize}

\begin{itemize}
	\item \textsc{Archimedean Copulas}
\begin{itemize}
    \item \textbf{Clayton Copula:}
\begin{equation*}
	C_{\theta}(u,v) = (u^{-\theta} + v^{-\theta} - 1)^{-1/\theta}, \quad \theta > 0
\end{equation*}

    \item \textbf{Gumbel Copula:}
\begin{equation*}
	C_{\theta}(u,v) = \exp\{-[(-\ln u)^{\theta} + (-\ln v)^{\theta}]^{1/\theta}\}, \quad \theta \geq 1
\end{equation*}

    \item \textbf{Frank Copula:}
\begin{equation*}
	C_{\theta}(u,v) = -\frac{1}{\theta}\ln\left(1 + \frac{(e^{-\theta u}-1)(e^{-\theta v}-1)}{e^{-\theta}-1}\right), \quad \theta \neq 0
\end{equation*}
\end{itemize}
\end{itemize}


\subsubsection{Model Selection}

For model selection, we employ information criteria to choose the most appropriate copula specification. Given a fitted copula with parameter(s) $\boldsymbol{\theta}$, we compute:
\begin{align*}
\text{AIC} &= -2\ln L(\boldsymbol{\theta}) + 2k \\
\text{BIC} &= -2\ln L(\boldsymbol{\theta}) + k\ln(n) \\
\text{HQIC} &= -2\ln L(\boldsymbol{\theta}) + 2k\ln(\ln(n))
\end{align*}

where $L(\boldsymbol{\theta})$ is the likelihood function, $k$ is the number of parameters, and $n$ is the sample size.

\subsubsection{Trading Strategy}
The trading strategy is based on the conditional expectation of one asset's returns given the other's. For a given copula $C$, we define:
\begin{equation*}
	\mathbb{E}[V|U=u] = \int_0^1 h(v|u)dv
\end{equation*}

where $h(v|u)$ is the conditional density derived from the copula:
\begin{equation*}
	h(v|u) = \frac{\partial C(u,v)}{\partial u} = c(u,v)
\end{equation*}

For specific copulas, this conditional expectation has closed forms. For example, for the Clayton copula:
\begin{equation*}
	\mathbb{E}[V|U=u] = (1 + \theta u^{-\theta})^{-1/\theta}
\end{equation*}


The trading signals are generated based on the deviation from this conditional expectation. Let $\epsilon_t$ denote the normalized residual at time $t$:

\red{This introuces lookahead bias! -> at time $t$, you don't know what the maximum or minimum of the whole residual series is going to be! In other words, 

$$\begin{array}{ll}
\max(residuals), \min(residuals)\in \F_T
\\
\eps_t = f(v_t, \E[V|U=u_t]) \in \F_t
\end{array}$$

}
\begin{equation*}
	\epsilon_t = \frac{v_t - \mathbb{E}[V|U=u_t] - \min(\text{residuals})}{\max(\text{residuals}) - \min(\text{residuals})}
\end{equation*}

For a given threshold $\alpha$, the trading signals $s_t$ are defined as:
\begin{equation*}
	s_t = \begin{cases}
1 & \text{if } \epsilon_t < \alpha \\
-1 & \text{if } \epsilon_t > 1-\alpha \\
0 & \text{otherwise}
\end{cases}
\end{equation*}

where $s_t = 1$ indicates a long position in MSFT and short in AAPL, and $s_t = -1$ indicates the opposite position.

\red{
Important: check that you're not introducing Lookahead bias in the computation!
Note that $s_t$ requires information up to time $t$ to be computed. In other words, $s_t\in \F_t$. Thus, we cannot get $r_t$ since $r_t$ is produced by taking a position at time $t-1$ and cancelling it at time $t$ (i.e: $r_t = P_t/P_{t-1} -1$. Therefore, if I oberve $s_t$ at time $t$, I cannot possibly receive $r_t$, as this means that I took a position at time $t-1$, that is, before even observing the signal!. That is wrong. 
Hence, a $s_t$ observed signal should translate into getting $r_{t+1}$, as you: 
\begin{enumerate}
	\item Observe $s_t$ at time $t$	
	\item Place a trade at the closing price of $t$ or at the open price of $t+1$
	\item Close the trade at the closing price of $t+1$
\end{enumerate}

}


\subsubsection{Portfolio Value and Performance Metrics}

The portfolio value $P_t$ evolves according to:
\begin{equation*}
	P_t = P_{t-1}(1 + s_{t-1}(r_{Y,t} - r_{X,t}))
\end{equation*}
where $r_{X,t}$ and $r_{Y,t}$ are the returns of AAPL and MSFT respectively at time $t$.
