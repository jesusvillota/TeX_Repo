\subsection{KMeans Algorithm}
%----------------------------------------------------
% Alternative 1: Algorithmic Setup
\input{alg_KMeans_1.tex}
%----------------------------------------------------
% Alternative 2: More organized setup
%\section*{KMeans Clustering Algorithm}

\subsection*{Inputs}
\begin{itemize}
    \item Embedding vectors: $\{\mathbf{e}^1, \mathbf{e}^2, \ldots, \mathbf{e}^{N_{tr}}\}$
    \item Number of clusters: $k$
\end{itemize}

\subsection*{Outputs}
\begin{itemize}
    \item Cluster assignments: $\{C_1, C_2, \ldots, C_k\}$
    \item Centroids: $\{\mathbf{c}_1, \mathbf{c}_2, \ldots, \mathbf{c}_k\}$
\end{itemize}

\subsection*{Algorithm}
\begin{enumerate}
    \item \textbf{Initialize} centroids $\{\mathbf{c}_1, \mathbf{c}_2, \ldots, \mathbf{c}_k\}$ randomly.
    \item \textbf{Repeat until convergence:}
    \begin{enumerate}
        \item \textbf{Assignment Step:}
        \[
        C_j = \left\{ \mathbf{e}^i \mid j = \arg \min_{l} \|\mathbf{e}^i - \mathbf{c}_l\|^2, \quad l = 1, 2, \ldots, k \right\}
        \]
        \item \textbf{Update Step:}
        \[
        \mathbf{c}_j = \frac{1}{|C_j|} \sum_{\mathbf{e}^i \in C_j} \mathbf{e}^i, \quad \forall j = 1, 2, \ldots, k
        \]
    \end{enumerate}
    \item \textbf{Convergence Criterion:} Repeat steps 2(a) and 2(b) until the cluster assignments do not change, i.e.,
    \[
    C_j^{(t+1)} = C_j^{(t)}, \quad \forall j = 1, 2, \ldots, k
    \]
    where $t$ denotes the iteration number.
\end{enumerate}

\subsection*{Objective}
The KMeans algorithm aims to minimize the within-cluster sum of squares (WCSS):
\[
\min_{C, \mathbf{c}} \sum_{j=1}^k \sum_{\mathbf{e}^i \in C_j} \|\mathbf{e}^i - \mathbf{c}_j\|^2
\]


%----------------------------------------------------

\newpage
\subsection{Hyperparameter Choice}
Our hyperparameters are $L$ and $\theta$. Recall that $L$ denotes the number of trading days over which we hold the positions in the beta-neutral strategy, while $\theta$ represents the upper bound on each side (long and short) for the amount of clusters we select for the trading strategy. The specific choice of hyperparameters we made for the results presented in the paper were:
\begin{align*}
L &= 4
\\
\theta &= \integer{0.5k}
\end{align*}
where $k$ represents the number of clusters (26 for KMeans clustering, and 20 for LLM clustering). This choice is not arbitrary nor opportunistic. Instead, it results from the maximization of the Sharpe Ratio of the portfolio in the train and validation samples for both KMeans and LLM clustering. This choice procedure is completely based on \textit{in-sample} criteria and it prevents lookahead bias. The justification for such choices is made below.

\subsubsection{KMeans Clustering}

In \cref{fig:KMeans_hyperparameter_justification_L} we can see that a choice of $L=4$ in the training and validation splits generates the most stable Sharpe Ratio. Namely, In the train set (\cref{fig:K_hyp_1}), it makes more sense to choose low values of $L$ (less than 4) to maximize the $SR$. However, in the validation set (\cref{fig:K_hyp_2}), it makes more sense to choose higher values of $L$. The choice of $L=4$ stands as a middle ground between this contradiction, generating a stable choice and a stable profile of earnings in sample.

%----------------------------------------------------
\begin{figure}[H]
  \caption{Sharpe Ratios in the train and validation splits as a function of $L$}
  \centering
  
  \begin{subfigure}[b]{0.46\textwidth}
    \centering
    \includegraphics[width=\textwidth]{/Users/jesusvillotamiranda/Library/CloudStorage/OneDrive-UniversidaddeLaRioja/CEMFI/Rest/__Second_year__/MasterThesis/__Output/KMeans_RobustnessCheck_SR_Train_Set_vs_L_[Change_L].pdf}
    \caption{Plot of $SR^{\mathcal P^{tr}}(L)$ over a grid of $L$}
    \label{fig:K_hyp_1}
  \end{subfigure}
  \hspace{0.05\textwidth} % Add horizontal space between the subfigures
  \begin{subfigure}[b]{0.46\textwidth}
    \centering
    \includegraphics[width=\textwidth]{/Users/jesusvillotamiranda/Library/CloudStorage/OneDrive-UniversidaddeLaRioja/CEMFI/Rest/__Second_year__/MasterThesis/__Output/KMeans_RobustnessCheck_SR_Validation_Set_vs_L_[Change_L].pdf}
    \caption{Plot of $SR^{\mathcal P^{val}}(L)$ over a grid of $L$}
    \label{fig:K_hyp_2}
  \end{subfigure}  
  \label{fig:KMeans_hyperparameter_justification_L}
\end{figure}
%----------------------------------------------------

On the other hand, the choice of $\theta=\integer{0.5\cd 26}=13$ is a choice that pursues stability in the Sharpe Ratio of the train and validation portfolios. As we can see from \cref{fig:KMeans_hyperparameter_justification_theta}, the Sharpe Ratios tend to converge to the highest and most stable value when we choose the highest possible value of $\theta$. 

 %----------------------------------------------------
\begin{figure}[H]
  \caption{Sharpe Ratios in the train and validation splits as a function of $\theta$}
  \centering
    \begin{subfigure}[b]{0.46\textwidth}
    \centering
    \includegraphics[width=\textwidth]{/Users/jesusvillotamiranda/Library/CloudStorage/OneDrive-UniversidaddeLaRioja/CEMFI/Rest/__Second_year__/MasterThesis/__Output/KMeans_RobustnessCheck_SR_Train_Set_vs_theta_[Change_theta].pdf}
    \caption{Plot of $SR^{\mathcal P^{tr}}(\theta)$ over a grid of $\theta$}
    \label{fig:K_hyp_3}
  \end{subfigure}
  \hspace{0.05\textwidth} % Add horizontal space between the subfigures
  \begin{subfigure}[b]{0.46\textwidth}
    \centering
    \includegraphics[width=\textwidth]{/Users/jesusvillotamiranda/Library/CloudStorage/OneDrive-UniversidaddeLaRioja/CEMFI/Rest/__Second_year__/MasterThesis/__Output/KMeans_RobustnessCheck_SR_Validation_Set_vs_theta_[Change_theta].pdf}
    \caption{Plot of $SR^{\mathcal P^{val}}(\theta)$ over a grid of $\theta$}
    \label{fig:K_hyp_4}
  \end{subfigure}
  \label{fig:KMeans_hyperparameter_justification_theta}
\end{figure}
%----------------------------------------------------


\subsubsection{LLM Clustering}
Following a similar logic as below, the choice of $L=4$ sets a consensus between the maximization of $SR^{\mathcal P^{tr}}$ and $SR^{\mathcal P^{val}}$. That is, maximizing $SR^{\mathcal P^{tr}}$ requires lower holding period lengths (the maximizer is $L=4$), while maximizing $SR^{\mathcal P^{val}}$ requires increasing the window length. Among this contradiction, $L=4$ standing as a perfect choice to balance the maximization requirements in both samples, generating a stable choice for the holding period window length.

%----------------------------------------------------
\begin{figure}[H]
  \caption{Sharpe Ratios in the train and validation splits as a function of hyperparameters}
  \centering
  
  \begin{subfigure}[b]{0.46\textwidth}
    \centering
    \includegraphics[width=\textwidth]{/Users/jesusvillotamiranda/Library/CloudStorage/OneDrive-UniversidaddeLaRioja/CEMFI/Rest/__Second_year__/MasterThesis/__Output/LLAMA_RobustnessCheck_SR_Train_Set_vs_L_[Change_L].pdf}
    \caption{Plot of $SR^{\mathcal P^{tr}}(L)$ over a grid of $L$}
    \label{fig:LLM_hyp_1}
    
  \end{subfigure}
  \hspace{0.05\textwidth} % Add horizontal space between the subfigures
  \begin{subfigure}[b]{0.46\textwidth}
    \centering
    \includegraphics[width=\textwidth]{/Users/jesusvillotamiranda/Library/CloudStorage/OneDrive-UniversidaddeLaRioja/CEMFI/Rest/__Second_year__/MasterThesis/__Output/LLAMA_RobustnessCheck_SR_Validation_Set_vs_L_[Change_L].pdf}
    \caption{Plot of $SR^{\mathcal P^{val}}(L)$ over a grid of $L$}
    \label{fig:LLM_hyp_2}
  \end{subfigure}
  
  \label{fig:LLM_hyperparameter_justification_L}
\end{figure}
%----------------------------------------------------

Finally, the same conclusion as in KMeans applies here. By selecting $\theta=\integer{0.5\cd 20}=10$, we get a stable Sharpe Ratio. Even though we observe that $SR^{\mathcal P^{tr}}(L)$ falls momentarily at $\theta=10$ for the Greedy algorithm, it still constitutes a good choice. Conversely, at $\theta=10$ the greedy algorithm sees a jump in $SR^{\mathcal P^{val}}(L)$. All in all, we can easily conclude that $\theta=\integer{0.5k}$ arises as a good hyperpamrameter choice also for LLM clustering.

\begin{figure}[H]
  \centering

    \begin{subfigure}[b]{0.46\textwidth}
    \centering
    \includegraphics[width=\textwidth]{/Users/jesusvillotamiranda/Library/CloudStorage/OneDrive-UniversidaddeLaRioja/CEMFI/Rest/__Second_year__/MasterThesis/__Output/LLAMA_RobustnessCheck_SR_Train_Set_vs_Theta_[Change_theta].pdf}
    \caption{Plot of $SR^{\mathcal P^{tr}}(\theta)$ over a grid of $\theta$}
    \label{fig:LLM_hyp_3}
  \end{subfigure}
  \hspace{0.05\textwidth} % Add horizontal space between the subfigures
  \begin{subfigure}[b]{0.46\textwidth}
    \centering
    \includegraphics[width=\textwidth]{/Users/jesusvillotamiranda/Library/CloudStorage/OneDrive-UniversidaddeLaRioja/CEMFI/Rest/__Second_year__/MasterThesis/__Output/LLAMA_RobustnessCheck_SR_Validation_Set_vs_Theta_[Change_theta].pdf}
    \caption{Plot of $SR^{\mathcal P^{val}}(\theta)$ over a grid of $\theta$}
    \label{fig:LLM_hyp_4}
  \end{subfigure}

\label{fig:LLM_hyperparameter_justification_theta}
\end{figure}





\newpage
%%%%%%%%%%%%%%%%%%%%%%%%%%%%%%%%%%%%%%%%%%%%%%%%%%%%%
\subsection{Optimal Cluster Selection Algorithms}
%%%%%%%%%%%%%%%%%%%%%%%%%%%%%%%%%%%%%%%%%%%%%%%%%%%%%
%----------------------------------------------------
\begin{algorithm}
\caption{
\textsc{Greedy Selection} 
~|~
{{Top average Sharpe Ratio in Validation Set}}
}
%%%%%%%%%%%%%%%%%%%%%%%%%%%%%%%%%%%%%%%%%%%%%%%%%%%%%
\label{alg:greedy_selection}
%%%%%%%%%%%%%%%%%%%%%%%%%%%%%%%%%%%%%%%%%%%%%%%%%%%%%
\begin{algorithmic}[1]
\mx 
\State \textbf{Input:} Set of clusters $\mathcal{G} = \{1, 2, \ldots, k^*\}$, Sharpe Ratios in the validation sample $\{SR_L^{(i,j)}\}_{(i,j)\in \mathcal B^{val}}$, maximum number of traded clusters $\theta\in\mathbb{N}$ (usually, $\theta\propto k^*)$

\mx 
\State \textbf{Output:} Set of long-traded clusters $\mathcal{G}_{\theta}^{+}$ and set of short-traded clusters $\mathcal{G}_{\theta}^{-}$
%----------------------------------------------------
%\Statex
\vspace{0.4cm}
\Statex \underline{\textit{Step \#1: Compute Cluster Average Sharpe Ratios in Validation Set}}
\For{each $g \in \mathcal{G}$}
    \State Compute average Sharpe Ratio ~
$
\overline{S R}_g^{val} \leftarrow \frac{1}{|\mathcal{B}_g^{val} |} \sum_{(i,j) \in \mathcal{B}_g^{val}} S R_{{{L}}}^{(i,j)}
$
\EndFor
%----------------------------------------------------
%\Statex
\vspace{0.4cm}
\Statex \underline{\textit{Step \#2: Identify Positive and Negative Sharpe Ratio Clusters}}
\State Define $\mathcal{G}_{SR^+}^{val} \leftarrow \{ g \in \mathcal{G} \mid \overline{SR}_g^{val} > 0 \}$
\State Define $\mathcal{G}_{SR^-}^{val} \leftarrow \{ g \in \mathcal{G} \mid \overline{SR}_g^{val} < 0 \}$
%----------------------------------------------------
\vspace{0.4cm}
\Statex \underline{\textit{Step \#3: Rank Clusters by Average Sharpe Ratio in the Validation Set}}
\For{each $g \in \mathcal{G}$}
	\State Rank the average Sharpe Ratio~~
$
\mathfrak{R}_g^{val} \leftarrow  \sum_{h \in \mathcal{G}} 
\mathbf{1}\1{
\overline{S R}_h^{val} \geq \overline{S R}_g^{val} 
}
$
\EndFor
%\State Sort clusters in descending order of $\overline{SR}_g^{val}$
%%\State
%$$\overline{SR}_{\varkappa_1}^{val} \geq \overline{SR}_{\varkappa_2}^{val} \geq \ldots \geq \overline{SR}_{\varkappa_{k^*}}^{val}$$
%----------------------------------------------------
%\Statex
\vspace{0.4cm}
\Statex \underline{\textit{Step \#4: Select Top $\theta$ Clusters}}
\State Define $\theta^+ \leftarrow \min(\theta, |\mathcal{G}_{SR^+}^{val}|)$
;~~
$\mathcal{G}_{\theta}^{+} \leftarrow \{ g\in\G \mid 1 \leq \mathfrak{R}_g^{val} \leq \theta^+ \}$
%\State 

\State Define $\theta^- \leftarrow \min(\theta, |\mathcal{G}_{SR^-}^{val}|)$
;~~
%\State 
 $\mathcal{G}_{\theta}^{-} \leftarrow \{ g \in\G \mid k^* - \theta^- < \mathfrak{R}_g^{val} \leq k^* \}$
%----------------------------------------------------
%\Statex
\vspace{0.5cm}
\State \textbf{Return} Long-traded clusters $\mathcal{G}_{\theta}^{+}$, Short-traded clusters $\mathcal{G}_{\theta}^{-}$

\end{algorithmic}
\end{algorithm}


%
%\begin{algorithm}[H]
%\caption{Greedy Selection of Clusters Based on Average Sharpe Ratio}
%%%%%%%%%%%%%%%%%%%%%%%%%%%%%%%%%%%%%%%%%%%%%%%%%%%%%%
%\label{alg:greedy_selection}
%%%%%%%%%%%%%%%%%%%%%%%%%%%%%%%%%%%%%%%%%%%%%%%%%%%%%%
%\begin{algorithmic}[1]
%\State \textbf{Input:} Set of clusters $\mathcal{G} = \{1, 2, \ldots, k^*\}$, Sharpe Ratios in the validation sample $\{SR_L^{(i,j)}\}_{(i,j)\in \mathcal{B}^{val}}$, maximum number of traded clusters $\theta \in \mathbb{N}$ (usually, $\theta \propto k^*$)
%\State \textbf{Output:} Set of long-traded clusters $\mathcal{G}_{\theta}^{+}$ and set of short-traded clusters $\mathcal{G}_{\theta}^{-}$
%
%\Statex
%\Statex \underline{\textit{Step \#1: Compute Cluster Average Sharpe Ratios in Validation Set}}
%\For{each $g \in \mathcal{G}$}
%    \State Compute average Sharpe Ratio ~
%    \[
%    \overline{SR}_g^{val} \leftarrow \frac{1}{|\mathcal{B}_g^{val} |} \sum_{(i,j) \in \mathcal{B}_g^{val}} SR_{L}^{(i,j)}
%    \]
%\EndFor
%
%\Statex
%\Statex \underline{\textit{Step \#2: Identify Positive and Negative Sharpe Ratio Clusters}}
%\State Define $\mathcal{G}_{SR^+}^{val} \leftarrow \{ g \in \mathcal{G} \mid \overline{SR}_g^{val} > 0 \}$
%\State Define $\mathcal{G}_{SR^-}^{val} \leftarrow \{ g \in \mathcal{G} \mid \overline{SR}_g^{val} < 0 \}$
%
%\Statex
%\Statex \underline{\textit{Step \#3: Rank Clusters by Average Sharpe Ratio}}
%\State Sort clusters in descending order of $\overline{SR}_g^{val}$
%%\State
%\[
%\overline{SR}_{\varkappa_1}^{val} \geq \overline{SR}_{\varkappa_2}^{val} \geq \ldots \geq \overline{SR}_{\varkappa_{k^*}}^{val}
%\]
%
%\Statex
%\Statex \underline{\textit{Step \#4: Select Top $\theta$ Clusters}}
%\State Define $\theta^+ \leftarrow \min(\theta, |\mathcal{G}_{SR^+}^{val}|)$
%\State Define $\theta^- \leftarrow \min(\theta, |\mathcal{G}_{SR^-}^{val}|)$
%
%\State Define $\mathcal{G}_{\theta}^{+} \leftarrow \{ \varkappa_{\ell} \in \mathcal{G} \mid 1 \leq \ell \leq \theta^+ \}$
%\State Define $\mathcal{G}_{\theta}^{-} \leftarrow \{ \varkappa_{\ell} \in \mathcal{G} \mid k^* - \theta^- < \ell \leq k^* \}$
%
%\Statex
%\State \textbf{Return} Long-traded clusters $\mathcal{G}_{\theta}^{+}$, Short-traded clusters $\mathcal{G}_{\theta}^{-}$
%
%\end{algorithmic}
%\end{algorithm}

%----------------------------------------------------


%----------------------------------------------------
\input{alg_trading_cluster_selection_rank.tex}
%----------------------------------------------------




%%%%%%%%%%%%%%%%%%%%%%%%%%%%%%%%%%%%%%%%%%%%%%%%%%%%%
\subsection{Sample of articles for each cluster}
%%%%%%%%%%%%%%%%%%%%%%%%%%%%%%%%%%%%%%%%%%%%%%%%%%%%%
\setcounter{table}{0}
\renewcommand{\thetable}{A\arabic{table}} % To ensure the tables in the appendix are numbered as A1, A2, etc.

%%----------------------------------------------------
\input{tab_KMeans_Cluster_Articles_3_English.tex}
%%----------------------------------------------------

%----------------------------------------------------
\input{tab_LLM_Cluster_Articles_3_English.tex}
%----------------------------------------------------




%%%%%%%%%%%%%%%%%%%%%%%%%%%%%%%%%%%%%%%%%%%%%%%%%%%%%
\subsection{Function Calling with LLaMA-3}
%%%%%%%%%%%%%%%%%%%%%%%%%%%%%%%%%%%%%%%%%%%%%%%%%%%%%
%----------------------------------------------------
\definecolor{lightgray}{gray}{0.6} % Define a light gray color

\begin{algorithm}[H]
\caption{Function Calling Workflow for LLaMA-3}
%%%%%%%%%%%%%%%%%%%%%%%%%%%%%%%%%%%%%%%%%%%%%%%%%%%%%
\label{alg:function_calling}
%%%%%%%%%%%%%%%%%%%%%%%%%%%%%%%%%%%%%%%%%%%%%%%%%%%%%
\begin{algorithmic}[1]
\Require $\D$: Dataset of news articles
\Ensure Structured JSON output for each article
\State Initialize LLaMA-3 model via GroqCloud API
\For{each article $i \in \D$} \Comment{\scalebox{0.9}{\textcolor{lightgray}{Iterate over each article in the dataset}}}
%    \State Set up system message with instructions for LLM
    \State \textbf{Message: System} \Comment{\scalebox{0.9}{\textcolor{lightgray}{Define the role and task for the LLM}}}
%        \Statex \hspace{1cm} You are a function calling LLM that analyzes business news in Spanish. For every article, 
%
%\hspace{0.3cm} identify the firms that are directly affected by the news and classify the shocks in type, 
%
%\hspace{0.3cm} magnitude and direction

		\begin{quote}
			\qquote{You are a function calling LLM that analyzes business news in Spanish. For every article, identify the firms that are directly affected by the news and classify the shocks in type, magnitude and direction}
		\end{quote}


%as follows:
%    \Statex \hspace{1cm} \textit{Type}: \{demand, supply, financial, policy, technology\}
%    \Statex \hspace{1cm} \textit{Magnitude}: \{minor, major\}
%    \Statex \hspace{1cm} \textit{Direction}: \{positive, negative\}
%    \State Prepare user prompt $P_i$ containing the text of article $i$ \Comment{\scalebox{0.9}{\textcolor{lightgray}{Input the article text}}
    \State \textbf{Message: User} \Comment{\scalebox{0.9}{\textcolor{lightgray}{User provides the article text as input}}}
    \Statex \hspace{1cm} Content: prompt $P_i$ containing the text of article $i$
%    \State Define tools, including the \texttt{news\_parser} function \Comment{\scalebox{0.9}{\textcolor{lightgray}{Specify the functions to be used}}}
    \State \textbf{Tool: news\_parser} \Comment{\scalebox{0.9}{\textcolor{lightgray}{Define the \texttt{news\_parser} function}}}
%    \Statex \hspace{1cm} Parameters: \{firms: array of objects\}
%    \Statex \hspace{1cm} \textit{Each object contains:}
%    
\begin{quote}
\begin{quote}
Parameters: \{\texttt{firms}: \bblue{\texttt{array}} of objects\}, where each object contains:
            \begin{itemize}
                \item \texttt{firm}: \hspace{2.3cm} \bblue{\texttt{string}} (each one firm in \texttt{firms})
                \item \texttt{ticker}: \hspace{1.9cm} \bblue{\texttt{string}} (stock market ticker)
                \item \texttt{shock\_type}: \hspace{1cm} \{demand, supply, financial, policy, technology\}
                \item \texttt{shock\_magnitude}: \hspace{0cm} \{minor, major\}
                \item \texttt{shock\_direction}: \hspace{0.cm} \{positive, negative\}
            \end{itemize} 
\end{quote} 
\end{quote} 
 
     
%    \Statex \hspace{1cm} - \texttt{firm}: string (publicly listed Spanish firm)
%    \Statex \hspace{1cm} - \texttt{ticker}: string (e.g., TICKER.MC)
%    \Statex \hspace{1cm} - \texttt{shock\_type}: \{demand, supply, financial, policy, technology\}
%    \Statex \hspace{1cm} - \texttt{shock\_magnitude}: \{minor, major\}
%    \Statex \hspace{1cm} - \texttt{shock\_direction}: \{positive, negative\}
    \State Send initial messages and tools to LLaMA-3 \Comment{\scalebox{0.9}{\textcolor{lightgray}{Initiate interaction with the LLM}}}
    \State Retrieve response from LLaMA-3 \Comment{\scalebox{0.9}{\textcolor{lightgray}{Get the initial response from the LLM}}}
    \If{Function call is requested by LLaMA-3} \Comment{\scalebox{0.9}{\textcolor{lightgray}{Check if the LLM needs to call a function}}}
        \State Execute \texttt{news\_parser} function with provided arguments \Comment{\scalebox{0.9}{\textcolor{lightgray}{Run the function}}}
        \State Append function response to the conversation \Comment{\scalebox{0.9}{\textcolor{lightgray}{Include function output in the dialogue}}}
        \State Send updated messages to LLaMA-3 \Comment{\scalebox{0.9}{\textcolor{lightgray}{Continue the conversation with new information}}}
        \State Retrieve final response from LLaMA-3 \Comment{\scalebox{0.9}{\textcolor{lightgray}{Get the final output from the LLM}}}
    \EndIf
    \State Extract and store structured JSON output \Comment{\scalebox{0.9}{\textcolor{lightgray}{Save the processed data}}}
\EndFor
\end{algorithmic}
\end{algorithm}

%----------------------------------------------------




%\subsection{Function schema}
%\begin{landscape}
%\input{tab_GPT_functions.tex}
%\end{landscape}

%\begin{landscape}
%\input{code_GPT_functions.tex} 
%\end{landscape}


%
%%-------------- CLUSTER MAPPING --------------------
%\input{/Users/jesusvillotamiranda/Library/CloudStorage/OneDrive-UniversidaddeLaRioja/CEMFI/Rest/__Second_year__/MasterThesis/__Output/LLAMA_Cluster_Mapping.tex}
%%----------------------------------------------------


%%-------------- CLUSTER MAPPING --------------------
%\input{/Users/jesusvillotamiranda/Library/CloudStorage/OneDrive-UniversidaddeLaRioja/CEMFI/Rest/__Second_year__/MasterThesis/__Output/LLAMA_Cluster_Mapping_Extended.tex}
%%----------------------------------------------------